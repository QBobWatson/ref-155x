
\usetikzlibrary{matrix,shapes.geometric,angles}

\titleframe{Chapter 2}{Matrix Algebra}
\titleframe{Section 2.1}{Matrix Operations}


%%%%%%%%%%%%%%%%%%%%%%%%%%%%%%%%%%%%%%%%%%%%%%%%%%%%%%%%%%%%%%%%%%%

\begin{frame}
\frametitle{Motivation}

\alert{Recall:} we can turn any system of linear equations into a matrix
equation
\[ Ax = b. \]
\pause
This notation is suggestive.  Can we solve the equation by ``dividing by A''?
\[ x \overset{??}= \frac bA \]
\pause
\alert{Answer:} Sometimes, but you have to know what you're doing.

\pause\bigskip
Today we'll study \emph{matrix algebra}: adding and multiplying matrices.

\end{frame}


%%%%%%%%%%%%%%%%%%%%%%%%%%%%%%%%%%%%%%%%%%%%%%%%%%%%%%%%%%%%%%%%%%%

\long\def\row#1#2{\hbox to \linewidth{\hss%
    \begin{minipage}[t]{.5\linewidth}#1\end{minipage}\hskip 1em plus 1fill%
    \begin{minipage}[t]{.45\linewidth}\centering#2\end{minipage}\hss}%
  \vskip 1.5ex%
}

\tikzstyle{circle entry} = [draw,rounded corners,thick,blue!50]

\begin{frame}
\frametitle{More Notation for Matrices}

Let $A$ be an $m\times n$ matrix.

\pause\medskip
\row{
  We write $a_{ij}$ for the entry in the $i$th row and the $j$th
  column.  It is called the \textbf{$ij$th entry} of the matrix.
  % NB: A_{ij} is the i,j *minor*
}{
  \begin{tikzpicture}[picture align top]
    \matrix[math matrix]  (aij)
      {
        a_{11} \& \cdots \& a_{1j} \& \cdots \& a_{1n} \\
        \spvdots \&        \& \spvdots \&        \& \spvdots \\
        a_{i1} \& \cdots \& a_{ij} \& \cdots \& a_{in} \\
        \spvdots \&        \& \spvdots \&        \& \spvdots \\
        a_{m1} \& \cdots \& a_{mj} \& \cdots \& a_{mn} \\
      };
      \node[fit=(aij-1-3) (aij-5-3), inner sep=0pt,
          draw=blue!50, thick, rounded corners,
          label={[text=blue!50]below:\small$j$th column}] {};
      \node[fit=(aij-3-1) (aij-3-5), inner sep=0pt,
          draw=green!70!black, thick, rounded corners] (row) {};
      \node[text=green!70!black, rotate=90, anchor=north, yshift=1mm, font=\small]
        at (row.east) {$i$th row};
  \end{tikzpicture}
}\pause
\row{The entries $a_{11},a_{22},a_{33},\ldots$ are the
  \textbf{diagonal entries}; they form the \textbf{main diagonal} of the matrix.
}{
  \begin{tikzpicture}[picture align top]
    \matrix[math matrix]  (aij2)
      {
        \node[circle entry]{a_{11}}; \& a_{12} \& a_{13} \\
        a_{21} \& \node[circle entry]{a_{22}}; \& a_{23} \\
      };
    \matrix[math matrix, right=20pt of aij2] 
      {
        \node[circle entry]{a_{11}}; \& a_{12} \\
        a_{21} \& \node[circle entry]{a_{22}}; \\
        a_{31} \& a_{32} \\
      };
  \end{tikzpicture}
}\pause\vskip 1ex
\row{
  A \textbf{diagonal matrix} is a \emph{square} matrix whose only nonzero
  entries are on the main diagonal.
}{
  \vskip-\aboveheight
  $\mat{a_{11} 0 0; 0 a_{22} 0; 0 0 a_{33}}$
}\pause
\row{
  The $n\times n$ \textbf{identity matrix} $I_n$ is the diagonal matrix with all
  diagonal entries equal to $1$.  It is special because
  $\color{red}I_nv = \blankuntil{6}{v}$ for \emph{all} $v$ in $\R^n$.
}{
  \vskip-\aboveheight
  \uncover<4->{$I_3 = \mat{1 0 0; 0 1 0; 0 0 1}$}
}

\end{frame}


%%%%%%%%%%%%%%%%%%%%%%%%%%%%%%%%%%%%%%%%%%%%%%%%%%%%%%%%%%%%%%%%%%%

\begin{frame}
\frametitle{More Notation for Matrices}
\framesubtitle{Continued}

\row{
  The \textbf{zero matrix} (of size $m\times n$) is the $m\times n$ matrix $0$
  with all zero entries.
}{
  \vskip-\aboveheight
  $0 = \mat{0 0 0; 0 0 0}$
}\pause
\row{
  The \textbf{transpose} of an $m\times n$ matrix $A$ is the $n\times m$ matrix
  $A^T$ whose rows are the columns of $A$.  In other words, the $ij$ entry of
  $A^T$ is $a_{ji}$.
}{
  \begin{tikzpicture}[
      picture align top,
      every matrix/.append style={nodes={shape=regular polygon,regular polygon sides=4,
        inner sep=-.8pt}}
      ]
    \matrix[math matrix, label={[yshift=1mm]above:$A$}] (aij)
      {
        a_{11} \& a_{12} \& a_{13} \\
        a_{21} \& a_{22} \& a_{23} \\
      };
    \matrix[math matrix, right=15mm of aij,
        label={[yshift=1mm]above:$A^T$}] (aijT)
      {
        a_{11} \& a_{21} \\
        a_{12} \& a_{22} \\
        a_{13} \& a_{23} \\
      };
    \draw[->, decorate,
        decoration={
          snake, amplitude=.5mm, segment length=1mm, post length=1mm}] 
      ($(aij.east) + (3mm,0)$) -- ($(aijT.west) - (3.2mm,0)$);
    \draw[green!50!black, opacity=.5, shorten >=-4mm]
      (aij-1-1.north west) -- (aij-2-2.south east)
      coordinate[pos=1.3, below left=3mm] (left)
      coordinate[pos=1.3, above right=3mm] (right)
      node[pos=1.2, below right, opaque] {\small flip};
    \draw[<->, green!50!black] (left) to[bend left] (right);
    \draw[green!50!black, opacity=.5, shorten >=-4mm]
      (aijT-1-1.north west) -- (aijT-2-2.south east);
      
  \end{tikzpicture}
}

\end{frame}


%%%%%%%%%%%%%%%%%%%%%%%%%%%%%%%%%%%%%%%%%%%%%%%%%%%%%%%%%%%%%%%%%%%

\begin{frame}
\frametitle{Addition and Scalar Multiplication}

You add two matrices component by component, like with vectors.
\[ \loopmat23{\append{a_{\i\j}}} + \loopmat23{\append{b_{\i\j}}}
= \loopmat23{\append{a_{\i\j}+b_{\i\j}}} \]
\pause
Note you can only add two matrices \emph{of the same size}.

\pause\bigskip
You multiply a matrix by a scalar by multiplying each component, like with
vectors.
\[ \def\r{\textcolor{red}}
\r c\loopmat23{\append{a_{\i\j}}}
= \loopmat23{\appendnoexp{\textcolor{red}c}\append{a_{\i\j}}}. \]

\pause\bigskip
These satisfy the expected rules, like with vectors:
\begin{webonly}
\[\syseq{A+B = B+A\hfill, \qquad,
  (A+B)+C = A+(B+C);
  c(A+B) = cA+cB \qquad,
  (c+d)A = cA+dA\hfill ;
  (cd)A = c(dA)\hfill, \qquad,
  A+0 = A\hfill}
\]
\end{webonly}

\end{frame}


%%%%%%%%%%%%%%%%%%%%%%%%%%%%%%%%%%%%%%%%%%%%%%%%%%%%%%%%%%%%%%%%%%%

\begin{frame}
\frametitle{Matrix Multiplication}

\alert{Beware:} matrix multiplication is more subtle than addition and scalar
multiplication.

\pause\bigskip
Let $A$ be an $m\times\namedbox{cols}{\color<4->{red}n}$ matrix and let $B$ be an
$\namedbox{rows}{\color<4->{red}n}\times p$ matrix with
columns $v_1,v_2\ldots,v_p$:
\[ B = \mat{| | ,, |; v_1 v_2 \cdots, v_p; | | ,, |}. \]
The \textbf{product} $AB$ is the $m\times p$ matrix with columns
$Av_1,Av_2,\ldots,Av_p$: 
\[ AB \namedbox{def}{{}\overset{\rm def}={}}
\mat{| | ,, |; Av_1 Av_2 \cdots, Av_p; | | ,, |}. \]
\begin{tikzpicture}[remember picture, overlay]
  \path (def.west) ++ (-1cm,0)
    node[blue!50, inner sep=0pt, left, align=center] (expl)
      {The equality is\\a definition};
  \draw[->, blue!50] (expl.north) to[bend left] (def.north west);
\end{tikzpicture}%
\pause
In order for $Av_1,Av_2,\ldots,Av_p$ to make sense, the number of
\blankuntil{4}{\textcolor{red}{columns}} of $A$ has to be the same as the number of
\blankuntil{4}{\textcolor{red}{rows}} of $B$.

\begin{tikzpicture}[overlay, remember picture]
  \draw<4->[<->, red, shorten=1.5pt, rounded corners] 
    (cols.north) |- ($(rows.north) + (0,6mm)$) 
    node[below, pos=.75, thin border] {\small must be equal} -- (rows.north) ;
\end{tikzpicture}

\pause[5]\vskip-.3cm
\begin{eg}
  \null
  \vskip-1cm
  \small\[\begin{aligned} \mat{1 2 3; 4 5 6} \mat{1 -3; 2 -2; 3 -1}
    &= \webonlycmd{\mat{\mat{1 2 3; 4 5 6}\cdot\vec{1 2 3}
      \mat{1 2 3; 4 5 6}\cdot\vec{-3 -2 -1}}}\\
    &\webonlycmd{= \mat{14 -10; 32 -28}}
  \end{aligned}\]
\end{eg}

\end{frame}


%%%%%%%%%%%%%%%%%%%%%%%%%%%%%%%%%%%%%%%%%%%%%%%%%%%%%%%%%%%%%%%%%%%

\begin{frame}
\frametitle{Composition of Transformations}

\alert{Why} is this the correct definition of matrix multiplication?
\pause

\begin{defn}
  Let $T\colon\R^n\to\R^m$ and $U\colon\R^p\to\R^n$ be transformations.  The
  \textbf{composition} is the transformation
  \[ T\circ U\colon\R^p\to\R^m \sptxt{defined by}
  T\circ U(x) = T(U(x)). \]
\end{defn}
\pause
This makes sense because $U(x)$ (the output of $U$) is in \blankuntil{4}{$\R^n$},
which is the domain of $T$ (the inputs of $T$).

\pause[5]
\begin{center}
\begin{tikzpicture}[scale=.3, every label/.append style={text=black, thin border}]
  \draw[grid lines] (-3,-3) grid (3,3)
    (0,-4) node[black] {$\R^p$}
    (2,1)  node[point, "$x$" left] (x) {};
  \draw[grid lines,xshift=12cm] (-3,-3) grid (3,3)
    (0,-4) node[black] {$\R^n$}
    (-1,-2) node[point, "$U(x)$" above] (Ux) {};
  \draw[grid lines,xshift=24cm] (-3,-3) grid (3,3)
    (0,-4) node[black] {$\R^m$}
    (-2,2) node[point, "$T\circ U(x)$" below right] (TUx) {};
  \draw[|->, shorten=1.5pt, out=0, in=180]
    (x) to node[below,midway] {$U$} (Ux);
  \draw[|->, shorten=1.5pt, out=0, in=180]
    (Ux) to node[below,midway] {$T$} (TUx);
  \draw[shorten=1.5pt, out=0, in=180]
    (x) to node[above=2pt,midway,thin border] {$T\circ U$} (TUx);
\end{tikzpicture}
\end{center}

\pause
\alert{Fact:}
If $T$ and $U$ are linear then so is $T\circ U$.  
% We have to check:\pause
% \small
% \[\begin{split}
%   T\circ U(v+w) &\webonlycmd{= T(U(v+w)) = T(U(v)+U(w)) 
%   = T(U(v))+T(U(w))} \\
%   &= T\circ U(v) + T\circ U(w) \\
%   T\circ U(cv) &\webonlycmd{= T(U(cv)) = T(cU(v)) = cT(U(v))} = cT\circ U(v)
% \end{split}\]

\pause\medskip
\alert{Guess:}
If $A$ is the matrix for $T$, and $B$ is the matrix for $U$, what is the matrix
for $T\circ U$?


\end{frame}


%%%%%%%%%%%%%%%%%%%%%%%%%%%%%%%%%%%%%%%%%%%%%%%%%%%%%%%%%%%%%%%%%%%

\begin{frame}
\frametitle{Composition of Linear Transformations}

Let $T\colon\R^n\to\R^m$ and $U\colon\R^p\to\R^n$ be \emph{linear}
transformations.  Let $A$ and $B$ be their matrices:
\pause
\[ A = \mat{| | ,, |; T(e_1) T(e_2) \cdots, T(e_n); | | ,, |}
\quad
B = \mat{| | ,, |; U(e_1) U(e_2) \cdots, U(e_p); | | ,, |} \]

\pause
\begin{ques}
  What is the matrix for $T\circ U$?
\end{ques}

\pause
\begin{webonly}
We find the matrix for $T\circ U$ by plugging in the unit coordinate vectors:
\[ T\circ U(e_1) = T(U(e_1)) \mathbin{\namedbox{whyeq}{=}} T(Be_1) = A(Be_1) = (AB)e_1 \]
\namedbox{because}{because} $Be_1$ is the first column of $B$, which is $U(e_1)$.
\tikz[remember picture, overlay]
  \draw[<-,blue!50] (because) to[out=90,in=-90,looseness=.5] (whyeq);
For any other $i$, the same works:
\[ T\circ U(e_i) = T(U(e_i)) = T(Be_i) = A(Be_i) = (AB)e_i. \]
This says that the $i$th column of the matrix for $T\circ U$ is the $i$th column
of $AB$.
\end{webonly}

\begin{bluebox}{.8\linewidth}
  The matrix of the composition is the product of the matrices!
\end{bluebox}

\end{frame}


%%%%%%%%%%%%%%%%%%%%%%%%%%%%%%%%%%%%%%%%%%%%%%%%%%%%%%%%%%%%%%%%%%%

\begin{frame}
\frametitle{Composition of Linear Transformations}
\framesubtitle{Example}

Let $T\colon\R^2\to\R^2$ be rotation by $45^\circ$, and let $U\colon\R^2\to\R^2$
be projection onto the $x$-axis.  Let's compute their standard matrices
$A$ and~$B$:\\[1.5mm]
\begin{webonly}\hfill
\begin{tikzpicture}[picture align center, thin border nodes, scale=1.5]
  \draw[grid lines, step=.5] (-1.1,-.3) grid (1.1,1.1);
  \draw[grid lines] (1,0) arc[start angle=0, end angle=180, radius=1];
  \draw[vector] (0,0) to["$T(e_1)$"] ++(45:1) node[coordinate] (y) {};
  \point (o) at (0,0);
  \draw (o) -| 
     node[pos=.5,coordinate] (z) {}
     node[pos=.25,below] {$\frac 1{\sqrt 2}$} 
     node[pos=.75,right] {$\frac 1{\sqrt 2}$} 
     (y);
  \pic[draw] {right angle=(o)--(z)--(y)};
  \pic[draw, "$45$" font=\tiny, angle eccentricity=.7] {angle=z--o--y};
\end{tikzpicture}\hfill
\hbox to 3cm{\hss$\displaystyle
\begin{aligned}[c]
T(e_1) &= \frac 1{\sqrt 2}\vec{1 1}\\
T(e_2) &= \frac 1{\sqrt 2}\vec{-1 1}\\
\end{aligned}$\hss}\hfill
\begin{tikzpicture}[picture align center, thin border nodes, scale=1.5]
  \draw[grid lines, step=.5] (-1.1,-.3) grid (1.1,1.1);
  \draw[grid lines] (1,0) arc[start angle=0, end angle=180, radius=1];
  \draw[vector] (0,0) to["$T(e_2)$"'] ++(135:1) node[coordinate] (y) {};
  \point (o) at (0,0);
  \draw (o) -| 
     node[pos=.5,coordinate] (z) {}
     node[pos=.25,below] {$\frac 1{\sqrt 2}$} 
     node[pos=.75,left] {$\frac 1{\sqrt 2}$} 
     (y);
  \pic[draw] {right angle=(o)--(z)--(y)};
  \pic[draw, "$45$" font=\tiny, angle eccentricity=.7] {angle=y--o--z};
\end{tikzpicture}\hfill\null

\medskip
\hfill
\begin{tikzpicture}[picture align center, thin border nodes, scale=1.5]
  \draw[grid lines, step=.5] (-1.1,-.3) grid (1.1,1.1);
  \draw (-1.1,0) -- (1.1,0);
  \draw[vector] (0,0) to["$U(e_1)$"] (1,0) node[coordinate] (x) {};
  \point (o) at (0,0);
\end{tikzpicture}\hfill
\hbox to 3cm{\hss$\displaystyle
\begin{aligned}[c]
U(e_1) &= \vec{1 0}\\
U(e_2) &= \vec{0 0}\\
\end{aligned}$\hss}\hfill
\begin{tikzpicture}[picture align center, thin border nodes, scale=1.5]
  \draw[grid lines, step=.5] (-1.1,-.3) grid (1.1,1.1);
  \draw (-1.1,0) -- (1.1,0);
  \point["$U(e_2)$"] (o) at (0,0);
\end{tikzpicture}\hfill\null
\end{webonly}

\pause\medskip
\[ \implies\quad
A = \frac 1{\sqrt 2}\mat{1 -1; 1 1} \qquad
B = \mat{1 0; 0 0} \]

\end{frame}


%%%%%%%%%%%%%%%%%%%%%%%%%%%%%%%%%%%%%%%%%%%%%%%%%%%%%%%%%%%%%%%%%%%

\begin{frame}
\frametitle{Composition of Linear Transformations}
\framesubtitle{Example, continued}

So the matrix $C$ for $T\circ U$ is
\webonlycmd{
\[\begin{split} C = AB &= \frac 1{\sqrt 2}\mat{1 -1; 1 1}\mat{1 0; 0 0} \\
&= \mat{
  \frac 1{\sqrt 2}\mat{1 -1; 1 1}\vec{1 0}
  \frac 1{\sqrt 2}\mat{1 -1; 1 1}\vec{0 0}
} 
= \frac 1{\sqrt 2}\mat{1 0; 1 0}.
\end{split}\]
}

\pause
\alert{Check:}\\[1.5mm]
\begin{webonly}
\hbox to \linewidth{\hss
\begin{tikzpicture}[picture align center, thin border nodes, scale=1]
  \draw[grid lines, step=.5] (-1.1,-.3) grid (1.1,1.1);
  \draw[grid lines] (1,0) arc[start angle=0, end angle=180, radius=1];
  \draw[vector] (0,0) to["$e_1$"] (1,0);
  \point (o) at (0,0);
\end{tikzpicture}\;\longsquiggly\;
\begin{tikzpicture}[picture align center, thin border nodes, scale=1]
  \draw[grid lines, step=.5] (-1.1,-.3) grid (1.1,1.1);
  \draw[grid lines] (1,0) arc[start angle=0, end angle=180, radius=1];
  \draw[vector] (0,0) to["$U(e_1)$"] (1,0);
  \point (o) at (0,0);
\end{tikzpicture}\;\longsquiggly\;
\begin{tikzpicture}[picture align center, thin border nodes, scale=1]
  \draw[grid lines, step=.5] (-1.1,-.3) grid (1.1,1.1);
  \draw[grid lines] (1,0) arc[start angle=0, end angle=180, radius=1];
  \draw[vector] (0,0) to["$T(U(e_1))$"] ++(45:1) node[coordinate] (y) {};
  \point (o) at (0,0);
  \coordinate (z) at (1,0);
\end{tikzpicture}
\hbox to 3cm{$\displaystyle T\circ U(e_1)= \frac 1{\sqrt 2}\vec{1 1}$\hss}\hss}
  
\smallskip\hbox to \linewidth{\hss
\begin{tikzpicture}[picture align center, thin border nodes, scale=1]
  \draw[grid lines, step=.5] (-1.1,-.3) grid (1.1,1.1);
  \draw[grid lines] (1,0) arc[start angle=0, end angle=180, radius=1];
  \draw[vector] (0,0) to["$e_2$"] (0,1);
  \point (o) at (0,0);
\end{tikzpicture}\;\longsquiggly\;
\begin{tikzpicture}[picture align center, thin border nodes, scale=1]
  \draw[grid lines, step=.5] (-1.1,-.3) grid (1.1,1.1);
  \draw[grid lines] (1,0) arc[start angle=0, end angle=180, radius=1];
  \point["$U(e_2)$"] at (0,0);
\end{tikzpicture}\;\longsquiggly\;
\begin{tikzpicture}[picture align center, thin border nodes, scale=1]
  \draw[grid lines, step=.5] (-1.1,-.3) grid (1.1,1.1);
  \draw[grid lines] (1,0) arc[start angle=0, end angle=180, radius=1];
  \point["$T(U(e_2))$"] at (0,0);
\end{tikzpicture}
\hbox to 3cm{$\displaystyle T\circ U(e_2)= \vec{0 0}$\hss}\hss}
\end{webonly}

\pause\medskip
\[ \implies C = \frac 1{\sqrt 2}\mat{1 0; 1 0} \bigcheck[\quad] \]

\end{frame}



%%%%%%%%%%%%%%%%%%%%%%%%%%%%%%%%%%%%%%%%%%%%%%%%%%%%%%%%%%%%%%%%%%%
% JDR: probably skip this slide in class for time

\begin{frame}
\frametitle{Composition of Linear Transformations}
\framesubtitle{Another example}

Let 
\[ A = \mat{1 2 3; 4 5 6} \qquad B = \mat{1 -3; 2 -2; 3 -1}. \]
Let $T(x) = Ax$ and $U(y) = By$, so
\[ T\colon\R^{\blankuntil{2}{3}}\To\R^{\blankuntil{2}{2}}
\qquad
U\colon\R^{\blankuntil{3}{2}}\To\R^{\blankuntil{3}{3}}
\qquad
T\circ U\colon\R^{\blankuntil{4}{2}}\To\R^{\blankuntil{4}{2}}. \]
\pause[5]%
Let's find the matrix for $T\circ U$:
\[\begin{split} T\circ U(e_1) &=\webonlycmd{ T(U(e_1)) = T(Be_1) = T\vec{1 2 3}
= A\vec{1 2 3} = \vec{14 32}} \\
T\circ U(e_2) &=\webonlycmd{ T(U(e_2)) = T(Be_2) = T\vec{-3 -2 -1}
= A\vec{-3 -2 -1} = \vec{-10 -28}} 
\end{split}\]
\pause
Before we computed $AB = \mat{14 -10; 32 -28}$, so $AB$ is the matrix of $T\circ U$.

\end{frame}


%%%%%%%%%%%%%%%%%%%%%%%%%%%%%%%%%%%%%%%%%%%%%%%%%%%%%%%%%%%%%%%%%%%

\begin{pollframe}

\vskip 5mm

\begin{poll}
\begin{bluebox}[Poll]{.8\textwidth}
  Do there exist \emph{nonzero} matrices $A$ and $B$ with $AB=0$?
\end{bluebox}

\pause\bigskip

Yes!  Here's an example:
\[ \mat{1 0; 1 0}\mat{0 0; 1 1} 
= \mat{\mat{1 0; 1 0}\vec{0 1} \mat{1 0;1 0}\vec{0 1}}
= \mat{0 0; 0 0}. \]
\end{poll}


\end{pollframe}


%%%%%%%%%%%%%%%%%%%%%%%%%%%%%%%%%%%%%%%%%%%%%%%%%%%%%%%%%%%%%%%%%%%

\begin{frame}
\frametitle{The Row-Column Rule for Matrix Multiplication}

\displayskips{3pt}
\alert{Recall:} A row vector of length $n$ times a column vector of length $n$
is a scalar:
\[ \mat{a_1 \cdots, a_n}\vec{b_1 \vdots, b_n} = a_1b_1 + \cdots + a_nb_n. \]
\pause
Another way of multiplying a matrix by a vector is:
\[ Ax = 
\mat[c]{ \matrow{r_1};
     \vdots ;
     \matrow{r_m}}
     x
= \vec{r_1x \vdots, r_mx}.
\]
\pause
On the other hand, you multiply two matrices by
\[ AB = A\mat{| ,, |; c_1 \cdots, c_p; | ,, |} =
\mat{| ,, |; Ac_1 \cdots, Ac_p; | ,, |}. \]
\pause
It follows that
\[ AB = 
\mat[c]{ \matrow{r_1};
     \vdots ;
     \matrow{r_m}} 
   \mat{ | ,, |; c_1 \cdots, c_p; | ,, |}
= \mat{ r_1c_1 r_1c_2 \cdots, r_1c_p;
         r_2c_1 r_2c_2 \cdots, r_2c_p;
         \vdots, \vdots, , \vdots;
        r_mc_1 r_mc_2 \cdots, r_mc_p}
 \]

\end{frame}


%%%%%%%%%%%%%%%%%%%%%%%%%%%%%%%%%%%%%%%%%%%%%%%%%%%%%%%%%%%%%%%%%%%

\begin{frame}
\frametitle{The Row-Column Rule for Matrix Multiplication}

\begin{bluebox}{.9\textwidth}
  The $ij$ entry of $C = AB$ is the $i$th row of $A$ times the $j$th column of $B$:
  \abovedisplayskip=2pt
  \[ c_{ij} = (AB)_{ij} = a_{i1}b_{1j} + a_{i2}b_{2j} + \cdots + a_{in}b_{nj}. \]
\end{bluebox}

\pause
This is how everybody on the planet actually computes $AB$.
\pause
Diagram ($AB=C$):
\[
\begin{tikzpicture}[baseline]
  \matrix[math matrix]  (aij)
    {
      a_{11} \& \cdots \& a_{1k} \& \cdots \& a_{1n} \\
      \spvdots \&        \& \spvdots \&        \& \spvdots \\
      a_{i1} \& \cdots \& a_{ik} \& \cdots \& a_{in} \\
      \spvdots \&        \& \spvdots \&        \& \spvdots \\
      a_{m1} \& \cdots \& a_{mk} \& \cdots \& a_{mn} \\
    };
  \node[fit=(aij-3-1) (aij-3-5), inner sep=0pt,
      draw=green!70!black, thick, rounded corners] (row) {};
  \node[text=green!70!black, rotate=90, anchor=north, yshift=1.1mm, font=\small]
      at (row.east) {$i$th row};
\end{tikzpicture}\cdot
\begin{tikzpicture}[baseline]
  \matrix[math matrix]  (bij)
    {
      b_{11} \& \cdots \& b_{1j} \& \cdots \& b_{1p} \\
      \spvdots \&        \& \spvdots \&        \& \spvdots \\
      b_{k1} \& \cdots \& b_{kj} \& \cdots \& b_{kp} \\
      \spvdots \&        \& \spvdots \&        \& \spvdots \\
      b_{n1} \& \cdots \& b_{nj} \& \cdots \& b_{np} \\
    };
  \node[fit=(bij-1-3) (bij-5-3), inner sep=0pt,
      draw=blue!50, thick, rounded corners,
      label={[text=blue!50]below:\small$j$th column}] {};
\end{tikzpicture}=
\begin{tikzpicture}[baseline]
  \matrix[math matrix,
      label=below:$\textcolor{green!70!black}i\textcolor{blue!50}j$ entry]
        (cij)
    {
      c_{11} \& \cdots \& c_{1j} \& \cdots \& c_{1p} \\
      \spvdots \&        \& \spvdots \&        \& \spvdots \\
      c_{i1} \& \cdots \& c_{ij} \& \cdots \& c_{ip} \\
      \spvdots \&        \& \spvdots \&        \& \spvdots \\
      c_{m1} \& \cdots \& c_{mj} \& \cdots \& c_{mp} \\
    };
  \draw[thick, green!70!black]
    (cij-3-3.center) circle[radius=1.3ex];
  \draw[thick, blue!50]
    (cij-3-3.center) circle[radius=1.3ex+\pgflinewidth];
\end{tikzpicture}
\]

\pause\vskip -.3cm
\begin{eg}
  \null
  \vskip-.8cm
  \begin{webonly}
  \small\[\fboxsep=0pt
  \def\g{\textcolor{green!70!black}}
  \def\b{\textcolor{blue!50}}
  \begin{aligned}
    \mat{\g1 \g2 \g3; 4 5 6}
    \mat{\b1 -3; \b2 -2; \b3 -1}
    &= \mat{\g1\cdot\b1+\g2\cdot\b2+\g3\cdot\b3 \fbox{\phantom 8};
      \fbox{\phantom 8} \fbox{\phantom 8}}
    = \mat{14 \fbox{\phantom 8}; \fbox{\phantom 8} \fbox{\phantom 8}} \\
    \mat{1 2 3; \g4 \g5 \g6}
    \mat{\b1 -3; \b2 -2; \b3 -1}
    &= \mat{\fbox{\phantom 8} \fbox{\phantom 8};
      \g4\cdot\b1+\g5\cdot\b2+\g6\cdot\b3 \fbox{\phantom 8}}
    = \mat{\fbox{\phantom 8} \fbox{\phantom 8}; 32 \fbox{\phantom 8}}
  \end{aligned}\]
  \end{webonly}
\end{eg}

\end{frame}


%%%%%%%%%%%%%%%%%%%%%%%%%%%%%%%%%%%%%%%%%%%%%%%%%%%%%%%%%%%%%%%%%%%

\begin{frame}
\frametitle{Properties of Matrix Multiplication}

Mostly matrix multiplication works like you'd expect.  Suppose $A$ has size
$m\times n$, and that the other matrices below have the right size to make
multiplication work. 
\webonlycmd{
  \[\syseq{A(BC) = (AB)C\hfill, \qquad,
    A(B+C) = ({AB+AC});
    (B+C)A = BA+CA \qquad,
    c(AB) = (cA)B\hfill ;
    c(AB) = A(cB)\hfill, \qquad,
    I_nA = A\hfill;
    AI_m = A\hfill}
\]}\pause
Most of these are easy to verify.  

\pause\medskip
\textbf{Associativity} is $A(BC) = (AB)C$.  It is a pain to verify using the row-column
rule!
\pause
Much easier: use associativity of linear transformations:
\[ S\circ(T\circ U) = (S\circ T)\circ U. \]

\pause\medskip
This is a good example of an instance where having a conceptual viewpoint saves
you a lot of work.

\pause\medskip
\alert{Recommended:} Try to verify all of them on your own.

\end{frame}


%%%%%%%%%%%%%%%%%%%%%%%%%%%%%%%%%%%%%%%%%%%%%%%%%%%%%%%%%%%%%%%%%%%

\begin{frame}
\frametitle{Properties of Matrix Multiplication}
\framesubtitle{Caveats}

\alert{\large Warnings!}
\begin{itemize}
\item $AB$ is usually not equal to $BA$.
  \webonlycmd{
  \[\hss \mat{0 -1; 1 0}\mat{2 0; 0 1} = \mat{0 -1; 2 0} \qquad
  \mat{2 0; 0 1}\mat{0 -1; 1 0} = \mat{0 -2; 1 0} \hss\]
  }\pause
  In fact, $AB$ may be defined when $BA$ is not.

\pause\medskip
\item $AB = AC$ does not imply $B = C$, even if $A\neq 0$.
  \webonlycmd{
  \[ \mat{1 0; 0 0}\mat{1 2; 3 4} = \mat{1 2; 0 0}
     = \mat{1 0; 0 0}\mat{1 2; 5 6} \]
   }%

\pause
\item $AB = 0$ does not imply $A=0$ or $B=0$.
  \webonlycmd{
  \[ \mat{1 0; 1 0}\mat{0 0; 1 1} 
  = \mat{0 0; 0 0} \]
}

\end{itemize}

\end{frame}


%%%%%%%%%%%%%%%%%%%%%%%%%%%%%%%%%%%%%%%%%%%%%%%%%%%%%%%%%%%%%%%%%%%

\begin{frame}
\frametitle{Other Reading}

\vfill

\begin{bluebox}{.8\linewidth}
  Read about powers of a matrix and multiplication of transposes in \S2.1.
\end{bluebox}

\vfill

\end{frame}



%%% Local Variables:
%%% TeX-master: "../slides"
%%% End:
