
% JDR: These are the topics I thought my students found most confusing from
%   chapter 1, and the ones they asked me to cover.  They can probably be covered
%   in 50 minutes, but aren't necessarily meant to--the students can always review
%   the rest online.

\titleframe{Review for Midterm 1}{Selected Topics}


%%%%%%%%%%%%%%%%%%%%%%%%%%%%%%%%%%%%%%%%%%%%%%%%%%%%%%%%%%%%%%%%%%%

\begin{frame}
\frametitle{Linear Equations}

We have \emph{four} equivalent ways of writing (and thinking about) linear
systems:
\pause
\begin{enumerate}
\item As a system of equations:
\[ \syseq{2x_1 + 3x_2 = 7; x_1 - x_2 = 5} \]
\vskip-5mm\pause
\item As an augmented matrix:
  \vskip -3mm
  \[ \amat{ 2 3 7; 1 -1 5} \]
\vskip-5mm\pause
\item As a vector equation ($x_1v_1 + \cdots + x_nv_n = b$):
  \[ x_1\vec{2 1} + x_2\vec{3 -1} = \vec{7 5} \]
\vskip-7mm\pause
\item As a matrix equation ($Ax = b$):
  \[ \mat{ 2 3; 1 -1}\vec{x_1 x_2} = \vec{7 5} \]
  \note{(4) is by definition equal to (3).}%
\end{enumerate}
\pause

In particular, \emph{all four have the same solution set}.

\end{frame}


%%%%%%%%%%%%%%%%%%%%%%%%%%%%%%%%%%%%%%%%%%%%%%%%%%%%%%%%%%%%%%%%%%%

\begin{frame}
\frametitle{Number of Solutions}

There are \emph{three possibilities} for the reduced row echelon form of the
augmented matrix of a linear system.\pause

\begin{enumerate}
\item \alert{The last column is a pivot column.}\\
  \pause
  There are \emph{zero} solutions, i.e.\ the solution set is \emph{empty}.
  \pause
  In this case, the system is called \textbf{inconsistent}.  
  Picture:
  \pause
  \[ \amat{ 1 0 {\color{red}0}; 0 1 {\color{red}0}; 0 0 {\color{red}1}} \]

\pause
\item \alert{Every column except the last column is a pivot column.}\\
  In this case, the system has a \emph{unique solution}.  Picture:
  \[ \amat{
      1  0  0  \star ;
      0  1  0  \star ;
      0  0  1  \star
    } \]
\note{Ask: what are $x,y,z$?}%

\pause
\item \alert{The last column is not a pivot column, and some other column isn't either.}\\
  In this case, the system has \emph{infinitely many} solutions, corresponding
  to the infinitely many possible values of the free variable(s).  Picture:
  \[\amat{
      1  \color{seq-red}\star,  0  \color{seq-blue}\star,  \star ;
      0  \color{seq-red}0   1  \color{seq-blue}\star, \star
    }\]

\end{enumerate}

\end{frame}


%%%%%%%%%%%%%%%%%%%%%%%%%%%%%%%%%%%%%%%%%%%%%%%%%%%%%%%%%%%%%%%%%%%

\begin{frame}
\frametitle{Span}

The \textbf{span} of vectors $v_1,v_2,\ldots,v_n$ is the set of all linear
combinations of these vectors:
\[ \Span\{v_1,v_2,\ldots,v_n\}
= \bigl\{ a_1v_1+a_2v_2+\cdots+a_nv_n\mid a_1,a_2,\ldots,a_n\text{ in }\R \bigr\}. \]

\pause
\begin{thm}
  Let $v_1,v_2,\ldots,v_n$, and $b$ be vectors in $\R^m$, and let $A$ be the
  $m\times n$ matrix with columns $v_1,v_2,\ldots,v_n$.  The following are
  equivalent\namedbox{colon}{:}
  \begin{tikzpicture}[remember picture, overlay, blue!50]
    \path (colon.east) ++(1.2cm,.2cm) node[anchor=north west, text width=3cm]
      (expl) {either they're all true, or they're all false, for the given vectors};
    \draw[->, thick] (expl.west) to[out=180, in=0] (colon.east);
  \end{tikzpicture}
  \begin{enumerate}
    \pause
  \item $Ax = b$ is consistent.
    \pause
  \item $(\,A\mid b\,)$ does not have a pivot in \pause the last column.
    \pause
  \item $b$ is in $\Span\{v_1,v_2,\ldots,v_n\}$ (the span of the columns of
    $A$).
  \end{enumerate}
\end{thm}

\pause
In this case, a solution to the matrix equation
\[ A\vec{x_1 x_2 \vdots, x_n} = b
\parbox{3cm}{\centering gives the linear\\combination}
x_1v_1 + x_2v_2 + \cdots + x_nv_n = b. \]

\end{frame}


%%%%%%%%%%%%%%%%%%%%%%%%%%%%%%%%%%%%%%%%%%%%%%%%%%%%%%%%%%%%%%%%%%%

\begin{frame}
\frametitle{Transformations}

\vskip-3mm
\begin{defn}
  A \textbf{transformation} (or \textbf{function} or \textbf{map}) from $\R^n$
  to $\R^m$ is a rule $T$ that assigns to each vector $x$ in $\R^n$ a vector
  $T(x)$ in $\R^m$.
\end{defn}

\pause
Picture and vocabulary words:

\begin{center}
\begin{tikzpicture}[scale=0.35, baseline, thin border nodes]
  \path[use as bounding box] (-3,-5.5) -- (15,3);
  \draw[grid lines] (-3,-3) grid (3,3);

  \node (A) at (0,-4) {$\R^n$};
  \node (B) at (12,-4) {$\R^m$};
  \node at (0,-5) {domain};
  \node at (12,-5) {codomain};
  \draw[->] (A.east) +(5mm,0) -- node[midway,above=1pt] {$T$}
    ($(B.west)-(5mm,0)$);

  \point["$x$" left] (P) at (1,1.5);

  \begin{scope}[myxyz, xshift=12cm]
    \path[clip, resetxy] (-3,-3) rectangle (3,3);

    \def\v{(2,-1,1)}
    \def\w{(1,0,-1)}
  
    \node[coordinate] (X) at \v {};
    \node[coordinate] (Y) at \w {};

    \draw[very thin] (-2,0,0) -- (0,0,0);
    \draw[very thin] (0,-2,0) -- (0,0,0);
    \draw[very thin] (0,0,-2) -- (0,0,0);

    \begin{scope}[x=(X), y=(Y), transformxy]
      \fill[seq4!10, nearly opaque] (-1,-1) rectangle (1,1);
      \draw[step=.5cm, seq4!30, very thin] (-1,-1) grid (1,1);
      \point[label={[font=\tiny,fill=none]below:$T(x)$}]
        (Q) at (-.5,.5);
      \node[coordinate,
          pin={[pin edge={very thin,-},pin distance=3mm,anchor=north,font=\small,xshift=1mm,text=seq4]-90:range}]
        at (0.1,1) {};
    \end{scope}

    \draw[->, very thin] (0,0,0) -- (2,0,0);
    \draw[->, very thin] (0,0,0) -- (0,2,0);
    \draw[->, very thin] (0,0,0) -- (0,0,2);

    \draw[resetxy] (-3,-3) rectangle (3,3);

  \end{scope}

  \pic[scale=.4, "$T$"] (machine) at (6,0) {machine};
  \draw[->, shorten >=.35mm, shorten <=.35mm] (P.east)
    .. controls +(0:1cm) and +(180:1cm) .. (machine-input);
  \draw[->, shorten >=.35mm, shorten <=.35mm] (machine-output)
    .. controls +(0:1cm) and +(190:2cm) .. (Q.west);

\end{tikzpicture}
\end{center}

\pause
It is \textbf{one-to-one} if different vectors in the domain go to different
vectors in the codomain: $x\neq y\implies T(x)\neq T(y)$.

\pause\medskip
It is \textbf{onto} if every vector in the codomain is $T(x)$ for some $x$.
\pause
In other words, the range equals the codomain.

\end{frame}


%%%%%%%%%%%%%%%%%%%%%%%%%%%%%%%%%%%%%%%%%%%%%%%%%%%%%%%%%%%%%%%%%%%

\begin{frame}
\frametitle{Linear Transformations}

A transformation $T\colon\R^n\to\R^m$ is \textbf{linear} if it satisfies:
\[ T(u+v) = T(u) + T(v) \sptxt{and} T(cu) = cT(u) \]
for every $u,v$ in $\R^n$ and every $c$ in $\R$.

\pause\smallskip
\begin{bluebox}{.85\linewidth}\centering
  Linear transformations are the same as matrix transformations.
\end{bluebox}

\pause
\hbox to \linewidth{\hfil\alert{Dictionary}\hfil}
\pause\leavevmode
\parbox{.3\linewidth}{\centering
  Linear transformation\\$T\colon\R^n\to\R^m$}%
\;\longsquiggly\;
\hbox to .5\linewidth{$m\times n$ matrix
\small$A = \mat{| | ,, |; T(e_1) T(e_2) \cdots, T(e_n); | | ,, | }$\hss}\\[1mm]
\pause\leavevmode
\hbox to .3\linewidth{\hfil
  $\begin{aligned}&T(x) = Ax\\&T\colon\R^n\to\R^m\end{aligned}$\hfil}%
\;\rlongsquiggly\;
\hbox to .5\linewidth{$m\times n$ matrix $A$\hfil}

\pause\smallskip
As always, $e_1,e_2,\ldots,e_n$ are the \textbf{unit coordinate vectors}\small
\[ e_1 = \vec{1 0 \vdots, 0 0},\quad e_2 = \vec{0 1 \vdots, 0 0},\quad\ldots,\quad
   e_{n-1} = \vec{0 0 \vdots, 1 0},\quad e_n = \vec{0 0 \vdots, 0 1}. \]

\end{frame}


%%%%%%%%%%%%%%%%%%%%%%%%%%%%%%%%%%%%%%%%%%%%%%%%%%%%%%%%%%%%%%%%%%%

\begin{frame}
\frametitle{Linear Transformations and Matrices}

Let $A$ be an $m\times n$ matrix and let $T$ be the linear transformation
$T(x)=Ax$.
\begin{itemize}
  \pause
\item The domain of $T$ is $\R^n$.  (Inputs are vectors with $n$ entries.)
  \pause
\item The codomain of $T$ is $\R^m$.  (Outputs are vectors with $m$ entries.)
  \pause
\item The range of $T$ is span of the columns of $A$.\\
  \pause
  (This is the set of all
  $b$ in $\R^m$ such that $Ax=b$ has a solution.)
\end{itemize}

\pause\leavevmode
\begin{minipage}[t]{.5\textwidth}
\begin{eg}
  \vskip -.5cm
  \[ A = \mat[r]{2 1; -1 0; 1 -1} \qquad T(x) = Ax \]
  \begin{itemize}
  \item<7-> The domain of $T$ is $\R^{\blankuntil{8}2}$.
  \item<9-> The codomain of $T$ is $\R^{\blankuntil{10}3}$.
  \item<11-> The range of $T$ is 
    \uncover<12->{\[ \Span\left\{\textcolor{seq-red}{\vec{2 -1 1}},\,
        \textcolor{seq-blue}{\vec{1 0 -1}}\right\}. \]}
  \end{itemize}
\end{eg}
\end{minipage}\quad
\begin{uncoverenv}<12->
\begin{tikzpicture}[myxyz, scale=.6, baseline=3cm]
  \begin{scope}
  \path[clip, resetxy] (-4,-3) rectangle (4,3);

  \def\v{(2,-1,1)}
  \def\w{(1,0,-1)}

  \node[coordinate] (X) at \v {};
  \node[coordinate] (Y) at \w {};

  \begin{scope}[x=(X), y=(Y), transformxy]
    \path[clip] (-5, -5) -- (5, 5) -- (5, 10) -- (-5, 10) -- cycle;
    \fill[seq4!10, nearly opaque] (-1.5,-1) rectangle (1.5,2);
    \draw[step=.5cm, seq4!30, very thin] (-1.5,-1) grid (1.5,2);
  \end{scope}

  \begin{scope}[transformxy]
    \fill[white, semitransparent] (-2, -2) rectangle (3, 3);
    \draw[step=1cm, grid lines] (-2, -2) grid (3, 3);
  \end{scope}

  \begin{scope}[x=(X), y=(Y), transformxy]
    \path[clip] (-5, -5) -- (5, 5) -- (5, -10) -- (-5, -10) -- cycle;
    \fill[seq4!10, nearly opaque] (-1.5,-1) rectangle (1.5,2);
    \draw[step=.5cm, seq4!30, very thin] (-1.5,-1) grid (1.5,2);
  \end{scope}

  \draw[vector, seq1] (0,0,0) -- \v;
  \draw[thin, densely dotted] \v -- \projxy\v;

  \draw[vector, seq2!50!white] (0,0,0) -- \w;
  \draw[thin, densely dotted, black!50!white] \w -- \projxy\w;

  \node[seq4] at (-1.5cm, 1.5cm) {$\range(T)$};

  \point at (0,0,0) {};
  \draw[thin,resetxy] (-4,-3) rectangle (4,3);
  \end{scope}
  \node at (0,-3.5cm) {$\R^3 = {\rm codomain}(T)$};
\end{tikzpicture}
\end{uncoverenv}

\end{frame}


%%%%%%%%%%%%%%%%%%%%%%%%%%%%%%%%%%%%%%%%%%%%%%%%%%%%%%%%%%%%%%%%%%%

\begin{frame}
\frametitle{When the Span is Everything}

\vskip-3mm

\begin{thm}
  Let $A$ be an $m\times n$ matrix, and let $T\colon\R^n\to\R^m$ be the linear
  transformation $T(x) = Ax$.  The following are equivalent:
  \begin{enumerate}
    \pause
  \item $T$ is onto.
    \pause
  \item $T(x) = b$ has a solution for every $b$ in $\R^m$.
    \pause
  \item $Ax = b$ is consistent for every $b$ in $\R^m$.
    \pause
  \item The columns of $A$ span $\R^m$.
    \pause
  \item A has a pivot in each \emph{row}.
  \end{enumerate}
\end{thm}

\pause
\alert{Moral:}
If $A$ has a pivot in each row then its reduced row echelon form looks like this:
\[\def\r{\color{red}} \mat{
\r1   0   \star,   0   \star ;
0   \r1   \star , 0   \star ;
0   0   0   \r1   \star 
}
\pause\quad\parbox{.2\textwidth}
{\centering and $(\,A\mid b\,)$ reduces to this:}\quad
\amat[c]{
\r1   0   \star,   0   \star, \star ;
0   \r1   \star , 0   \star, \star ;
0   0   0   \r1   \star, \star 
}.\]
\pause
There's no $b$ that makes it inconsistent, so there's always a solution.

\pause\medskip
\alert{Refer:} slides for \S1.4 and \S1.9.

\end{frame}


%%%%%%%%%%%%%%%%%%%%%%%%%%%%%%%%%%%%%%%%%%%%%%%%%%%%%%%%%%%%%%%%%%%

\begin{frame}
\frametitle{Linear Independence}

A \emph{set} of vectors $\{v_1,v_2,\ldots,v_n\}$ is
\textbf{linearly independent} if
\[ a_1v_1 + a_2v_2 + \cdots + a_nv_n = 0 
\sptxt{only when} a_1 = a_2 = \cdots = a_n = 0. \]
\pause
Otherwise they are \textbf{linearly dependent}, and an equation
$a_1v_1 + a_2v_2 + \cdots + a_nv_n = 0$
with some $a_i\neq0$ is a \textbf{linear dependence relation}.

\pause
\begin{thm}
  Let $v_1,v_2,\ldots,v_n$ be vectors in $\R^m$, and let $A$ be the
  $m\times n$ matrix with columns $v_1,v_2,\ldots,v_n$.  The following are
  equivalent:
  \begin{enumerate}
    \pause
  \item The set $\{v_1,v_2,\ldots,v_n\}$ is linearly independent.
    \pause
  \item For every $i$ between $1$ and $n$, $v_i$ is not in
    $\Span\{v_1,v_2,\ldots,v_{i-1}\}$.
    \pause
  \item $Ax=0$ only has the trivial solution.
    \pause
  \item $A$ has a pivot in every \blankuntil{8}{column}\pause.
  \end{enumerate}
\end{thm}

\pause
If the vectors are linearly \emph{de}pendent, a nontrivial solution to the
matrix equation
\[ A\vec{x_1 x_2 \vdots, x_n} = 0
\parbox{3.5cm}{\centering gives the linear\\dependence relation}
x_1v_1 + x_2v_2 + \cdots + x_nv_n = 0. \]

\end{frame}


%%%%%%%%%%%%%%%%%%%%%%%%%%%%%%%%%%%%%%%%%%%%%%%%%%%%%%%%%%%%%%%%%%%

\begin{frame}
\frametitle{More Criteria for Linear Independence}

\vskip-3mm

\begin{thm}
  Let $A$ be an $m\times n$ matrix, and let $T\colon\R^n\to\R^m$ be the linear
  transformation $T(x) = Ax$.  The following are equivalent:
  \begin{enumerate}
    \pause
  \item $T$ is one-to-one.
    \pause
  \item $T(x) = b$ has one or zero solutions for every $b$ in $\R^m$.
    \pause
  \item $Ax = b$ has a unique solution or is inconsistent for every $b$ in $\R^m$.
    \pause
  \item $Ax = 0$ has a unique solution.
    \pause
  \item The columns of $A$ are linearly independent.
    \pause
  \item A has a pivot in each \emph{column}.
  \end{enumerate}
\end{thm}

\pause
\alert{Moral:}
If $A$ has a pivot in each column then its reduced row echelon form looks like this:
\[\def\r{\color{red}} \mat{
\r1   0   0 ;
0   \r1   0 ;
0     0   \r1 ;
0     0   0 
}
\pause\quad\parbox{.2\textwidth}
{\centering and $(\,A\mid b\,)$ reduces to this:}\quad
\amat[c]{
\r1   0   0 \star;
0   \r1   0 \star;
0     0   \r1 \star ;
0     0   0 \star
}.\]
\pause
This can be inconsistent, but if it is consistent, it has a unique solution.

\pause\medskip
\alert{Refer:} slides for \S1.4, \S1.8, \S1.9.

\end{frame}


%%%%%%%%%%%%%%%%%%%%%%%%%%%%%%%%%%%%%%%%%%%%%%%%%%%%%%%%%%%%%%%%%%%

\begin{frame}
\frametitle{Parametric Form of Solution Sets}

To find the solution set to $Ax=b$, first form the augmented matrix
$(\,A\mid b\,)$, then row reduce.
\pause
\def\r{\color<5->{red}}\def\g{\color<5->{green!70!black}}
\[ \amat{1 \r3 0  \g4 0 2;
         0 \r0 1 \g{-1} 0 3;
         0 \r0 0  \g0 1 -7} \]
\pause
This translates into
\spalignsysdelims..
\[ \syseq{x_1 + \r3x_2 \+ \.  \+ \g{x_4} \+ \.  = 2;
          \. \+  \.  \+ x_3  - \g{x_4}  \+ \.  = 3;
          \. \+  \.  \+ \.  \+  \.  \+ x_5 = -7} \]
\pause
The variables correspond to the non-augmented columns of the matrix.\\
\pause
The \emph{\r{free} \g{variables}} correspond to the non-augmented columns
\pause
\emph{without pivots}.

\pause\medskip
Move the free variables to the other side, get the \emph{parametric form}:
\[ \syseq{x_1 = 2 - 3x_2 - x_4;
          %x_2 = \. \+ x_2;
          x_3 = 3 \+ \. + x_4;
          %x_4 = \. \+ \. \+ x_4;
          x_5 = -7
        } \]
This is a solution for every value of $x_3$ and $x_4$.
\end{frame}


%%%%%%%%%%%%%%%%%%%%%%%%%%%%%%%%%%%%%%%%%%%%%%%%%%%%%%%%%%%%%%%%%%%

\begin{frame}
\frametitle{Parametric Vector Form of Solution Sets}

\def\r{\color<3->{seq1}}\def\g{\color<3->{seq2}}
\def\b{\color<3->{seq3}}\def\a{\color<3->{seq4}}
Parametric form:
\[ \syseq{x_1 = 2 - 3x_2 - x_4;
          %x_2 = \. \+ x_2;
          x_3 = 3 \+ \. + x_4;
          %x_4 = \. \+ \. \+ x_4;
          x_5 = -7
        } 
 \pause
 \quad\longsquiggly[add free variables]\quad
 \syseq{\r{x_1} = \g2 - \b3x_2 - \a{x_4};
        \r{x_2} = \. \+ \b{x_2};
        \r{x_3} = \g3 \+ \. + \a{x_4};
        \r{x_4} = \. \+ \. \+ \a{x_4};
        \r{x_5} = \g-7
      } 
\]
\pause
Now collect all of the equations into a vector equation:
\[ x = {\r\vec{x_1 x_2 x_3 x_4 x_5}}
= {\g\vec{2 0 3 0 -7}}
+ {\b x_2\vec{-3 1 0 0 0}}
+ {\a x_4\vec{-1 0 1 1 0}}. \]
This is the \textbf{parametric vector form} of the solution set.
\pause
This means that the
\[ \text{(solution set)}
= \vec{2 0 3 0 -7}
  + \Span\left\{ \vec{-3 1 0 0 0},\, \vec{-1 0 1 1 0}\right\}. \]
\end{frame}


%%%%%%%%%%%%%%%%%%%%%%%%%%%%%%%%%%%%%%%%%%%%%%%%%%%%%%%%%%%%%%%%%%%

\begin{frame}
\frametitle{Homogeneous and Non-Homogeneous Equations}

The equation $Ax=b$ is called \textbf{homogeneous} if $b=0$, and
\textbf{non-homogeneous} otherwise.
\pause
A homogeneous equation always has the \textbf{trivial solution} $x=0$:
\[ A0 = 0. \]

\pause\medskip
The solution set to a homogeneous equation is always a span:
\[ \text{(solutions to $Ax=0$)} = \Span\{v_1,v_2,\ldots,v_r\} \]
where $r$ is the
\pause
number of free variables.
\pause
The solution set to a consistent non-homogeneous equation is
\[ \text{(solutions to $Ax=b$)} = p + \Span\{v_1,v_2,\ldots,v_r\} \]
where $p$ is a \textbf{specific solution} (i.e.\ some vector such that $Ap=b$),
\pause
and $\Span\{v_1,\ldots,v_r\}$ is the solution set to the homogeneous equation
$Ax=0$.
\pause
This is a \emph{translate of a span}.

\pause\bigskip
Both expressions can be read off from the parametric vector form.

\end{frame}


%%% Local Variables:
%%% TeX-master: "../slides"
%%% End:
