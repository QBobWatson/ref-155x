
% JDR: This is meant to be half a lecture, the other half being 6.2.

\titleframe{Section 6.3}{Orthogonal Projections}


%%%%%%%%%%%%%%%%%%%%%%%%%%%%%%%%%%%%%%%%%%%%%%%%%%%%%%%%%%%%%%%%%%%

\begin{frame}
\frametitle{Idea Behind Orthogonal Projections}

If $x$ is not in a subspace $W$, then $y$ in $W$ is the closest to $x$ if
\pause
$x-y$ is in $W^\perp$:
\begin{center}
\begin{tikzpicture}[myxyz, scale=.5, thin border nodes,
    right angle len=1.5mm]
  \coordinate (u) at (1,0,0);
  \coordinate (v) at (0,1.1,-.2);
  \coordinate (uxv) at (0,.2,1.1);
  \coordinate (x) at ($-1.1*(u)+(v)+1.5*(uxv)$);
  \begin{scope}[x=(u),y=(v),transformxy]
    \fill[seq-violet!30] (-2,-2) rectangle (2,2);
    \draw[seq-violet, help lines] (-2,-2) grid (2,2);
    \node[seq-violet] at (2.6,1) {$W$};
  \end{scope}
  \point<-2| handout:0>[seq-blue, "$y$" {below,text=seq-blue}] (y) at ($-1.1*(u)+1*(v)$);
  \point<3->[seq-blue, "$x_W$" {below,text=seq-blue}] (y) at ($-1.1*(u)+1*(v)$);
  \point[seq-red, "$x$" {above,text=seq-red}] (xx) at (x);
  \draw<-2| handout:0>[vector, thin] (y) -- node[auto] {$x-y$} (xx);
  \draw<3->[vector, thin] (y) -- node[auto] {$x_{W^\perp}$} (xx);
  \point at (0,0,0);
  \coordinate (yu) at ($(y)+(u)$);
  \coordinate (yv) at ($(y)+(v)$);
  \pic[draw] {right angle=(x)--(y)--(yu)};
  \pic[draw] {right angle=(x)--(y)--(yv)};
\end{tikzpicture}
\end{center}

\pause\medskip
\alert{Reformulation:}
Every vector $x$ can be decompsed uniquely as 
\[ x = x_W + x_{W^\perp} \]
where $x_W\textcolor{black!50}{{} = y}$ is the closest vector to $x$ in $W$, and
$x_{W^\perp}\textcolor{black!50}{{} = x-y}$ is in
$W^\perp$. 

\pause\medskip
\alert{Example:}
Let $u = {3\choose 2}$ and let $L = \Span\{u\}$.  Let
$x = {-6\choose 4}$.
\pause
Then the closest point to $x$ in
$L$ is
\pause
$\proj_L(x) = \frac{x\cdot u}{u\cdot u}u$, so
\displayskips{3pt}
\[ x_L =\pause \proj_L(x) = -\frac{10}{13}\vec{3 2}
\qquad
x_{L^\perp} = \pause x - \proj_L(x) = \vec{-6 4} + \frac{10}{13}\vec{3 2}. \]
\begin{center}
\begin{tikzpicture}[scale=.5, thin border nodes,
    right angle len=1.5mm]
  \draw[seq-violet] (-3,-2) -- node[below right, very near start] {$L$} (3,2);
  \draw[vector] (0,0) --  node[below right] {$u$} (1.5,1);
  \point[seq-red] (x) at (-3,2);
  \point (o) at (0,0);
  \draw[vector,seq-red] (o) -- node[auto,swap] {$x$} (x);
  \point[seq-blue, "$x_L = \proj_L(x)$" {below right,seq-blue}]
    (p) at (${-2.5/(1.5*1.5+1)}*(1.5,1)$);
  \draw[vector, seq-green] (p) -- node[below left] {$x_{L^\perp}$} (x);
  \pic[draw] {right angle=(x)--(p)--(o)};
\end{tikzpicture}
\end{center}

\end{frame}


%%%%%%%%%%%%%%%%%%%%%%%%%%%%%%%%%%%%%%%%%%%%%%%%%%%%%%%%%%%%%%%%%%%

\begin{frame}
\frametitle{Orthogonal Projections}

\vskip-3mm
\begin{defn}
  Let $W$ be a subspace of $\R^n$, and let $\{u_1,u_2,\ldots,u_m\}$ be an
  \emph{orthogonal} basis for $W$.  The \textbf{orthogonal projection} of a
  vector $x$ onto $W$ is
  \[ \proj_W(x) \overset{\rm def}= \sum_{i=1}^m \frac{x\cdot u_i}{u_i\cdot u_i}\,u_i. \]
\end{defn}

\pause
\alert{Question:} What is the difference between this and the formula for
$[x]_\cB$ from before?

\pause\medskip
\begin{thm}
  Let $W$ be a subspace of $\R^n$, and let $x$ be a vector in $\R^n$.  Then
  $\proj_W(x)$ is the closest point to $x$ in $W$.
  \pause
  Therefore
  \[ \namedbox{start}{x_W} = \proj_W(x) \qquad x_{W^\perp}
  = \namedbox{end}{x-\proj_W(x)}. \]
\end{thm}%
\begin{tikzpicture}[remember picture, overlay]%
  \node[fit=(start) (end), inner sep=1mm, thick, rounded corners, seq-red, draw] {};
\end{tikzpicture}%

\vskip-5mm
\begin{webonly}
\alert{Why?}
Let $y = \proj_W(x)$.  We need to show that $x-y$ is in $W^\perp$.
In other words, $u_i\cdot(x-y)=0$ for each $i$.
Let's do $u_1$:
\[ u_1\cdot(x-y)
= u_1\cdot\left( x-\sum_{i=1}^m\frac{x\cdot u_i}{u_i\cdot u_i}\,u_i \right)
= u_1\cdot x - \frac{x\cdot u_1}{u_1\cdot u_1}(u_1\cdot u_1) - 0 - \cdots
= 0. \]
\end{webonly}

\end{frame}


%%%%%%%%%%%%%%%%%%%%%%%%%%%%%%%%%%%%%%%%%%%%%%%%%%%%%%%%%%%%%%%%%%%

\begin{frame}
\frametitle{Orthogonal Projections}
\framesubtitle{Easy example}

What is the projection of $x = \vec{1 2 3}$ onto the $xy$-plane?

\medskip
\begin{webonly}
\alert{Answer:}
The $xy$-plane is $W=\Span\{e_1,e_2\}$, and $\{e_1,e_2\}$ is an orthogonal basis.
\[ x_W = \proj_W\vec{1 2 3}
= \frac{x\cdot e_1}{e_1\cdot e_1}e_1 + \frac{x\cdot e_2}{e_2\cdot e_2}e_2
= \frac{1\cdot 1}{1^2} e_1 + \frac{1\cdot 2}{1^2} e_2
= \vec{1 2 0}.
 \] 
\end{webonly}
\pause
So this is the same projection as before.

\begin{center}
\begin{tikzpicture}[myxyz, scale=.75, thin border nodes]
  \draw (0,0,-2)--(0,0,0);
  \fill[transformxy, help lines, seq-violet!20, opacity=.7]
    (-2.5,-2.5) rectangle (2.5,2.5);
  \draw[transformxy, help lines, seq-violet!50] (-2.5,-2.5) grid (2.5,2.5);
  \draw[->] (-2.5,0,0)--(2.5,0,0);
  \draw[->] (0,-2.5,0)--(0,2.5,0);
  \draw[->] (0,0,0)--(0,0,2);
  \point[seq-red, "$x$" {above, text=seq-red}] (x) at (1,2,3);
  \point[seq-blue, "$x_W$" {below, text=seq-blue}] (xW) at (1,2,0);
  \point (o) at (0,0,0);
  \draw[vector, seq-green] (xW) --
    node[auto, seq-green!80!black] {$x_{W^\perp}$} (x);
  \coordinate (xWu) at ($(xW)+(1,0,0)$);
  \coordinate (xWv) at ($(xW)+(0,1,0)$);
  \pic[draw] {right angle=(x)--(xW)--(xWu)};
  \pic[draw] {right angle=(x)--(xW)--(xWv)};
  \node[seq-violet] at (3,1,0) {$W$};
    
\end{tikzpicture}
\end{center}

\end{frame}


%%%%%%%%%%%%%%%%%%%%%%%%%%%%%%%%%%%%%%%%%%%%%%%%%%%%%%%%%%%%%%%%%%%

\begin{frame}
\frametitle{Orthogonal Projections}
\framesubtitle{More complicated example}

\hbox to \linewidth{
What is the projection of $x = \vec[r]{-1.1\phantom0 1.4\phantom0 1.45}$ onto
$W = \Span\left\{ \vec{1 0 0},\;\vec{0 1.1 -.2} \right\}$?\hss}

\medskip
\begin{webonly}
\alert{Answer:}
The basis is orthogonal, so
\[\begin{split}
  x_W 
  &= \proj_W\vec[r]{-1.1\phantom0 1.4\phantom0 1.45}
  = \frac{x\cdot u_1}{u_1\cdot u_1}u_1 + \frac{x\cdot u_2}{u_2\cdot u_2}u_2 \\
  &= \frac{(-1.1)(1)}{1^2}\vec{1 0 0} + 
  \frac{(1.4)(1.1)+(1.45)(-.2)}{1.1^2+(-.2)^2}\vec{0 1.1 -.2} \\
\end{split}\]
This turns out to be equal to $u_2-1.1u_1$.
\end{webonly}
\vfill\pause
\begin{center}
\begin{tikzpicture}[myxyz, scale=.5, thin border nodes,
    right angle len=1.5mm]
  \coordinate (u) at (1,0,0);
  \coordinate (v) at (0,1.1,-.2);
  \coordinate (uxv) at (0,.2,1.1);
  \coordinate (x) at ($-1.1*(u)+(v)+1.5*(uxv)$);
  \begin{scope}[x=(u),y=(v),transformxy]
    \fill[seq-violet!30] (-2,-2) rectangle (2,2);
    \draw[seq-violet, help lines] (-2,-2) grid (2,2);
    \node[seq-violet] at (2.6,1) {$W$};
    \draw[vector] (0,0) -- (1,0) node[right] {$u_1$};
    \draw[vector] (0,0) -- (0,1) node[below] {$u_2$};
  \end{scope}
  \point[seq-blue, "$x_W$" {below,text=seq-blue}] (y) at ($-1.1*(u)+1*(v)$);
  \point[seq-red, "$x$" {above,text=seq-red}] (xx) at (x);
  \draw[vector, seq-green] (y) -- node[auto] {$x_{W^\perp}$} (xx);
  \point at (0,0,0);
  \coordinate (yu) at ($(y)+(u)$);
  \coordinate (yv) at ($(y)+(v)$);
  \pic[draw] {right angle=(x)--(y)--(yu)};
  \pic[draw] {right angle=(x)--(y)--(yv)};
\end{tikzpicture}
\end{center}\vfill

\end{frame}


%%%%%%%%%%%%%%%%%%%%%%%%%%%%%%%%%%%%%%%%%%%%%%%%%%%%%%%%%%%%%%%%%%%

\begin{frame}
\frametitle{Orthogonal Projections}
\framesubtitle{Picture}

Let $W$ be a subspace of $\R^n$, and let $\{u_1,u_2,\ldots,u_m\}$ be an
orthogonal basis for $W$.  Let $L_i = \Span\{u_i\}$.  Then
\[ \proj_W(x) = \sum_{i=1}^m \frac{x\cdot u_i}{u_i\cdot u_i}\,u_i
= \sum_{i=1}^m \proj_{L_i}(x). \]
\pause
So the orthogonal projection is formed by adding orthogonal projections onto
perpendicular lines.

\vfill
\begin{center}
\begin{tikzpicture}[myxyz, scale=1, thin border nodes]
  \coordinate (u) at (1,0,0);
  \coordinate (v) at (0,1.1,-.2);
  \coordinate (uxv) at (0,.2,1.1);
  \coordinate (x) at ($-1.1*(u)+(v)+1.5*(uxv)$);
  \begin{scope}[x=(u),y=(v),transformxy]
    \fill[seq-violet!30] (-2,-2) rectangle (2,2);
    \draw[seq-violet, help lines] (-2,-2) grid (2,2);
    \node[seq-violet] at (2.6,1) {$W$};
    \draw (-2,0) -- node[near end, above] {$L_1$} (2,0);
    \draw (0,-2) -- node[near start, above left] {$L_2$} (0,2);
    \point (p1) at (-1.1,0);
    \point["$\proj_{L_2}(x)$" below right] (p2) at (0,1);
    \pic[draw] {right angle=(p1)--(o)--(p2)};
  \end{scope}
  \pgfmathanglebetweenpoints{\pgfpoint{0cm}{0cm}}{\pgfpointanchor{v}{center}}
  \node[anchor=mid west, inner xsep=1.5mm, rotate={\pgfmathresult+180}] 
    at (p1) {$\proj_{L_1}(x)$};
  \draw[densely dashed] (x) -- (p1);
  \draw[densely dashed] (x) -- (p2);
  \point (o) at (0,0,0);
  \coordinate (oo) at (-2,0,0);
  \pic[draw] {right angle=(x)--(p1)--(oo)};
  \pic[draw] {right angle=(x)--(p2)--(o)};
  \point[seq-blue, "$x_W$" {below,text=seq-blue}] (y) at ($-1.1*(u)+1*(v)$);
  \point[seq-red, "$x$" {above,text=seq-red}] (xx) at (x);
  \draw[densely dashed] (p1) -- (y);
  \draw[densely dashed] (p2) -- (y);
  \draw (xx) -- (y);
  \pic[draw] {right angle=(p1)--(y)--(xx)};
  \pic[draw] {right angle=(p2)--(y)--(xx)};
\end{tikzpicture}
\end{center}

\vfill

\end{frame}


%%%%%%%%%%%%%%%%%%%%%%%%%%%%%%%%%%%%%%%%%%%%%%%%%%%%%%%%%%%%%%%%%%%

\begin{frame}
\frametitle{Orthogonal Projections}
\framesubtitle{Properties}

First we restate the property we've been using all along.

\begin{oneoffthm}{Best Approximation Theorem}
  Let $W$ be a subspace of $\R^n$, and let $x$ be a vector in $\R^n$.  Then
  $y = \proj_W(x)$ is the closest point in $W$ to $x$, in the sense that
  \[ \dist(x,y') \geq \dist(x,y) \sptxt{for all} y'\text{ in }W. \]
\end{oneoffthm}

\pause
We can think of orthogonal projection as a \emph{transformation}:
\[ \proj_W\colon\R^n\To\R^n \qquad x \mapsto \proj_W(x). \]

\pause
\begin{thm}
  Let $W$ be a subspace of $\R^n$.
  \pause
  \begin{enumerate}
  \item $\proj_W$ is a \emph{linear} transformation.
    \pause
  \item For every $x$ in $W$, we have $\proj_W(x) = \pause x$.
    \pause
  \item For every $x$ in $W^\perp$, we have $\proj_W(x) = \pause 0$.
    \pause
  \item The range of $\proj_W$ is
    \pause
    $W$.
  \end{enumerate}
\end{thm}

\end{frame}


%%%%%%%%%%%%%%%%%%%%%%%%%%%%%%%%%%%%%%%%%%%%%%%%%%%%%%%%%%%%%%%%%%%

\begin{pollframe}

Let $W$ be a subspace of $\R^n$.

\begin{bluebox}[Poll]{.9\linewidth}
  Let $A$ be the matrix for $\proj_W$.  What is/are the eigenvalue(s) of $A$?
  \[
  \text{\alert{A. }} 0 \quad
  \text{\alert{B. }} 1 \quad
  \text{\alert{C. }} -1 \quad
  \text{\alert{D. }} 0,\, 1 \quad
  \text{\alert{E. }} 1,\,-1 \quad
  \text{\alert{F. }} 0,\,-1 \quad
  \text{\alert{G. }} -1,\,0,\,1
  \]
\end{bluebox}

\pause\medskip
The $1$-eigenspace is
\pause
$W$.

\pause\medskip
The $0$-eigenspace is
\pause
$W^\perp$.

\pause\medskip
We have $\dim W+\dim W^\perp = n$, so that gives $n$ linearly independent
eigenvectors already.

\pause\medskip
So the answer is \alert{D}.

\end{pollframe}


%%%%%%%%%%%%%%%%%%%%%%%%%%%%%%%%%%%%%%%%%%%%%%%%%%%%%%%%%%%%%%%%%%%

\begin{frame}
\frametitle{Orthogonal Projections}
\framesubtitle{Matrices}

\displayskips{2pt}
What is the matrix for $\proj_W\colon\R^3\to\R^3$, where 
\[ W = \Span\left\{ \vec{1 0 -1},\;\vec{1 1 1} \right\}? \]

\footnotesize
\begin{webonly}%
\alert{Answer:}
Recall how to compute the matrix for a linear transformation:
\[ A = \mat{| | |; \proj_W(e_1) \proj_W(e_2) \proj_W(e_3); | | |}. \]
We compute:
\[\begin{split}
  \proj_W(e_1) &= \frac{e_1\cdot u_1}{u_1\cdot u_1}u_1 +
  \frac{e_1\cdot u_2}{u_2\cdot u_2}u_2
  = \frac 12\vec{1 0 -1} + \frac 13\vec{1 1 1}
  = \vec{5/6 1/3 -1/6} \\
  \proj_W(e_2) &= \frac{e_2\cdot u_1}{u_1\cdot u_1}u_1 +
  \frac{e_2\cdot u_2}{u_2\cdot u_2}u_2
  = 0 + \frac 13\vec{1 1 1}
  = \vec{1/3 1/3 1/3} \\
  \proj_W(e_3) &= \frac{e_3\cdot u_1}{u_1\cdot u_1}u_1 +
  \frac{e_3\cdot u_2}{u_2\cdot u_2}u_2
  = -\frac 12\vec{1 0 -1} + \frac 13\vec{1 1 1}
  = \vec{-1/6 1/3 5/6}
\end{split}\]
Therefore $A = \mat{5/6 1/3 -1/6; 1/3 1/3 1/3; -1/6 1/3 5/6}$.
  
\end{webonly}

\end{frame}


%%%%%%%%%%%%%%%%%%%%%%%%%%%%%%%%%%%%%%%%%%%%%%%%%%%%%%%%%%%%%%%%%%%

\begin{frame}
\frametitle{Orthogonal Projections}
\framesubtitle{Matrix facts}

Let $W$ be an $m$-dimensional subspace of $\R^n$, let $\proj_W\colon\R^n\to W$
be the projection, and let $A$ be the matrix for $\proj_L$.

\pause\medskip
\alert{Fact 1:} $A$ is diagonalizable with eigenvalues $0$ and $1$; it is
similar to the diagonal matrix with $m$ ones and $n-m$ zeros on the diagonal.

\medskip
\begin{webonly}
  \alert{Why?} Let $v_1,v_2,\ldots,v_m$ be a basis for $W$, and let
  $v_{m+1},v_{m+2},\ldots,v_n$ be a basis for $W^\perp$.  These are (linearly
  independent) eigenvectors with eigenvalues $1$ and $0$, respectively, and they
  form a basis for $\R^n$ because there are $n$ of them.

  \medskip
  \alert{Example:} If $W$ is a plane in $\R^3$, then $A$ is similar
  to projection onto the $xy$-plane:
  \[ \mat{1 0 0; 0 1 0; 0 0 0}. \]
\end{webonly}

\pause\medskip
\alert{Fact 2:} $A^2 = A$.

\medskip
\begin{webonly}
  \alert{Why?} Projecting twice is the same as projecting once:
  \[ \proj_W\circ\proj_W = \proj_W \implies A\cdot A = A. \]
\end{webonly}

\end{frame}



%%%%%%%%%%%%%%%%%%%%%%%%%%%%%%%%%%%%%%%%%%%%%%%%%%%%%%%%%%%%%%%%%%%

\begin{frame}
\frametitle{Orthogonal Projections}
\framesubtitle{Minimum distance}

What is the distance from $e_1$ to 
$W = \Span\left\{ \vec{1 0 -1},\;\vec{1 1 1} \right\}$? 

\medskip
\begin{webonly}%
\alert{Answer:}
The closest point on $W$ to $e_1$ is 
$\proj_W(e_1) = \vec{5/6 1/3 -1/6}$.\\[1mm]
The distance from $e_1$ to this point is\\[2mm]
$\;\displaystyle\begin{aligned}[c] \dist&\bigl(e_1,\proj_W(e_1)\bigr) 
= \|(e_1)_{W^\perp}\| \\
&= \left\|\vec{1 0 0}-\vec{5/6 1/3 -1/6}\right\| \\
&= \left\|\vec{1/6 -1/3 -1/6}\right\| \\
&= \sqrt{(1/6)^2+(-1/3)^2+(-1/6)^2}  \\
&= \frac 1{\sqrt 6}.
\end{aligned}$
\end{webonly}\hfill\pause
\begin{tikzpicture}[myxyz, scale=2, thin border nodes,
    right angle len=1.5mm, baseline=-.5cm]
  \coordinate (u) at (1,0,-1);
  \coordinate (v) at (1,1,1);
  \coordinate (x) at (1,0,0);
  \coordinate (y) at (5/6,1/3,-1/6);
  \begin{scope}[x=(u),y=(v),transformxy]
    \fill[seq-violet!30] (0,0) rectangle (1,1);
    \draw[seq-violet, help lines, step=.25] (0,0) grid (1,1);
    \node[seq-violet] at (1.1,.5) {$W$};
  \end{scope}
  \draw[vector] (0,0,0) to["$\vec{1 0 -1}$"'] (u);
  \draw[vector] (0,0,0) to["$\vec{1 1 1}$"] (v);
  \draw (x) to["$(e_1)_{W^\perp}$", "$\frac 1{\sqrt 6}$"' pos=.4] (y);
  \point[seq-red, "$e_1$" seq-red] at (x);
  \point[seq-blue, "$y$" {left,seq-blue}] at (y);
  \coordinate (yu) at ($(y)+(u)$);
  \coordinate (yv) at ($(y)+(v)$);
  \pic[draw] {right angle=(x)--(y)--(yu)};
  \pic[draw] {right angle=(x)--(y)--(yv)};
  \point at (0,0,0);
\end{tikzpicture}\hfill\null

\end{frame}


%%% Local Variables:
%%% TeX-master: "../slides"
%%% End:
