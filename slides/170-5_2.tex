
% JDR: This is less than 50 minutes; it can be combined with 5.1.

\usetikzlibrary{decorations.pathreplacing}

\titleframe{Section 5.2}{The Characteristic Equation}


%%%%%%%%%%%%%%%%%%%%%%%%%%%%%%%%%%%%%%%%%%%%%%%%%%%%%%%%%%%%%%%%%%%

\begin{frame}
\frametitle{The Invertible Matrix Theorem}
\framesubtitle{Addenda}

We have a couple of new ways of saying ``$A$ is invertible'' now:
\pause\medskip

\begin{oneoffthm}{The Invertible Matrix Theorem}
  Let $A$ be a square $n\times n$ matrix, and let $T\colon\R^n\to\R^n$ be the
  linear transformation $T(x) = Ax$.  The following statements are equivalent.\\
  \begin{enumerate}
  \item $A$ is invertible.
  \end{enumerate}
  \pause
  \rlap{
  \begin{tikzpicture}
    \node[scale=.7, anchor=west] (L) {\begin{minipage}{\linewidth}
      \begin{enumerate}
      \setcounter{enumi}{1}
      \item $T$ is invertible.
      \item $A$ is row equivalent to $I_n$.
      \item $A$ has $n$ pivots.
      \item $Ax=0$ has only the trivial solution.
      \item The columns of $A$ are linearly independent.
      \item $T$ is one-to-one.
      \item $Ax = b$ is consistent for all $b$ in $\R^n$.
      \item The columns of $A$ span $\R^n$.
      \item $T$ is onto.
      \end{enumerate}
      \end{minipage}};
    \node[scale=.7, right=-2.5cm of L] {
      \begin{minipage}{\linewidth}
      \begin{enumerate}
      \setcounter{enumi}{10}
      \item $A$ has a left inverse (there exists $B$ such that $BA = I_n$).
      \item $A$ has a right inverse (there exists $B$ such that $AB = I_n$).
      \item $A^T$ is invertible.
      \item The columns of $A$ form a basis for $\R^n$.
      \item $\Col A = \R^n$.
      \item $\dim\Col A = n$.
      \item $\rank A = n$.
      \item $\Nul A = \{0\}$.
      \item $\dim\Nul A = 0$.
      \end{enumerate}
    \end{minipage}};
  \useasboundingbox (-1mm,0);
  \end{tikzpicture}}\par\vskip-1mm
  \begin{enumerate}
    \setcounter{enumi}{18}
    \pause
  \item The determinant of $A$ is \emph{not} equal to zero.
    \pause
  \item The number $0$ is \emph{not} an eigenvalue of $A$.
  \end{enumerate}
\end{oneoffthm}

\end{frame}


%%%%%%%%%%%%%%%%%%%%%%%%%%%%%%%%%%%%%%%%%%%%%%%%%%%%%%%%%%%%%%%%%%%

\begin{frame}
\frametitle{The Characteristic Polynomial}

Let $A$ be a square matrix.
\[\begin{split}
  \text{$\lambda$ is an eigenvalue of $A$}
  &\iff\uncover<2->{Ax=\lambda x \text{ has a nontrivial solution}} \\
  &\uncover<3->{\iff (A-\lambda I)x = 0  \text{ has a nontrivial solution}} \\
  &\uncover<4->{\iff A-\lambda I \text{ is not invertible}} \\
  &\uncover<5->{\iff \det(A-\lambda I) = 0.}
\end{split}\]
\pause[6]%
This gives us a way to compute the eigenvalues of $A$.

\pause\medskip
\begin{defn}
  Let $A$ be a square matrix.  The \textbf{characteristic polynomial} of $A$ is
  \[ f(\lambda) = \det(A-\lambda I). \]
  \pause
  The \textbf{characteristic equation} of $A$ is the equation
  \[ f(\lambda) = \det(A-\lambda I) = 0. \]
\end{defn}

\pause
\begin{bluebox}[Important]{.7\linewidth}
  The eigenvalues of $A$ are the roots of the characteristic polynomial
  $f(\lambda) = \det(A-\lambda I)$.
\end{bluebox}

\end{frame}


%%%%%%%%%%%%%%%%%%%%%%%%%%%%%%%%%%%%%%%%%%%%%%%%%%%%%%%%%%%%%%%%%%%

\begin{frame}
\frametitle{The Characteristic Polynomial}
\framesubtitle{Example}

\alert{Question:} What are the eigenvalues of 
\[ A = \mat{5 2; 2 1}? \]

\begin{webonly}
\alert{Answer:}
First we find the characteristic polynomial:
\[\begin{split} f(\lambda) &= \det(A-\lambda I)
= \det\left[ \mat{5 2; 2 1} - \mat{\lambda, 0; 0 \lambda} \right]
= \det\mat{5-\lambda, 2; 2 1-\lambda} \\
&= (5-\lambda)(1-\lambda) - 2\cdot 2 \\
&= \lambda^2 - 6\lambda + 1.
\end{split}\]
The eigenvalues are the roots of the characteristic polynomial, which we can
find using the quadratic formula:
\[ \lambda = \frac{6\pm\sqrt{36-4}}2
= 3\pm 2\sqrt 2. \]
\end{webonly}

\end{frame}


%%%%%%%%%%%%%%%%%%%%%%%%%%%%%%%%%%%%%%%%%%%%%%%%%%%%%%%%%%%%%%%%%%%

\begin{frame}
\frametitle{The Characteristic Polynomial}
\framesubtitle{Example}

\alert{Question:} What is the characteristic polynomial of
\[ A = \mat{a b; c d}? \]

\displayskips{3pt}
\begin{webonly}
\alert{Answer:}\vskip -5mm
\[\begin{split} f(\lambda) &= \det(A-\lambda I)
= \det\mat{a-\lambda, b; c d-\lambda} 
= (a-\lambda)(d-\lambda) - bc \\
&= \lambda^2 - (a+d)\lambda + (ad-bc)
\end{split}\]
\end{webonly}

\pause
What do you notice about $f(\lambda)$?

\begin{itemize}
\pause
\item The constant term is $\det(A)$, which is zero if and only if
  \pause
  $\lambda=0$ is a root.

  \pause
\item The linear term $-(a+d)$ is the negative of the sum of the diagonal
  entries of $A$.
\end{itemize}

\pause
\begin{defn}
  The \textbf{trace} of a square matrix $A$ is 
  $\Tr(A) = \text{sum of the diagonal entries of $A$}$.
\end{defn}

\pause
\begin{bluebox}[Shortcut]{.7\linewidth}
  \displayskips{3pt}
  The characteristic polynomial of a $\color{seq-red}2\times 2$ matrix $A$ is
  \[ f(\lambda) = \lambda^2 - \Tr(A)\,\lambda + \det(A). \]
\end{bluebox}

\end{frame}


%%%%%%%%%%%%%%%%%%%%%%%%%%%%%%%%%%%%%%%%%%%%%%%%%%%%%%%%%%%%%%%%%%%

\begin{frame}
\frametitle{The Characteristic Polynomial}
\framesubtitle{Example}

\alert{Question:} What are the eigenvalues of the rabbit population matrix
\[ A = \mat{0 6 8; \frac 12 0 0; 0 \frac 12 0}? \]

\begin{webonly}
\alert{Answer:}
First we find the characteristic polynomial:
\abovedisplayskip=3pt\belowdisplayskip=\abovedisplayskip
\[\begin{split} f(\lambda) &= \det(A-\lambda I)
= \det\mat{-\lambda, 6 8; \frac 12 -\lambda, 0; 0 \frac 12 -\lambda} \\
&= 8\biggl(\frac 14 - 0\cdot-\lambda\biggr)
- \lambda\biggl( \lambda^2 - 6\cdot\frac 12 \biggr) \\
&= -\lambda^3 + 3\lambda + 2.
\end{split}\]
We know from before that one eigenvalue is $\lambda=2$: indeed,
$f(2) = -8 + 6 + 2 = 0$.  Doing polynomial long division, we get:
\[ \frac{-\lambda^3 + 3\lambda + 2}{\lambda-2} = -\lambda^2 - 2\lambda - 1
= -(\lambda+1)^2. \]
Hence $\lambda=-1$ is also an eigenvalue.
\end{webonly}

\end{frame}


%%%%%%%%%%%%%%%%%%%%%%%%%%%%%%%%%%%%%%%%%%%%%%%%%%%%%%%%%%%%%%%%%%%

\begin{frame}
\frametitle{Algebraic Multiplicity}

\vskip-3mm
\begin{defn}
  The \textbf{(algebraic) multiplicity} of an eigenvalue $\lambda$ is its
  multiplicity as a root of the characteristic polynomial.
\end{defn}

\pause\medskip
This is not a very interesting notion \emph{yet}.  
\pause
It will become interesting when we also define \emph{geometric} multiplicity
later. 

\pause\medskip
\begin{eg}
In the rabbit population matrix, 
$f(\lambda) = -(\lambda-2)(\lambda+1)^2$,
so the algebraic multiplicity of the eigenvalue
$2$ is \blankuntil{5}{1},
\pause[6]%
and the algebraic multiplicity of the eigenvalue
$-1$ is \blankuntil{7}{2}.
\end{eg}

\pause[8]\medskip
\begin{eg}
In the matrix $\mat{5 2; 2 1}$, 
$f(\lambda) = (\lambda-(3-2\sqrt2))(\lambda-(3+2\sqrt 2))$, so
the algebraic multiplicity of
$3+2\sqrt 2$ is \blankuntil{9}{1},
\pause[10]%
and the algebraic multiplicity of
$3-2\sqrt 2$ is \blankuntil{11}{1}.
\end{eg}

\end{frame}


%%%%%%%%%%%%%%%%%%%%%%%%%%%%%%%%%%%%%%%%%%%%%%%%%%%%%%%%%%%%%%%%%%%

\begin{frame}
\frametitle{The Characteristic Polynomial}

\displayskips{3mm}
\alert{Fact:}
If $A$ is an $n\times n$ matrix, the characteristic polynomial
\[ f(\lambda) = \det(A-\lambda I) \]
turns out to be a polynomial of degree $n$, and its roots are the eigenvalues of
$A$:
\[ f(\lambda) 
= (-1)^n\lambda^n + a_{n-1}\lambda^{n-1} + a_{n-2}\lambda^{n-2} + \cdots +
a_1\lambda + a_0. \]

\ifx\slidesmode\blankmode\else{
\framesubtitle{Poll}
\pause
\begin{bluebox}[Poll]{.7\linewidth}
  If you count the eigenvalues of $A$, with their algebraic multiplicities, you
  will get:
  \begin{eAlpherate}
  \item Always $n$.
  \item Always at most $n$, but sometimes less.
  \item Always at least $n$, but sometimes more.
  \item None of the above.
  \end{eAlpherate}
\end{bluebox}

\pause\smallskip
The answer depends on whether you allow \emph{complex} eigenvalues.
\pause
If you only allow real eigenvalues, the answer is \alert{B.}
\pause
Otherwise it is \alert{A,} because any degree-$n$ polynomial has exactly $n$
\emph{complex} roots, counted with multiplicity.
\pause
Stay tuned.
}\fi

\end{frame}


%%%%%%%%%%%%%%%%%%%%%%%%%%%%%%%%%%%%%%%%%%%%%%%%%%%%%%%%%%%%%%%%%%%

\begin{frame}
\frametitle{Similarity}

\vskip-3mm
\begin{defn}
  Two $n\times n$ matrices $A$ and $B$ are \textbf{similar} if there is an
  invertible $n\times n$ matrix $C$ such that
  \[ A = CBC\inv. \]
\end{defn}

\pause
\alert{What does this mean?}
\begin{webonly}
Say the columns of $C$ are $v_1,v_2,\ldots,v_n$.
These form a basis $\cB = \{v_1,v_2,\ldots,v_n\}$ for $\R^n$ because $C$ is invertible.
If $x = c_1v_1 + c_2v_2 + \cdots + c_nv_n$ then 
\displayskips{5pt}
\[ [x]_\cB = 
\vec{c_1 c_2 \vdots, c_n}
\implies x = c_1v_1 + c_2v_2 + c_nv_n
= C[x]_\cB. \]
Since $x = C[x]_\cB$ we have $[x]_\cB = C\inv x$.
\[ B[x]_\cB = [y]_\cB \implies Ax = CBC\inv x = CB[x]_\cB = C[y]_\cB = y. \]
\end{webonly}

\pause
\begin{bluebox}{.8\linewidth}
  $A$ acts on the standard coordinates of $x$ in the same way that $B$ acts
  on the $\cB$-coordinates of $x$: $B[x]_\cB = [Ax]_\cB$.
\end{bluebox}

\end{frame}


%%%%%%%%%%%%%%%%%%%%%%%%%%%%%%%%%%%%%%%%%%%%%%%%%%%%%%%%%%%%%%%%%%%

\begin{frame}
\frametitle{Similarity}
\framesubtitle{Example}

\vskip -5mm
\[ A = \mat{1 2; -1 4} \quad
B = \mat{2 0; 0 3}\quad
C = \mat{2 1; 1 1}
\quad\implies\quad
A = CBC\inv. \]
\pause
\alert{What does $B$ do geometrically?}
\pause
It scales the $x$-direction by $2$ and the $y$-direction by $3$.

\pause\smallskip
So $A$ does to the standard coordinates what $B$ does to the $\cB$-coordinates, where
\[ \cB = \left\{ \vec{2 1},\,\vec{1 1} \right\}. \]

\pause
\begin{center}
\begin{tikzpicture}[scale=.5, thin border nodes]
  \node at (5,3.75) {$B$ acting on the usual coordinates};

  \begin{scope}[scale=.6]
    \draw[grid lines] (-5,-5) grid (5,5);
    \draw[thin] (-5,0) -- (5,0) (0,-5) -- (0,5);
    \draw[<-] (2.5,0) -- ++(0,-1.8) node[below, font=\scriptsize] {2-eigenspace};
    \draw[<-] (0,2.5) -- ++(-.8,0) node[left, font=\scriptsize] {3-eigenspace};
    \draw[thin, dashed] (-5,-5) rectangle (5,5);
    \draw[vector, seq-orange] (0,0) to["$e_1$" swap] (1,0);
    \draw[vector, seq-blue] (0,0) to["$e_2$"] (0,1);
    \point at (0,0);
    \point[seq-red, "$x$" {above,text=seq-red}] at (1,1);
    \point[seq-green, "$y$" {below,text=seq-green}] at (-2,-1);
    \coordinate (X) at (5,1);
  \end{scope}

  \begin{scope}[xshift=10cm, scale=.6]
    \draw[grid lines] (-5,-5) grid (5,5);
    \draw[thin] (-5,0) -- (5,0) (0,-5) -- (0,5);
    \draw[thin, dashed] (-5,-5) rectangle (5,5);
    \draw[vector, seq-orange] (0,0) to["$Be_1$" swap] (2,0);
    \draw[vector, seq-blue] (0,0) to["$Be_2$"] (0,3);
    \point at (0,0);
    \point[seq-red, "$Bx$" {above,text=seq-red}] at (2,3);
    \point[seq-green, "$By$" {below,text=seq-green}] at (-4,-3);
    \coordinate (Y) at (-5,1);
  \end{scope}[scale=.5]

  \draw[->, shorten=1mm] (X) to[bend left, "$B$" yshift=2pt] (Y);

\end{tikzpicture}
\end{center}

\end{frame}


%%%%%%%%%%%%%%%%%%%%%%%%%%%%%%%%%%%%%%%%%%%%%%%%%%%%%%%%%%%%%%%%%%%

\begin{frame}

\vskip2mm
\leavevmode
\hbox to \linewidth{\hss
\begin{tikzpicture}[scale=.25, thin border nodes, baseline, remember picture]
  \node at (0,15) 
    {$A$ does to the usual coordinates what $B$ does to the $\cB$-coordinates};
  \clip (-16,-14) rectangle (16,14);
  \pgftransformcm2111{\pgfpointorigin}
  \draw[grid lines] (-50,-50) grid (50,50);
  \draw[thin] (-50,0) -- (50,0) (0,-50) -- (0,50);
  \draw[thin, dashed] (-5,-5) rectangle (5,5);
  \draw[vector, seq-orange] (0,0) to["$v_1$" swap] (1,0);
  \draw[vector, seq-blue] (0,0) to["$v_2$"] (0,1);
  \point at (0,0);
  \point[seq-red, "$x$" {above right,text=seq-red}] at (1,1);
  \point[seq-green, "$y$" {below left,text=seq-green}] at (-2,-1);
  \pgftransformreset
  \begin{scope}[x={(2cm,1cm)},y={(1cm,1cm)}]
    \draw[<-] (1.6,0) -- ++(0cm,-.6cm) node[below, font=\scriptsize] {2-eigenspace};
    \draw[<-] (0,2) -- ++(155:.6cm) node[left, font=\scriptsize] {3-eigenspace};
  \end{scope}
\end{tikzpicture}
\begin{minipage}[c]{0.3\linewidth}
  \def\r{\textcolor{seq-red}}
  \def\g{\textcolor{seq-green}}
  $\displaystyle
  \begin{aligned}[c]
    \textcolor{seq-orange}{v_1} &= \namedbox{v1}{\vec{2 1}} \\
    \textcolor{seq-blue}{v_2} &= \namedbox{v2}{\vec{1 1}} \\
    \phantom B[\r x]_\cB &= \vec{1 1} \\
    \r x &= \webonlycmd{v_1 + v_2 = \vec{3 2}} \\
    [\g y]_\cB &= \vec{-2 -1} \\
    \g y &= \webonlycmd{-2v_1 - v_2} \\
    &\webonlycmd{\quad= \vec{-5 -3}}
  \end{aligned}$
\end{minipage}\hss}
\begin{tikzpicture}[remember picture, overlay]
  \draw[decoration={brace,amplitude=2mm}, decorate, thick]
    let \p1=($(v1.north east) + (4mm,0)$) in
      (\p1) -- 
        node[rotate=-90, yshift=4mm] {vectors in $\cB$}
      (\p1 |- v2.south);
\end{tikzpicture}

\end{frame}


%%%%%%%%%%%%%%%%%%%%%%%%%%%%%%%%%%%%%%%%%%%%%%%%%%%%%%%%%%%%%%%%%%%

\begin{frame}

\vskip2mm
\leavevmode\hbox to \linewidth{\hss
\begin{tikzpicture}[scale=.25, thin border nodes, baseline]
  \node at (0,15) 
    {$A$ does to the usual coordinates what $B$ does to the $\cB$-coordinates};
  \clip (-16,-14) rectangle (16,14);
  \pgftransformcm2111{\pgfpointorigin}
  \draw[grid lines] (-50,-50) grid (50,50);
  \draw[thin] (-50,0) -- (50,0) (0,-50) -- (0,50);
  \draw[thin, dashed] (-5,-5) rectangle (5,5);
  \draw[vector, seq-orange] (0,0) to["$Av_1$" swap] (2,0);
  \draw[vector, seq-blue] (0,0) to["$Av_2$"] (0,3);
  \point at (0,0);
  \point[seq-red, "$Ax$" {above right,text=seq-red}] at (2,3);
  \point[seq-green, "$Ay$" {below left,text=seq-green}] at (-4,-3);
  \pgftransformreset
  \begin{scope}[x={(2cm,1cm)},y={(1cm,1cm)}]
    \draw[<-] (1.6,0) -- ++(0cm,-.6cm) node[below, font=\scriptsize] {2-eigenspace};
    \draw[<-] (0,2) -- ++(155:.6cm) node[left, font=\scriptsize] {3-eigenspace};
  \end{scope}
\end{tikzpicture}
\begin{minipage}[c]{0.3\linewidth}
  \def\r{\textcolor{seq-red}}
  \def\g{\textcolor{seq-green}}
  $\displaystyle
  \begin{aligned}[c]
    \textcolor{seq-orange}{Av_1} &=\webonlycmd{2v_1 = \vec{4 2}} \\
    \textcolor{seq-blue}{Av_2} &=\webonlycmd{3v_2 = \vec{3 3}} \\
    B[\r x]_\cB &= \webonlycmd{\vec{2 3}} = [\r{Ax}]_\cB \\
    \r{Ax} &= \webonlycmd{2v_1 + 3v_2 = \vec{7 5}} \\
    B[\g y]_\cB &= \webonlycmd{\vec{-4 -3}} = [\g{Ay}]_\cB \\
    \g{Ay} &= \webonlycmd{-4v_1 - 3v_2} \\
    &\webonlycmd{\quad= \vec{-11 -7}}
  \end{aligned}$
\end{minipage}\hss}\\[3mm]
\hskip-3mm{\alert{Check:}}
\hfill
\webonlycmd{$\displaystyle
\textcolor{seq-red}{Ax} = \mat{1 2; -1 4}\vec{3 2} = \vec{7 5} \quad
\textcolor{seq-green}{Ay} = \mat{1 2; -1 4}\vec{-5 -3} = \vec{-11 -7} \bigcheck$}
\hfill\null

\end{frame}


%%%%%%%%%%%%%%%%%%%%%%%%%%%%%%%%%%%%%%%%%%%%%%%%%%%%%%%%%%%%%%%%%%%

\begin{frame}
\frametitle{Similar Matrices Have the Same Characteristic Polynomial}

\alert{Fact:} If $A$ and $B$ are similar, then they have the same characteristic
polynomial.

\pause\medskip
\alert{Why?}
Suppose $A = CBC\inv$.
\begin{webonly}
\[\begin{split}
  A-\lambda I &= CBC\inv - \lambda I \\
  &= CBC\inv - C(\lambda I)C\inv \\
  &= C(B-\lambda I)C\inv.
\end{split}\]
Therefore,
\[\begin{split} \det(A-\lambda I) &= \det\bigl(C(B-\lambda I)C\inv\bigr) \\
&= \det(C)\det(B-\lambda I)\det(C\inv) \\
&= \det(B-\lambda I), \end{split}\]
because $\det(C\inv) = \det(C)\inv$.
\end{webonly}

\pause\medskip
\begin{bluebox}{.7\linewidth}
  \alert{Conseqence:} similar matrices have the same eigenvalues!  
  \pause
  (But different eigenvectors in general.)
\end{bluebox}

\end{frame}


%%%%%%%%%%%%%%%%%%%%%%%%%%%%%%%%%%%%%%%%%%%%%%%%%%%%%%%%%%%%%%%%%%%

\begin{frame}
\frametitle{Similarity}
\framesubtitle{Caveats}

\vfill
\begin{bluebox}[Warning]{.8\linewidth}
  \begin{enumerate}
  \item Matrices with the same eigenvalues need not be similar.  For instance,
    \[ \mat{2 1; 0 2} \sptxt{and} \mat{2 0; 0 2} \]
    both only have the eigenvalue $2$, but they are not similar.

    \pause

  \item Similarity has nothing to do with row equivalence.  For instance,
    \[ \mat{2 1; 0 2} \sptxt{and} \mat{1 0; 0 1} \]
    are row equivalent, but they have different eigenvalues.
  \end{enumerate}
\end{bluebox}
\vfill

\end{frame}


%%% Local Variables:
%%% TeX-master: "../slides"
%%% End:
