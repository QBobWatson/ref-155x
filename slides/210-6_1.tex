
\usetikzlibrary{decorations.pathreplacing}

\titleframe{Chapter 6}{Orthogonality and Least Squares}
\titleframe{Section 6.1}{Inner Product, Length, and Orthogonality}


%%%%%%%%%%%%%%%%%%%%%%%%%%%%%%%%%%%%%%%%%%%%%%%%%%%%%%%%%%%%%%%%%%%

\begin{frame}
\frametitle{Orientation}

\alert{Recall:} This course is about learning to:

\begin{itemize}
\item \textcolor<2->{gray}{Solve the matrix equation $Ax = b$}\\
  \pause
\item \textcolor<3->{gray}{Solve the matrix equation $Ax = \lambda x$}\\
  \pause
\item Almost solve the equation $Ax = b$
\end{itemize}

\pause
We are now aiming at the last topic.

\pause\medskip
\alert{Idea:} In the real world, data is imperfect.
\pause
Suppose you measure a data point $x$ which you know for theoretical reasons must lie
on a plane spanned by two vectors $u$ and $v$.
\pause
\begin{center}
\begin{tikzpicture}[myxyz, scale=.5, thin border nodes]
  \coordinate (u) at (1,0,0);
  \coordinate (v) at (0,1.1,-.2);
  \coordinate (uxv) at (0,.2,1.1);
  \coordinate (x) at ($-1.1*(u)+(v)+1.5*(uxv)$);
  \coordinate (px) at ($-1.1*(u)+(v)$);
  \begin{scope}[x=(u),y=(v),transformxy]
    \fill[seq-violet!30] (-2,-2) rectangle (2,2);
    \draw[help lines] (-2,-2) grid (2,2);
  \end{scope}
  \draw[vector] (0,0,0) -- (u) node[right] {$u$};
  \draw[vector] (0,0,0) -- (v) node[left] {$v$};
  \coordinate (xu) at ($(px) + .3*(u)$);
  \coordinate (xv) at ($(px) + .3*(v)$);
  \pic<13->[draw] {right angle=(x)--(px)--(xu)};
  \pic<13->[draw] {right angle=(x)--(px)--(xv)};
  \draw<13-> (x) -- (px);
  \point<13-> at (px);
  \point[seq-red, "$x$" {above,text=seq-red}] at (x);
  \point at (0,0,0);
\end{tikzpicture}
\end{center}
\pause
Due to measurement error, though, the measured $x$ is not actually in
$\Span\{u,v\}$.
\pause
In other words, the equation
$au + bv = x$ has no solution.
\pause
What do you do?
\pause
The real value is probably the \emph{closest} point to $x$ on $\Span\{u,v\}$.
\pause
Which point is that?

\end{frame}


%%%%%%%%%%%%%%%%%%%%%%%%%%%%%%%%%%%%%%%%%%%%%%%%%%%%%%%%%%%%%%%%%%%

\begin{frame}
\frametitle{The Dot Product}

We need a notion of \emph{angle} between two vectors, and in particular, a
notion of \emph{orthogonality} (i.e.\ when two vectors are perpendicular).
\pause
This is the purpose of the dot product.

\pause
\begin{defn}
  The \textbf{dot product} of two vectors $x,y$ in $\R^n$ is
  \[ x\cdot y = \vec{x_1 x_2 \vdots, x_n} \cdot \vec{y_1 y_2 \vdots, y_n}
  \overset{\rm def}= x_1y_1 + x_2y_2 + \cdots + x_ny_n. \]
  \pause
  Thinking of $x,y$ as column vectors, this is the same as $x^Ty$.
\end{defn}

\pause
\begin{eg}
  \[ \vec{1 2 3}\cdot\vec{4 5 6} = \mat{1 2 3}\vec{4 5 6}
  = \webonlycmd{1\cdot 4 + 2\cdot 5 + 3\cdot 6 = 32.} \]
\end{eg}

\end{frame}


%%%%%%%%%%%%%%%%%%%%%%%%%%%%%%%%%%%%%%%%%%%%%%%%%%%%%%%%%%%%%%%%%%%

\begin{frame}
\frametitle{Properties of the Dot Product}

Many usual arithmetic rules hold, as long as you remember you can only dot two
vectors together, and that the result is a scalar.

\pause
\begin{itemize}
\item $x\cdot y = y \cdot x$
  \pause
\item $(x + y)\cdot z = x\cdot z+y\cdot z$
  \pause
\item $(cx)\cdot y = c(x\cdot y)$
\end{itemize}

\pause\medskip
Dotting a vector with itself is special:
\[ \vec{x_1 x_2 \vdots, x_n}\cdot\vec{x_1 x_2 \vdots, x_n}
= x_1^2 + x_2^2 + \cdots + x_n^2. \]
Hence:
\pause
\begin{itemize}
\item $x\cdot x\geq 0$
  \pause
\item $x\cdot x = 0$ if and only if $x = 0$.
\end{itemize}

\pause\medskip
\alert{Important:} $x\cdot y = 0$ does \emph{not} imply $x = 0$ or $y = 0$.
\pause
For example, ${1\choose 0} \cdot {0\choose 1} = 0$.

\end{frame}


%%%%%%%%%%%%%%%%%%%%%%%%%%%%%%%%%%%%%%%%%%%%%%%%%%%%%%%%%%%%%%%%%%%

\begin{frame}
\frametitle{The Dot Product and Length}

\vskip -3mm
\begin{defn}
  The \textbf{length} or \textbf{norm} of a vector $x$ in $\R^n$ is
  \[\|x\| = \sqrt{x\cdot x} = \sqrt{x_1^2 + x_2^2 + \cdots + x_n^2}.\]
\end{defn}

\pause
Why is this a good definition?
\pause
The Pythagorean theorem!
\begin{center}
\begin{tikzpicture}[scale=.5,decoration={brace,raise=3pt}]
  \draw[grid lines] (0,0) grid (5,5);
  \point at (0,0);
  \point["$\vec{3 4}$" right] (x) at (3,4);
  \draw[vector] (0,0) 
    -- node[sloped,above,inner sep=1pt] {$\sqrt{3^2 + 4^2} = 5$}
    (x);
  \draw[decorate,decoration={mirror}] (0,0) -- node[below=4pt] {3} (3,0);
  \draw[decorate] (0,0) -- node[left=3pt] {4} (0,4);

  \node at (10,2.5) {$\left\|\vec{3 4}\right\| = \sqrt{3^2+4^2} = 5$};
  
\end{tikzpicture}
\end{center}

\pause\vskip -3mm
\begin{fact2}
  If $x$ is a vector and $c$ is a scalar, then
  $\|cx\| = |c|\cdot\|x\|$.
\end{fact2}

\pause
\[ \displaystyle\left\|\vec{6 8}\right\| = \left\|2\vec{3 4}\right\| = 
\webonlycmd{2\left\|\vec{3 4}\right\| = 10} \]

\end{frame}


%%%%%%%%%%%%%%%%%%%%%%%%%%%%%%%%%%%%%%%%%%%%%%%%%%%%%%%%%%%%%%%%%%%

\begin{frame}
\frametitle{The Dot Product and Distance}

\vskip -3mm
\begin{defn}
  The \textbf{distance} between two points $x,y$ in $\R^n$ is
  \[ \dist(x,y) = \pause \|y-x\|. \]
\end{defn}

\pause
This is just the length of the vector from $x$ to $y$.

\pause
\begin{eg}
  Let $x = (1,2)$ and $y = (4,4)$.  Then
  \[ \dist(x,y) = \webonlycmd{
    \|y-x\| = \left\|\vec{3,2}\right\| = \sqrt{3^2+2^2} = \sqrt{13}.
  } \]
  \pause
  \begin{center}
  \begin{tikzpicture}[scale=.5, thin border nodes]
    \draw[grid lines] (0,0) grid (5,5);
    \point["$0$" left] at (0,0);
    \point["$x$" left] (x) at (1,2);
    \point["$y$" right] (y) at (4,4);
    \draw[vector] (x) --
      node[sloped,above] {$y-x$} (y);

  \end{tikzpicture}
  \end{center}

\end{eg}

\end{frame}


%%%%%%%%%%%%%%%%%%%%%%%%%%%%%%%%%%%%%%%%%%%%%%%%%%%%%%%%%%%%%%%%%%%

\begin{frame}
\frametitle{Unit Vectors}

\vskip-3mm
\begin{defn}
  A \textbf{unit vector} is a vector $v$ with length $\|v\|=1$.
\end{defn}

\pause
\begin{eg}
  The unit coordinate vectors are unit vectors:
  \[
    \|e_1\| = \left\|\vec{1 0 0}\right\| = \sqrt{1^2 + 0^2 + 0^2} = 1 
  \]
\end{eg}

\pause
\begin{defn}
  Let $x$ be a nonzero vector in $\R^n$.  The
  \textbf{unit vector in the direction of $x$} is the vector
  $\displaystyle\frac{\mathstrut x}{\|x\|}$.
\end{defn}

\pause\medskip
This is in fact a unit vector:
\[ \left\|\frac{x}{\namedbox{scalar}{\|x\|}}\right\|
= \frac 1{\|x\|}\|x\| = 1. \]
\begin{tikzpicture}[remember picture, overlay]
  \node[draw,blue!50,circle,fit=(scalar),inner sep=.5pt] (scalarbox) {};
  \path (scalar) ++ (-2cm,2mm) node[blue!50] (expl) {scalar};
  \draw[->,blue!50] (expl) to[out=-45,in=225] (scalarbox);
\end{tikzpicture}

\end{frame}


%%%%%%%%%%%%%%%%%%%%%%%%%%%%%%%%%%%%%%%%%%%%%%%%%%%%%%%%%%%%%%%%%%%

\begin{frame}
\frametitle{Unit Vectors}
\framesubtitle{Example}

\vskip-3mm
\begin{eg}
  What is the unit vector in the direction of $x = \vec{3 4}$?
  \begin{webonly}
  \[ u = \frac{x}{\|x\|} = \frac 1{\sqrt{3^2+4^2}}\vec{3 4}
  = \frac{1}{5}\vec{3 4}. \]
  \begin{center}
    \medskip
  \begin{tikzpicture}[scale=.5, thin border nodes]
    \draw[grid lines] (-1,-1) grid (5,5);
    \coordinate["$x$" above] (x) at (3,4);
    \draw[vector] (0,0) -- (x);
    \draw[vector,seq-blue] (0,0) to["$u$"] ($1/5*(x)$);
    \point["$0$" left] at (0,0);
  \end{tikzpicture}
  \end{center}

  \end{webonly}

\end{eg}

\end{frame}


%%%%%%%%%%%%%%%%%%%%%%%%%%%%%%%%%%%%%%%%%%%%%%%%%%%%%%%%%%%%%%%%%%%

\begin{frame}
\frametitle{Orthogonality}

\vskip 1mm
\namedbox{ortho}{
\begin{minipage}{\linewidth}
\vskip-2mm
\begin{defn}
  Two vectors $x,y$ are \textbf{orthogonal} or \textbf{perpendicular} if
  $x\cdot y = 0$.\\
  \uncover<2->{\emph{Notation:} $x\perp y$ means $x\cdot y = 0$.}
\end{defn}
\end{minipage}}
\tikz[remember picture, overlay]
  \node[redbox, fit=(ortho)] {};

\pause[3]\bigskip
Why is this a good definition?
\pause
The Pythagorean theorem / law of cosines!
\begin{center}
\begin{tikzpicture}[scale=.75]
  \point ["$x$" right] (x) at (3,1);
  \point ["$y$" above] (y) at (-1,3);
  \coordinate (o) at (0,0);
  \draw[vector] (0,0) -- node[below] {$\|x\|$} (x);
  \draw[vector] (0,0) -- node[left] {$\|y\|$} (y);
  \draw[vector] (y) -- node[sloped,above] {$\|x-y\|$} (x);
  \pic[draw] {right angle=(x)--(o)--(y)};
\end{tikzpicture}
\end{center}
\begin{webonly}
  \[\begin{split}
    \parbox{2cm}{\centering $x$ and $y$ are\\perpendicular}
    &\iff \|x\|^2 + \|y\|^2 = \|x-y\|^2 \\
    &\iff x\cdot x + y\cdot y = (x-y)\cdot(x-y) \\
    &\iff x\cdot x + y\cdot y = x\cdot x + y\cdot y -2x\cdot y \\
    &\iff x\cdot y = 0
  \end{split}\]
\end{webonly}

\pause
\alert{Fact:} $x\perp y \iff \|x-y\|^2=\|x\|^2+\|y\|^2$ 

\end{frame}


%%%%%%%%%%%%%%%%%%%%%%%%%%%%%%%%%%%%%%%%%%%%%%%%%%%%%%%%%%%%%%%%%%%

\begin{frame}
\frametitle{Orthogonality}
\framesubtitle{Example}

\alert{Problem:} Find \emph{all} vectors orthogonal to $v = \vec{1,1,-1}$.

\medskip
\begin{webonly}
We have to find all vectors $x$ such that $x\cdot v = 0$.  This means solving
the equation
\[ 0 = x\cdot v = \vec{x_1 x_2 x_3}\cdot\vec{1 1 -1} = x_1 + x_2 - x_3. \]
The parametric form for the solution is $x_1 = -x_2 + x_3$, so the parametric
vector form of the general solution is
\[ x = \vec{x_1 x_2 x_3} = x_2\vec{-1 1 0} + x_3\vec{1 0 1}. \]
For instance, $\vec{-1 1 0}\perp\vec{1 1 -1}$ because
$\vec{-1 1 0}\cdot\vec{1 1 -1} = 0$.
\end{webonly}

\end{frame}


%%%%%%%%%%%%%%%%%%%%%%%%%%%%%%%%%%%%%%%%%%%%%%%%%%%%%%%%%%%%%%%%%%%

\begin{frame}
\frametitle{Orthogonality}
\framesubtitle{Example}

\alert{Problem:} Find \emph{all} vectors orthogonal to both $v = \vec{1 1 -1}$ and 
$w = \vec{1 1 1}$.

\medskip
\begin{webonly}
Now we have to solve the system of two homogeneous equations
\spalignsysdelims..
\[ \syseq{
  0 = x\cdot v = \vec{x_1 x_2 x_3}\cdot\vec{1 1 -1} = x_1 + x_2 - x_3;
  0 = x\cdot w = \vec{x_1 x_2 x_3}\cdot\vec{1 1 1}  = x_1 + x_2 + x_3\rlap.} \]
In matrix form:
\[ \namedbox{vw}{\mat{1 1 -1; 1 1 1}} \;\longsquiggly[rref]\;\mat{1 1 0; 0 0 1}. \]
The parametric vector form of the solution is 
\[ \vec{x_1 x_2 x_3} = x_2\vec{-1 1 0}. \]
\begin{tikzpicture}[remember picture, overlay]
  \path (vw.west) ++(-5mm,0)
    node[blue!50, anchor=mid east, inner sep=3pt] (expl) {The rows are $v$ and $w$};
  \draw[->, blue!50] (expl.mid east) -- (vw.west);
\end{tikzpicture}
\end{webonly}

\end{frame}


%%%%%%%%%%%%%%%%%%%%%%%%%%%%%%%%%%%%%%%%%%%%%%%%%%%%%%%%%%%%%%%%%%%

\begin{frame}
\frametitle{Orthogonality}
\framesubtitle{General procedure}

\alert{Problem:} Find all vectors orthogonal to some number of vectors
$v_1,v_2,\ldots,v_m$ in $\R^n$.

\pause\medskip
This is the same as finding all vectors $x$ such that 
\[ 0 = v_1^Tx = v_2^Tx = \cdots = v_m^Tx. \]
\pause\leavevmode
\begin{minipage}[c]{0.5\linewidth}
  Putting the \emph{row} vectors $v_1^T,v_2^T,\ldots,v_m^T$ into a matrix, this
  is the same as finding all $x$ such that
\end{minipage}%
\begin{minipage}[c]{0.5\linewidth}
  \[
  \mat{ \matrow{v_1^T};
    \matrow{v_2^T}; \spvdots;
    \matrow{v_m^T}} x = 
  \vec{v_1\cdot x v_2\cdot x \spvdots, v_m\cdot x} = 0. \]
\end{minipage}%
\pause\medskip
\begin{bluebox}[Important]{.8\linewidth}
  \vskip-2mm
  \begin{minipage}[c]{0.7\linewidth}
    The set of all vectors orthogonal to some vectors $v_1,v_2,\ldots,v_m$ in
    $\R^n$ is the \emph{null space} of the $m\times n$ matrix
  \end{minipage}
  \begin{minipage}[c]{0.25\linewidth}
  \[ \mat{
    \matrow{v_1^T};
    \matrow{v_2^T};
    \spvdots;
    \matrow{v_m^T}}. \]
  \end{minipage}\\
  \pause
  In particular, this set is a subspace!
\end{bluebox}

\end{frame}


%%%%%%%%%%%%%%%%%%%%%%%%%%%%%%%%%%%%%%%%%%%%%%%%%%%%%%%%%%%%%%%%%%%

\begin{frame}
\frametitle{Orthogonal Complements}

\vskip-3mm
\begin{defn}
  Let $W$ be a subspace of $\R^n$.  Its \textbf{orthogonal complement} is
  \[ W^{\namedbox{perp}{\scriptstyle\perp}} = \bigl\{ \text{$v$ in $\R^n$}\mid
  v\cdot w=0 \text{ for all $w$ in $W$} \bigr\}
  \rlap{\qquad\text{read ``$W$ perp''}.} \]
\end{defn}
\pause
\begin{tikzpicture}[remember picture, overlay]
  \node[blue!50, draw, circle, inner sep=.5pt, fit=(perp)] (perpbox) {};
  \path (perpbox.south) ++(3mm,-7mm)
    node[right, blue!50, align=left] (expl)
      {$W^{\color{seq-red}{\scriptstyle\perp}}$ is orthogonal complement\\
       $A^{\color{seq-red}{\scriptstyle T}}$ is transpose};
  \draw[->, blue!50] (expl.west) to[bend left] (perpbox.south);
\end{tikzpicture}

\pause
\alert{Pictures}:\vskip-5mm
\begin{columns}[c,onlytextwidth]
  \column{.7\linewidth}
    The orthogonal complement of a \textcolor{seq-blue}{line} in $\R^2$ is
    \uncover<4->{the perpendicular \textcolor{seq-green}{line}.}
  \column{.3\linewidth}\centering
    \begin{tikzpicture}[scale=.5, thin border nodes]
      \draw (-2,-2) rectangle (2,2);
      \clip (-2,-2) rectangle (2,2);
      \draw[seq-blue] (-2,-3) -- node[right,pos=.6] {$W$} (2,3);
      \draw<4->[seq-green] (3,-2) -- node[below left,pos=.6] {$W^\perp$} (-3,2);
      \draw<4->[very thin]
        let \p1=($(0,0)!4mm!(-2,-3)$), \p2=($(0,0)!4mm!(3,-2)$) in
          (\p1) -- ($(\p1) + (\p2)$) -- (\p2);
    \end{tikzpicture}
\end{columns}
\begin{columns}[c,onlytextwidth]
  \column{.7\linewidth}
  \pause[5]%
    The orthogonal complement of a \textcolor{seq-blue}{line} in $\R^3$ is
    \uncover<6->{the perpendicular \textcolor{seq-green}{plane}.}
  \column{.3\linewidth}\centering
    \begin{tikzpicture}[scale=.5, myxyz, thin border nodes]
      \clip[resetxy] (-2,-2) rectangle (2,2);
      \coordinate (x) at ($1/sqrt(1.09)*(.3,0,1)$);
      \coordinate (y) at (0.957826, 0., -0.287348);
      \coordinate (z) at (0,1,0);
      \draw[seq-blue] (0,0) -- ($-3*(x)$);
      \begin{scope}[x=(y), y=(z), transformxy]
        \filldraw<6->[seq-green,fill opacity=.5]
          (-1.5,-1.5) rectangle (1.5,1.5);
        \node<6->[seq-green] at (-2,-1) {$W^\perp$};
      \end{scope}
      \draw[seq-blue] (0,0) -- node[right,pos=.5] {$W$} ($3*(x)$);
      \point<6-> (o) at (0,0,0);
      \pic<6->[draw] {right angle=(z)--(o)--(x)};
      \pic<6->[draw] {right angle=(y)--(o)--(x)};
    \end{tikzpicture}
\end{columns}
\begin{columns}[c,onlytextwidth]
  \column{.7\linewidth}
  \pause[7]%
    The orthogonal complement of a \textcolor{seq-blue}{plane} in $\R^3$ is
    \uncover<8->{the perpendicular \textcolor{seq-green}{line}.}
  \column{.3\linewidth}\centering
    \begin{tikzpicture}[scale=.5, myxyz, thin border nodes]
      \clip[resetxy] (-2,-2) rectangle (2,2);
      \coordinate (x) at ($1/sqrt(1.09)*(.3,0,1)$);
      \coordinate (y) at (0.957826, 0., -0.287348);
      \coordinate (z) at (0,1,0);
      \draw<8->[seq-green] (0,0) -- ($-3*(x)$);
      \begin{scope}[x=(y), y=(z), transformxy]
        \filldraw[seq-blue,fill opacity=.5]
          (-1.5,-1.5) rectangle (1.5,1.5);
        \node[seq-blue] at (-2,-1) {$W$};
      \end{scope}
      \draw<8->[seq-green] (0,0) -- node[right,pos=.5] {$W^\perp$} ($3*(x)$);
      \point<8-> (o) at (0,0,0);
      \pic<8->[draw] {right angle=(z)--(o)--(x)};
      \pic<8->[draw] {right angle=(y)--(o)--(x)};
    \end{tikzpicture}
\end{columns}

\end{frame}


%%%%%%%%%%%%%%%%%%%%%%%%%%%%%%%%%%%%%%%%%%%%%%%%%%%%%%%%%%%%%%%%%%%

\begin{pollframe}

\vskip 1cm

\begin{bluebox}[Poll]{.8\linewidth}
  Let $W$ be a plane in $\R^4$.  How would you describe $W^\perp$?
  \smallskip
  \begin{eAlpherate}
  \item The zero space $\{0\}$.
  \item A line in $\R^4$.
  \item A plane in $\R^4$.
  \item A $3$-dimensional space in $\R^4$.
  \item All of $\R^4$.
  \end{eAlpherate}
\end{bluebox}

\end{pollframe}


%%%%%%%%%%%%%%%%%%%%%%%%%%%%%%%%%%%%%%%%%%%%%%%%%%%%%%%%%%%%%%%%%%%

\begin{frame}
\frametitle{Orthogonal Complements}
\framesubtitle{Basic properties}

Let $W$ be a subspace of $\R^n$.

\medskip
\alert{Facts:}\pause
\begin{enumerate}
\item $W^\perp$ is also a subspace of $\R^n$
\pause
\item $(W^\perp)^\perp = W$
\pause
\item $\dim W + \dim W^\perp = n$
\pause
\item If $W = \Span\{v_1,v_2,\ldots,v_m\}$, then
  \[\begin{split}
    W^\perp &= \text{all vectors orthogonal to each $v_1,v_2,\ldots,v_m$}\\
    &= \bigl\{ \text{$x$ in $\R^n$}\mid
  x\cdot v_i = 0 \text{ for all } i=1,2,\ldots,m \bigr\}\\
    &= \Nul\mat{
    \matrow{v_1^T};
    \matrow{v_2^T};
    \spvdots;
    \matrow{v_m^T}}.
  \end{split}\]

\end{enumerate}

\medskip\small
\begin{webonly}
Let's check \alert{1.}
\begin{itemize}
\item Is $0$ in $W^\perp$?  
  Yes: $0\cdot w = 0$ for any $w$ in $W$.
\item Suppose $x,y$ are in $W^\perp$.
  So $x\cdot w = 0$ and $y\cdot w = 0$ for all $w$ in $W$.
  Then $(x+y)\cdot w = x\cdot w + y\cdot w = 0+0 = 0$ for all $w$ in $W$.
  So $x+y$ is also in $W^\perp$.
\item Suppose $x$ is in $W^\perp$.
  So $x\cdot w = 0$ for all $w$ in $W$.
  If $c$ is a scalar, then $(cx)\cdot w = c(x\cdot 0) = c(0) = 0$ for any $w$ in
  $W$.
  So $cx$ is in $W^\perp$.
\end{itemize}
\end{webonly}

\end{frame}


%%%%%%%%%%%%%%%%%%%%%%%%%%%%%%%%%%%%%%%%%%%%%%%%%%%%%%%%%%%%%%%%%%%

\begin{frame}
\frametitle{Orthogonal Complements}
\framesubtitle{Computation}

\alert{Problem:} if $W = \Span\left\{ \vec{1 1 -1},\;\vec{1 1 1} \right\}$,
compute $W^\perp$.

\medskip
\begin{webonly}
By property \alert{4}, we have to find the null space of the matrix whose rows
are $\mat{1 1 -1}$ and $\mat{1 1 1}$, which we did before:
\[ \Nul\mat{1 1 -1; 1 1 1} = \Span\left\{ \vec{-1 1 0} \right\}. \]
\end{webonly}

\pause\medskip
\begin{bluebox}{.8\linewidth}
  \[ \Span\{v_1,v_2,\ldots,v_m\}^\perp=\Nul\mat{
    \matrow{v_1^T};
    \matrow{v_2^T};
              \spvdots;
    \matrow{v_m^T}} \]
\end{bluebox}

\end{frame}


%%%%%%%%%%%%%%%%%%%%%%%%%%%%%%%%%%%%%%%%%%%%%%%%%%%%%%%%%%%%%%%%%%%

\begin{frame}
\frametitle{Orthogonal Complements}
\framesubtitle{Row space, column space, null space}

\vskip-3mm
\begin{defn}
  The \textbf{row space} of an $m\times n$ matrix $A$ is the span of the
  \emph{rows} of $A$.  It is denoted $\Row A$.
  \pause
  Equivalently, it is the column span of $A^T$:
  \[ \Row A = \Col A^T. \]
  \pause
  It is a subspace of $\R^{\blankuntil{4}{n}}$.
\end{defn}

\pause[5]\medskip
We showed before that if $A$ has rows $v_1^T,v_2^T,\ldots,v_m^T$, then 
\[ \Span\{v_1,v_2,\ldots,v_m\}^\perp = \pause\Nul A. \]
\pause
Hence we have shown:

\medskip
\alert{Fact:} $(\Row A)^\perp = \Nul A$.

\pause\medskip
Replacing $A$ by $A^T$, and remembering $\Row A^T = \Col A$:

\pause\medskip
\alert{Fact:} $(\Col A)^\perp = \Nul A^T$.

\pause\medskip
Using property \alert{2} and taking the orthogonal complements of both sides, 
we get:

\pause\medskip
\alert{Fact:} $(\Nul A)^\perp = \Row A$ and
$\Col A = (\Nul A^T)^\perp$.

\end{frame}


%%%%%%%%%%%%%%%%%%%%%%%%%%%%%%%%%%%%%%%%%%%%%%%%%%%%%%%%%%%%%%%%%%%

\begin{frame}
\frametitle{Orthogonal Complements}
\framesubtitle{Reference sheet}

\hbox to \linewidth{
\hss\alert{\large Orthogonal Complements of Most of the
  Subspaces We've Seen}\hss
}

\bigskip\centering
For any vectors $v_1,v_2,\ldots,v_m$:
\[\color{seq-orange} \Span\{v_1,v_2,\ldots,v_m\}^\perp = \Nul\mat{
  \matrow{v_1^T};
  \matrow{v_2^T};
  \spvdots;
  \matrow{v_m^T}} \]
\medskip
For any matrix $A$:
\[\color{seq-orange} \Row A = \Col A^T \]
\medskip
and
\[\color{seq-orange}
\begin{aligned}
  (\Row A)^\perp &= \Nul A    &  \Row A &= (\Nul A)^\perp \\
  (\Col A)^\perp &= \Nul A^T  &  \Col A &= (\Nul A^T)^\perp
\end{aligned}\]

\end{frame}



%%% Local Variables:
%%% TeX-master: "../slides"
%%% End:
