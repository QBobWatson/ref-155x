
% JDR: At (my) normal speed, this takes less than 50 minutes.

\titleframe{Section 1.7}{Linear Independence}


%%%%%%%%%%%%%%%%%%%%%%%%%%%%%%%%%%%%%%%%%%%%%%%%%%%%%%%%%%%%%%%%%%%

\begin{frame}
\frametitle{Motivation}

Sometimes the span of a set of vectors is ``smaller'' than you expect from the
number of vectors.

\pause\vskip .5cm
\hfill
\begin{tikzpicture}[scale=0.5, thin border nodes]
  \draw[grid lines, light gray] (-3,-3) grid (4, 4);
  \path[clip] (-3,-3) rectangle (4, 4);
  \node[seq4] at (.5,2.5) {$\Span\{v,w\}$};
  \draw[thin, seq4] ($-2*(2,2)$) -- ($2*(2,2)$);
  \draw[vector, seq1] (0,0)
    -- node[midway,above left,text=seq1] {$v$} (2,2);
  \draw[vector, seq2] (0,0)
    -- node[pos=.4,above left,text=seq2] {$w$} (-1,-1);
  \point at (0,0);
\end{tikzpicture}
\hfill\pause
\begin{tikzpicture}[myxyz, scale=0.5, thin border nodes]
  \path[clip, resetxy] (-4,-3) rectangle (4,3);

  \def\v{(2,-1,1)}
  \def\w{(2,1,.3)}
  \def\u{(2,0,.65)}

  \node[coordinate] (X) at \v {};
  \node[coordinate] (Y) at \w {};

  \begin{scope}[x=(X), y=(Y), transformxy]
    \path[clip] (-1.5, 5) -- (1.5, -5) -- (1.5, -7) -- (-1.5, -7) -- cycle;
    \fill[seq4!30, nearly opaque] (-1.5,-1) rectangle (1.5,2);
    \draw[step=.5cm, seq4, very thin] (-1.5,-1) grid (1.5,2);
  \end{scope}

  \begin{scope}[transformxy]
    \fill[white, nearly opaque] (-2, -2) rectangle (3, 3);
    \draw[grid lines] (-2, -2) grid (3, 3);
  \end{scope}

  \begin{scope}[x=(X), y=(Y), transformxy]
    \path[clip] (-1.5, 5) -- (1.5, -5) -- (1.5, 7) -- (-1.5, 7) -- cycle;
    \fill[seq4!30, nearly opaque] (-1.5,-1) rectangle (1.5,2);
    \draw[step=.5cm, seq4, very thin] (-1.5,-1) grid (1.5,2);
  \end{scope}

  \node[seq4] at (-1cm, 2cm) {$\Span\{u,v,w\}$};

  \draw[vector, seq1] (0,0,0) -- 
    node [midway, above] {$v$} \v;
  \draw[thin, densely dotted] \v -- \projxy\v;
  \draw[vector, seq2] (0,0,0) -- 
    node [midway, below] {\strut$w$} \w;
  \draw[thin, densely dotted] \w -- \projxy\w;
  \draw[vector, seq3] (0,0,0) -- 
    \u node [right] {$u$};
  \draw[thin, densely dotted] \u -- \projxy\u;

  \draw[thin, densely dotted] \v -- \w;

  \point at (0,0,0);

\end{tikzpicture}
\hfill\null

\pause\vskip .5cm
This can mean many things.  For example, it can mean you're using too many
vectors to write your solution set.

\pause\medskip
Notice in each case that one vector in the set is already in the span of the
others---so it doesn't make the span bigger.

\pause\medskip
Today we will formalize this idea in the concept of \emph{linear (in)dependence}.

\end{frame}


%%%%%%%%%%%%%%%%%%%%%%%%%%%%%%%%%%%%%%%%%%%%%%%%%%%%%%%%%%%%%%%%%%%

\begin{frame}
\frametitle{Linear Independence}

\vskip .14cm

\namedbox{redbox}{
\begin{minipage}{\textwidth}
\vskip -.14cm
\begin{defn}
  A set of vectors $\{v_1,v_2,\ldots,v_p\}$ in $\R^n$ is
  \textbf{linearly independent} if the vector equation
  \[ x_1v_1 + x_2v_2 + \cdots + x_pv_p = 0 \]
  has only the trivial solution $x_1=x_2=\cdots=x_p=0$.  
  \pause
  The set
  $\{v_1,v_2,\ldots,v_p\}$ is \textbf{linearly dependent} otherwise.
\end{defn}
\end{minipage}
}

\bigskip
\begin{uncoverenv}<3->
In other words, $\{v_1,v_2,\ldots,v_p\}$ is linearly dependent if
\pause[4]%
there exist
numbers $x_1,x_2,\ldots,x_p$, not all equal to zero, such that
\[ x_1v_1 + x_2v_2 + \cdots + x_pv_p = 0. \]
This is called a \textbf{linear dependence relation}.

\pause\smallskip
\begin{center}
\begin{tikzpicture}[remember picture]
  \node[bluebox, align=justify, text width=.9\textwidth]
    (bluebox)
    {
      Like span, linear (in)dependence is another one of those big vocabulary words
      that you absolutely need to learn.  Much of the rest of the course will be
      built on these concepts, and you need to know exactly what they mean in order
      to be able to answer questions on quizzes and exams (and solve real-world
      problems later on).
    };
  \node[overlay, redbox, fit=(redbox)] (x) {};
  \draw<5->[overlay, ->, very thick, black, rounded corners, shorten >=1pt] 
    let \p1=($(bluebox.west -| current page.west)!.3cm!(bluebox.west)$) in
    (bluebox.west)
      -- (\p1)  { -- (\p1 |- x.west) }
      -- (x.west);
\end{tikzpicture}
\end{center}
\end{uncoverenv}

\end{frame}


%%%%%%%%%%%%%%%%%%%%%%%%%%%%%%%%%%%%%%%%%%%%%%%%%%%%%%%%%%%%%%%%%%%

\begin{frame}
\frametitle{Linear Independence}

\begin{minipage}{\textwidth}
\begin{defn}
  A set of vectors $\{v_1,v_2,\ldots,v_p\}$ in $\R^n$ is
  \textbf{linearly independent} if the vector equation
  \[ x_1v_1 + x_2v_2 + \cdots + x_pv_p = 0 \]
  has only the trivial solution $x_1=x_2=\cdots=x_p=0$.  
  The set
  $\{v_1,v_2,\ldots,v_p\}$ is \textbf{linearly dependent} otherwise.
\end{defn}
\end{minipage}

\bigskip
\begin{bluebox}{.7\linewidth}
  Note that linear (in)dependence is a notion that applies to a
  \emph{collection of vectors}, not to a single vector, or to one vector in the
  presence of some others.
\end{bluebox}

\end{frame}


%%%%%%%%%%%%%%%%%%%%%%%%%%%%%%%%%%%%%%%%%%%%%%%%%%%%%%%%%%%%%%%%%%%

\begin{frame}
\frametitle{Checking Linear Independence}

\alert{Question: } 
Is $\left\{ \vec{1 1 1},\, \vec{1 -1 2},\, \vec{3 1 4} \right\}$ linearly
independent?

\medskip
\begin{webonly}
Equivalently, does the (homogeneous) the vector equation
\[ x\vec{1 1 1} + y\vec{1 -1 2} + z\vec{3 1 4} = \vec{0 0 0} \]
have a nontrivial solution?
How do we solve this kind of vector equation?
\[ \mat{1 1 3; 1 -1 1; 1 2 4} 
\quad\longsquiggly[row reduce]\quad
\mat{1 0 2; 0 1 1; 0 0 0}
\]
So $x = -2z$ and $y = -z$.  So the vectors are
linearly \textit{de}pendent, and an equation of linear dependence is
(taking $z=1$)
\[ -2\vec{1 1 1} - \vec{1 -1 2} + \vec{3 1 4} = \vec{0 0 0}. \]
\end{webonly}

\end{frame}


%%%%%%%%%%%%%%%%%%%%%%%%%%%%%%%%%%%%%%%%%%%%%%%%%%%%%%%%%%%%%%%%%%%

\begin{frame}
\frametitle{Checking Linear Independence}

\alert{Question: } 
Is $\left\{ \vec{1 1 0},\, \vec{1 -1 2},\, \vec{3 1 4} \right\}$ linearly
independent?

\medskip
\begin{webonly}
Equivalently, does the (homogeneous) the vector equation
\[ x\vec{1 1 0} + y\vec{1 -1 2} + z\vec{3 1 4} = \vec{0 0 0} \]
have a nontrivial solution?
\[ \mat{1 1 3; 1 -1 1; 0 2 4} 
\quad\longsquiggly[row reduce]\quad
\mat{1 0 0; 0 1 0; 0 0 1}
\]
The trivial solution $\vec{x y z} = \vec{0 0 0}$ is the unique solution. 
So the vectors are linearly \textit{in}dependent.
\end{webonly}

\end{frame}


%%%%%%%%%%%%%%%%%%%%%%%%%%%%%%%%%%%%%%%%%%%%%%%%%%%%%%%%%%%%%%%%%%%

\begin{frame}
\frametitle{Linear Independence and Matrix Columns}

In general, $\{v_1,v_2,\ldots,v_p\}$ is linearly independent 
\pause
if and only if the
vector equation
\[ x_1v_1 + x_2v_2 + \cdots + x_pv_p = 0 \]
has only the trivial solution, 
\pause
if and only if the matrix equation
\[ Ax = 0 \]
has only the trivial solution, where $A$ is the matrix
\pause
with columns $v_1,v_2,\ldots,v_p$:
\[ A = \mat{| |, , |; v_1 v_2 \cdots, v_p; | |, , |}. \]
\pause
This is true if and only if the matrix $A$ has a pivot
in each \blankuntil{6}{column}.

\pause[7]\smallskip

\begin{bluebox}[Important]{.9\textwidth}
  \begin{itemize}
  \item The vectors $v_1,v_2,\ldots,v_p$ are linearly independent if and only if
    the matrix with columns $v_1,v_2,\ldots,v_p$ has a pivot in each column.
    \pause
  \item Solving the matrix equation $Ax=0$ will either verify that the columns
    $v_1,v_2,\ldots,v_p$ of $A$ are linearly independent, or will produce
    a linear dependence relation.
  \end{itemize}
\end{bluebox}

\end{frame}


%%%%%%%%%%%%%%%%%%%%%%%%%%%%%%%%%%%%%%%%%%%%%%%%%%%%%%%%%%%%%%%%%%%

\begin{frame}
\frametitle{Linear Independence}
\framesubtitle{Criterion}

Suppose that one of the vectors $\{v_1,v_2,\ldots,v_p\}$ is a linear combination
of the other ones (that is, it is in the span of the other ones):
\pause
\[ v_3 = 2v_1 - \frac 12v_2 + 6v_4 \]
Then the vectors are linearly \textit{de}pendent:
\webonlycmd{
\[ 2v_1 - \frac 12v_2 - v_3 + 6v_4 = 0. \]}

\pause
Conversely, if the vectors are linearly dependent
\[ 2v_1 - \frac 12v_2 + 6v_4 = 0. \]
\pause
then one vector is a linear combination of (in the span of) the other ones:
\webonlycmd{\[ v_2 = 4v_1 + 12v_4. \]}
\pause
\vskip -.75cm

\begin{thm}
  A set of vectors $\{v_1,v_2,\ldots,v_p\}$ is linearly \textit{de}pendent if and only if
  one of the vectors is in the span of the other ones.
\end{thm}

% \pause
% (In fact, we can find $j$ such that $v_j$ is in
% $\Span\{v_1,v_2,\ldots,v_{j-1}\}$: later.)

% \note{Ask: how to do this with the above example?}

\end{frame}


%%%%%%%%%%%%%%%%%%%%%%%%%%%%%%%%%%%%%%%%%%%%%%%%%%%%%%%%%%%%%%%%%%%

\begin{frame}<handout:2,4,7>
\frametitle{Linear Independence}
\framesubtitle{Pictures in $\R^2$}

\vfill

\begin{tikzpicture}[scale=1, baseline, thin border nodes]
  \path[clip] (-3,-3) rectangle (3,3);
  \fill<all:5->[seq4!10] (-3,-3) rectangle (3,3);
  \draw[seq4!30] (-4,-2) -- (4,2);
  \draw<all:3->[seq4!30] (-3,3) -- (3,-3);
  \path (-2,-1) node[below right, seq4] {$\Span\{v\}$};
  \path<all:3-> (-2,2) node[above right, seq4] {$\Span\{w\}$};
  \node<all:5->[seq4] at (1,2) {$\Span\{v,w\}$};
  \draw[vector, seq1] (0,0) --
    node[midway,above left]{$v$} (2,1);
  \draw<all:3->[vector, seq2] (0,0) --
    node[midway,below left]{$w$} (-1,1);
  \draw<all:5->[vector, seq3] (0,0) --
    node[midway,right]{$u$} (0,-1.5);
  \point at (0,0);
  \draw[thin] (-3,-3) rectangle (3,3);
\end{tikzpicture}
\hfill
\begin{minipage}[c]{.4\textwidth}\raggedright
  \alert{In this picture}\\
  \hrulefill\medskip

  \alert{One vector $\{\textcolor{seq1}v\}$:}\\
  \uncover<all:2->
  {Linearly independent if $v\neq 0$.}
  \note{What if $v=0$?}

  \medskip
  \uncover<all:3->{
  \alert{Two vectors
    $\{\textcolor{seq1}v,\textcolor{seq2}w\}$:}}\\
  \uncover<all:4->{
  Linearly independent: neither is in the span of the other.}

  \medskip
  \uncover<all:5->{
  \alert{Three vectors
    $\{\textcolor{seq1}v,\textcolor{seq2}w, \textcolor{seq3}u\}$:}}\\
  \uncover<all:6->{
  Linearly dependent: $u$ is in $\Span\{v,w\}$.}

  \medskip
  \uncover<all:7->{
  Also $v$ is in $\Span\{u,w\}$ and $w$ is in $\Span\{u,v\}$.}
\end{minipage}

\vfill

\end{frame}


%%%%%%%%%%%%%%%%%%%%%%%%%%%%%%%%%%%%%%%%%%%%%%%%%%%%%%%%%%%%%%%%%%%

\begin{frame}<handout:2,7>
\frametitle{Linear Independence}
\framesubtitle{Pictures in $\R^2$}

\vfill

\begin{tikzpicture}[scale=1, thin border nodes, baseline]
  \path[clip] (-3,-3) rectangle (3,3);
  \draw[seq4!30] (-4,-2) -- (4,2);
  \path (-2,-1) node[below right, seq4] {$\Span\{v\}$};
  \draw[vector, seq1] (0,0) --
    node[midway,above left]{$v$} (2,1);
  \draw[vector, seq2] (0,0) --
    node[pos=.4,above left]{$w$} (-1,-.5);
  \draw<all:5->[vector, seq3] (0,0) --
    node[midway,right]{$u$} (0,-1.5);
  \point at (0,0);
  \draw[thin] (-3,-3) rectangle (3,3);
\end{tikzpicture}
\hfill
\begin{minipage}[c]{.4\textwidth}\raggedright
  \uncover<all:1->{
  \alert{Two collinear vectors
    $\{\textcolor{seq1}v,\textcolor{seq2}w\}$:}}\\
  \uncover<all:2->{
  Linearly dependent: $w$ is in $\Span\{v\}$ (and vice-versa).}

  \medskip
  \uncover<all:3->{
  \alert{Observe:} \emph{Two} vectors are linearly \textit{de}pendent if and only if}
  \uncover<all:4->{they are \emph{collinear}.}

  \medskip
  \uncover<all:5->{
  \alert{Three vectors
    $\{\textcolor{seq1}v,\textcolor{seq2}w, \textcolor{seq3}u\}$:}}\\
  \uncover<all:6->{
  Linearly dependent: $w$ is in $\Span\{v\}$ (and vice-versa).}

  \medskip
  \uncover<all:7->{
  \alert{Observe:} If a set of vectors is linearly dependent, then so is any
  larger set of vectors!
  }
  
\end{minipage}

\vfill

\end{frame}


%%%%%%%%%%%%%%%%%%%%%%%%%%%%%%%%%%%%%%%%%%%%%%%%%%%%%%%%%%%%%%%%%%%

\begin{frame}<handout:2,4,6>
\frametitle{Linear Independence}
\framesubtitle{Pictures in $\R^3$}

\vfill

\begin{tikzpicture}[myxyz, scale=.75, baseline, thin border nodes]
  \path[clip, resetxy] (-4,-4) rectangle (4,4);

  \def\v{(2,-1,1)}
  \def\w{(1,0,-1)}
  \def\u{(.5,-.5,1)}
  \def\uu{(0,2,2)}

  \node[coordinate] (X) at \v {};
  \node[coordinate] (Y) at \w {};

  \begin{uncoverenv}<all:3->
  \begin{scope}[x=(X), y=(Y), transformxy]
    \path[clip] (-5, -5) -- (5, 5) -- (5, 10) -- (-5, 10) -- cycle;
    \fill[seq4!10, nearly opaque] (-1.5,-1) rectangle (1.5,2);
    \draw[step=.5cm, seq4!30, very thin] (-1.5,-1) grid (1.5,2);
  \end{scope}
  \end{uncoverenv}

  \draw[seq4!30] ($-3*(2,-1,1)$) -- (0,0,0);
  \draw[seq4!30] ($5*(1,0,-1)$) -- (0,0,0);

  \begin{scope}[transformxy]
    \fill[white, semitransparent] (-2, -2) rectangle (3, 3);
    \draw[help lines, light gray] (-2, -2) grid (3, 3);
  \end{scope}

  \draw[seq4!30] ($3*(2,-1,1)$) -- (0,0,0);
  \draw[seq4!30] ($-5*(1,0,-1)$) -- (0,0,0);

  \begin{uncoverenv}<all:3->
  \begin{scope}[x=(X), y=(Y), transformxy]
    \path[clip] (-5, -5) -- (5, 5) -- (5, -10) -- (-5, -10) -- cycle;
    \fill[seq4!10, nearly opaque] (-1.5,-1) rectangle (1.5,2);
    \draw[step=.5cm, seq4!30, very thin] (-1.5,-1) grid (1.5,2);
  \end{scope}
  \end{uncoverenv}

  \draw[vector, seq1] (0,0,0) -- 
    node [midway, below right] {$v$} \v;
  \draw[thin, densely dotted] \v -- \projxy\v;

  \draw[vector, seq2!50!white] (0,0,0) -- 
    node [midway, below left] {$w$} \w;
  \draw[thin, densely dotted, black!50!white] \w -- \projxy\w;

  \draw<all:3-4>[vector, seq3] (0,0,0) -- 
    node [midway, below left] {$u$} \uu;
  \draw<all:3-4>[thin, densely dotted] \uu -- \projxy\uu;

  \draw<all:5->[vector, seq5] (0,0,0) -- 
    node [midway, above left] {$x$} \u;
  \draw<all:5->[thin, densely dotted] \u -- \projxy\u;

  \path[seq4] ($.9*(-3,1.5,-1.5)$) node[below right] {$\Span\{v\}$};
  \path[seq4] (-2,0,2) node[below left] {$\Span\{w\}$};
  \node<all:3->[seq4] at (.8cm, 2.8cm) {$\Span\{v,w\}$};

  \point at (0,0);
  \draw[thin,resetxy] (-4,-4) rectangle (4,4);

\end{tikzpicture}
\hfill
\begin{minipage}[c]{.4\textwidth}\raggedright
  \alert{In this picture}\\
  \hrulefill\medskip

  \uncover<all:1->{
  \alert{Two vectors
    $\{\textcolor{seq1}v,\textcolor{seq2}w\}$:}}\\
  \uncover<all:2->{
    Linearly independent: neither is in the span of the other.}

  \begin{overlayarea}{\textwidth}{3cm}
    \medskip
  \only<all:3-4>{
  \alert{Three vectors
    $\{\textcolor{seq1}v,\textcolor{seq2}w, \textcolor{seq3}u\}$:\\}}
  \only<all:4>{
  Linearly independent: no one is in the span of the other two.}

  \only<all:5->{
  \alert{Three vectors
    $\{\textcolor{seq1}v,\textcolor{seq2}w, \textcolor{seq5}{x}\}$:\\}}
  \only<all:6->{
    Linearly dependent: $x$ is in $\Span\{v,w\}$.
  }
  \end{overlayarea}
  
\end{minipage}

\vfill

\end{frame}


%%%%%%%%%%%%%%%%%%%%%%%%%%%%%%%%%%%%%%%%%%%%%%%%%%%%%%%%%%%%%%%%%%%

\begin{pollframe}

\vskip 0pt plus .5fill

\begin{bluebox}[Poll]{.8\textwidth}
  Are there four vectors $u,v,w,x$ in $\R^3$ which are linearly dependent, but
  such that $u$ is \emph{not} a linear combination of $v,w,x$?  If so, draw a
  picture; if not, give an argument.
\end{bluebox}

\vfill\pause
\alert{Yes:} actually the pictures on the previous slides provide such an example.

\vfill\pause
Linear dependence of $\{v_1,\ldots,v_p\}$ means \emph{some} $v_i$ is a linear
combination of the others, not \emph{any}.

\vfill

\end{pollframe}


%%%%%%%%%%%%%%%%%%%%%%%%%%%%%%%%%%%%%%%%%%%%%%%%%%%%%%%%%%%%%%%%%%%

\begin{frame}
\frametitle{Linear Independence}
\framesubtitle{Stronger criterion}

\vskip-3mm
\begin{thm}
  A set of vectors $\{v_1,v_2,\ldots,v_p\}$ is linearly \textit{de}pendent if and only if
  one of the vectors is in the span of the other ones.
\end{thm}

\pause\medskip
Take the largest $j$ such that $v_j$ is in the span of the others.
\pause
Then $v_j$ is in the span of $v_1,v_2,\ldots,v_{j-1}$.  Why?
\pause
If not ($j=3$):
\[ v_3 = 2v_1 - \frac 12v_2 + 6v_4 \]
\pause
Rearrange:
\webonlycmd{
\[ v_4 = -\frac 16\biggl( 2v_1 - \frac 12v_2 - v_3 \biggr) \]}%
\pause
so $v_4$ works as well, but $v_3$ was supposed to be the last one that was in
the span of the others.

\pause\medskip

\begin{oneoffthm}[definition]{Better Theorem}
  A set of vectors $\{v_1,v_2,\ldots,v_p\}$ is linearly dependent if and only if
  there is some $j$ such that $v_j$ is in $\Span\{v_1,v_2,\ldots,v_{j-1}\}$.
\end{oneoffthm}


\end{frame}


%%%%%%%%%%%%%%%%%%%%%%%%%%%%%%%%%%%%%%%%%%%%%%%%%%%%%%%%%%%%%%%%%%%

\begin{frame}
\frametitle{Linear Independence}
\framesubtitle{Increasing span criterion}

\vskip-3mm
\begin{thm}
  A set of vectors $\{v_1,v_2,\ldots,v_p\}$ is linearly dependent if and only if
  there is some $j$ such that $v_j$ is in $\Span\{v_1,v_2,\ldots,v_{j-1}\}$.
\end{thm}

\pause\bigskip
Equivalently, $\{v_1,v_2,\ldots,v_p\}$ is linearly \textit{in}dependent if for every $j$,
the vector $v_j$ is not in $\Span\{v_1,v_2,\ldots,v_{j-1}\}$.

\pause\medskip
This means $\Span\{v_1,v_2,\ldots,v_j\}$ is \emph{bigger} than
$\Span\{v_1,v_2,\ldots,v_{j-1}\}$.

\pause\bigskip
\begin{thm}
  A set of vectors $\{v_1,v_2,\ldots,v_p\}$ is linearly independent if
  and only if, for every $j$, the span of $v_1,v_2,\ldots,v_{j}$ is strictly
  larger than the span of $v_1,v_2,\ldots,v_{j-1}$.
\end{thm}

\pause\medskip
\begin{bluebox}[Translation]{.7\linewidth}
  A set of vectors is linearly independent if and only if, every time you add
  another vector to the set, the span gets bigger.
\end{bluebox}

\end{frame}


%%%%%%%%%%%%%%%%%%%%%%%%%%%%%%%%%%%%%%%%%%%%%%%%%%%%%%%%%%%%%%%%%%%

\begin{frame}<handout:1->
\frametitle{Linear Independence}
\framesubtitle{Increasing span criterion: pictures}

\vskip-3mm
\begin{thm}
  A set of vectors $\{v_1,v_2,\ldots,v_p\}$ is linearly independent if
  and only if, for every $j$, the span of $v_1,v_2,\ldots,v_{j}$ is strictly
  larger than the span of $v_1,v_2,\ldots,v_{j-1}$.
\end{thm}

\vfill

\quad
\begin{tikzpicture}[myxyz, scale=.65, baseline, thin border nodes]
  \path[clip, resetxy] (-4,-4) rectangle (4,4);

  \def\v{(2,-1,1)}
  \def\w{(1,0,-1)}
  \def\u{(0,2,2)}
  \def\x{(.5,-.5,1)}

  \node[coordinate] (X) at \v {};
  \node[coordinate] (Y) at \w {};

  \fill<all:3>[seq4!10, resetxy] (-4,-4) rectangle (4,4);

  \begin{uncoverenv}<all:2->
  \begin{scope}[x=(X), y=(Y), transformxy]
    \path[clip] (-5, -5) -- (5, 5) -- (5, 10) -- (-5, 10) -- cycle;
    \fill[seq4!10, nearly opaque] (-1.5,-1) rectangle (1.5,2);
    \draw[step=.5cm, seq4!30, very thin] (-1.5,-1) grid (1.5,2);
  \end{scope}
  \end{uncoverenv}

  \draw[seq4!30] ($-3*(2,-1,1)$) -- (0,0,0);

  \begin{scope}[transformxy]
    \fill[white, semitransparent] (-2, -2) rectangle (3, 3);
    \draw[help lines, light gray] (-2, -2) grid (3, 3);
  \end{scope}

  \draw[seq4!30] ($3*(2,-1,1)$) -- (0,0,0);

  \begin{uncoverenv}<all:2->
  \begin{scope}[x=(X), y=(Y), transformxy]
    \path[clip] (-5, -5) -- (5, 5) -- (5, -10) -- (-5, -10) -- cycle;
    \fill[seq4!10, nearly opaque] (-1.5,-1) rectangle (1.5,2);
    \draw[step=.5cm, seq4!30, very thin] (-1.5,-1) grid (1.5,2);
  \end{scope}
  \end{uncoverenv}

  \draw[vector, seq1] (0,0,0) -- 
    node [midway, below right] {$v$} \v;
  \draw[thin, densely dotted] \v -- \projxy\v;

  \draw<all:2->[vector, seq2!50!white] (0,0,0) -- 
    node [midway, below left] {$w$} \w;
  \draw<all:2->[thin, densely dotted, black!50!white] \w -- \projxy\w;

  \draw<all:3>[vector, seq3] (0,0,0) -- 
    node [midway, below left] {$u$} \u;
  \draw<all:3>[thin, densely dotted] \u -- \projxy\u;

  \draw<all:4>[vector, seq5] (0,0,0) -- 
    node [midway, above left] {$x$} \x;
  \draw<all:4>[thin, densely dotted] \x -- \projxy\x;

  \path[seq4] ($.9*(-3,1.5,-1.5)$) node[below right] {$\Span\{v\}$};
  \node<all:2-3>[seq4] at (.8cm, 2.8cm) {$\Span\{v,w\}$};
  \node<all:3>[seq4] at (-2cm,2cm) {$\Span\{v,w,u\}$};
  \node<all:4>[seq4] at (.5cm, 2.8cm) {$\Span\{v,w,x\}$};

  \point at (0,0,0);
  \draw[thin,resetxy] (-4,-4) rectangle (4,4);

\end{tikzpicture}
\hfill
\begin{minipage}[c]{.4\textwidth}
  \alert{One vector $\{\textcolor{seq1}v\}$:}\\
  Linearly independent: span got bigger (than $\{0\}$).

  \medskip
  \begin{overlayarea}{\textwidth}{3cm}
  \uncover<all:2->{
    \alert{Two vectors $\{\textcolor{seq1}v, \textcolor{seq2}w\}$:}\\
    Linearly independent: span got bigger.}

  \medskip
  \only<all:3>{
    \alert{Three vectors $\{\textcolor{seq1}v, \textcolor{seq2}w,
      \textcolor{seq3}u\}$:}\\
    Linearly independent: span got bigger.}

  \only<all:4>{
    \alert{Three vectors $\{\textcolor{seq1}v, \textcolor{seq2}w,
      \textcolor{seq5}x\}$:}\\
    Linearly dependent: span didn't get bigger.}
  \end{overlayarea}

\end{minipage}
\quad\null

\vfill

\end{frame}


%%%%%%%%%%%%%%%%%%%%%%%%%%%%%%%%%%%%%%%%%%%%%%%%%%%%%%%%%%%%%%%%%%%

\begin{frame}
\frametitle{Linear Independence}
\framesubtitle{Two more facts}

\displayskips{3pt}
\alert{Fact 1:} Say $v_1,v_2,\ldots,v_n$ are in $\R^m$.  If $n > m$ then
$\{v_1,v_2,\ldots,v_n\}$ is linearly
\pause
\textit{de}pendent: the matrix
\[ A = \mat{| |, , |; v_1 v_2 \cdots, v_n; | |, , |}. \]
cannot have a pivot in each column
\pause
(it is too wide).

\pause\medskip
This says you can't have $4$ linearly independent vectors in $\R^3$, for
instance.

\pause
\begin{bluebox}{.8\linewidth}
  A wide matrix can't have linearly independent columns.
\end{bluebox}

\pause\medskip
\alert{Fact 2:} If one of $v_1,v_2,\ldots,v_n$ is zero, then 
$\{v_1,v_2,\ldots,v_n\}$ is linearly
\pause
\textit{de}pendent.  For instance, if $v_1 = 0$, then
\pause
\[ 1\cdot v_1 + 0\cdot v_2 + 0\cdot v_3 + \cdots + 0\cdot v_n = 0 \]
is a linear dependence relation.

\pause
\begin{bluebox}{.8\linewidth}
  A set containing the zero vector is linearly dependent.
\end{bluebox}

\end{frame}


%%% Local Variables:
%%% TeX-master: "../slides"
%%% End:
