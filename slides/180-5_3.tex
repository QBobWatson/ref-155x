
\lecture{}{11-28}
\titleframe{Section 5.3}{Diagonalization}


%%%%%%%%%%%%%%%%%%%%%%%%%%%%%%%%%%%%%%%%%%%%%%%%%%%%%%%%%%%%%%%%%%%

\begin{frame}
\frametitle{Motivation}
\framesubtitle{Difference equations}

Many real-word linear algebra problems have the form:
\[ v_1 = Av_0, \quad v_2 = Av_1 = A^2 v_0, \quad
v_3 = Av_2 = A^3v_0, \quad \ldots \quad v_n = A v_{n-1} = A^nv_0. \]
This is called a \textbf{difference equation}.

\pause\bigskip
Our toy example about rabbit populations had this form.

\pause\bigskip
The question is, what happens to $v_n$ as $n\to\infty$?

\pause\bigskip
\begin{itemize}
\item Taking powers of diagonal matrices is easy!
  \pause
\item Taking powers of \emph{diagonalizable} matrices is still easy!
  \pause
\item Diagonalizing a matrix is an eigenvalue problem.
\end{itemize}

\end{frame}


%%%%%%%%%%%%%%%%%%%%%%%%%%%%%%%%%%%%%%%%%%%%%%%%%%%%%%%%%%%%%%%%%%%

\begin{frame}
\frametitle{Powers of Diagonal Matrices}

If $D$ is diagonal, then $D^n$ is also diagonal; its diagonal entries are the
$n$th powers of the diagonal entries of $D$:

\begin{webonly}
\bigskip
\[ \hss D = \mat{2 0 ; 0 3}, \quad
D^2 = \mat{4 0; 0 9}, \quad
D^3 = \mat{8 0; 0 27}, \quad \ldots \quad
D^n = \mat{2^n 0; 0 3^n}. \hss \]
\bigskip
\[\begin{split}
  D &= \mat{-1 0 0; 0 \frac 12 0; 0 0 \frac 13}, \quad
  D^2 = \mat{1 0 0; 0 \frac 14 0; 0 0 \frac 19}, \quad
  D^3 = \mat{-1 0 0; 0 \frac 18 0; 0 0 \frac 1{27}}, \\
  \quad &\ldots\quad D^n = \mat{(-1)^n 0 0; 0 \frac 1{2^n} 0; 0 0 \frac 1{3^n}}
\end{split}\]
\end{webonly}

\end{frame}


%%%%%%%%%%%%%%%%%%%%%%%%%%%%%%%%%%%%%%%%%%%%%%%%%%%%%%%%%%%%%%%%%%%

\begin{frame}
\frametitle{Powers of Matrices that are Similar to Diagonal Ones}

What if $A$ is not diagonal?

\pause
\begin{eg}
  Let $A =\mat{1 2; -1 4}$.  Compute $A^n$.
\end{eg}

\pause\smallskip
In \S5.2 lecture we saw that $A$ is similar to a diagonal matrix:
\[ A = PDP\inv \sptxt{where} P = \mat{2 1; 1 1} \sptxt{and} D = \mat{2 0; 0 3}. \]
\pause
Then
\[\begin{split}
  A^2 &= \webonlycmd{(PDP\inv)(PDP\inv) = PD(P\inv P)DP\inv = PDIDP\inv = PD^2P\inv}\\
  A^3 &= \webonlycmd{(PDP\inv)(PD^2P\inv) = PD(P\inv P)D^2P\inv
    = PDID^2P\inv = PD^3P\inv}\\
  &\;\;\vdots \\
  A^n &= \webonlycmd{PD^nP\inv}
\end{split}\]
\pause
Therefore
\[ A^n = \webonlycmd{\mat{2 1; 1 1}\mat{2^n 0; 0 3^n}\mat{1 -1; -1 2}
= \namedbox{An}{\mat{2^{n+1}-3^n -2^{n+1}+2\cdot 3^n; 2^n-3^n -2^n+2\cdot 3^n}}.}  \]
\begin{webonly}
\begin{tikzpicture}[remember picture, overlay]
  \path (An.north) ++(-8mm,5mm)
    node[blue!50, anchor=south east, align=center] (expl) 
      {Closed formula in terms of $n$:\\easy to compute};
  \draw[->, shorten >=1pt, blue!50] (expl) to[out=0,in=90] (An);
\end{tikzpicture}
\end{webonly}

\end{frame}


%%%%%%%%%%%%%%%%%%%%%%%%%%%%%%%%%%%%%%%%%%%%%%%%%%%%%%%%%%%%%%%%%%%

\begin{frame}
\frametitle{Diagonalizable Matrices}

\vskip-3mm
\begin{defn}
  An $n\times n$ matrix $A$ is \textbf{diagonalizable} if it is similar to a
  diagonal matrix:
  \[ A = PDP\inv \sptxt{for $D$ diagonal.} \]
\end{defn}

\pause
\begin{bluebox}[Important]{.7\linewidth}
  If $A = PDP\inv$ for
  $D = \mat{d_{11} 0 \cdots, 0;
    0 d_{22} \cdots, 0;
    \vdots, \vdots, \ddots, \vdots;
    0 0 \cdots, d_{nn}}$
  then
  \[ A^k = PD^KP\inv = P\mat{d_{11}^k 0 \cdots, 0;
    0 d_{22}^k \cdots, 0;
    \vdots, \vdots, \ddots, \vdots;
    0 0 \cdots, d_{nn}^k}P\inv. \]
\end{bluebox}

\pause
So diagonalizable matrices are easy to raise to any power.

\end{frame}


%%%%%%%%%%%%%%%%%%%%%%%%%%%%%%%%%%%%%%%%%%%%%%%%%%%%%%%%%%%%%%%%%%%

\begin{frame}
\frametitle{Diagonalization}

\vskip1mm
\namedbox{bigthm}{\begin{minipage}{\linewidth}\vskip-2mm
\begin{oneoffthm}{The Diagonalization Theorem}
  An $n\times n$ matrix $A$ is diagonalizable if and only if $A$ has $n$
  linearly independent eigenvectors.

  \smallskip
  \begin{uncoverenv}<2->
  In this case, $A = PDP\inv$ for
  \[ P = \mat{| |,, |; v_1 v_2 \cdots, v_n; | |,, |}
  \qquad
  D = \mat{\lambda_1 0 \cdots, 0; 0 \lambda_2 \cdots, 0;
    \vdots, \vdots, \ddots, \vdots; 0 0 \cdots, \lambda_n},
  \]
  where $v_1,v_2,\ldots,v_n$ are linearly independent eigenvectors,
  and $\lambda_1,\lambda_2,\ldots,\lambda_n$ are the corresponding eigenvalues
  (in the same order).
  \end{uncoverenv}
\end{oneoffthm}\end{minipage}}
\pause[3]%
\tikz[remember picture, overlay] \node[redbox, fit=(bigthm)] {}; 

\pause\medskip
\begin{oneoffthm}{\namedbox{cor}{Corollary}}
  An $n\times n$ matrix with $n$ distinct eigenvalues is diagonalizable.
\begin{tikzpicture}[remember picture, overlay]
  \path (cor.east) ++(1cm,1mm)
    node[right, blue!50, font=\small] (expl)
      {a theorem that follows easily from another theorem};
  \draw[->, blue!50, shorten >=1mm] (expl.west) to[out=180,in=0] (cor.east);
\end{tikzpicture}
\end{oneoffthm}

\pause\medskip
The Corollary is true because eigenvectors with distinct eigenvalues are always
linearly independent.
\pause
We will see later that a diagonalizable matrix need not have $n$ distinct
eigenvalues though.

\end{frame}


%%%%%%%%%%%%%%%%%%%%%%%%%%%%%%%%%%%%%%%%%%%%%%%%%%%%%%%%%%%%%%%%%%%

\begin{frame}
\frametitle{Diagonalization}
\framesubtitle{Example}

\alert{Problem:} Diagonalize $A = \mat{1 2; -1 4}$.

\medskip
\begin{webonly}%
The characteristic polynomial is
\[ f(\lambda) = \lambda^2 - \Tr(A)\,\lambda + \det(A)
= \lambda^2 - 5\lambda + 6 = (\lambda-2)(\lambda-3). \]
Therefore the eigenvalues are
$2$ and $3$.  Let's compute some eigenvectors:
\[ (A-2I)x = 0 \iff
  \mat{-1 2; -1 2}x = 0
  \;\longsquiggly[rref]\;
  \mat{1 -2; 0 0}x = 0
  \]
The parametric form is $x = 2y$, so $v_1 = {2\choose 1}$ is an eigenvector with
eigenvalue $2$.
\[ (A-3I)x = 0 \iff
  \mat{-2 2; -1 1}x = 0
  \;\longsquiggly[rref]\;
  \mat{1 -1; 0 0}x = 0
\]
The parametric form is $x = y$, so $v_2 = {1\choose 1}$ is an eigenvector with
eigenvalue $3$.

\medskip
The eigenvectors $v_1,v_2$ are linearly independent, so the Diagonalization
Theorem says
\[ A = PDP\inv \sptxt{for} P = \mat{2 1; 1 1} \qquad D = \mat{2 0; 0 3}. \]
\end{webonly}

\end{frame}


%%%%%%%%%%%%%%%%%%%%%%%%%%%%%%%%%%%%%%%%%%%%%%%%%%%%%%%%%%%%%%%%%%%

\begin{frame}
\frametitle{Diagonalization}
\framesubtitle{Another example}

\alert{Problem:} Diagonalize $A = \mat{4 -3 0; 2 -1 0; 1 -1 1}$.

\medskip
\begin{webonly}%
\displayskips{3pt}
The characteristic polynomial is
\[ f(\lambda) = \det(A-\lambda I) = -\lambda^3+4\lambda^2-5\lambda+2
= -(\lambda-1)^2(\lambda-2). \]
Therefore the eigenvalues are $1$ and $2$, with respective multiplicities $2$
and $1$.  Let's compute the $1$-eigenspace:
\[ (A-I)x = 0 \iff
\mat{3 -3 0; 2 -2 0; 1 -1 0}x = 0
\;\longsquiggly[rref]\;
\mat{1 -1 0; 0 0 0; 0 0 0}x = 0
\]
The parametric vector form is
\[ \syseq{x = y; y = y; z = \. \+ z}
\implies \vec{x y z} = y\vec{1 1 0} + z\vec{0 0 1}  \]
Hence a basis for the $1$-eigenspace is
\[ \cB_1 = \bigl\{ v_1,v_2 \bigr\} \sptxt{where}
v_1 = \vec{1 1 0}, \quad v_2 = \vec{0 0 1}. \]

\end{webonly}

\end{frame}


%%%%%%%%%%%%%%%%%%%%%%%%%%%%%%%%%%%%%%%%%%%%%%%%%%%%%%%%%%%%%%%%%%%

\begin{frame}
\frametitle{Diagonalization}
\framesubtitle{Another example, continued}

\alert{Problem:} Diagonalize $A = \mat{4 -3 0; 2 -1 0; 1 -1 1}$.

\begin{webonly}%
\displayskips{3pt}
Now let's compute the $2$-eigenspace:
\[ (A-2I)x = 0 \iff
\mat{2 -3 0; 2 -3 0; 1 -1 -1}x = 0
\;\longsquiggly[rref]\;
\mat{1 0 -3; 0 1 -2; 0 0 0}x = 0
\]
The parametric form is $x = 3z, y = 2z$, so an eigenvector with eigenvalue $2$ is
\[ v_3 = \vec{3 2 1}. \]
The eigenvectors $v_1,v_2,v_3$ are linearly independent: $v_1,v_2$ form a basis
for the $1$-eigenspace, and $v_3$ is not contained in the $1$-eigenspace.
Therefore the Diagonalization Theorem says
\[ A = PDP\inv \sptxt{for}  
P = \mat{1 0 3; 1 0 2; 0 1 1} \qquad D = \mat{1 0 0; 0 1 0; 0 0 2}. \]
\end{webonly}%
\pause
\alert{Note:} In this case, there are three linearly independent eigenvectors, but
only two distinct eigenvalues.
\end{frame}


%%%%%%%%%%%%%%%%%%%%%%%%%%%%%%%%%%%%%%%%%%%%%%%%%%%%%%%%%%%%%%%%%%%

\begin{frame}
\frametitle{Diagonalization}
\framesubtitle{A non-diagonalizable matrix}

\alert{Problem:} Show that $A = \mat{1 1; 0 1}$ is not diagonalizable.

\medskip
\begin{webonly}%
This is an upper-triangular matrix, so the only eigenvalue is $1$.
Let's compute the $1$-eigenspace:
\[ (A-I)x = 0 \iff \mat{0 1; 0 0}x = 0. \]
This is row reduced, but has only one free variable $x$; a basis for the
$1$-eigenspace is $\{{1\choose 0}\}$.
So \emph{all eigenvectors} of $A$ are multiples of ${1\choose 0}$.
\end{webonly}

\pause\bigskip
\alert{Conclusion:}
$A$ has only one linearly independent eigenvector, so by the ``only if'' part of
the diagonalization theorem, $A$ is not diagonalizable.

\end{frame}


%%%%%%%%%%%%%%%%%%%%%%%%%%%%%%%%%%%%%%%%%%%%%%%%%%%%%%%%%%%%%%%%%%%

\begin{pollframe}

\begin{bluebox}[Poll]{.8\linewidth}
  Which of the following matrices are diagonalizable, and why?
  \[
  \alert{\text{A. }} \mat{1 2; 0 1} \quad
  \alert{\text{B. }} \mat{1 2; 0 2} \quad
  \alert{\text{C. }} \mat{2 1; 0 2} \quad
  \alert{\text{D. }} \mat{2 0; 0 2}
  \]
\end{bluebox}

\pause\bigskip
Matrix \alert{A} is not diagonalizable: its only eigenvalue is $1$, and its
$1$-eigenspace is spanned by ${1\choose 0}$.

\pause\bigskip
Similarly, matrix~\alert{C} is not diagonalizable.

\pause\bigskip
Matrix~\alert{B} is diagonalizable because it is a $2\times 2$ matrix with
distinct eigenvalues.

\pause\bigskip
Matrix~\alert{D} is already diagonal!

\end{pollframe}


%%%%%%%%%%%%%%%%%%%%%%%%%%%%%%%%%%%%%%%%%%%%%%%%%%%%%%%%%%%%%%%%%%%

\begin{frame}
\frametitle{Diagonalization}
\framesubtitle{Procedure}

\alert{\large How to diagonalize a matrix $A$:}
\begin{enumerate}
  \pause
\item Find the eigenvalues of $A$ using the characteristic polynomial. 
  \pause
\item For each eigenvalue $\lambda$ of $A$, compute a basis $\cB_\lambda$ for the
  $\lambda$-eigenspace.
  \pause
\item If there are fewer than $n$ total vectors in the union of all of the
  eigenspace bases $\cB_\lambda$, then the matrix is not diagonalizable.
  \pause
\item Otherwise, the $n$ vectors $v_1,v_2,\ldots,v_n$ in your eigenspace bases are
  linearly independent, and $A = PDP\inv$ for
  \[ P = \mat{| |,, |; v_1 v_2 \cdots, v_n; | | ,, |} \sptxt{and}
  D = \mat{\lambda_1 0 \cdots, 0;
    0 \lambda_2 \cdots, 0;
    \vdots, \vdots, \ddots, \vdots;
    0 0 \cdots, \lambda_n}, \]
  where $\lambda_i$ is the eigenvalue for $v_i$.

\end{enumerate}

\end{frame}


%%%%%%%%%%%%%%%%%%%%%%%%%%%%%%%%%%%%%%%%%%%%%%%%%%%%%%%%%%%%%%%%%%%

\begin{frame}
\frametitle{Diagonalization}
\framesubtitle{Proof}

Why is the Diagonalization Theorem true?

\medskip
\begin{webonly}
\alert{$A$ diagonalizable implies $A$ has $n$ linearly independent eigenvectors:}
Suppose $A = PDP\inv$, where $D$ is diagonal with diagonal entries $\lambda_1,\lambda_2,\ldots,\lambda_n$.
Let $v_1,v_2,\ldots,v_n$ be the columns of $P$.
They are linearly independent because
$P$ is invertible.
So $Pe_i = v_i$, hence $P\inv v_i = e_i$.
\[ Av_i = PDP\inv v_i = PDe_i = P(\lambda_i e_i) = \lambda_i Pe_i =
\lambda_i v_i. \]
Hence $v_i$ is an eigenvector of $A$ with eigenvalue $\lambda_i$.
So the columns of $P$ form $n$ linearly independent eigenvectors of $A$, and the
diagonal entries of $D$ are the eigenvalues.

\medskip
\alert{$A$ has $n$ linearly independent eigenvectors implies $A$ is diagonalizable:}
Suppose $A$ has $n$ linearly independent eigenvectors $v_1,v_2,\ldots,v_n$, with
eigenvalues $\lambda_1,\lambda_2,\ldots,\lambda_n$.
Let $P$ be the invertible matrix with columns $v_1,v_2,\ldots,v_n$.
Let $D = P\inv A P$.
\[ De_i = P\inv A Pe_i = P\inv A v_i = P\inv(\lambda_i v_i)
= \lambda_i P\inv v_i = \lambda_i e_i. \]
Hence $D$ is diagonal, with diagonal entries
$\lambda_1,\lambda_2,\ldots,\lambda_n$.
Solving $D = P\inv A P$ for $A$ gives $A = PDP\inv$.
\end{webonly}

\end{frame}


%%%%%%%%%%%%%%%%%%%%%%%%%%%%%%%%%%%%%%%%%%%%%%%%%%%%%%%%%%%%%%%%%%%

\begin{frame}
\frametitle{Non-Distinct Eigenvalues}

\vskip -3mm
\begin{defn}
  Let $\lambda$ be an eigenvalue of a square matrix $A$.  The 
  \textbf{geometric multiplicity} of $\lambda$ is the dimension of the
  $\lambda$-eigenspace. 
\end{defn}

\pause
\begin{thm}
  Let $\lambda$ be an eigenvalue of a square matrix $A$.  Then
  \displayskips{3pt}
  \[ 1 \leq \text{(the geometric multiplicity of $\lambda$)}
  \leq \text{(the algebraic multiplicity of $\lambda$)}. \]
\end{thm}

\pause\vskip-1mm
The proof is beyond the scope of this course.

\pause\smallskip
\begin{cor}
  Let $\lambda$ be an eigenvalue of a square matrix $A$.  If the algebraic
  multiplicity of $\lambda$ is $1$, then the geometric multiplicity is also $1$.
\end{cor}

\pause
\begin{oneoffthm}{The Diagonalization Theorem (Alternate Form)}
  Let $A$ be an $n\times n$ matrix.  The following are equivalent:
  \begin{enumerate}
    \pause
  \item $A$ is diagonalizable.
    \pause
  \item The sum of the geometric multiplicities of the eigenvalues of $A$ equals
    $n$.
    \pause
  \item The sum of the algebraic multiplicities of the eigenvalues of $A$ equals
    $n$, and \emph{the geometric multiplicity equals the algebraic multiplicity} of
    each eigenvalue.
  \end{enumerate}
\end{oneoffthm}

\end{frame}


%%%%%%%%%%%%%%%%%%%%%%%%%%%%%%%%%%%%%%%%%%%%%%%%%%%%%%%%%%%%%%%%%%%
% JDR: I prefer to skip the following two slides in favor of the 
%   subsequent application to difference equations.

\begin{frame}
\frametitle{Non-Distinct Eigenvalues}
\framesubtitle{Examples}

\vskip-3mm
\begin{eg}
  If $A$ has $n$ distinct eigenvalues, then the algebraic multiplicity of each
  equals $1$, hence so does the geometric multiplicity, and therefore $A$ is
  diagonalizable. 
  
  \pause\medskip
  For example, $A = \mat{1 2; -1 4}$ has eigenvalues $2$ and $3$, so it is
  diagonalizable.
\end{eg}

\pause\medskip
\begin{eg}
  The matrix $A = \mat{4 -3 0; 2 -1 0; 1 -1 1}$ has characteristic polynomial
  \[ f(\lambda) = -(\lambda-1)^2(\lambda-2). \]  
  \pause
  The algebraic multiplicities of $1$ and $2$ are $2$ and $1$, respectively.
  \pause
  They sum to $3$.

  \pause\smallskip
  We showed before that the geometric multiplicity of $1$ is $2$ (the
  $1$-eigenspace has dimension $2$).
  \pause
  The eigenvalue $2$ automatically has geometric multiplicity $1$.

  \pause\smallskip
  Hence the geometric multiplicities add up to $3$, so $A$ is diagonalizable.
\end{eg}

\end{frame}


%%%%%%%%%%%%%%%%%%%%%%%%%%%%%%%%%%%%%%%%%%%%%%%%%%%%%%%%%%%%%%%%%%%

\begin{frame}
\frametitle{Non-Distinct Eigenvalues}
\framesubtitle{Another example}

\vskip-3mm
\begin{eg}
  The matrix $A = \mat{1 1; 0 1}$ has characteristic polynomial
  $f(\lambda) = (\lambda-1)^2$.

  \pause\smallskip
  It has one eigenvalue $1$ of algebraic multiplicity $2$.

  \pause\smallskip
  We showed before that the geometric multiplicity of $1$ is $1$ (the
  $1$-eigenspace has dimension $1$).

  \pause\smallskip
  Since the geometric multiplicity is smaller than the algebraic multiplicity,
  the matrix is \emph{not} diagonalizable.
\end{eg}

\end{frame}


%%%%%%%%%%%%%%%%%%%%%%%%%%%%%%%%%%%%%%%%%%%%%%%%%%%%%%%%%%%%%%%%%%%
% JDR: The following is a good transition into the stochastic matrix material

\begin{frame}
\frametitle{Applications to Difference Equations}

Let $D = \mat{1 0; 0 1/2}$.\\[2mm]
Fix a vector $v_0$, and let
$v_1 = Dv_0,\, v_2=Dv_1$, etc., so
$v_n = D^nv_0$.

\pause\bigskip
\alert{Question:} What happens to the $v_i$'s for different choices of $v_0$?

\bigskip
\begin{webonly}
\alert{Answer:} Note that $D$ is diagonal, so
\[ D^n\vec{a b} = \mat{1^n 0; 0 1/2^n}\vec{a b} = \vec{a b/2^n}. \]
So the $x$-coordinate of $v_n$ equals the $x$-coordinate of $v_0$, and the
$y$-coordinate gets halved every time.
\end{webonly}

\end{frame}


%%%%%%%%%%%%%%%%%%%%%%%%%%%%%%%%%%%%%%%%%%%%%%%%%%%%%%%%%%%%%%%%%%%

\begin{frame}
\frametitle{Applications to Difference Equations}
\framesubtitle{Picture}

\[ D\vec{a b} = \mat{1 0 ; 0 1/2}\vec{a b} = \vec{a b/2} \]

\begin{center}
\begin{tikzpicture}[scale=.5, thin border nodes]
  \draw[grid lines] (-5,-5) grid (5,5);
  \draw[thin] (-5,0) -- (5,0) (0,-5) -- (0,5);
  \draw[<-] (4.5,0) to[out=-90,in=180] ++(1,-.8)
    node[right, font=\scriptsize] {1-eigenspace};
  \draw[<-] (0,-4.5) to[out=180,in=0] ++(-.8,-1)
    node[left, font=\scriptsize] {$1/2$-eigenspace};

  \draw[vector] (0,0) -- (1,0) node[below] {$e_1$};
  \draw[vector] (0,0) -- (0,1) node[left] {$e_2$};

  \point[seq1] (p1) at (.5, 3);
  \point[seq1] (p2) at (2, -4);
  \point[seq1] (p3) at (3.5, 3.5);
  \point[seq1] (p4) at (3, -3);
  \point[seq1] (p5) at (-1.5, -2);
  \point[seq1] (p6) at (-3.33, 3);
  \point[seq1] (p7) at (-4.5, 4.5);
  \point[seq1] (p8) at (-3, -4);

  \foreach \i in {1,...,8} 
    \foreach \j in {2,...,5} {
      \point<\j->[seq\j] at ($(-5,0)!(p\i)!(5,0)!{.5^(\j-1)}!(p\i)$);
    }
    
  \node at (-6.5,0) {
    \begin{minipage}{1cm}
      \foreach \j in {1,...,5} {
        \pgfmathtruncatemacro{\i}{\j-1}
        \uncover<\j->{\color{seq\j}$v_{\i}$}\par
      }
    \end{minipage}
  };
  
  \point at (0,0);

\end{tikzpicture}
\end{center}
\pause[6]%
So all vectors get ``sucked into the $x$-axis,'' which is the $1$-eigenspace.

\end{frame}


%%%%%%%%%%%%%%%%%%%%%%%%%%%%%%%%%%%%%%%%%%%%%%%%%%%%%%%%%%%%%%%%%%%

\begin{frame}
\frametitle{Applications to Difference Equations}
\framesubtitle{More complicated example}

Let $A = \mat{3/4 1/4; 1/4 3/4}$.\\[2mm]
Fix a vector $v_0$, and let
$v_1 = Av_0,\, v_2=Av_1$, etc., so
$v_n = A^nv_0$.

\pause\medskip
\alert{Question:} What happens to the $v_i$'s for different choices of $v_0$?

\medskip
\begin{webonly}
\alert{Answer:} We want to compute powers of $A$, so this is a diagonalization
question.  The characteristic polynomial is
\[ f(\lambda) = \lambda^2 - \Tr(A)\,\lambda + \det(A)
= \lambda^2 - \frac 32\lambda + \frac 12
= (\lambda-1)\Bigl(\lambda-\frac 12\Bigr). \]
We compute eigenvectors with eigenvalues $1$ and $1/2$ to be, respectively,
\[ w_1 = \vec{1 1} \qquad w_2 = \vec{1 -1}. \]
\rlap{Therefore,}\hfill\qquad
$A = PDP\inv \sptxt{for} P = \mat{1 1; 1 -1} \qquad D = \mat{1 0; 0 1/2}$.%
\hfill\null\\[1mm]
This is the same matrix $D$ from before.  Hence
\[ v_n = A^nv_0 = PD^nP\inv v_0. \]
\end{webonly}

\end{frame}


%%%%%%%%%%%%%%%%%%%%%%%%%%%%%%%%%%%%%%%%%%%%%%%%%%%%%%%%%%%%%%%%%%%

\begin{frame}
\frametitle{Applications to Difference Equations}
\framesubtitle{Picture of the more complicated example}

\alert{Recall:} $A^n = PD^nP\inv$ acts on the usual coordinates of $v_0$ in the
same way that $D^n$ acts on the $\cB$-coordinates, where $\cB=\{w_1,w_2\}$.
\note{Show the previous picture on the other screen}
\smallskip

\begin{center}
\begin{tikzpicture}[scale=.3, thin border nodes]
  \begin{scope}
    \clip (-9,-9) rectangle (9,9);
    \draw[grid lines, cm={1,1,1,-1,(0,0)}] (-9,-9) grid (9,9);
  \end{scope}
  \begin{scope}[cm={1,1,1,-1,(0,0)}]
    \draw[thin] (-9,0) -- (9,0) (0,-9) -- (0,9);
    \draw[<-] (4.5,0) to[out=90,in=225] ++(.5,.8)
      node[right, font=\scriptsize] {1-eigenspace};
    \draw[<-] (0,-4.5) to[out=180,in=45] ++(-.8,-.5)
      node[left, font=\scriptsize] {$1/2$-eigenspace};

    \draw[vector] (0,0) -- (1,0) node[anchor=-30] {$w_1$};
    \draw[vector] (0,0) -- (0,1) node[anchor=30] {$w_2$};

    \point[seq1] (p1) at (.5, 3);
    \point[seq1] (p2) at (2, -4);
    \point[seq1] (p3) at (3.5, 3.5);
    \point[seq1] (p4) at (3, -3);
    \point[seq1] (p5) at (-1.5, -2);
    \point[seq1] (p6) at (-3.33, 3);
    \point[seq1] (p7) at (-4.5, 4.5);
    \point[seq1] (p8) at (-3, -4);
  
    \foreach \i in {1,...,8} 
      \foreach \j in {2,...,5} {
        \point<\j->[seq\j] at ($(-5,0)!(p\i)!(5,0)!{.5^(\j-1)}!(p\i)$);
      }
      
    \point at (0,0);
  \end{scope}

  \node at (-10,0) {
    \begin{minipage}{1cm}
      \foreach \j in {1,...,5} {
        \pgfmathtruncatemacro{\i}{\j-1}
        \uncover<\j->{\color{seq\j}$v_{\i}$}\par
      }
    \end{minipage}
  };
 
\end{tikzpicture}
\end{center}
\pause[6]
So all vectors get ``sucked into the $1$-eigenspace.''

\end{frame}


%%%%%%%%%%%%%%%%%%%%%%%%%%%%%%%%%%%%%%%%%%%%%%%%%%%%%%%%%%%%%%%%%%%

\begin{frame}
\frametitle{Applications to Difference Equations}
\framesubtitle{Remark}

The matrix $A = \mat{3/4 1/4; 1/4 3/4}$ is called a \textbf{stochastic matrix}.

\pause\bigskip
We will study such matrices in detail next time.

\end{frame}


%%% Local Variables:
%%% TeX-master: "../slides"
%%% End:
