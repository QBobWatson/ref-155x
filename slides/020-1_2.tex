
% JDR: This is probably more than 50 minutes worth of material.  A natural break
%   point is after "Inconsistent Matrices"; then do parametric form the next
%   day.

\titleframe{Section 1.2}{Row Reduction and Echelon Forms}

\def\rowop#1#2{%
  \hfill%
  \hbox to 0.2\linewidth{\hss\longsquiggly[#1]}%
  \hbox to 0.4\linewidth{\hss#2}%
  }

%%%%%%%%%%%%%%%%%%%%%%%%%%%%%%%%%%%%%%%%%%%%%%%%%%%%%%%%%%%%%%%%%%%

\begin{frame}
\frametitle{Row Echelon Form}

Let's come up with an \emph{algorithm} for turning an arbitrary matrix into a
``solved'' matrix.
\pause
What do we mean by ``solved''?

\pause

\begin{bluebox}{.8\textwidth}
  A matrix is in \textbf{row echelon form} if

  \pause\smallskip
  \begin{enumerate}
  \item All zero rows are at the bottom.
    \pause
  \item Each leading nonzero entry of a row is to the \emph{right} of the
    leading entry of the row above.
    \pause
  \item Below a leading entry of a row, all entries are \emph{zero}.
  \end{enumerate}
\end{bluebox}

\begin{uncoverenv}<7->
Picture:
\[\mat{
\color{red}\boxed\star,  \star,  \star,  \star,  \star ;
0  \color{red}\boxed\star,  \star,  \star,  \star ;
0  0  0  \color{red}\boxed\star, \star ;
0 0 0 0 0}\qquad
\begin{aligned}
  \star &= \text{any number} \\
  \color{red}\boxed\star &= \text{any nonzero number}
\end{aligned}
\]

\pause[8]
\begin{defn}
  A \textbf{pivot} $\color{red}\boxed\star$ is the first nonzero entry of a row
  of a matrix in row echelon form.
\end{defn}
\end{uncoverenv}

\end{frame}


%%%%%%%%%%%%%%%%%%%%%%%%%%%%%%%%%%%%%%%%%%%%%%%%%%%%%%%%%%%%%%%%%%%

\begin{frame}
\frametitle{Reduced Row Echelon Form}

\vskip-6mm\null
\begin{bluebox}{.8\textwidth}
  A matrix is in \textbf{reduced row echelon form} if it is in row echelon form,
  and in addition,

  \pause\smallskip
  \begin{enumerate}
    \setcounter{enumi}{3}
  \item The pivot in each nonzero row is equal to $1$.
    \pause
    \item Each pivot is the only nonzero entry in its column.
  \end{enumerate}
\end{bluebox}

\begin{uncoverenv}<4->
Picture:
\[\mat{
\color{red}1   0   \star,   0   \star ;
0   \color{red}1   \star , 0   \star ;
0   0   0   \color{red}1   \star ;
0 0 0 0 0
} \qquad
\begin{aligned}
  \star &= \text{any number} \\
  \color{red}1 &= \text{pivot}
\end{aligned}\]

\pause[5]\medskip
\alert{Note:} Echelon forms do not care whether or not a column is augmented.
Just ignore the vertical line.

\pause\smallskip
\begin{ques}
  Can every matrix be put into reduced row echelon form only using row operations?
\end{ques}

\pause
\alert{Answer:} Yes!  Stay tuned.
\end{uncoverenv}

\end{frame}


%%%%%%%%%%%%%%%%%%%%%%%%%%%%%%%%%%%%%%%%%%%%%%%%%%%%%%%%%%%%%%%%%%%

\begin{frame}
\frametitle{Reduced Row Echelon Form}
\framesubtitle{Continued}

\alert{Why is this the ``solved'' version of the matrix?}
\pause
\[\amat{
1  0  0  1 ;
0  1  0  -2 ;
0  0  1  3
}\]
is in reduced row echelon form.  It translates into
\webonlycmd{\[ \syseq{x = 1; y = -2; z = 3\rlap{,}} \]}\pause
which is clearly the solution.

\pause

\vfill

But what happens if there are fewer pivots than rows?
\pause \ldots parametrized solution set (later).

\end{frame}


%%%%%%%%%%%%%%%%%%%%%%%%%%%%%%%%%%%%%%%%%%%%%%%%%%%%%%%%%%%%%%%%%%%

\begin{pollframe}

\begin{poll}
\begin{bluebox}[Poll]{.8\textwidth}
  Which of the following matrices are in reduced row echelon form?
  \[ \textrm{A.}~ \mat{1 0; 0 2} \qquad \textrm{B.}~ \mat{0 0 0;0 0 0} \]
  \[ \textrm{C.}~ \vec{0 1 0 0} \qquad \textrm{D.}~ \mat{0 1 0 0} \qquad
  \textrm{E.}~ \mat{0 1 8 0} \]
  \[ \textrm{F.}~ \amat{1 17 0;0 0 1} \]
\end{bluebox}
\end{poll}

\note{Matrix A is in row echelon form though.}

\end{pollframe}


%%%%%%%%%%%%%%%%%%%%%%%%%%%%%%%%%%%%%%%%%%%%%%%%%%%%%%%%%%%%%%%%%%%

\begin{frame}
\frametitle{Reduced Row Echelon Form}

\vskip-3mm
\begin{thm}
  Every matrix is row equivalent to one and only one matrix in reduced row
  echelon form.
\end{thm}

\pause
\bigskip
We'll give an algorithm, called \textbf{row reduction}, which demonstrates that
every matrix is row equivalent to \emph{at least one} matrix in reduced row
echelon form.

\pause
\bigskip
\alert{Note:} Like echelon forms, the row reduction algorithm does not care if a
column is augmented: ignore the vertical line when row reducing.

\pause\bigskip
The uniqueness statement is interesting---%
\pause
it means that, nomatter \emph{how\/} you row reduce, you \emph{always\/} get the
same matrix in reduced row echelon form.
\pause
(Assuming you only do the three legal row operations.)
\pause
(And you don't make any arithmetic errors.)

\pause\bigskip
Maybe you can figure out why it's true!

\end{frame}


%%%%%%%%%%%%%%%%%%%%%%%%%%%%%%%%%%%%%%%%%%%%%%%%%%%%%%%%%%%%%%%%%%%

\begin{frame}
\frametitle{Row Reduction Algorithm}

\begin{itemize}[<+->]
\item [Step 1a] Swap the 1st row with a lower one so a leftmost nonzero entry is
  in 1st row (if necessary).
\item [Step 1b] Scale 1st row so that its leading entry is equal to 1.
\item [Step 1c] Use row replacement so all entries above and below this 1 are 0.
\item [Step 2a] Cover the first row, swap the 2nd row with a lower one so that the leftmost nonzero (uncovered) entry is in 2nd row; uncover 1st row.
\item [Step 2b] Scale 2nd row so that its leading entry is equal to 1.
\item [Step 2c] Use row replacement so all entries above and below this 1 are 0.
\item [Step 3a] Cover the first two rows, swap the 3rd row with a lower one so
  that the leftmost nonzero (uncovered) entry is in 3rd row; uncover first two rows.
\item [etc.]
\end{itemize}

\note{Do example concurrently, on the board.}

\begin{eg}<1->\vskip-5mm
  \[\amat{
    0   -7   -4   2 ;
    2   4   6   12 ;
    3   1   -1   -2
  }\]
\end{eg}

\end{frame}


%%%%%%%%%%%%%%%%%%%%%%%%%%%%%%%%%%%%%%%%%%%%%%%%%%%%%%%%%%%%%%%%%%%

\begin{frame}
\frametitle{Row Reduction}
\framesubtitle{Example}

\def\r{\color{red}}
$\amat{
\phantom-\namedbox{a11}{0}   -7   -4   2 ;
\namedbox{a21}{2}   4   6   12 ;
\namedbox{a31}{3}   1   -1   -2
}$%
\begin{webonly}%
\begin{tikzpicture}[remember picture, overlay]
  \node [draw,thick,rounded corners,blue!50,fit=(a11)] (x) {};
  \node [blue!50, below=1.5mm of a31, anchor=north west, xshift=-1.3cm] (expl) 
    {Step 1a: Row swap to make this nonzero.};
  \draw[->, blue!50, rounded corners] 
    let \p1=($(expl.north west) + (5mm,0)$) in
      (\p1) |- (x.west);
\end{tikzpicture}%
\rowop{$R_1 \ToT R_2$}{$\amat{
      \namedbox{b11}{\r2}   4   6   12 ;
      \phantom-0   -7   -4   2 ;
      \namedbox{b31}3   1   -1   -2
    }$}%
\begin{tikzpicture}[remember picture, overlay]
  \node [draw,thick,rounded corners,blue!50,fit=(b11)] (y) {};
  \node [blue!50, below=1.5mm of b31, anchor=north west, xshift=-1.3cm] (expl2) 
    {Step 1b: Scale to make this $1$.};
  \draw[->, blue!50, rounded corners] 
    let \p1=($(expl2.north west) + (5mm,0)$) in
      (\p1) |- (y.west);
\end{tikzpicture}\\[5mm]%
\rowop{$R_1 = R_1\divsymb 2$}{$\amat{
      \r1   2   3   6 ;
      \phantom-0   -7   -4   2 ;
      \namedbox{c31}3 1 -1 -2
    }$}%
\begin{tikzpicture}[remember picture, overlay]
  \node [draw,thick,rounded corners,blue!50,fit=(c31)] (z) {};
  \node [blue!50, below=1.5mm of c31, anchor=north west, xshift=-1.3cm, align=left] (expl3) 
    {Step 1c: Subtract a multiple of\\
     ~the first row to clear this.};
  \draw[->, blue!50, rounded corners] 
    let \p1=($(expl3.north west) + (5mm,0)$) in
      (\p1) |- (z.west);
\end{tikzpicture}%
\\[8mm]%
\rowop{$R_3= R_3 - 3R_1$}{$\amat{
    1   2   3   6 ;
    0   -7   -4   2;
    \r0 -5 -10 -20
  }$}\\[3mm]%
\leavevmode\rlap{\parbox{.4\linewidth}{\small\color{blue!50}
    Optional: swap rows $2$ and $3$ to
    make Step 2b easier later on.}}%
\rowop{$R_2 \ToT R_3$}{$\amat{
    1  2    3   6;
    0 -5  -10 -20;
    0 -7   -4   2;
  }$}
\end{webonly}

\end{frame}


%%%%%%%%%%%%%%%%%%%%%%%%%%%%%%%%%%%%%%%%%%%%%%%%%%%%%%%%%%%%%%%%%%%

\begin{frame}
\frametitle{Row Reduction}
\framesubtitle{Example, continued}
\def\r{\color{red}}\def\g{\color{black!15}}

$\amat{
    \g1  \g2    \g3   \g6;
    0 \namedbox{a22}{-5}  -10 -20;
    \namedbox{a31}{0} -7   -4   2;
  }$%
\begin{webonly}
\begin{tikzpicture}[remember picture, overlay]
  \node [draw,thick,rounded corners,blue!50,fit=(a22)] (x) {};
  \node [blue!50, below=1.5mm of a31, anchor=north west, xshift=-1cm, align=left] (expl) 
    {Step 2a: This is already nonzero.\\
     Step 2b: Scale to make this $1$.\\[1mm]
     \hskip1em (There are no fractions because\\
     \hskip1em of the optional step before.)};
  \draw[->, blue!50, rounded corners] 
    let \p1=($(expl.north west) + (5mm,0)$) in
      (\p1) |- (x.west);
\end{tikzpicture}%
\rowop{$R_2 = R_2 \divsymb -5$}{$\amat{
    1 \namedbox{b12}{\phantom-2} 3 6 ;
    0 \r1 2 4 ;
    \namedbox{b31}{0} \namedbox{b32}{-7} -4 2
  }$}%
\begin{tikzpicture}[remember picture, overlay]
  \node [draw,thick,rounded corners,blue!50,fit=(b12)] (y) {};
  \node [draw,thick,rounded corners,blue!50,fit=(b32)] (y') {};
  \node [blue!50, below=1.5mm of b31, anchor=north west, xshift=-1cm, align=left] (expl2) 
    {Step 2c: Add multiples of\\
      ~the second row to clear these.};
  \draw[->, blue!50, rounded corners] 
    let \p1=($(expl2.north west) + (5mm,0)$) in
      (\p1) |- (y.west);
  \draw[->, blue!50, rounded corners] 
    let \p1=($(expl2.north west) + (5mm,0)$) in
      (\p1) |- (y'.west);
\end{tikzpicture}%
\\[8mm]%
\rowop{$R_1 = R_1 - 2R_2$}{$\amat{
    1 \r0 -1 -2;
    0 1 2 4;
    0 -7 -4 2}$}\\
\rowop{$R_3 = R_3 + 7R_2$}{$\amat{
    1 0 -1 -2;
    0 1 2 4;
    0 \r0 10 30}$}
\end{webonly}

\pause\bigskip
\alert{Note:} Step~2 never messes up the first (nonzero) column of the matrix,
because it looks like this:

\[ \amat{
  \phantom11 \star, \star, \star;
  \namedbox{b21}{0} \star, \star, \namedbox{b24}{\star};
  0 \star, \star, \star} \]
\begin{tikzpicture}[remember picture, overlay]
  \node [draw,thick,rounded corners,blue!50,fit=(b21) (b24)] (z) {};
  \node [blue!50, left=8mm of b21, anchor=east, align=left] (expl3) 
    {``Active'' row};
  \draw[->, blue!50] (expl3.east) -- (z.west);
\end{tikzpicture}

\end{frame}


%%%%%%%%%%%%%%%%%%%%%%%%%%%%%%%%%%%%%%%%%%%%%%%%%%%%%%%%%%%%%%%%%%%

\begin{frame}
\frametitle{Row Reduction}
\framesubtitle{Example, continued}
\def\r{\color{red}}\def\g{\color{black!15}}

$\amat{
    \g1 \g0 \g-1 \g-2;
    \g0 \g1 \g2  \g4;
    \namedbox{a31}{0} 0 \namedbox{a33}{10} 30
  }$%
\begin{webonly}
\begin{tikzpicture}[remember picture, overlay]
  \node [draw,thick,rounded corners,blue!50,fit=(a33)] (x) {};
  \node [blue!50, below=1.5mm of a31, anchor=north west, xshift=-1cm, align=left] (expl) 
    {Step 3a: This is already nonzero.\\
     Step 3b: Scale to make this $1$.};
  \draw[->, blue!50, rounded corners] 
    let \p1=($(expl.north west) + (5mm,0)$) in
      (\p1) |- (x.west);
\end{tikzpicture}%
\rowop{$R_3 = R_3 \divsymb 10$}{$\amat{
    1 0 \namedbox{b13}{-1} -2;
    0 1 \namedbox{b23}{2}  4;
    \namedbox{b31}{0} 0 \r1 3
  }$}%
\begin{tikzpicture}[remember picture, overlay]
  \node [draw,thick,rounded corners,blue!50,fit=(b13) (b23)] (y) {};
  \node [blue!50, below=1.5mm of b31, anchor=north west, xshift=-1cm, align=left] (expl) 
    {Step 3c: Add multiples of\\
      ~the third row to clear these.};
  \draw[->, blue!50, rounded corners] 
    let \p1=($(expl.north west) + (5mm,0)$) in
      (\p1) |- (y.west);
\end{tikzpicture}%
\\[8mm]%
\rowop{$R_1 = R_1 + R_3$}{$\amat{
    1 0 \r0 1;
    0 1 2 4;
    0 0 1 3
  }$}
\\
\rowop{$R_2 = R_2 - 2R_3$}{$\amat{
    1 0 0 1;
    0 1 \r0 -2;
    0 0 1 3
  }$}
\end{webonly}

\pause\medskip
\alert{Note:} Step~3 never messes up the columns to the left.

\pause\medskip
\alert{Success!}  
The reduced row echelon form is
\[ \amat{
  1 0 0 1;
  0 1 0 -2;
  0 0 1 3
} \qquad\implies\qquad \spalignsysdelims\{.
 \syseq{
  x \+ \. \+ \. = 1 ;
  \. \+ y \+ \. = -2;
  \. \+ \. \+ z = 3
} \]
\pause
\alert{Step 4:} profit?

\end{frame}


%%%%%%%%%%%%%%%%%%%%%%%%%%%%%%%%%%%%%%%%%%%%%%%%%%%%%%%%%%%%%%%%%%%

\begin{frame}
\frametitle{Row Reduction}
\framesubtitle{Another example}

The linear system

\smallskip

\hfill$\syseq{2x + 10y = -1; 3x + 15y = 2}$\hfill
gives rise to the matrix\hfill
\webonlycmd{$\amat{ 2 10 -1; 3 15 2}.$\hskip 0pt plus .3fil}\hfill\null

\pause\smallskip
\note[item]{Students say what the next step is.}
\note[item]{Put the algorithm up on the second screen.}
Let's row reduce it:

\def\r{\color{red}}
\begin{center}
\begin{minipage}{.7\linewidth}
$\amat{ 2 {10} {-1} ;3 {15} 2}$
\begin{webonly}
  \rowop{$R_1 = R_1\divsymb 2$}{$\amat{
      \r1 5 {-\frac 12}; 3 {15} 2}$}\rlap{\qquad\textcolor{blue}{(Step 1b)}}\par
  \rowop{$R_2 = R_2 - 3R_1$}{$\amat{
      1 5 {-\frac 12} ; 
      \r0 0 {\frac 72}}$}\rlap{\qquad\textcolor{blue}{(Step 1c)}}\par
  \rowop{$R_2 = R_2 \times \frac 27$}{$\amat{
      1 5 {-\frac 12} ;0 0 \r1}$}\rlap{\qquad\textcolor{blue}{(Step 2b)}}\par
  \rowop{$R_1 = R_1 + \frac 12R_2$}{$\amat{
      1 5 \r0 ;0 0 1}$}\rlap{\qquad\textcolor{blue}{(Step 2c)}}
\end{webonly}%
\end{minipage}\qquad\null%
\end{center}

\pause
The row reduced matrix

\medskip
\hfill
$\amat{1 5 0 ;0 0 1}$ \hfill
\parbox{.35\linewidth}{\centering corresponds to the\\\emph{inconsistent} system}
\hfill
$\syseq{x + 5y = 0; \. \+ 0 = 1\rlap.}$
\hfill\null

\end{frame}


%%%%%%%%%%%%%%%%%%%%%%%%%%%%%%%%%%%%%%%%%%%%%%%%%%%%%%%%%%%%%%%%%%%

\begin{frame}
\frametitle{Inconsistent Matrices}

\vskip-3mm
\begin{ques}
  What does an augmented matrix in reduced row echelon form look like, if its
  system of linear equations is inconsistent? 
  \note{Ask what it means in terms of pivots}
\end{ques}

\pause\bigskip
\alert{Answer:}
\[\amat{
1  0  \star,  \star,  \color{red}0 ;
0  1  \star,  \star,  \color{red}0 ;
0  0  0  0  \color{red}1 
}\]

\pause\bigskip
\begin{bluebox}{.75\linewidth}
  An augmented matrix corresponds to an inconsistent system of equations if and
  only if \emph{the last \textup{(i.e., the augmented)} column is a pivot column.}
\end{bluebox}

\end{frame}


%%%%%%%%%%%%%%%%%%%%%%%%%%%%%%%%%%%%%%%%%%%%%%%%%%%%%%%%%%%%%%%%%%%

\begin{frame}
\frametitle{Another Example}

The linear system

\smallskip
\hfill
$\syseq{2x + y + 12z = 1; x + 2y + 9z = -1}$\hfill
gives rise to the matrix\hfill
$\amat{ 2 1 {12} 1 ;1 2 9 {-1}}.$\hfill\null

\pause\smallskip
Let's row reduce it:

\def\r{\color{red}}
\begin{center}
\begin{minipage}{.8\linewidth}
  $\amat{ 2 1 {12} 1 ;1 2 9 {-1}}$
\begin{webonly}
  \rowop{$R_1 \ToT R_2$}{$\amat{
      \r1 2 9 {-1}; 2 1 {12} 1}$}\rlap{\qquad\textcolor{blue}{(Optional)}}\par
  \rowop{$R_2 = R_2 - 2R_1$}{$\amat{
      1 2 9 {-1} ; \r0 {-3} {-6} 3}$}\rlap{\qquad\textcolor{blue}{(Step 1c)}}\par
  \rowop{$R_2 = R_2 \divsymb -3$}{$\amat{
      1 2 9 {-1} ;0 \r1 2 {-1}}$}\rlap{\qquad\textcolor{blue}{(Step 2b)}}\par
  \rowop{$R_1 = R_1 - 2R_2$}{$\amat{
      1 \r0 5 1 ;0 1 2 {-1}}$}\rlap{\qquad\textcolor{blue}{(Step 2c)}}
\end{webonly}%
\end{minipage}\qquad\qquad\null
\end{center}

\pause
The row reduced matrix
\medskip

\hfill $\amat{1 0 5 1 ;0 1 2 {-1}}$ \hfill
\parbox{.35\linewidth}{\centering corresponds to the\\linear system}
\hfill\spalignsysdelims\{.
$\syseq{x + 5z = 1; y + 2z = -1}$
\hfill\null

\end{frame}


%%%%%%%%%%%%%%%%%%%%%%%%%%%%%%%%%%%%%%%%%%%%%%%%%%%%%%%%%%%%%%%%%%%

\begin{frame}
\frametitle{Another Example}
\framesubtitle{Continued}
The system
\[ \syseq{x + 5z = 1; y + 2z = -1} \]
comes from a matrix in reduced row echelon form.
\pause
Are we done?  Is the system solved?

\pause\medskip
\alert{Yes!}  Rewrite:
\[ \syseq{x = 1 - 5z; y = -1 - 2z} \]
For any value of $z$, there is exactly one value of $x$ and $y$ that makes the
equations true.  But $z$ can be \emph{anything we want\/}!

\pause\medskip
So we have found the solution set: it is all values $x,y,z$ where
\[ \syseq{x = 1 - 5z; y = -1 - 2z; \llap(z = \. \+ z\rlap)}
\qquad \text{for $z$ any real number.} \]
\pause
This is called the \textbf{parametric form} for the solution.

\medskip
\webonlycmd{For instance, $(1, -1, 0)$ and $(-4, -3, 1)$ are solutions.}

\end{frame}


%%%%%%%%%%%%%%%%%%%%%%%%%%%%%%%%%%%%%%%%%%%%%%%%%%%%%%%%%%%%%%%%%%%

\begin{frame}
\frametitle{Free Variables}

\vskip-3mm
\begin{defn}
  Consider a \emph{consistent} linear system of equations in the variables
  $x_1,\ldots,x_n$.
  Let $A$ be a row echelon form of the matrix for this system.

  \pause\smallskip
  We say that $x_i$ is a \textbf{free variable} if its
  corresponding column in $A$ is \emph{not} a pivot column.
\end{defn}

\pause
\begin{bluebox}[Important]{.8\textwidth}
  \begin{enumerate}
  \item You can choose \emph{any value} for the free variables in a (consistent)
    linear system.
  \item<4-> Free variables come from \emph{columns without pivots\/} in a matrix
    in row echelon form.
  \end{enumerate}
\end{bluebox}

\pause[5]\vskip-2mm
In the previous example, $\color{seq-red}z$ was free because the reduced row echelon form
matrix was
\[ \amat{ 1 0 {\color{seq-red}5} 4 ; 0 1 {\color{seq-red}2} {-1}}. \]

\pause
In this matrix:
\[\amat{
1  \color<7->{seq-red}\star,  0  \color<7->{seq-blue}\star,  \star ;
0  \color<7->{seq-red}0   1  \color<7->{seq-blue}\star,  \star
}\]
the free variables are
\blankuntil{7}{$\color{seq-red}x_2$ and $\color{seq-blue}x_4$}.
\pause[8]%
(What about the last column?)
\note{It doesn't correspond to a variable!}

\end{frame}


%%%%%%%%%%%%%%%%%%%%%%%%%%%%%%%%%%%%%%%%%%%%%%%%%%%%%%%%%%%%%%%%%%%

\begin{frame}
\frametitle{One More Example}

\vskip-1mm
The reduced row echelon form of the matrix for a linear system in
$x_1,x_2,x_3,x_4$ is
\[ \amat{
    1 \color<3->{seq-red}0 0 \color<3->{seq-blue}3 2 ;
    0 \color<3->{seq-red}0 1 \color<3->{seq-blue}4 -1
  } \]
\pause
The free variables are
\blankuntil{3}{$\color{seq-red}x_2$ and $\color{seq-blue}x_4$}:
\pause
they are the ones whose
columns are \emph{not} pivot columns.

\pause\medskip
This translates into the system of equations
\[ \spalignsysdelims\{.
\syseq{x_1 \+ \. \+ \. + 3x_4 = 2;
  \.  \+ \. \+ x_3 + x_4 = -1} 
\pause\qquad\implies\qquad\spalignsysdelims..
\framebox{$\syseq{x_1 = 2 - 3x_4; x_3 = -1 - 4x_4\llap.}$}\]
\pause
What happened to $x_2$?  What is it allowed to be?
\pause
Anything!  The general solution is
\webonlycmd{
\[ (x_1,\,x_2,\,x_3,\,x_4) = (2-3x_4,\,x_2,\,-1-4x_4,\,x_4) \]
}%
for any values of $x_2$ and $x_4$.
\begin{webonly}
  For instance, $(2,0,-1,0)$ is a solution ($x_2=x_4=0$), and
  $(5,1,3,-1)$ is a solution ($x_2=1$, $x_4=-1$).
\end{webonly}

\pause
\begin{bluebox}{.8\linewidth}
  The boxed equation is called the \textbf{parametric form} of the general solution to
  the system of equations.  It is obtained by moving all free variables to the
  right-hand side of the $=$.
\end{bluebox}

\end{frame}


%%%%%%%%%%%%%%%%%%%%%%%%%%%%%%%%%%%%%%%%%%%%%%%%%%%%%%%%%%%%%%%%%%%

\begin{pollframe}

\begin{poll}
\vskip 1cm

\begin{bluebox}[Poll]{.6\textwidth}
  Is it possible for a system of linear equations to have exactly two solutions?
\end{bluebox}
\end{poll}

\end{pollframe}


%%%%%%%%%%%%%%%%%%%%%%%%%%%%%%%%%%%%%%%%%%%%%%%%%%%%%%%%%%%%%%%%%%%

\begin{frame}
\frametitle{Summary}

There are \emph{three possibilities} for the reduced row echelon form of the
augmented matrix of a linear system.\pause

\begin{enumerate}
\item \alert{The last column is a pivot column.}\\
  In this case, the system is \emph{inconsistent}.  
  \pause
  There are \emph{zero} solutions, i.e.\ the solution set is \emph{empty}.  
  Picture:
  \pause
  \[ \amat{ 1 0 {\color{red}0}; 0 1 {\color{red}0}; 0 0 {\color{red}1}} \]

\pause
\item \alert{Every column except the last column is a pivot column.}\\
  In this case, the system has a \emph{unique solution}.  Picture:
  \[ \amat{
      1  0  0  \star ;
      0  1  0  \star ;
      0  0  1  \star
    } \]
\note{Ask: what are $x,y,z$?}

\pause
\item \alert{The last column is not a pivot column, and some other column isn't either.}\\
  In this case, the system has \emph{infinitely many} solutions, corresponding
  to the infinitely many possible values of the free variable(s).  Picture:
  \[\amat{
      1  \color{seq-red}\star,  0  \color{seq-blue}\star,  \star ;
      0  \color{seq-red}0   1  \color{seq-blue}\star, \star
    }\]

\end{enumerate}

\end{frame}


%%% Local Variables:
%%% TeX-master: "../slides"
%%% End:
