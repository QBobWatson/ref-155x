
% JDR: Intro to the course.  Some instructors skip this lecture.

\begin{frame}
  \titlepage
\end{frame}

\titleframe{Introduction to Linear Algebra}{Motivation and Overview}

%%%%%%%%%%%%%%%%%%%%%%%%%%%%%%%%%%%%%%%%%%%%%%%%%%%%%%%%%%%%%%%%%%%

\begin{frame}
\frametitle{Linear.  Algebra.}
What is Linear Algebra?

\vfill

\alert{Linear}
\begin{itemize}
  \begin{webonly}
    \item having to do with lines/planes/etc. 
    \item For example, $x+y+3z=7$, not $\sin, \log, x^2$, etc.
  \end{webonly}
\end{itemize}

\vfill

\pause
\alert{Algebra}
\begin{itemize}
\webonlycmd{\item solving equations involving numbers and symbols}
\pause
\item from al-jebr (Arabic), meaning reunion of broken parts
\item 9$^{th}$ century Abu Ja'far Muhammad ibn Muso al-Khwarizmi
\end{itemize}

\vfill

\end{frame}


%%%%%%%%%%%%%%%%%%%%%%%%%%%%%%%%%%%%%%%%%%%%%%%%%%%%%%%%%%%%%%%%%%%

\begin{frame}
  \frametitle{Why a whole course?}
  But these are the easiest kind of equations!  I learned how to solve them in
  7th grade!

  \bigskip
  \pause
  Ah, but engineers need to solve \emph{lots} of equations in \emph{lots} of
  variables.
  \[\syseq{
    3x_1 + 4x_2 + 10x_3 + 19x_4 - 2x_5 - 3x_6 = 141;
    7x_1 + 2x_2 - 13x_3 - 7x_4 + 21x_5 + 8x_6 = 2567;
    -x_1 + 9x_2 + \frac 32x_3 + x_4 + 14x_5 + 27x_6 = 26;
    \frac 12x_1 + 4x_2 + 10x_3 + 11x_4 + 2x_5 + x_6 = -15 
  }\]

  \bigskip
  \pause
  Often, it's enough to know some information about the set of solutions without
  having to solve the equations at all!

  \bigskip
  \pause
  Also, what if one of the coefficients of the $x_i$ is itself a parameter---
  like an unknown real number $t$?

  \bigskip
  \pause
  In real life, the difficult part is often in recognizing that a problem can be
  solved using linear algebra in the first place: need \emph{conceptual}
  understanding. 

\end{frame}


%%%%%%%%%%%%%%%%%%%%%%%%%%%%%%%%%%%%%%%%%%%%%%%%%%%%%%%%%%%%%%%%%%%

\begin{frame}
\frametitle{Linear Algebra in Engineering}

\vfill

\begin{bluebox}{.5\textwidth}
  Large classes of engineering problems, no matter how huge, can be reduced to
  linear algebra:
  \begin{align*}
    Ax &= b \quad \text{ or}\\
    Ax &= \lambda x
  \end{align*}
\end{bluebox}

\bigskip
\pause
\begin{center}
  ``\ldots and now it's just linear algebra''
\end{center}

\note[item]{Linear algebra is extremely well-understood.}
\note[item]{If you can reduce to a linear algebra problem, you can probably solve it.}  \note[item]{Unlike reducing to solving a PDE\ldots}

 \vfill \vfill

\end{frame}


%%%%%%%%%%%%%%%%%%%%%%%%%%%%%%%%%%%%%%%%%%%%%%%%%%%%%%%%%%%%%%%%%%%

\usetikzlibrary{decorations.markings}

\tikzset{
  mid arrow/.style={
    postaction={
      decoration={
        markings,
        mark=at position #1 with {\arrow{Stealth[scale=1]}},
      },
      decorate
    },
  },
  rmid arrow/.style={
    postaction={
      decoration={
        markings,
        mark=at position #1 with {\arrowreversed{Stealth[scale=1]}},
      },
      decorate
    }
  },
  mid arrow/.default={0.5},
  rmid arrow/.default={0.5},
}

\begin{frame}
\frametitle{Applications of Linear Algebra}
\alert{Civil Engineering:} How much traffic flows through the four labeled segments?

\vfill
\begin{columns}[onlytextwidth]
  \column[c]{.5\textwidth}
  \uncover<2->{\longsquiggly~ system of linear equations:}    
  \begin{webonly}
    \[\syseq{
      w + 120 = x + 250;
      x + 120 = y + 70;
      y + 630 = z + 390;
      z + 115 = w + 175
    }\]
  \end{webonly}

  \column[c]{.5\textwidth}
  \centering
  \begin{tikzpicture}[scale=1, thick,
      every node/.style={inner sep=3pt, label distance=1mm}]
    \node at (0,2.4) {Traffic flow (cars/hr)};
    \point[scale=1.5] (A) at (-1,1);
    \point[scale=1.5] (B) at (1,1);
    \point[scale=1.5] (C) at (1,-1);
    \point[scale=1.5] (D) at (-1,-1);
    \draw[mid arrow] (A.center) to["$x$"] (B.center);
    \draw[mid arrow] (B.center) to["$y$"] (C.center);
    \draw[mid arrow] (C.center) to["$z$"] (D.center);
    \draw[mid arrow] (D.center) to["$w$"] (A.center);
    \draw[rmid arrow=.3] (A.center) to["$120$"] +(-1,0);
    \draw[mid arrow=.7] (A.center) to["$250$"] +(0,1);
    \draw[mid arrow=.7] (B.center) to["$70$" swap] +(1,0);
    \draw[rmid arrow=.3] (B.center) to["$120$" swap] +(0,1);
    \draw[rmid arrow=.3] (C.center) to["$630$"] +(1,0);
    \draw[mid arrow=.7] (C.center) to["$390$"] +(0,-1);
    \draw[mid arrow=.7] (D.center) to["$175$" swap] +(-1,0);
    \draw[rmid arrow=.3] (D.center) to["$115$" swap] +(0,-1);
  \end{tikzpicture}
\end{columns}
\vfill

\end{frame}




%%%%%%%%%%%%%%%%%%%%%%%%%%%%%%%%%%%%%%%%%%%%%%%%%%%%%%%%%%%%%%%%%%%

\begin{frame}
\frametitle{Applications of Linear Algebra}
\alert{Chemistry:} Balancing reaction equations 
\bigskip

\begin{center}
  \underline{\hbox to 5mm{\hss\uncover<2->{$x$}\hss}} C$_2$H$_6$ +
  \underline{\hbox to 5mm{\hss\uncover<2->{$y$}\hss}} O$_2$ $\rightarrow$
  \underline{\hbox to 5mm{\hss\uncover<2->{$z$}\hss}} CO$_2$ +
  \underline{\hbox to 5mm{\hss\uncover<2->{$w$}\hss}} H$_2$O
\end{center}
\uncover<2->{$\longsquiggly$ system of linear equations, one equation for each
  element.}

\begin{webonly}
  \[\begin{split}
    2x &= z \\
    6x &= 2w \\
    2y &= 2z
  \end{split}\]
\end{webonly}

\vfill

\end{frame}


%%%%%%%%%%%%%%%%%%%%%%%%%%%%%%%%%%%%%%%%%%%%%%%%%%%%%%%%%%%%%%%%%%%

\begin{frame}
\frametitle{Applications of Linear Algebra}
\alert{Biology:} In a population of rabbits\ldots
\begin{itemize}
\item half of the new born rabbits survive their first year
\item of those, half survive their second year
\item the maximum life span is three years
\item rabbits produce 0, 6, 8 rabbits in their first, second, and third years
\end{itemize}

\pause\medskip
If I know the population in 2016 (in terms of the number of first, second, and
third year rabbits), then what is the population in 2017?
\pause\\[5mm]
$\longsquiggly$ system of linear equations:

\begin{webonly}
  \[\syseq{
    \. \+ 6y_{2016} + 8z_{2016} = x_{2017};
    \frac 12x_{2016} \+ \. \+ \. = y_{2017};
    \. \+ \frac 12y_{2016} \+ \. = z_{2017} 
  }\]
\end{webonly}

\pause

\begin{ques}
  Does the rabbit population have an asymptotic behavior?  Is this even a linear
  algebra question?
  \pause
  Yes, it is!
\end{ques}

\end{frame}


%%%%%%%%%%%%%%%%%%%%%%%%%%%%%%%%%%%%%%%%%%%%%%%%%%%%%%%%%%%%%%%%%%%

\begin{frame}
\frametitle{Applications of Linear Algebra}
\alert{Geometry and Astronomy:} Find the equation of a circle passing through 3
given points, say $(1,0)$, $(0,1)$, and $(1,1)$. 
\pause
The general form of a circle is
$a(x^2+y^2)+bx+cy+d=0$.  
\pause\\[5mm]
$\longsquiggly$ system of linear equations: 

\note{Say we're solving for $a,b,c,d$.}

\begin{webonly}
  \[\syseq{
    a + b \+ \. + d = 0;
    a \+ \. + c + d = 0;
    2a + b + c + d = 0
  }\]
\end{webonly}

\vfill \vfill

\pause
Very similar to: compute the orbit of a planet: 
\[ ax^2+by^2+cxy+dx+ey+f=0 \]

\end{frame}


%%%%%%%%%%%%%%%%%%%%%%%%%%%%%%%%%%%%%%%%%%%%%%%%%%%%%%%%%%%%%%%%%%%

\begin{frame}
\frametitle{Applications of Linear Algebra}
\alert{Google:} ``The 25 billion dollar eigenvector.'' Each web page has some
importance, which it shares via outgoing links to other pages
\pause\\[1mm]
$\longsquiggly$ system of linear equations
\pause
(in gazillions of variables).

\pause
\vfill

Larry Page flies around in a private 747 because he paid attention in his
linear algebra class!

\vfill
\pause

Stay tuned!

\vfill

\end{frame}


%%%%%%%%%%%%%%%%%%%%%%%%%%%%%%%%%%%%%%%%%%%%%%%%%%%%%%%%%%%%%%%%%%%

\begin{frame}
\frametitle{Overview of the Course}

\begin{itemize}
\item<1-> Solve the matrix equation $Ax = b$
  \begin{itemize}
  \item<2-> \alert{Solve systems of linear equations} using matrices, row
    reduction, and inverses.
  \item<3-> \alert{Solve systems of linear equations with varying parameters}
    using parametric forms for solutions, the geometry of linear transformations,
    the characterizations of invertible matrices, and determinants.
  \end{itemize}
  \bigskip
\item<4-> Solve the matrix equation $Ax = \lambda x$
  \begin{itemize}
  \item<5-> \alert{Solve eigenvalue problems} through the use of the
    characteristic polynomial.
  \item<6-> \alert{Understand the dynamics of a linear transformation} via the
    computation of eigenvalues, eigenvectors, and diagonalization.
  \end{itemize}
  \bigskip
\item<7-> Almost solve the equation $Ax = b$
  \begin{itemize}
  \item<8->
    \alert{Find best-fit solutions to systems of linear equations that have no
      actual solution} using least squares approximations.
  \end{itemize}
\end{itemize}

\end{frame}


%%%%%%%%%%%%%%%%%%%%%%%%%%%%%%%%%%%%%%%%%%%%%%%%%%%%%%%%%%%%%%%%%%%

\begin{frame}
\frametitle{What to Expect This Semester}

Your previous math courses probably focused on how to do (sometimes rather
involved) computations.
\begin{webonly}
  \begin{itemize}
  \item Compute the derivative of $\sin(\log x)\cos(e^x)$.  
  \item Compute $\int_0^1 (1-\cos(x))\,dx$.
  \end{itemize}
\end{webonly}
\pause\medskip
This is important, \textbf{but} Wolfram Alpha can do all these problems better
than any of us can.
\pause
Nobody is going to hire you to do something a computer can do better.

\pause
\begin{bluebox}{.8\linewidth}
  If a computer can do the problem better than you can, then it's just an
  algorithm: this is not real problem solving.
\end{bluebox}

\pause
So what are we going to do?
\pause
\begin{itemize}
\item About half the material focuses on how to do linear algebra
  computations---that is still important.
\pause
\item The other half is on \emph{conceptual} understanding of linear algebra.
  \pause
  This is much more subtle: it's about figuring out \emph{what question} to ask
  the computer, or whether you actually need to do any computations at all.
\end{itemize}
\end{frame}


%%%%%%%%%%%%%%%%%%%%%%%%%%%%%%%%%%%%%%%%%%%%%%%%%%%%%%%%%%%%%%%%%%%

\begin{frame}
\frametitle{Resources}

\begin{bluebox}{.8\textwidth}
  \LARGE\centering Everything is on the course web page.
\end{bluebox}

\pause\smallskip
Including these slides.
\pause
There's a link from T-Square.

\pause\medskip
On the webpage you'll find:
\pause
\begin{itemize}
\item\alert{Course administration:} the names of your TAs, their office hours,
  your recitation location, etc.
\pause
\item\alert{Course organization:} grading policies, details about homework and
  exams, etc.
\pause
\item\alert{Help and advice:} how to succeed in this course, resources available
  to you.
\pause
\item\alert{Calendar:} what will happen on which day, links to daily slides,
  quizzes, practice exams, solutions, etc.
\end{itemize}

\pause\medskip
\alert{T-Square:} your grades, link to WeBWorK.

\pause\medskip
\alert{Piazza:} this is where to ask questions, and where I'll post announcements.

\end{frame}



%%% Local Variables:
%%% TeX-master: "../slides"
%%% End:
