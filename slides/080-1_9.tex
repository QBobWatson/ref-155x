
\usetikzlibrary{math}

\titleframe{Section 1.9}{The Matrix of a Linear Transformation}


%%%%%%%%%%%%%%%%%%%%%%%%%%%%%%%%%%%%%%%%%%%%%%%%%%%%%%%%%%%%%%%%%%%
% JDR: Students will be asking you until the day of the final what do e_1, e_2,
%   etc. mean...

\begin{frame}
\frametitle{Unit Coordinate Vectors}

\vskip-.3cm

\begin{defn}
  The \textbf{unit coordinate vectors} in $\R^n$ are
  \[ \namedbox{ei}{
    e_1 = \vec{1 0 \vdots, 0 0},\quad e_2 = \vec{0 1 \vdots, 0 0},\quad\ldots,\quad
    e_{n-1} = \vec{0 0 \vdots, 1 0},\quad e_n = \vec{0 0 \vdots, 0 1}.} \]
  \pause
  \begin{tikzpicture}[remember picture, overlay]
    \node[orangebox, fit=(ei)] (ei') {};
    \path (ei'.north) ++(10mm,2.5mm)
      node[anchor=south west, orange, align=left] (expl)
        {This is what $e_1,e_2,\ldots$ mean,\\
          \emph{for the rest of the class}.};
    \draw[->, orange] (expl.west) to[out=180, in=90] ($(ei'.north)+(5mm,0)$);
      
  \end{tikzpicture}
\end{defn}

\pause\hfill
\begin{tikzpicture}[scale=.5, thin border nodes]
  \draw[grid lines] (-2,-2) grid (2,2);
  \draw[->] (-2,0) -- (2,0);
  \draw[->] (0,-2) -- (0,2);
  \draw[vector, seq1] (0,0) to["$e_1$" below=1pt] (1,0);
  \draw[vector, seq2] (0,0) to["$e_2$"] (0,1);
  \node[font=\normalsize, above] at (0,2.1) {in $\R^2$};
  \point at (0,0);
\end{tikzpicture}
\hfill\pause
\begin{tikzpicture}[scale=.5, myxyz, thin border nodes]
  \node[font=\normalsize, above] at (0,0,2.1) {in $\R^3$};
  \useasboundingbox[resetxy] (-2,-2) rectangle (2,2);
  \draw[grid lines, transformxy] (-2,-2) grid (2,2);
  \draw[->] (-2,0,0) -- (2,0,0);
  \draw[->] (0,-2,0) -- (0,2,0);
  \draw[->] (0,0,-2) -- (0,0,2);
  \draw[vector, seq1] (0,0,0) to["$e_1$" below=1pt] (1,0,0);
  \draw[vector, seq2] (0,0,0) to["$e_2$" pos=.8] (0,1,0);
  \draw[vector, seq3] (0,0,0) to["$e_3$"] (0,0,1);
  \point at (0,0,0);
\end{tikzpicture}
\hfill\null

\pause\medskip
\alert{Note:}
if $A$ is an $m\times n$ matrix with columns $v_1,v_2,\ldots,v_n$, then
\textcolor{seq-red}{$Ae_i = v_i$} for $i=1,2,\ldots,n$:
\pause
multiplying a matrix by $e_i$ gives you the $i$th column.

\webonlycmd{\def\A{\mat{1 2 3; 4 5 6; 7 8 9}}\small
\smallskip
\hbox to \linewidth{\hss
$ 
\A\vec{1 0 0} = \vec{1 4 7} \qquad
\A\vec{0 1 0} = \vec{2 5 8} \qquad
\A\vec{0 0 1} = \vec{3 6 9}
$
\hss}
}

\end{frame}


%%%%%%%%%%%%%%%%%%%%%%%%%%%%%%%%%%%%%%%%%%%%%%%%%%%%%%%%%%%%%%%%%%%

\begin{frame}
\frametitle{Linear Transformations are Matrix Transformations}

\alert{Recall:} A matrix $A$ defines a linear transformation $T$ by $T(x) = Ax$.

\pause
\begin{thm}
  Let $T\colon\R^n\to\R^m$ be a linear transformation.  Let
  \[ A = \mat{| | ,, |; T(e_1) T(e_2) \cdots, T(e_n); | | ,, | }. \]
  This is an $\blankuntil{3}{m\times n}$ matrix, 
  \pause 
  and $T$ is the matrix transformation for $A$: $T(x) = Ax$.

  \pause\smallskip
  The matrix $A$ is called the \textbf{standard matrix} for $T$.
\end{thm}

\pause
\begin{bluebox}[Take-Away]{.85\linewidth}\centering
  Linear transformations are the same as matrix transformations.
\end{bluebox}
\note[item]{These are essentially the only transformations we'll deal with in
  this class.}
\note[item]{The added abstraction of linear transformations is useful: sometimes
  a transformation isn't given a priori as a matrix transformation, like in the
  1.8 slides.}

\pause
\hbox to \linewidth{\hfil\alert{Dictionary}\hfil}
\pause\leavevmode
\parbox{.3\linewidth}{\centering
  Linear transformation\\$T\colon\R^n\to\R^m$}%
\;\longsquiggly\;
\hbox to .5\linewidth{$m\times n$ matrix
\small$A = \mat{| | ,, |; T(e_1) T(e_2) \cdots, T(e_n); | | ,, | }$\hss}\\[1mm]
\pause\leavevmode
\hbox to .3\linewidth{\hfil
  $\begin{aligned}&T(x) = Ax\\&T\colon\R^n\to\R^m\end{aligned}$\hfil}%
\;\rlongsquiggly\;
\hbox to .5\linewidth{$m\times n$ matrix $A$\hfil}

\end{frame}


%%%%%%%%%%%%%%%%%%%%%%%%%%%%%%%%%%%%%%%%%%%%%%%%%%%%%%%%%%%%%%%%%%%

\begin{frame}
\frametitle{Linear Transformations are Matrix Transformations}
\framesubtitle{Continued}

\alert{Why is a linear transformation a matrix transformation?}  

\pause\medskip
Suppose for simplicity that $T\colon\R^3\to\R^2$.
\begin{webonly}
\[\begin{split}
  T\vec{x y z} &= T\left( x\vec{1 0 0} + y\vec{0 1 0} + z\vec{0 0 1} \right) \\
  &= T\bigl( xe_1 + ye_2 + ze_3 \bigr) \\
  &= xT(e_1) + yT(e_2) + zT(e_3) \\
  &= \mat{| | |; T(e_1) T(e_2) T(e_3); | | |}\vec{x y z}\\
  &= A\vec{x y z}.
\end{split} \]
\end{webonly}

\end{frame}


%%%%%%%%%%%%%%%%%%%%%%%%%%%%%%%%%%%%%%%%%%%%%%%%%%%%%%%%%%%%%%%%%%%

\begin{frame}
\frametitle{Linear Transformations are Matrix Transformations}
\framesubtitle{Example}

Before, we defined a \textbf{dilation} transformation $T\colon\R^2\to\R^2$ by
$T(x) = 1.5x$.  What is its standard matrix?
\begin{webonly}
\[\left.\begin{aligned}
    T(e_1) &= 1.5e_1 = \vec{1.5 0} \\
    T(e_2) &= 1.5e_2 = \vec{0 1.5}
\end{aligned}\right\}
\implies A = \mat{1.5 0; 0 1.5}.\]
\end{webonly}

\pause
\alert{Check:}
\[ \mat{1.5 0; 0 1.5}\vec{x y} = \pause \vec{1.5x 1.5y}
= \pause 1.5\vec{x y} = T\vec{x y}. \]

\end{frame}


%%%%%%%%%%%%%%%%%%%%%%%%%%%%%%%%%%%%%%%%%%%%%%%%%%%%%%%%%%%%%%%%%%%

\begin{frame}
\frametitle{Linear Transformations are Matrix Transformations}
\framesubtitle{Example}

\vskip -3mm
\begin{ques}
  What is the matrix for the linear transformation
  $T\colon\R^2\to\R^2$ defined by
  \[ T(x) = x \text{ rotated counterclockwise by an angle } \theta? \]
  \pause
  (Check linearity\ldots)
\end{ques}

\begin{webonly}
\begin{center}
\begin{tikzpicture}[scale=2, thin border nodes]
  \clip (-.5,-.2) rectangle (3.5,1.3);

  \draw[help lines] (0,0) circle[radius=1];
  \draw[->] (-1.3,0) -- (1.3,0);
  \draw[->] (0,-1.3) -- (0,1.3);

  \tikzmath{
    coordinate \c;
    \x = cos(37);
    \y = sin(37);
    \c = (\x,\y);
  }

  \coordinate (Te) at (\c);
  \coordinate (cos) at (\cx,0);
  \point (o) at (0,0);
  \pic[draw] {right angle=(o)--(cos)--(Te)};
  \draw[thick vector] (0,0) -- (1,0) 
    node[pos=.9, below=1mm] {$e_1$};
  \draw[thick vector] (0,0) -- (Te)
    node[pos=.7, above left] {$T(e_1)$};
  \draw (.3,0) arc[start angle=0, delta angle=37, radius=.3cm]
    node[midway, right=.2mm, yshift=.5mm] {$\theta$};
  \draw[seq1,thick] (Te) -- (\cx,0) node[midway,right] {$\sin(\theta)$};
  \draw[seq2,thick] (0,0) -- (\cx,0) node[midway,below=1mm] {$\cos(\theta)$};

  \begin{scope}[xshift=3cm]
    \draw[help lines] (0,0) circle[radius=1];
    \draw[->] (-1.3,0) -- (1.3,0);
    \draw[->] (0,-1.3) -- (0,1.3);
  
    \tikzmath{
      coordinate \c;
      \x = -sin(37);
      \y = cos(37);
      \c = (\x,\y);
    }
  
    \coordinate (Te) at (\c);
    \coordinate (sin) at (\cx, 0);
    \point (o) at (0,0);
    \pic[draw] {right angle=(o)--(sin)--(Te)};
    \draw[thick vector] (0,0) -- (0,1) 
      node[pos=.6, right] {$e_2$};
    \draw[thick vector] (0,0) -- (Te)
      node[pos=1.15, inner sep=1pt] {$T(e_2)$};
    \draw (0,.3) arc[start angle=90, delta angle=37, radius=.3cm]
      node[midway, above=.21mm] {$\theta$};
    \draw[seq2,thick] (Te) -- (\cx,0) node[midway,left] {$\cos(\theta)$};
    \draw[seq1,thick] (0,0) -- (\cx,0) node[midway,below=1mm] {$\sin(\theta)$};
      
  \end{scope}

\end{tikzpicture}
\end{center}

\hbox to \linewidth{\hss
$\left.\begin{aligned}
    T(e_1) &= \vec{\color{seq2}\cos(\theta) \color{seq1}\sin(\theta)} \\
    T(e_2) &= \vec{-\color{seq1}\sin(\theta) \color{seq2}\cos(\theta)}
\end{aligned}\right\}
\implies A = 
\mat{\color{seq2}\cos(\theta) -\color{seq1}\sin(\theta);
     \color{seq1}\sin(\theta) \color{seq2}\cos(\theta)}
\quad
\mat{\theta=90^\circ\implies; A=\mat{0 -1; 1 0}; \text{(from before)}}
$\hss}
\end{webonly}

\end{frame}


%%%%%%%%%%%%%%%%%%%%%%%%%%%%%%%%%%%%%%%%%%%%%%%%%%%%%%%%%%%%%%%%%%%

\def\drawarrow#1{
\begin{tikzpicture}[myxyz, scale=0.6, thin border nodes, baseline]
  \path[clip, resetxy] (-2,-2) rectangle (2,2);

  \begin{scope}[transformxy]
    \fill[blue, opacity=.05] (-1.5,-1.5) rectangle (1.5,1.5);
    \draw[step=1cm, help lines, blue!40!black]
      (-1.5, -1.5) grid (1.5, 1.5);
  \end{scope}

  \node[blue!40!black] at (-.9,1,-.5) {$xy$};

  \begin{scope}[transformyz]
    \fill[green, opacity=.05] (-1.5,-1.5) rectangle (1.5,1.5);
    \draw[step=1cm, help lines, green!40!black]
      (-1.5, -1.5) grid (1.5, 1.5);
  \end{scope}

  \node[green!40!black] at (.75,-1,1) {$yz$};

  #1

\end{tikzpicture}
}

\begin{frame}
\frametitle{Linear Transformations are Matrix Transformations}
\framesubtitle{Example}

\vskip-3mm
\begin{ques}
  What is the matrix for the linear transformation $T\colon\R^3\to\R^3$ that
  reflects through the $xy$-plane and then projects onto the $yz$-plane?
\end{ques}

\pause\vfill

\hbox to \linewidth{\hss
\drawarrow{\draw[vector] (0,0,0) -- (1,0,0) node[above] {$e_1$};}
\longsquiggly[reflect $xy$]
\pause
\drawarrow{\draw[vector] (0,0,0) -- (1,0,0);}
\longsquiggly[project $yz$]
\pause
\drawarrow{\point at (0,0,0);}
\hss}

\pause\medskip

\[ T(e_1) = \vec{0 0 0}. \]

\vfill

\end{frame}

\begin{frame}
\frametitle{Linear Transformations are Matrix Transformations}
\framesubtitle{Example, continued}

\vskip-3mm
\begin{ques}
  What is the matrix for the linear transformation $T\colon\R^3\to\R^3$ that
  reflects through the $xy$-plane and then projects onto the $yz$-plane?
\end{ques}

\vfill

\hbox to \linewidth{\hss
\drawarrow{\draw[vector] (0,0,0) -- (0,1,0) node[below right=-.5mm] {$e_2$};}
\longsquiggly[reflect $xy$]
\pause
\drawarrow{\draw[vector] (0,0,0) -- (0,1,0);}
\longsquiggly[project $yz$]
\pause
\drawarrow{\draw[vector] (0,0,0) -- (0,1,0);}
\hss}

\pause\medskip
\[ T(e_2) = e_2 = \vec{0 1 0}. \]

\vfill

\end{frame}


\begin{frame}
\frametitle{Linear Transformations are Matrix Transformations}
\framesubtitle{Example, continued}

\vskip-3mm
\begin{ques}
  What is the matrix for the linear transformation $T\colon\R^3\to\R^3$ that
  reflects through the $xy$-plane and then projects onto the $yz$-plane?
\end{ques}

\vfill

\hbox to \linewidth{\hss
\drawarrow{\draw[vector] (0,0,0) -- (0,0,1) node[right] {$e_3$};}
\longsquiggly[reflect $xy$]
\pause
\drawarrow{\draw[vector] (0,0,0) -- (0,0,-1);}
\longsquiggly[project $yz$]
\pause
\drawarrow{\draw[vector] (0,0,0) -- (0,0,-1);}
\hss}

\pause\medskip
\[ T(e_3) = \vec{0 0 -1}. \]

\vfill

\end{frame}


\begin{frame}
\frametitle{Linear Transformations are Matrix Transformations}
\framesubtitle{Example, continued}

\vskip-3mm
\begin{ques}
  What is the matrix for the linear transformation $T\colon\R^3\to\R^3$ that
  reflects through the $xy$-plane and then projects onto the $yz$-plane?
\end{ques}

\vfill

\[\left.\begin{aligned}
  T(e_1) &= \vec{0 0 0} \\
  T(e_2) &= \vec{0 1 0} \\
  T(e_1) &= \vec{0 0 -1} \\
\end{aligned}\right\}
\implies A = \pause\mat{0 0 0; 0 1 0; 0 0 -1}.
\]

\vfill

\end{frame}


%%%%%%%%%%%%%%%%%%%%%%%%%%%%%%%%%%%%%%%%%%%%%%%%%%%%%%%%%%%%%%%%%%%

\begin{frame}
\frametitle{Other Geometric Transformations}

\vfill

\begin{bluebox}{.75\textwidth}
  There is a long list of geometric transformations of $\R^2$ in \S1.9 of Lay.
  (Reflections over the diagonal, contractions and expansions along different
  axes, shears, projections, \ldots)
  Please look them over.
\end{bluebox}

\vfill

\end{frame}


%%%%%%%%%%%%%%%%%%%%%%%%%%%%%%%%%%%%%%%%%%%%%%%%%%%%%%%%%%%%%%%%%%%

\begin{frame}
\frametitle{Onto Transformations}

\vskip -.3cm
\begin{defn}
  A transformation $T\colon\R^n\to\R^m$ is \textbf{onto} (or
  \textbf{surjective}) if the range of $T$ is equal to $\R^m$ (its codomain).  
  \pause
  In other words, each $b$
  in $\R^m$ is the image of \emph{at least one} $x$ in $\R^n$:
  \pause
  every possible output has an input.
  \pause
  Note that \emph{not} onto means 
  \pause
  there is some $b$ in $\R^m$ which is not the image of any $x$ in $\R^n$.
\end{defn}
\note{Beware quantifiers}

\pause

\begin{center}
\begin{tikzpicture}[scale=.4, thin border nodes]
  \draw (-3,-3) rectangle (3,3);

  \draw[->, very thin, myxyz] (-2,0,0) -- (2,0,0);
  \draw[->, very thin, myxyz] (0,-2,0) -- (0,2,0);
  \draw[->, very thin, myxyz] (0,0,-2) -- (0,0,2);

  \node (source) at (0,-3.6) {$\R^n$};
  \point["$x$" above left] (x) at (1,2);

  \begin{scope}[xshift=12cm, every label/.append style={fill=seq4!10}]
    \fill[seq4!10] (-3,-3) rectangle (3,3);
    \draw[grid lines] (-3,-3) grid (3,3);
    \point["$T(x)$" above right] (Tx) at (1,-1);
    \node[fill=seq4!10, text=seq4]
      at (0,2) {$\range(T)$};

    \node (target) at (0,-3.6) {$\R^m =$ codomain};
  \end{scope}

  \draw[->] (source) -- (target) node[midway,above=1pt] {$T$};
  \draw[|->, shorten=.35mm] (x) to[out=0, in=180] (Tx);

  \node[seq-blue] at (6,2.5) {onto};

\end{tikzpicture}

\bigskip\pause

\begin{tikzpicture}[scale=.4, thin border nodes]
  \draw (-3,-3) rectangle (3,3);

  \draw[->, very thin, myxyz] (-2,0,0) -- (2,0,0);
  \draw[->, very thin, myxyz] (0,-2,0) -- (0,2,0);
  \draw[->, very thin, myxyz] (0,0,-2) -- (0,0,2);

  \node (source) at (0,-3.6) {$\R^n$};
  \point["$x$" above left] (x) at (1,-2);

  \begin{scope}[xshift=12cm]
    \draw[grid lines] (-3,-3) grid (3,3);
    \draw[seq4] (-3,-1.5) -- (3,1.5); 
   \point["$T(x)$" below right] (Tx) at (1,.5);
    \node[fill=white, text=seq4]
      at (-1,-1.8) {$\range(T)$};

    \node (target) at (0,-3.6) {$\R^m =$ codomain};
  \end{scope}

  \draw[->] (source) -- (target) node[midway,above=1pt] {$T$};
  \draw[|->, shorten=.35mm] (x) to[out=0, in=150] (Tx);

  \node[seq-blue] at (6,2.5) {not onto};

\end{tikzpicture}
\end{center}

\end{frame}


%%%%%%%%%%%%%%%%%%%%%%%%%%%%%%%%%%%%%%%%%%%%%%%%%%%%%%%%%%%%%%%%%%%

\begin{frame}
\frametitle{Characterization of Onto Transformations}

\vskip -3mm
\begin{thm}
  Let $T\colon\R^n\to\R^m$ be a linear transformation with matrix $A$.  Then the
  following are equivalent:
  \begin{itemize}
  \item $T$ is onto
    \pause
  \item $T(x) = b$ has a solution for every $b$ in $\R^m$
    \pause
  \item $Ax = b$ is consistent for every $b$ in $\R^m$
    \pause
  \item The columns of $A$ span $\R^m$
    \pause
  \item $A$ has a pivot in every \blankuntil{6}{row}
  \end{itemize}
\end{thm}

\pause[7]

\begin{ques}
  If $T\colon\R^n\to\R^m$ is onto, what can we say about the relative sizes of
  $n$ and $m$?
\end{ques}

\pause
\alert{Answer:} $T$ corresponds to an \blankuntil{9}{$m\times n$} matrix $A$.
\pause[10]
In order for $A$ to have a pivot in every row, it must have
\emph{at least as many} columns as rows: $m\leq n$.
\pause
\[ \mat{
\color{red}1   0   \star,   0   \star ;
0   \color{red}1   \star , 0   \star ;
0   0   0   \color{red}1   \star 
} \]
\pause
For instance, $\R^2$ is ``too small'' to map \emph{onto\/} $\R^3$.

\end{frame}


%%%%%%%%%%%%%%%%%%%%%%%%%%%%%%%%%%%%%%%%%%%%%%%%%%%%%%%%%%%%%%%%%%%

\begin{frame}
\frametitle{One-to-one Transformations}

\vskip -3mm
\begin{defn}
  A transformation $T\colon\R^n\to\R^m$ is \textbf{one-to-one} (or
  \textbf{into}, or \textbf{injective}) if different vectors in $\R^n$ map to
  different vectors in $\R^m$.
  \pause
  In other words, each $b$ in $\R^m$ is the image of
  \emph{at most one} $x$ in $\R^n$:
  \pause
  different inputs have different outputs.
  \pause
  Note that \emph{not} one-to-one means
  \pause
  different vectors in $\R^n$ have the same image.
\end{defn}

\pause

\begin{center}
\begin{tikzpicture}[scale=0.4, thin border nodes]
  \draw[grid lines] (-3,-3) grid (3,3);

  \node (A) at (0,-3.6) {$\R^n$};
  \node (B) at (12,-3.6) {$\R^m$};
  \draw[->] (A) -- (B) node[midway,above=1pt] {$T$};

  \point["$x$" left] (x) at (1,1.5);
  \point["$y$" left] (y) at (-1,0);
  \point["$z$" left] (z) at (0,-2);

  \begin{scope}[myxyz, xshift=12cm]
    \path[clip, resetxy] (-3,-3) rectangle (3,3);

    \def\v{(2,-1,1)}
    \def\w{(1,0,-1)}
  
    \node[coordinate] (X) at \v {};
    \node[coordinate] (Y) at \w {};

    \draw[very thin] (-2,0,0) -- (0,0,0);
    \draw[very thin] (0,-2,0) -- (0,0,0);
    \draw[very thin] (0,0,-2) -- (0,0,0);

    \begin{scope}[x=(X), y=(Y), transformxy,
        every label/.style={font=\tiny,fill=none,inner sep=.5pt}]
      \fill[seq4!10, nearly opaque] (-1,-1) rectangle (1,1);
      \draw[step=.5cm, seq4!30, very thin] (-1,-1) grid (1,1);
      \point["$T(x)$" above] (Tx) at (.5,-.5);
      \point["$T(y)$" above] (Ty) at (.5,.5);
      \point["$T(z)$" above] (Tz) at (-.5,.75);
      \node[coordinate,
          pin={[pin edge={very thin,-},
            pin distance=3mm,anchor=north,text=seq4]-70:range}]
        at (0.1,1) {};
    \end{scope}

    \draw[->, very thin] (0,0,0) -- (2,0,0);
    \draw[->, very thin] (0,0,0) -- (0,2,0);
    \draw[->, very thin] (0,0,0) -- (0,0,2);

    \draw[resetxy] (-3,-3) rectangle (3,3);

  \end{scope}

  \begin{scope}[arrows={|->}, shorten=.35mm, every to/.style={out=0, in=180}]
    \draw (x) to (Tx);
    \draw (y) to (Ty);
    \draw (z) to (Tz);
  \end{scope}

  \node[seq-blue] at (6,2.5) {one-to-one};

\end{tikzpicture}

\bigskip\pause

\begin{tikzpicture}[scale=0.4, thin border nodes]
  \draw[grid lines] (-3,-3) grid (3,3);

  \node (A) at (0,-3.6) {$\R^n$};
  \node (B) at (12,-3.6) {$\R^m$};
  \draw[->] (A) -- (B) node[midway,above=1pt] {$T$};

  \point["$x$" left] (x) at (1,1.5);
  \point["$y$" left] (y) at (-1,0);
  \point["$z$" left] (z) at (0,-2);

  \begin{scope}[myxyz, xshift=12cm,
      every label/.style={font=\tiny, fill=none, inner sep=.5pt}]
    \path[clip, resetxy] (-3,-3) rectangle (3,3);

    \def\v{(1,-1,2)}
    \node[coordinate] (V) at \v {};
  
    \draw[very thin] (-2,0,0) -- (0,0,0);
    \draw[very thin] (0,-2,0) -- (0,0,0);
    \draw[very thin] (0,0,-2) -- (0,0,0);

    \draw[seq4] ($-4*(V)$) -- ($4*(V)$);
    \point[label={[xshift=.25cm,fill=white]above:{$T(x)=T(y)$}}]
      (Tx) at ($.5*(V)$);
    \point["$T(z)$" right] (Tz) at ($-1*(V)$);
    \node[coordinate,
        pin={[pin edge={very thin,-},
          pin distance=8mm,anchor=north,text=seq4]-30:range}]
        at ($-.2*(V)$) {};

    \draw[->, very thin] (0,0,0) -- (2,0,0);
    \draw[->, very thin] (0,0,0) -- (0,2,0);
    \draw[->, very thin] (0,0,0) -- (0,0,2);

    \draw[resetxy] (-3,-3) rectangle (3,3);

  \end{scope}

  \begin{scope}[arrows={|->}, shorten=.35mm, every to/.style={out=0, in=180}]
    \draw (x) to (Tx);
    \draw[|-] (y) to (Tx);
    \draw (z) to (Tz);
  \end{scope}

  \node[seq-blue] at (6,2.5) {not one-to-one};

\end{tikzpicture}

\end{center}

\end{frame}


%%%%%%%%%%%%%%%%%%%%%%%%%%%%%%%%%%%%%%%%%%%%%%%%%%%%%%%%%%%%%%%%%%%

\begin{frame}
\frametitle{Characterization of One-to-One Transformations}

\vskip -3mm
\begin{thm}
  Let $T\colon\R^n\to\R^m$ be a linear transformation with matrix $A$.  Then the
  following are equivalent:
  \begin{itemize}
  \item $T$ is one-to-one
    \pause
  \item $T(x) = b$ has one or zero solutions for every $b$ in $\R^m$
    \pause
  \item $Ax = b$ has a unique solution or is inconsistent for every $b$ in $\R^m$
    \pause
  \item $Ax = 0$ has a unique solution
    \pause
  \item The columns of $A$ are linearly independent
    \pause
  \item $A$ has a pivot in every \blankuntil{7}{column}.
  \end{itemize}
\end{thm}

\pause[8]

\begin{ques}
  If $T\colon\R^n\to\R^m$ is one-to-one, what can we say about the relative sizes of
  $n$ and $m$?
\end{ques}

\pause
\alert{Answer:} $T$ corresponds to an $m\times n$ matrix $A$.
\pause
In order for $A$ to have a pivot in every column, it must have
\emph{at least as many} rows as columns: $n\leq m$.

\pause
\hbox to \linewidth{\hss
\begin{minipage}{.4\textwidth}
\[ \mat{
\color{red}1   0   0 ;
0   \color{red}1  0 ;
0 0  \color{red}1 ;
0 0 0
} \]
\end{minipage}
\pause
\begin{minipage}{.6\textwidth}
For instance, $\R^3$ is ``too big'' to map \emph{into\/} $\R^2$.
\end{minipage}\hss}

\end{frame}


%%% Local Variables:
%%% TeX-master: "../slides"
%%% End:
