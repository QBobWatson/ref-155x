
\usetikzlibrary{decorations.pathreplacing}

\titleframe{Section 3.2}{Properties of Determinants}


%%%%%%%%%%%%%%%%%%%%%%%%%%%%%%%%%%%%%%%%%%%%%%%%%%%%%%%%%%%%%%%%%%%

\begin{frame}
\frametitle{Plan for Today}

Last time, we gave a recursive formula for determinants in terms of cofactor
expansions.

\pause\bigskip
\alert{Plan for today:}
\begin{itemize}
\item An abstract definition of the determinant in terms of its properties.
\pause
\item Computing determinants using row operations.
\pause
\item Determinants and products: $\det(AB) = \det(A)\det(B)$.
\pause
\item Determinants and volumes.
\pause
\item Determinants and linear transformations.
\end{itemize}

\pause\bigskip
The determinant is one of the most amazing functions ever devised.  Today is
about beginning to understand why.

\end{frame}


%%%%%%%%%%%%%%%%%%%%%%%%%%%%%%%%%%%%%%%%%%%%%%%%%%%%%%%%%%%%%%%%%%%

\begin{frame}
\frametitle{The Determinant is a Function}

We can think of the determinant as a function of the entries of a matrix:
\[ \det\mat{a_{11} a_{12} a_{13}; a_{21} a_{22} a_{23}; a_{31} a_{32} a_{33}}
= \begin{aligned}
& a_{11}a_{22}a_{33} + a_{12}a_{23}a_{31} + a_{13}a_{21}a_{32} \\
&\quad -a_{13}a_{22}a_{31} - a_{11}a_{23}a_{32} - a_{12}a_{21}a_{33}.
\end{aligned} \]
\pause
The formula for the determinant of an $n\times n$ matrix has $n!$ terms.
\pause
So the determinant of a $10\times 10$ matrix has $3{,}628{,}800$ terms!

\pause\bigskip
When mathematicians encounter a function whose formula is too difficult to write
down, we try to \emph{characterize} it in terms of its properties.

\pause\bigskip
The determinant function is characterized by how it is changed by row
operations.

\end{frame}


%%%%%%%%%%%%%%%%%%%%%%%%%%%%%%%%%%%%%%%%%%%%%%%%%%%%%%%%%%%%%%%%%%%

\begin{frame}
\frametitle{Defining the Determinant in Terms of its Properties}

\vskip-3mm
\begin{defn}
  The \textbf{determinant} is a function
  \[ \det\colon \{\text{square matrices}\} \To \R \]
  with the following \textbf{defining properties}:
  \pause
  \begin{enumerate}
  \item $\det(I_n) = 1$
    \pause
  \item If we do a row replacement on a matrix, the determinant does not change.
    \pause
  \item If we swap two rows of a matrix, the determinant scales by $-1$.
    \pause
  \item If we scale a row of a matrix by $k$, the determinant scales by $k$.
  \end{enumerate}
\end{defn}

\pause\medskip
Why would we think of these properties?
\pause
This is how volumes work!
\pause
\begin{enumerate}
\item The volume of the unit cube is $1$.
\pause
\item Volumes don't change under a shear.
\pause
\item Volume of a mirror image is negative of the volume?
\pause
\item If you scale one coordinate by $k$, the volume is multiplied by $k$.

\end{enumerate}

\end{frame}


%%%%%%%%%%%%%%%%%%%%%%%%%%%%%%%%%%%%%%%%%%%%%%%%%%%%%%%%%%%%%%%%%%%

\begin{frame}
\frametitle{Properties of the Determinant}
\framesubtitle{$2\times 2$ matrix}

\begin{webonly}
\begin{minipage}[c]{0.6\linewidth}
  \[ \det \mat{1 -2; 0 3 } = 3 \]
\end{minipage}\hfill
\begin{tikzpicture}[scale=.5, baseline=.75cm]
  \draw[help lines, black!25] (-3,-1) grid (3,4);
  \draw[vector] (0,0) -- (1,0);
  \draw[vector] (0,0) -- (-2,3);
  \filldraw[fill=seq-orange, fill opacity=.2, very thin]
    (0,0) -- (1,0) -- (-1, 3) -- (-2, 3) -- cycle;
  \node[anchor=south west, font=\small,
    inner sep=1pt, fill=white] at (-.25,2.1) {volume $= 3$};
  \point at (0,0);
\end{tikzpicture}

%\pause
\begin{minipage}[c]{0.6\linewidth}
  Scale: $R_2 = \frac 13 R_2$
  \[ \det\mat{1 -2; 0 1 } = 1 \]
\end{minipage}\hfill
\begin{tikzpicture}[scale=.5, baseline=.75cm]
  \draw[help lines, black!25] (-3,-1) grid (3,4);
  \draw[vector] (0,0) -- (1,0);
  \draw[vector] (0,0) -- (-2,1);
  \filldraw[fill=seq-orange, fill opacity=.2, very thin]
    (0,0) -- (1,0) -- (-1,1) -- (-2,1) -- cycle;
  \node[anchor=south west, font=\small,
    inner sep=1pt, fill=white] at (-.25,2.1) {volume $= 1$};
  \point at (0,0);
\end{tikzpicture}


%\pause
\begin{minipage}[c]{0.6\linewidth}
  Row replacement: $R_1 = R_1 + 2R_2$
  \[ \det\mat{1 -2; 0 1 } = 1 \]
  (This is a shear by the elementary matrix $\begin{psmm} 1&2\\0&1 \end{psmm}$.)
\end{minipage}\hfill
\begin{tikzpicture}[scale=.5, baseline=.75cm]
  \draw[help lines, black!25] (-3,-1) grid (3,4);
  \draw[vector] (0,0) -- (1,0);
  \draw[vector] (0,0) -- (0,1);
  \filldraw[fill=seq-orange, fill opacity=.2, very thin]
    (0,0) -- (1,0) -- (1,1) -- (0,1) -- cycle;
  \node[anchor=south west, font=\small,
    inner sep=1pt, fill=white] at (-1.25,2.1) {volume still $= 1$};
  \point at (0,0);
\end{tikzpicture}\\
\end{webonly}

\end{frame}


%%%%%%%%%%%%%%%%%%%%%%%%%%%%%%%%%%%%%%%%%%%%%%%%%%%%%%%%%%%%%%%%%%%

\begin{frame}
\frametitle{Properties of the Determinant}
\framesubtitle{Elementary matrices}

Since an elementary matrix differs from the identity matrix by one row
operation, and since $\det(I_n) = 1$, it is easy to calulate the determinant of
an elementary matrix:

\pause\bigskip
\[\begin{aligned}
  \det\mat{1 0 8; 0 1 0; 0 0 1} &=
  \webonlycmd{\hbox to 6cm{$\det(I_n) = 1$\hfil (properties 1 and 2)}} \\
  \det\mat{0 0 1; 0 1 0; 1 0 0} &=
  \webonlycmd{\hbox to 6cm{$-\det(I_n) = -1$\hfil (properties 1 and 3)}} \\
  \det\mat{1 0 0; 0 17 0; 0 0 1} &=
  \webonlycmd{\hbox to 6cm{$17\det(I_n) = 17$\hfil (properties 1 and 4)}}  
\end{aligned}\]

\end{frame}


%%%%%%%%%%%%%%%%%%%%%%%%%%%%%%%%%%%%%%%%%%%%%%%%%%%%%%%%%%%%%%%%%%%

\begin{frame}
\frametitle{Computing the Determinant by Row Reduction}

We can use the properties of the determinant and row reduction to compute the
determinant of any matrix!  
\pause
This means that $\det$ is completely characterized by its defining
properties.
\pause
\[\begin{aligned}
  \qquad\det\mat{0 1 0; 1 0 1; 5 7 -4} &=
  \webonlycmd{\hbox to 7cm{$-\det\mat{1 0 1; 0 1 0; 5 7 -4}$\hfil(property 3)}} \\
  &\webonlycmd{=\hbox to 7cm{$-\det\mat{1 0 1; 0 1 0; 0 7 -9}$\hfil(property 2)}} \\
  &\webonlycmd{=\hbox to 7cm{$-\,\namedbox{mat1}{\det
        \mat{1 0 1; 0 1 0; 0 0 -9}}$\hfil(property 2)}} \\
  &\webonlycmd{=\hbox to 7cm{$(-1)\cdot\namedbox{mat2}{(-9)\det
        \mat{1 0 1; 0 1 0; 0 0 1}}$
      \hfil(property 4)}} \\
  &\webonlycmd{=\hbox to 7cm{$(-1)\cdot(-9)\det\mat{1 0 0; 0 1 0; 0 0 1}$
      \hfil(property 2)}} \\
  &\webonlycmd{=\hbox to 7cm{$9$ \hfil(property 1)}} \\
\end{aligned}\]
\begin{webonly}
\begin{tikzpicture}[remember picture, overlay]
  \node[draw, thick, rounded corners, seq-green, fit=(mat1),
    inner ysep=2pt, inner xsep=1pt] (mat1box) {};
  \node[draw, thick, rounded corners, seq-blue,  fit=(mat2),
    inner ysep=2pt, inner xsep=1pt] (mat2box) {};
  \path let \p1=($.5*(mat1box.center) + .5*(mat2box.center)$),
            \p2=($(current page.west) + (5mm,0)$)
    in (\p1 -| \p2) node[right, align=justify, text width=3.5cm] (expl) {%
      The \textcolor{seq-blue}{second matrix} is obtained from the
      \textcolor{seq-green}{first matrix} by scaling by $-1/9$.  So the
      determinant of the \textcolor{seq-green}{first matrix} is $-9$ times the
      determinant of the \textcolor{seq-blue}{second matrix}.%
    };
  \draw[->, shorten >=1pt, seq-green]
    (expl.east) to[out=0,in=180] (mat1box.200);
  \draw[->, shorten >=1pt, seq-blue]
    (expl.east) to[out=0,in=180] (mat2box.170);
\end{tikzpicture}
\end{webonly}

\end{frame}


%%%%%%%%%%%%%%%%%%%%%%%%%%%%%%%%%%%%%%%%%%%%%%%%%%%%%%%%%%%%%%%%%%%

\begin{frame}
\frametitle{Computing the Determinant by Row Reduction}
\framesubtitle{Saving some work}

The determinant of an upper (or lower)
triangular matrix is the product of the diagonal entries,
\pause
so we can stop row reducing when we get to row echelon form.
\pause\medskip
\[\det\mat{0 1 0; 1 0 1; 5 7 -4} = \cdots
= -\det\mat{1 0 1; 0 1 0; 0 0 -9} = 9.
\]

\vfill
\pause
\begin{bluebox}{.9\linewidth}
  This is almost always the easiest way to compute the determinant of a large,
  complicated matrix, either by hand or by computer.

  \pause\medskip
  (Cofactor expansion is $O(n!)\sim O(n^n\sqrt n)$, row reduction is $O(n^3)$.)
\end{bluebox}
\vfill

\end{frame}


%%%%%%%%%%%%%%%%%%%%%%%%%%%%%%%%%%%%%%%%%%%%%%%%%%%%%%%%%%%%%%%%%%%

\begin{pollframe}

\begin{bluebox}[Poll]{.8\linewidth}
    Suppose that $A$ is a $4\times 4$ matrix satisfying
    \[ Ae_1 = e_2 \quad Ae_2 = e_3 \quad Ae_3 = e_4 \quad Ae_4 = e_1. \]
    What is $\det(A)$?
    \[ \text{A. } {-1} \qquad \text{B. } 0 \qquad \text{C. } 1 \]
\end{bluebox}

\pause\bigskip
These equations tell us the columns of $A$:
\[ A = \mat{0 0 0 1; 1 0 0 0; 0 1 0 0; 0 0 1 0} \]
\pause
You need $3$ row swaps to transform this to the identity matrix.\\[1mm]
\pause
So $\det(A) = (-1)^3 = -1$.

\end{pollframe}


%%%%%%%%%%%%%%%%%%%%%%%%%%%%%%%%%%%%%%%%%%%%%%%%%%%%%%%%%%%%%%%%%%%

\begin{frame}
\frametitle{A Mathematical IOU}

The characterization of the determinant function in terms of its properties is
very useful.  It gives us a fast way to compute determinants, and prove other
properties (later).  But\ldots

\pause\bigskip
The disadvantage of defining a function by its properties instead of a formula
is:
\pause
how do you know such a function exists?
\pause
and if it exists, why is there only one function satisfying those properties?

\pause\bigskip
In our case, we can compute the determinant of a matrix from its defining
properties, so if it exists, it is unique.
\pause
But how do we know that two different row reductions won't give
two different answers for the determinant?

\pause\bigskip
Here is a summary of the magical properties of the determinant.  Prof.\
Margalit's notes (on the website) have very understandable proofs.

\end{frame}


%%%%%%%%%%%%%%%%%%%%%%%%%%%%%%%%%%%%%%%%%%%%%%%%%%%%%%%%%%%%%%%%%%%

\begin{frame}
\frametitle{Magical Properties of the Determinant}

\begin{enumerate}
\item \namedbox{start}{There} is one and only one function
  $\det\colon\{\text{square matrices}\}\to\R$ satisfying the defining
  properties~(1)--(4).
\pause\smallskip
\item $A$ is invertible if and only if $\det(A) \neq 0$.
\pause\smallskip
\item If we row reduce $A$ without row scaling, then
  \[ \det(A) = (-1)^{\text{\#swaps}}
  \bigl( \text{product of diagonal entries in REF} \bigr). \]
\pause\null\vskip -9mm\null
\item The determinant can be computed using any of the $2n$ cofactor expansions.
  (You get the same number every time!)
\pause\smallskip
\item $\color{seq-orange}\det(AB) = \det(A)\det(B)$ \quad and \quad
  $\color{seq-orange}\det(A\inv) = \det(A)\inv$.
\pause\smallskip
\item $\color{seq-orange}\det(A) = \det(A^T)$.
\pause\smallskip
\item $|\det(A)|$ is the volume of the parallelepiped defined by the columns of
  $A$.
\pause\smallskip
\item If $A$ is an $n\times n$ matrix with transformation $T(x)=Ax$, and $S$ is a
  subset of $\R^n$, then the volume of $T(S)$ is $|\det(A)|$ times the volume of
  $S$.  (Even for curvy shapes $S$.)
\pause\smallskip
\item The determinant is multi-linear (we'll talk about this in a few slides\namedbox{end}).
\end{enumerate}
\pause
\begin{tikzpicture}[remember picture, overlay]
  \draw[decorate,decoration={brace, mirror, amplitude=3mm}, thick, red]
    let \p1=($(current page.west) + (1cm,0)$) in
      (start.north -| \p1) -- (end.south -| \p1)
      node[rotate=90, midway, anchor=south, yshift=.2cm]
        {you really have to know these};
\end{tikzpicture}

\end{frame}


%%%%%%%%%%%%%%%%%%%%%%%%%%%%%%%%%%%%%%%%%%%%%%%%%%%%%%%%%%%%%%%%%%%

\begin{frame}
\frametitle{Multiplicativity of the Determinant}

Why is \alert{Property 5} true?
\pause
In Lay, there's a proof using elementary matrices.
\pause
Here's a better one.
\note{Need \emph{some} example of a proof using uniqueness.}

\medskip
\begin{webonly}
Let $B$ be an $n\times n$ matrix.  There are two cases:
\begin{enumerate}
\item If $\det(B) = 0$, then $B$ is not inverible.
  So for any matrix $A$, $BA$ is not invertible. (Otherwise $B\inv =
  A(BA)\inv$.) So 
  \[ \det(BA) = 0 = 0\cdot\det(A) =  \det(B)\det(A). \]

\item If $A$ is invertible, define another function
  \displayskips{3pt}
  \[ f\colon\{\text{$n\times n$ matrices}\}\To\R
  \sptxt{by} f(B) = \frac{\det(BA)}{\det(A)}. \]
  Let's check the defining properties:\\[.5mm]
  \begin{enumerate}
  \item[1.] $f(I_n) = \det(I_nA)/\det(A) = 1$.
    \vskip.5mm
  \item[2--4.] Doing a row operation on $B$ and then multiplying by $A$, does
    the \emph{same row operation} on $BA$.
    This is because a row operation is left-multiplication by an elementary
    matrix $E$, and $(EB)A = E(AB)$.
    Hence $f$ scales like $\det$ with respect to row operations.
  \end{enumerate}
  By uniqueness, $f = \det$, i.e.,
  \[ \det(B) = f(B) = \frac{\det(AB)}{\det(A)}
  \sptxt{so}
  \det(A)\det(B) = \det(AB).
  \]

\end{enumerate}
\end{webonly}

\end{frame}


%%%%%%%%%%%%%%%%%%%%%%%%%%%%%%%%%%%%%%%%%%%%%%%%%%%%%%%%%%%%%%%%%%%

\def\curvy{(-37.3333, 2.6667)
     .. controls (-37.3333, -5.3333) and (-26.6667, -10.6667) .. (-16, -24)
     .. controls (-5.3333, -37.3333) and (5.3333, -58.6667) .. (18.6667, -48)
     .. controls (32, -37.3333) and (48, 5.3333) .. (45.3333, 18.6667)
     .. controls (42.6667, 32) and (21.3333, 16) .. (8, 13.3333)
     .. controls (-5.3333, 10.6667) and (-10.6667, 21.3333) .. (-18.6667, 21.3333)
     .. controls (-26.6667, 21.3333) and (-37.3333, 10.6667) .. cycle}

\begin{frame}
\frametitle{Determinants and Linear Transformations}

Why is \alert{Property 8} true?
\pause
For instance, if $S$ is the unit cube, then $T(S)$ is the parallelepiped defined
by the columns of $A$, since the columns of $A$ are
$T(e_1),T(e_2),\ldots,T(e_n)$.
\pause
In this case, Property~8 is the same as Property~7.
\pause

\begin{center}
\begin{tikzpicture}[scale=.7, thin border nodes]
  \draw[help lines, black!25] (-2,-2) grid (2,2);
  \draw[vector] (0,0) to["$e_1$\strut"'] (1,0);
  \draw[vector] (0,0) to["$e_2$"] (0,1);
  \filldraw[fill=seq-orange, fill opacity=.2, very thin]
    (0,0) -- (1,0) -- (1,1) -- (0,1) -- cycle;
  \node at (.5,.5) {$S$};
  \node[fill=white] at (0,-1) {$\vol(S)=1$};
  \point at (0,0);

  \draw[->, thick] (2.1,.5) to[bend left, "$T$" above=1mm]  
     node[align=center, below=2mm] {$A = \mat{1 1; -1 1}$\\[1mm]$\det(A)=2$}
     (10-2.1,.5);

  \begin{scope}[xshift=10cm]
    \draw[help lines, black!25] (-2,-2) grid (2,2);
    \draw[vector] (0,0) to["$T(e_1)$"'] (1,-1);
    \draw[vector] (0,0) to["$T(e_2)$"] (1,1);
    \filldraw[fill=seq-orange, fill opacity=.2, very thin]
      (0,0) -- (1,-1) -- (2,0) -- (1,1) -- cycle;
    \node at (1,0) {$T(S)$};
    \node[fill=white,anchor=center] at (0,-1.5) {$\vol(T(S)) = 2$};
    \point at (0,0);
  \end{scope}
\end{tikzpicture}
\end{center}

\pause
For curvy shapes, you break $S$ up into a bunch of tiny cubes.  Each one is
scaled by $|\det(A)|$; then you use \emph{calculus} to reduce to the previous
situation! 
\begin{center}
\begin{tikzpicture}[scale=.7, thin border nodes]
  \draw[help lines, black!25] (-2,-2) rectangle (2,2);
  \begin{scope}[ipe import, fill=seq-orange, fill opacity=.2, thin,
      x=.7bp, y=.7bp]
    \expandafter\filldraw\curvy;
    \expandafter\clip\curvy;
    \draw[very thin, step=4] (-60,-60) grid (60,60);
  \end{scope}

  \node at (0,.7) {$S$};
  \point at (0,0);

  \draw[->, thick] (2.1,0) to[bend left, "$T$" above=1mm] 
    node[below=5mm] {$\vol(T(S)) = 2\vol(S)$}
    (10-2.1,0);

  \begin{scope}[xshift=10cm]
    \draw[help lines, black!25] (-2,-2) rectangle (2,2);
    \begin{scope}[ipe import, cm={1,-1,1,1,(0,0)}, x=.7bp, y=.7bp,
        fill=seq-orange, fill opacity=.2, thin]
      \expandafter\filldraw\curvy;
      \expandafter\clip\curvy;
      \draw[very thin, step=4] (-60,-60) grid (60,60);
    \end{scope}

    \node at (.8,.5) {$T(S)$};
    \point at (0,0);
  \end{scope}

\end{tikzpicture}
\end{center}

\end{frame}


%%%%%%%%%%%%%%%%%%%%%%%%%%%%%%%%%%%%%%%%%%%%%%%%%%%%%%%%%%%%%%%%%%%

\begin{frame}
\frametitle{Multi-Linearity of the Determinant}

We can also think of $\det$ as a function of the columns (or the rows) of an
$n\times n$ matrix:
\[ \det\colon\underbrace{\R^n\times\R^n\times\cdots\times\R^n}_{\text{$n$ times}}
\To\R \]
\[ \det(v_1,v_2,\ldots,v_n) = \det\mat{| | ,, |; v_1 v_2 \cdots, v_n; | | ,, |}. \]

\pause
\alert{Property 9} says that for any $i$ and any vectors $v_1,v_2,\ldots,v_n$
and $v_i'$ and any scalar $c$,
\[\begin{split} \det(v_1,\ldots,v_i+v_i',\ldots,v_n) &=
\det(v_1,\ldots,v_i,\ldots,v_n) + \det(v_1,\ldots,v_i',\ldots,v_n) \\
\det(v_1,\ldots,cv_i,\ldots,v_n) &= c\det(v_1,\ldots,v_i,\ldots,v_n).
\end{split}\]
\pause
In other words, scaling one column (or row) by $c$ scales $\det$ by $c$ (which
we already knew), 
\pause
and if column $i$ is a sum of two vectors $v_i,v_i'$, then the
determinant is the sum of two determinants, one with $v_i$ in column $i$, and
one with $v_i'$ in column $i$.  
\pause
\emph{\color{red}This only works one column at a time.}

\pause\bigskip
\alert{Proof:} just expand cofactors along column $i$.

\end{frame}


%%% Local Variables:
%%% TeX-master: "../slides"
%%% End:
