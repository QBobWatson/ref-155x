
\usetikzlibrary{decorations.pathreplacing,math}

\titleframe{Section 6.5}{Least Squares Problems}


%%%%%%%%%%%%%%%%%%%%%%%%%%%%%%%%%%%%%%%%%%%%%%%%%%%%%%%%%%%%%%%%%%%

\begin{frame}
\frametitle{Motivation}

We now are in a position to solve the motivating problem of this third part of
the course:

\pause\medskip
\begin{bluebox}[Problem]{.7\linewidth}
  Suppose that $Ax=b$ does not have a solution.  What is the best possible
  approximate solution?
\end{bluebox}

\pause\medskip
To say $Ax=b$ does not have a solution means that $b$ is not in $\Col A$.

\pause\medskip
The closest possible $\hat b$ for which $Ax = \hat b$ does have a solution is
\pause
$\hat b = \proj_{\Col A}(b)$.

\pause\medskip
Then $A\hat x = \hat b$ is a consistent equation.

\pause\medskip
A solution $\hat x$ to $A\hat x = \hat b$ is a \textbf{least squares solution.}

\end{frame}


%%%%%%%%%%%%%%%%%%%%%%%%%%%%%%%%%%%%%%%%%%%%%%%%%%%%%%%%%%%%%%%%%%%

\begin{frame}
\frametitle{Least Squares Solutions}

Let $A$ be an $m\times n$ matrix.

\begin{defn}
  A \textbf{least squares solution} to $Ax=b$ is a vector $\hat x$ in $\R^n$
  such that
  \[ \|b-A\hat x\|\leq \|b-Ax\| \]
  for all $x$ in $\R^n$.
\end{defn}

\begin{center}
\begin{tikzpicture}[myxyz, scale=.6, thin border nodes]
  \coordinate (u) at (1,0,0);
  \coordinate (v) at (0,1.1,-.2);
  \coordinate (uxv) at (0,.2,1.1);
  \coordinate (x) at ($-1.1*(u)+(v)+1.5*(uxv)$);
  \begin{scope}[x=(u),y=(v),transformxy]
    \fill[seq-violet!30] (-2,-2) rectangle (2,2);
    \draw[seq-violet, help lines] (-2,-2) grid (2,2);
    \node[seq-violet] at (2.8,1) {$\Col A$};
    \point[seq-brown, "$Ax$" text=seq-brown] at (1,-1);
    \point[seq-brown, "$Ax$" text=seq-brown] at (-1,0);
    \point[seq-brown, "$Ax$" text=seq-brown] at (1,1);
  \end{scope}
  \point[seq-blue, pin={[
      text=seq-blue,
      pin edge={<-,very thin, seq-blue, shorten <=1pt},
      pin distance=.6cm]
      below:$A\hat x = \hat b = \proj_{\Col A}(b)$}]
    (y) at ($-1.1*(u)+1*(v)$);
  \coordinate (yu) at ($(y)+(u)$);
  \coordinate (yv) at ($(y)+(v)$);
  \pic[draw, right angle len=1.5mm] {right angle=(x)--(y)--(yu)};
  \pic[draw, right angle len=1.5mm] {right angle=(x)--(y)--(yv)};
  \point[seq-red, "$b$" {above,text=seq-red}] (xx) at (x);
  \draw[vector, thin] (y) -- node[auto,pos=.6] {$b-A\hat x$} (xx);
  \point at (0,0,0);

  \begin{scope}[xshift=-6.5cm,overlay,align=center]
    \node<2-> {Note that $b-A\hat x$\\is in $(\Col A)^\perp$.};
  \end{scope}
\end{tikzpicture}
\end{center}

\pause[3]
In other words, a least squares solution $\hat x$ solves $Ax=b$
\emph{as closely as possible}.

\pause\medskip
Equivalently, a least squares solution to $Ax=b$ is a vector $\hat x$ in $\R^n$
such that 
\[ A\hat x = \hat b = \proj_{\Col A}(b). \]
\pause
This is because $\hat b$ is the closest vector to $b$ such that
$A\hat x=\hat b$ is consistent.

\end{frame}


%%%%%%%%%%%%%%%%%%%%%%%%%%%%%%%%%%%%%%%%%%%%%%%%%%%%%%%%%%%%%%%%%%%

\begin{frame}
\frametitle{Least Squares Solutions}
\framesubtitle{Computation}

\vskip-3mm
\begin{thm}
  The least squares solutions to $Ax=b$ are the solutions to 
  \displayskips{3pt}
  \[ (A^TA)\hat x = A^Tb. \]
\end{thm}

\pause
This is just another $Ax=b$ problem, but with a \emph{square} matrix $A^TA$!
\pause\\
Note we compute $\hat x$ directly, without computing $\hat b$ first.

\pause\medskip
\alert{Why is this true?}
\begin{webonly}
\begin{itemize}
\item We want to find $\hat x$ such that $A\hat x = \proj_{\Col A}(b)$.
\item This means $b-A\hat x$ is in $(\Col A)^\perp$.
\item Recall that $(\Col A)^\perp = \Nul(A^T)$.
\item So $b-A\hat x$ is in $(\Col A)^\perp$ if and only if $A^T(b-A\hat x) = 0$.
\item In other words, $A^T A\hat x = A^T b$.
\end{itemize}
\end{webonly}

\pause\medskip
\alert{Alternative when $A$ has orthogonal columns $v_1,v_2,\ldots,v_n$:}
\displayskips{2pt}
\[ \hat b = \proj_{\Col A}(b)
= \sum_{i=1}^n \frac{b\cdot v_i}{v_i\cdot v_i}\,v_i \]
The right hand side equals $A\hat x$, where $\displaystyle\hat x = \left(
  \frac{b\cdot v_1}{v_1\cdot v_1},\;
  \frac{b\cdot v_2}{v_2\cdot v_2},\;
  \cdots,\;
  \frac{b\cdot v_n}{v_n\cdot v_n}
 \right).$

\end{frame}


%%%%%%%%%%%%%%%%%%%%%%%%%%%%%%%%%%%%%%%%%%%%%%%%%%%%%%%%%%%%%%%%%%%

\begin{frame}
\frametitle{Least Squares Solutions}
\framesubtitle{Example}

Find the least squares solutions to $Ax=b$ where:
\[ A = \mat{1 0; 1 1; 1 2} \qquad b = \vec{6 0 0}. \]

\begin{webonly}
  We have
  \[ A^TA = \mat{1 1 1; 0 1 2}\mat{1 0; 1 1; 1 2} = \mat{3 3; 3 5} \]
  and
  \[ A^T b = \mat{1 1 1; 0 1 2}\vec{6 0 0} = \vec{6 0}. \]
  Row reduce:
  \[ \amat{3 3 6; 3 5 0} \;\longsquiggly\; \amat{1 0 5; 0 1 -3}. \]
\end{webonly}%
\pause
So the only least squares solution is $\hat x = \vec{5 -3}$.

\end{frame}


%%%%%%%%%%%%%%%%%%%%%%%%%%%%%%%%%%%%%%%%%%%%%%%%%%%%%%%%%%%%%%%%%%%

\begin{frame}
\frametitle{Least Squares Solutions}
\framesubtitle{Example, continued}

How close did we get?
\begin{webonly}
  \[ \hat b = A\hat x = \mat{1 0; 1 1; 1 2}\vec{5 -3} = \vec{5 2 -1} \]
  The distance from $b$ is
  \[ \|b-A\hat x\| = \left\|\vec{6 0 0} - \vec{5 2 -1}\right\|
  = \left\|\vec{1 -2 1}\right\| = \sqrt{1^2+(-2)^2+1^2} = \sqrt 6. \]
\end{webonly}

\pause\leavevmode
\hbox to .5\linewidth{\hss
\begin{tikzpicture}[x={(1.2cm,.4cm)}, y={(1cm,-.8cm)}, z={(0cm,1cm)},
    scale=.25, thin border nodes, baseline]
  \coordinate (v1) at (1,1,1);
  \coordinate (v2) at (0,1,2);
  \coordinate (b)  at (6,0,0);
  \coordinate (bh) at ($5*(v1)-3*(v2)$);

  \begin{scope}[xshift=-5cm, yshift=1.5cm, scale=2]
    \draw[->] (0,0,0) -- (1,0,0) node[right] {$x$};
    \draw[->] (0,0,0) -- (0,1,0) node[right] {$y$};
    \draw[->] (0,0,0) -- (0,0,1) node[above] {$z$};
  \end{scope}

  \begin{scope}[x=(v1), y=(v2), transformxy]
    \draw[help lines, seq-violet!50] (-1,-4) grid (6,2);
    \fill[seq-violet!50, opacity=.7] (-1,-4) rectangle (6,2);
    \node[seq-violet] at (3,-5) {$\Col A$};
    \draw[vector] (0,0) to["$v_1$"'] (1,0);
    \draw[vector] (0,0) -- (0,1)
      node[anchor=south] {$v_2$};

    \draw<3->[dashed] (0,0) -| 
      node[pos=.25, above=1pt] {$5v_1$}
      node[pos=.75, right] {$-3v_2$}
        (5,-3);
  \end{scope}

  \point (o) at (0,0,0);
  \draw (b) -- node[font=\scriptsize,pos=.5,left=1pt] {$\sqrt 6$} (bh);
  \coordinate (b1) at ($(bh)-(v1)$);
  \coordinate (b2) at ($(bh)+(v2)$);
  \point[seq-blue, pin={[
      text=seq-blue,
      pin edge={<-,very thin, seq-blue, shorten <=1pt},
      pin distance=.6cm]
      below right:$\hat b=A\vec{5 -3}$}] at (bh);
  \point[seq-red, "$b$" text=seq-red] at (b);
  \pic[draw] {right angle=(b)--(bh)--(b1)};
  \pic[draw] {right angle=(b)--(bh)--(b2)};

\end{tikzpicture}\quad}%
\begin{minipage}[c]{.5\linewidth}\raggedright
  Let 
  \[ v_1 = \vec{1 1 1} \sptxt{and} v_2 = \vec{0 1 2} \]
  be the columns of $A$, and let $\cB=\{v_1,v_2\}$.
\end{minipage}

\pause\smallskip
Note $\hat x = {5\choose -3}$ is just the $\cB$-coordinates of $\hat b$,
in $\Col A = \Span\{v_1,v_2\}$.

\end{frame}


%%%%%%%%%%%%%%%%%%%%%%%%%%%%%%%%%%%%%%%%%%%%%%%%%%%%%%%%%%%%%%%%%%%

\begin{frame}
\frametitle{Least Squares Solutions}
\framesubtitle{Second example}

Find the least squares solutions to $Ax=b$ where:
\[ A = \mat[r]{2 0; -1 1; 0 2} \qquad b = \vec[r]{1 0 -1}. \]

\begin{webonly}
  We have
  \[ A^T A = \mat[r]{2 -1 0; 0 1 2}\mat[r]{2 0; -1 1; 0 2} =
  \mat[r]{5 -1; -1 5} \]
  and
  \[ A^T b = \mat[r]{2 -1 0; 0 1 2}\vec[r]{1 0 -1}
  = \vec[r]{2 -2}. \]
  Row reduce:
  \[ \amat{5 -1 2; -1 5 -2} \;\longsquiggly\;
  \amat{1 0 1/3; 0 1 -1/3}. \]
\end{webonly}%
\pause
So the only least squares solution is $\hat x = \vec{1/3 -1/3}$.

\end{frame}


%%%%%%%%%%%%%%%%%%%%%%%%%%%%%%%%%%%%%%%%%%%%%%%%%%%%%%%%%%%%%%%%%%%

\begin{frame}
\frametitle{Least Squares Solutions}
\framesubtitle{Uniqueness}

When does $Ax=b$ have a \emph{unique} least squares solution $\hat x$?

\pause
\begin{thm}
  Let $A$ be an $m\times n$ matrix.  The following are equivalent:
  \pause
  \begin{enumerate}
  \item $Ax=b$ has a \emph{unique} least squares solution for all $b$ in $\R^n$.
  \pause
  \item The columns of $A$ are linearly independent.
  \pause
  \item $A^TA$ is invertible.
  \end{enumerate}
  \pause
  In this case, the least squares solution is $(A^T A)\inv(A^T b)$.
\end{thm}

\pause\smallskip
\alert{Why?}
\pause
If the columns of $A$ are linearly \emph{de\/}pendent, then $A\hat x=\hat b$ has
many solutions:\\[-1mm]
\hfill\begin{tikzpicture}[x={(1.2cm,.4cm)}, y={(1cm,-.8cm)}, z={(0cm,1cm)},
    scale=.25, thin border nodes, baseline]
  \coordinate (v1) at (1,1,1);
  \coordinate (v2) at (0,1,2);
  \coordinate (b)  at (6,0,0);
  \coordinate (bh) at ($5*(v1)-3*(v2)$);

  \begin{scope}[x=(v1), y=(v2), transformxy]
    \draw[help lines, seq-violet!50] (-1,-4) grid (6,2);
    \fill[seq-violet!50, opacity=.7] (-1,-4) rectangle (6,2);
    \node[seq-violet] at (3,-5) {$\Col A$};
    \draw[vector] (0,0) to["$v_1$"'] (1,0);
    \draw[vector] (0,0) -- (0,1)
      node[anchor=south] {$v_2$};
    \draw[vector] (0,0) -- (1,-2)
      node[anchor=east] {$v_3$};
  \end{scope}

  \point (o) at (0,0,0);
  \draw (b) -- (bh);
  \coordinate (b1) at ($(bh)-(v1)$);
  \coordinate (b2) at ($(bh)+(v2)$);
  \point[seq-blue, pin={[
      text=seq-blue,
      pin edge={<-,very thin, seq-blue, shorten <=1pt},
      pin distance=.6cm]
      below right:$\hat b=A\hat x$}] at (bh);
  \point[seq-red, "$b$" text=seq-red] at (b);
  \pic[draw] {right angle=(b)--(bh)--(b1)};
  \pic[draw] {right angle=(b)--(bh)--(b2)};

\end{tikzpicture}\hfill\null

\pause\smallskip
\alert{Note:} $A^TA$ is always a square matrix, but it need not be invertible.

\end{frame}


%%%%%%%%%%%%%%%%%%%%%%%%%%%%%%%%%%%%%%%%%%%%%%%%%%%%%%%%%%%%%%%%%%%

\begin{frame}
\frametitle{Application}
\framesubtitle{Data modeling: best fit line}

Find the best fit line through $(0,6)$, $(1,0)$, and $(2,0)$.

\medskip
\begin{columns}[onlytextwidth]
\column[c]{.6\linewidth}
\begin{webonly}%
The general equation of a line is
\[ y = C + Dx. \]
So we want to solve:
\[\begin{split}
  6 &= C + D\cdot 0 \\
  0 &= C + D\cdot 1 \\
  0 &= C + D\cdot 2.
\end{split}\]
In matrix form:
\[ \mat{1 0; 1 1; 1 2}\vec{C D} = \vec{6 0 0}. \]
We already saw: the least squares solution is ${5\choose-3}$.
So the best fit line is
\[ y = -3x + 5. \]
\vss
\end{webonly}

\column[c]{.4\linewidth}\centering
\begin{tikzpicture}[scale=.6,decoration={brace,raise=1mm},
    thin border nodes]
  \draw[grid lines] (-2,-2) grid (4,7);
  \draw[->] (-2,0) -- (4,0);
  \draw[->] (0,-2) -- (0,7);
  \point["{$(0,6)$}" above right] at (0,6);
  \point["{$(1,0)$}" below] at (1,0);
  \point["{$(2,0)$}" anchor=-130] at (2,0);
  \draw<3->[decorate, seq-blue, thick, decoration={mirror}]
    (0,5) -- node[right=2mm] {$1$} (0,6);
  \draw<3->[decorate, seq-blue, thick]
    (1,0) -- node[left=2mm,yshift=1.5pt] {$-2$} (1,2);
  \draw<3->[decorate, seq-blue, thick]
    (2,0) -- node[right=2mm] {$1$} (2,-1);
  \draw<2->[thick, seq-red] (-2/3,7) --
    node[sloped,above=1pt] {$y=-3x+5$} (7/3,-2);
  \node<3->[text=seq-blue] at (1,-3.2) 
    {$A\vec{5 -3} - \vec{6 0 0} = \vec{1 -2 1}$};
  \useasboundingbox (1,-4.25);

\end{tikzpicture}
\end{columns}
\note{Don't reveal the last overlay until answering the poll.}

\end{frame}


%%%%%%%%%%%%%%%%%%%%%%%%%%%%%%%%%%%%%%%%%%%%%%%%%%%%%%%%%%%%%%%%%%%

\begin{pollframe}

\begin{poll}
\vskip 1cm

\begin{bluebox}[Poll]{.7\linewidth}
  What does the best fit line minimize?
  \smallskip
  \begin{eAlpherate}
  \item The sum of the squares of the distances from the data points to the line.
  \item The sum of the squares of the vertical distances from the data points to
    the line.
  \item The sum of the squares of the horizontal distances from the data points
    to the line.
  \item The maximal distance from the data points to the line.
  \end{eAlpherate}

\end{bluebox}

\pause\medskip
\alert{Answer: B.}  See the picture on the previous slide.
\end{poll}

\end{pollframe}


%%%%%%%%%%%%%%%%%%%%%%%%%%%%%%%%%%%%%%%%%%%%%%%%%%%%%%%%%%%%%%%%%%%

\def\eqline#1#2{(#1)^2 + A(#2)^2 + B(#1)(#2) + C(#1) + D(#2) + E = 0}
\edef\eqs{\eqline02;
  \eqline21;
  \eqline1{-1};
  \eqline{-1}{-2};
  \eqline{-3}1
}

\begin{frame}
\frametitle{Application}
\framesubtitle{Best fit ellipse}

Find the best fit ellipse for the points 
$(0,2)$, $(2,1)$, $(1,-1)$, $(-1,-2)$, $(-3,1)$.
\note{Tell the students not to bother writing this down.}

\pause\medskip
The general equation for an ellipse is
\[ x^2 + Ay^2 + Bxy + Cx + Dy + E = 0 \]
\pause
So we want to solve:
\[\spalignsysdelims..
\expandafter\syseq\expandafter{\eqs}
\]
\pause
In matrix form:
\[ \mat[r]{
  4 0 0 2 \phantom-1;
  1 2 2 1 1;
  1 -1 1 -1 1;
  4 2 -1 -2 1;
  1 -3 -3 1 1}
\vec{A B C D E} = \vec[r]{0 -4 -1 -1 -9}. \]

\end{frame}


%%%%%%%%%%%%%%%%%%%%%%%%%%%%%%%%%%%%%%%%%%%%%%%%%%%%%%%%%%%%%%%%%%%

\begin{frame}
\frametitle{Application}
\framesubtitle{Best fit ellipse, continued}

\small
\vskip-3mm
\[ A = \mat[r]{
  4 0 0 2 \phantom-1;
  1 2 2 1 1;
  1 -1 1 -1 1;
  4 2 -1 -2 1;
  1 -3 -3 1 1}
\qquad b = \vec[r]{0 -4 -1 -1 -9}. \]

\pause
\[ A^T A = \mat[r]{
  35 6 -4 1 11;
  6 18 10 -4 0;
  -4 10 15 0 -1;
  1 -4 0 11 1;
  11 0 -1 1 5
} \qquad
A^T b = \vec[r]{-18 18 19 -10 -15} \]
\pause
Row reduce:
\[ \amat{
  35 6 -4 1 11 -18;
  6 18 10 -4 0 18;
  -4 10 15 0 -1 19;
  1 -4 0 11 1 -10;
  11 0 -1 1 5 -15
}
\;\longsquiggly\;
\amat{
  1 0 0 0 0 16/7;
  0 1 0 0 0 -8/7;
  0 0 1 0 0 15/7;
  0 0 0 1 0 -6/7;
  0 0 0 0 1 -52/7
}
\]
\pause\normalsize
Best fit ellipse:
\displayskips{3pt}
\[ x^2 + \frac{16}7 y^2 -\frac87 xy + \frac{15}7x - \frac67y - \frac{52}7 = 0 \]
\pause
or
\[ 7 x^2 + 16 y^2 - 8 xy + 15 x - 6y - 52 = 0. \]

\end{frame}


%%%%%%%%%%%%%%%%%%%%%%%%%%%%%%%%%%%%%%%%%%%%%%%%%%%%%%%%%%%%%%%%%%%

\begin{frame}
\frametitle{Application}
\framesubtitle{Best fit ellipse, picture}

\begin{center}
\vskip-3mm
\begin{tikzpicture}[scale=.8,decoration={brace,raise=1mm},
    thin border nodes, whitebg nodes]
  \draw[grid lines] (-6,-4) grid (6,4);
  \draw[->] (-6,0) -- (6,0);
  \draw[->] (0,-4) -- (0,4);
  
  \point["{$(0,2)$}" below] at (0,2);
  \point["{$(2,1)$}" right] at (2,1);
  \point["{$(1,-1)$}" right] at (1,-1);
  \point["{$(-1,-2)$}" {above,xshift=-2mm}] at (-1,-2);
  \point["{$(-3,1)$}" {above,xshift=-1.5mm}] at (-3,1);

  \draw<2->[thick, seq-red]
    (-1.125,-.09375) ellipse[x radius=3.31346, y radius=1.85295, rotate=31.4296];
  \node<2->[text=seq-red, font=\normalsize] at (-1.125, -3.3)
    {$7 x^2 + 16 y^2 - 8 xy + 15 x - 6y - 52 = 0$};
\end{tikzpicture}
\end{center}

\pause[3]\vskip-1mm
\alert{Remark:} Gauss invented the method of least squares to do exactly this:
he predicted the (elliptical) orbit of the asteroid Ceres as it passed behind
the sun in 1801.
%
\note{Ask: what does the best fit ellipse minimize?}

\end{frame}


%%%%%%%%%%%%%%%%%%%%%%%%%%%%%%%%%%%%%%%%%%%%%%%%%%%%%%%%%%%%%%%%%%%

\begin{frame}
\frametitle{Application}
\framesubtitle{Best fit parabola}

What least squares problem $Ax=b$ finds the best parabola through the points
$(-1,0.5)$, $(1,-1)$, $(2,-0.5)$, $(3,2)$?

\medskip
\begin{webonly}
  The general equation for a parabola is
  \[ y = Ax^2 + Bx + C. \]
  So we want to solve:
  \[\spalignsysdelims..\syseq{
    0.5  = A(-1)^2 + B(-1) + C;
    -1   = A(1)^2  + B(1)  + C;
    -0.5 = A(2)^2  + B(2)  + C;
    2    = A(3)^2  + B(3)  + C
  }
  \]
  In matrix form:
  \[\mat[r]{
    1 -1 1;
    1  1 1;
    4  2 1;
    9  3 1}
  \vec{A B C} = \vec[r]{0.5 -1 -0.5 2}.
  \]

\end{webonly}

\pause
\hbox to \linewidth{
  \rlap{\alert{Answer:}}\hss
  $\displaystyle 88y = 53x^2 - \frac{379}5x - 82$  \hss
}

\end{frame}


%%%%%%%%%%%%%%%%%%%%%%%%%%%%%%%%%%%%%%%%%%%%%%%%%%%%%%%%%%%%%%%%%%%

\begin{frame}
\frametitle{Application}
\framesubtitle{Best fit parabola, picture}

\begin{center}
\vfill
\begin{tikzpicture}[scale=1,decoration={brace,raise=1mm},
    thin border nodes, whitebg nodes]
  \draw[grid lines] (-4,-2) grid (5,5);
  \draw[->] (-4,0) -- (5,0);
  \draw[->] (0,-2) -- (0,5);
  
  \point["{$(-1,0.5)$}" {right,yshift=2mm}] at (-1,0.5);
  \point["{$(1,-1)$}" {above,xshift=-2mm}] at (1,-1);
  \point["{$(2,-0.5)$}" right] at (2,-0.5);
  \point["{$(3,2)$}" {above,xshift=-4mm}] at (3,2);

  \clip (-4,-2) rectangle (5,5);

  \draw<2->[thick, seq-red, domain=-6:6, samples=50, smooth]
    plot (\x, 53/88*\x*\x - 379/440*\x - 41/44);
  \node<2->[text=seq-red, font=\normalsize] at (.8, 3.3)
    {$\displaystyle 88y = 53x^2 - \frac{379}5x - 82$};
\end{tikzpicture}
\vfill
\end{center}

\end{frame}


%%%%%%%%%%%%%%%%%%%%%%%%%%%%%%%%%%%%%%%%%%%%%%%%%%%%%%%%%%%%%%%%%%%

\begin{frame}
\frametitle{Application}
\framesubtitle{Best fit linear function}

\begin{columns}[onlytextwidth]
\column[t]{.7\linewidth}
What least squares problem $Ax=b$ finds the best linear function $f(x,y)$
fitting the following data?

\medskip\webonlycmd{
The general equation for a linear function in two variables is
\[ f(x,y) = Ax + By + C. \]
}
\column[t]{.3\linewidth}
\vskip-1cm
\[\begin{array}{r|r|c}
  x & y & f(x,y) \\\hline
  1 & 0 & 0 \\
  0 & 1 & 1 \\
  -1 & 0 & 3 \\
  0 & -1 & 4
\end{array}\]
\end{columns}

\begin{webonly}
  So we want to solve
  \[\spalignsysdelims..\syseq{
    A(1) + B(0) + C = 0;
    A(0) + B(1) + C = 1;
    A(-1) + B(0) + C = 3;
    A(0) + B(-1) + C = 4
  }\]
  In matrix form:
  \[\mat[r]{
    1 0 1;
    0 1 1;
    -1 0 1;
    0 -1 1
  }\vec{A B C} = \vec{0 1 3 4}.
  \]

\end{webonly}

\pause
\hbox to \linewidth{
  \rlap{\alert{Answer:}}\hss
  $\displaystyle f(x,y) = -\frac32 x -\frac 32y + 2$
  \hss
}

\end{frame}


%%%%%%%%%%%%%%%%%%%%%%%%%%%%%%%%%%%%%%%%%%%%%%%%%%%%%%%%%%%%%%%%%%%

\begin{frame}
\frametitle{Application}
\framesubtitle{Best fit linear function, picture}

\begin{center}
\vfill
\begin{tikzpicture}[scale=.5, myxyz, thin border nodes,
    vert lines/.style={very thin, black!40}]

  \begin{scope}[xshift=-8cm, yshift=1cm, scale=2, overlay]
    \draw<2->[->] (0,0,0) -- (1,0,0) node[right] {$x$};
    \draw<2->[->] (0,0,0) -- (0,1,0) node[left] {$y$};
    \draw<2->[->] (0,0,0) -- (0,0,1) node[above] {$f(x,y)$};
  \end{scope}

  \begin{scope}[xshift=8cm,, yshift=2cm, overlay]
    \node<2->[align=center]
      {Graph of\\[2mm]
        $\displaystyle\color{red}f(x,y) = -\frac32 x -\frac 32y + 2$};
  \end{scope}

  \begin{scope}[transformxy]
    \draw[help lines, black!90] (-2,-2) grid (2,2);
  \end{scope}

  \tikzmath {
    int \x, \y;
    function f(\x,\y) {
      return -3/2*\x-3/2*\y+2;
    };
    for \x in {-2,...,1}{
      for \y in {-2,...,1}{
        { \filldraw<2->[help lines, seq-red!70, fill=seq-red!40, fill opacity=.5]
             (\x,\y,{f(\x,\y)}) 
             -- (\x+1,\y,{f(\x+1,\y)}) 
             -- (\x+1,\y+1,{f(\x+1,\y+1)})
             -- (\x,\y+1,{f(\x,\y+1)}) 
             -- cycle;
          % \draw<2->[vert lines] (\x,\y,{f(\x,\y)}) -- (\x,\y,0);
          % \draw<2->[vert lines] (\x+1,\y,{f(\x+1,\y)}) -- (\x+1,\y,0);
          % \draw<2->[vert lines] (\x+1,\y+1,{f(\x+1,\y+1)}) -- (\x+1,\y+1,0);
          % \draw<2->[vert lines] (\x,\y+1,{f(\x,\y+1)}) -- (\x,\y+1,0);
        };
      };
    };
  }
  
  \draw<2->[vert lines] (-2,-2,{f(-2,-2)}) -- (-2,-2,0);
  \draw<2->[vert lines] (-2, 2,{f(-2, 2)}) -- (-2, 2,0);
  \draw<2->[vert lines] ( 2,-2,{f( 2,-2)}) -- ( 2,-2,0);
  \draw<2->[vert lines] ( 2, 2,{f( 2, 2)}) -- ( 2, 2,0);

  \begin{scope}[every node/.style={font=\footnotesize, inner sep=1pt}]
    \draw<2->[thick,seq-blue] (1,0,0)
      -- (1,0,{f(1,0)})
      node[point,seq-red,"${f(1,0)}$" text=seq-red] {};
    \point["${(1,0,0)}$" below] at (1,0,0);
    \draw[densely dotted] (0,1,1) -- (0,1,0);
    \draw<2->[thick,seq-blue] (0,1,1) 
      -- (0,1,{f(0,1)}) 
      node[point,seq-red,"${f(0,1)}$" {text=seq-red, below}] {};
    \point["${(0,1,1)}$"] at (0,1,1);
    \draw[densely dotted] (-1,0,3) -- (-1,0,0);
    \draw<2->[thick,seq-blue] (-1,0,3)
      -- (-1,0,{f(-1,0)}) 
      node[point,seq-red,"${f(-1,0)}$" text=seq-red] {};
    \point["${(-1,0,3)}$" below] at (-1,0,3);
    \draw[densely dotted] (0,-1,4) -- (0,-1,0);
    \draw<2->[thick,seq-blue] (0,-1,4) 
      -- (0,-1,{f(0,-1)}) 
      node[point,seq-red,"${f(0,-1)}$" {text=seq-red, below}] {};
    \point["${(0,-1,4)}$"] at (0,-1,4);
  \end{scope}

  \point[scale=.75, black!50] at (0,0,0);

  \useasboundingbox (2,2,{f(2,2)}) (-2,-2,{f(-2,-2)});
  
\end{tikzpicture}
\vfill
\end{center}

\end{frame}


%%% Local Variables:
%%% TeX-master: "../slides"
%%% End:
