
% JDR: This should take a bit more than one lecture as written.  It can bleed
%   into 2.9, which is shorter.

\titleframe{Section 2.8}{Subspaces of $\R^n$}


%%%%%%%%%%%%%%%%%%%%%%%%%%%%%%%%%%%%%%%%%%%%%%%%%%%%%%%%%%%%%%%%%%%

\begin{frame}
\frametitle{Motivation}

Today we will discuss \textbf{subspaces} of $\R^n$.

\pause\medskip
A subspace turns out to be the same as a span, except we don't know \emph{which}
vectors it's the span of.

\pause\medskip
This arises naturally when you have, say, a plane through the origin in $\R^3$
which is \emph{not} defined (a priori) as a span, but you still want to say
something about it.

\begin{center}
\begin{tikzpicture}[myxyz, scale=.6]
  \def\v{(2,-1,1)}
  \def\w{(1,0,-1)}

  \node[coordinate] (X) at \v {};
  \node[coordinate] (Y) at \w {};

  \begin{scope}[x=(X), y=(Y), transformxy]
    \fill[seq4!30, nearly opaque] (-1.5,-2) rectangle (1.5,2);
    \draw[step=.5cm, seq4!50, very thin] (-1.5,-2) grid (1.5,2);
  \end{scope}

  \node[seq4] at (-2.5cm, 2.9cm) {$x+3y+z=0$};

  \point at (0,0,0);
\end{tikzpicture}
\end{center}
\end{frame}


%%%%%%%%%%%%%%%%%%%%%%%%%%%%%%%%%%%%%%%%%%%%%%%%%%%%%%%%%%%%%%%%%%%

\begin{frame}
\frametitle{Definition of Subspace}

\vskip-3mm
\begin{defn}
  A \textbf{subspace} of $\R^n$ is a subset $V$ of $\R^n$ satisfying:
  \begin{enumerate}
  \item The zero vector is in $V$.
    \hfill \pause\textcolor{blue}{``not empty''}
    \pause
  \item If $u$ and $v$ are in $V$, then $u+v$ is also in $V$.
    \hfill \pause\textcolor{blue}{``closed under addition''}
    \pause
  \item If $u$ is in $V$ and $c$ is in $\R$, then $cu$ is in $V$.
    \hfill \pause\textcolor{blue}{``closed under $\times$ scalars''}
  \end{enumerate}
\end{defn}
\note{(3) doesn't imply (1): could be empty}

\pause\medskip
\alert{What does this mean?}
\begin{itemize}
\item If $v$ is in $V$, then all scalar multiples of $v$ are in $V$ by~(3).
  \pause
  That is, the line through $v$ is in $V$.

  \pause
\item If $u,v$ are in $V$, then $xu$ and $yv$ are in $V$ for scalars $x,y$ by~(3).
  \pause
  So $xu+yv$ is in $V$ by~(2).
  \pause
  So $\Span\{u,v\}$ is contained in $V$.

  \pause
\item Likewise, if $v_1,v_2,\ldots,v_n$ are all in $V$, then 
  $\Span\{v_1,v_2,\ldots,v_n\}$ is contained in $V$.

\end{itemize}

\pause\medskip
\begin{bluebox}{.75\linewidth}
  A subspace $V$ contains the span of any set of vectors in $V$.
\end{bluebox}

\end{frame}


%%%%%%%%%%%%%%%%%%%%%%%%%%%%%%%%%%%%%%%%%%%%%%%%%%%%%%%%%%%%%%%%%%%

\begin{frame}
\frametitle{Examples}

\vskip-3mm
\begin{eg}
  \begin{minipage}[t]{.5\linewidth}
    A line $L$ through the origin: this contains the span of any vector in $L$.
  \end{minipage}\hfill
  \begin{tikzpicture}[scale=.5, baseline]
    \useasboundingbox (-3,-1) -- (3,0);
    \draw[seq4] (-3,-1) -- (3,1);
    \draw[vector] (0,0) -- (1,1/3);
    \point at (0,0);
    \node[seq4] at (0,.5) {$L$};
  \end{tikzpicture}
\end{eg}

\pause
\begin{eg}
  \begin{minipage}[t]{.5\linewidth}
    A plane $P$ through the origin: this contains the span of any vectors in $P$.
  \end{minipage}\hfill
  \begin{tikzpicture}[myxyz, scale=.5, baseline=.5cm]
    \useasboundingbox[resetxy] (-3,-3) rectangle (3,0);
  
    \def\v{(2,-1,1)}
    \def\w{(1,0,-1)}
  
    \node[coordinate] (X) at \v {};
    \node[coordinate] (Y) at \w {};
  
    \begin{scope}[x=(X), y=(Y), transformxy]
      \fill[seq4!30, nearly opaque] (-1,-1) rectangle (1,1);
      \draw[step=.5cm, seq4!50, very thin] (-1,-1) grid (1,1);
    \end{scope}

    \draw[vector] (0,0,0) -- ($.5*(X)$);
    \draw[vector] (0,0,0) -- ($.5*(Y)$);
  
    \point at (0,0,0);
    \node[x=(X),y=(Y),seq4] at (.2,-1.4) {$P$};
  \end{tikzpicture}
\end{eg}

\pause
\begin{eg}
  All of $\R^n$: this contains $0$, and is closed under addition and scalar
  multiplication.
\end{eg}

\pause
\begin{eg}
  The subset $\{0\}$: this subspace contains only one vector.
\end{eg}

\pause\medskip
Note these are all pictures of spans!  (Line, plane, space, etc.)

\end{frame}


%%%%%%%%%%%%%%%%%%%%%%%%%%%%%%%%%%%%%%%%%%%%%%%%%%%%%%%%%%%%%%%%%%%

\begin{frame}
\frametitle{Non-Examples}

\vskip-3mm
\begin{noneg}
  \begin{minipage}[t]{.5\linewidth}
    A line $L$ (or any other set) that doesn't contain the origin is not a
    subspace.  Fails: \uncover<2->{\alert 1.}
  \end{minipage}\hfill
  \begin{tikzpicture}[scale=.5, baseline]
    \useasboundingbox (-3,-1) -- (3,0);
    \draw[seq4] (-3,-1) -- (3,1);
    \point<1> at (0,-.5);
    \point<2->[red] at (0,-.5);
  \end{tikzpicture}\hskip 0pt plus .5fill\null
\end{noneg}

\pause[3]%
\begin{noneg}
  \begin{minipage}[t]{.5\linewidth}
    A circle $C$ is not a subspace.  Fails: \uncover<4->{\alert{1,2,3}.}
    \uncover<5->{Think: a circle isn't a ``linear space.''}
  \end{minipage}\hfill
  \begin{tikzpicture}[scale=.5, baseline]
    \useasboundingbox (-2,-2) -- (2,0);
    \draw[seq4] (0,0) circle [radius=2cm];
    \draw<4->[vector,red] (0,0) -- (2.5,0);
    \draw<4->[vector] (0,0) -- (0,2);
    \draw<4->[vector] (0,0) -- (2,0);
    \draw<4->[vector,red] (0,0) -- (2,2);
    \point<3> at (0,0);
    \point<4->[red] at (0,0);
  \end{tikzpicture}\hskip 0pt plus .5fill\null
\end{noneg}

\pause[6]%
\begin{noneg}
  \begin{minipage}[t]{.5\linewidth}
    The first quadrant in $\R^2$ is not a subspace.
    Fails: \uncover<7->{\alert{3} only.}
  \end{minipage}\hfill
  \begin{tikzpicture}[scale=.5, baseline]
    \useasboundingbox (-2,-2) -- (2,1);
    \fill[seq4!20] (0,0) rectangle (2,2);
    \draw[->] (-2,0) -- (2,0);
    \draw[->] (0,-2) -- (0,2);
    \draw<7->[vector] (0,0) -- (1,1);
    \draw<7->[vector,red] (0,0) -- (-1,-1);
    \point at (0,0);
  \end{tikzpicture}\hskip 0pt plus .5fill\null
\end{noneg}

\pause[8]%
\begin{noneg}
  \begin{minipage}[t]{.5\linewidth}
    A line union a plane in $\R^3$ is not a subspace.
    Fails: \uncover<9->{\alert{2} only.}
  \end{minipage}\hfill
  \begin{tikzpicture}[scale=.5, baseline, myxyz]
    \useasboundingbox (-4,-2) -- (4,1);
    \path[clip, resetxy] (-4,-2) rectangle (4,2);
  
    \draw[seq4] (-1,-2,-4) -- (0,0,0);

    \begin{scope}[transformxy]
      \fill[seq4!20, semitransparent] (-1.5,-1.5) rectangle (1.5,1.5);
      \draw[step=.5cm, grid lines] (-1.5,-1.5) grid (1.5,1.5);
    \end{scope}
  
    \draw[seq4] (1,2,4) -- (0,0,0);

    \draw<9->[vector] (0,0,0) -- (.3,.6,1.2);
    \draw<9->[vector] (0,0,0) -- (1,0,0);
    \draw<9->[vector,red] (0,0,0) -- (1.3,.6,1.2);

    \point at (0,0,0);

  \end{tikzpicture}\hskip 0pt plus .5fill\null
\end{noneg}

\end{frame}


%%%%%%%%%%%%%%%%%%%%%%%%%%%%%%%%%%%%%%%%%%%%%%%%%%%%%%%%%%%%%%%%%%%

\begin{frame}
\frametitle{Subsets and Subspaces}
\framesubtitle{They aren't the same thing}

A \textbf{subset} of $\R^n$ is any collection of vectors whatsoever.

\pause\bigskip
All of the non-examples are still subsets.

\pause\bigskip
A \textbf{subspace} is a special kind of subset, which satisfies the three
defining properties.

\pause\vskip 1cm\hfill
\begin{tikzpicture}[scale=.5, baseline]
  \draw[grid lines] (-2.5,-2.5) grid (2.5,2.5);
  \draw (0,0) circle [radius=2cm];
\end{tikzpicture}
\quad
\begin{minipage}{.5\linewidth}\leavevmode
\hbox to 1.5cm{Subset:\hss} \emph{yes}\\
\hbox to 1.5cm{Subspace:\hss} \emph{no}
\end{minipage}\hfill\null

\end{frame}


%%%%%%%%%%%%%%%%%%%%%%%%%%%%%%%%%%%%%%%%%%%%%%%%%%%%%%%%%%%%%%%%%%%

\begin{frame}
\frametitle{Spans are Subspaces}

\vskip-3mm
\begin{thm}
  Any $\Span\{v_1,v_2,\ldots,v_n\}$ is a subspace.
\end{thm}

\pause\medskip
\begin{bluebox}[!!!]{.7\linewidth}
  Every subspace is a span, and every span is a subspace.
\end{bluebox}

\pause
\begin{defn}
  If $V = \Span\{v_1,v_2,\ldots,v_n\}$, we say that $V$ is the
  subspace \textbf{generated by} or \textbf{spanned by}
  the vectors $v_1,v_2,\ldots,v_n$.
\end{defn}

\medskip
\begin{webonly}
\alert{Check:}
\begin{enumerate}
\item $0 = 0v_1 + 0v_2 + \cdots + 0v_n$ is in the span.
\item If, say, $u = 3v_1 + 4v_2$ and $v = -v_1 - 2v_2$, then
  \[ u + v = 3v_1 + 4v_2 -v_1 - 2v_2 = 2v_1 + 2v_2 \]
  is also in the span.
\item Similarly, if $u$ is in the span, then so is $cu$ for any scalar $c$.
\end{enumerate}
\end{webonly}

\end{frame}


%%%%%%%%%%%%%%%%%%%%%%%%%%%%%%%%%%%%%%%%%%%%%%%%%%%%%%%%%%%%%%%%%%%

\begin{pollframe}

\vfill

\begin{bluebox}[Poll]{.9\linewidth}
  Is the empty set $\{\}$ a subspace?  If not, which property(ies) does it fail?
\end{bluebox}

\pause\medskip
The zero vector is not contained in the empty set, so it is \emph{not} a
subspace.

\pause\bigskip
\alert{Question:} What is the difference between $\{\}$ and $\{0\}$?

\vfill

\end{pollframe}


%%%%%%%%%%%%%%%%%%%%%%%%%%%%%%%%%%%%%%%%%%%%%%%%%%%%%%%%%%%%%%%%%%%

\begin{frame}
\frametitle{Subspaces}
\framesubtitle{Verification}

Let $V = \left\{ \vec{a b} \text{ in } \R^2\bigm| ab=0 \right\}$.  Let's check
if $V$ is a subspace or not.

\begin{webonly}
\begin{enumerate}
\item Does $V$ contain the zero vector?
  ${a\choose b} = {0\choose 0} \implies ab = 0$ \bigcheck[\quad]
\item[3.] Is $V$ closed under scalar multiplication?
  \begin{itemize}
  \item Let $a\choose b$ be in $V$.  
  \item \emph{This means:} $a$ and $b$ are numbers such that $ab=0$.
  \item Let $c$ be a scalar.  Is $c{a\choose b} = {ca\choose cb}$ in $V$?
  \item \emph{This means:} $(ca)(cb) = 0$.  
  \item Well, $(ca)(cb) = c^2(ab) = c^2(0) = 0 \bigcheck[\quad]\namedbox{mark}{}$
  \end{itemize}
\item[2.] Is $V$ closed under addition?
  \begin{itemize}
  \item Let $a\choose b$ and $a'\choose b'$ be in $V$.  
  \item \emph{This means:} $ab = 0$ and $a'b' = 0$.
  \item Is ${a\choose b} + {a'\choose b'} = {a+a'\choose b+b'}$ in $V$?
  \item \emph{This means:} $(a+a')(b+b')=0$.
  \item This is not true for all such $a,a',b,b'$: for instance,
    $1\choose 0$ and $0\choose 1$ are in $V$, but their sum 
    ${1\choose 0} + {0\choose 1} = {1\choose 1}$ is not in $V$, because
    $1\cdot 1\neq 0$. \bigcross
  \end{itemize}
\end{enumerate}
\end{webonly}

\pause
We conclude that $V$ is \emph{not} a subspace.
\pause
A picture is above.  (It doesn't look like a span.)
\begin{tikzpicture}[remember picture, overlay]
  \begin{scope}[shift=(mark), xshift=3.5cm]
    \draw[grid lines, step=.25] (-1,-1) grid (1,1);
    \draw[seq4, <->] (-1,0) -- (1,0);
    \draw[seq4, <->] (0,-1) -- (0,1);
    \node[seq4] at (.25,.25) {$V$};
  \end{scope}
\end{tikzpicture}

\end{frame}


%%%%%%%%%%%%%%%%%%%%%%%%%%%%%%%%%%%%%%%%%%%%%%%%%%%%%%%%%%%%%%%%%%%

\begin{frame}
\frametitle{Column Space and Null Space}

An $m\times n$ matrix $A$ naturally gives rise to \emph{two}
subspaces.

\pause
\begin{defn}
  \begin{itemize}
  \item The \textbf{column space} of $A$ is the subspace of
    $\R^{\blankuntil{3}{m}}$ spanned by the columns of $A$.  It is written 
    $\Col A$.
    \pause[4]
  \item The \textbf{null space} of $A$ is the set of all solutions of the
    homogeneous equation $Ax=0$:
    \[ \Nul A = \bigl\{ x \text{ in } \R^{\blankuntil{5}{n}}\mid Ax=0 \bigr\}. \]
    This is a subspace of $\R^{\blankuntil{5}{n}}$.
  \end{itemize}
\end{defn}

\vskip -.1cm
\pause[6]
The column space is defined as a span, so we know it is a subspace.
\pause
It is the range (as opposed to the codomain) of the transformation $T(x) = Ax$.

\pause\medskip
\alert{Check} that the null space is a subspace:
\begin{webonly}
\begin{enumerate}
\item $0$ is in $\Nul A$ because $A0 = 0$.
\item If $u$ and $v$ are in $\Nul A$, then $Au = 0$ and $Av = 0$.  Hence
  \abovedisplayskip=1pt\belowdisplayskip=1pt
  \[ A(u+v) = Au + Av = 0, \]
  so $u+v$ is in $\Nul A$.
\item If $u$ is in $\Nul A$, then $Au = 0$.  For any scalar $c$,
  $A(cu) = cAu = 0$.  So $cu$ is in $\Nul A$.
\end{enumerate}
\end{webonly}

\end{frame}


%%%%%%%%%%%%%%%%%%%%%%%%%%%%%%%%%%%%%%%%%%%%%%%%%%%%%%%%%%%%%%%%%%%

\begin{frame}
\frametitle{Column Space and Null Space}
\framesubtitle{Example}

Let $A = \mat{1 1; 1 1; 1 1}$.

\medskip
\begin{minipage}{.7\linewidth}
Let's compute the column space:
\begin{webonly}
\[ \Col A = \Span\left\{\vec{1 1 1},\,\vec{1 1 1}\right\}
= \Span\left\{\vec{1 1 1}\right\}. \]
This is a line in $\R^3$.
\end{webonly}
\end{minipage}\hfill\pause
\begin{tikzpicture}[scale=.5, baseline, myxyz]
  \useasboundingbox (-4,-2) -- (4,1);
  \path[clip, resetxy] (-4,-2) rectangle (4,2);

  \draw[seq4] (-4,-4,-4) -- (0,0,0);
   \begin{scope}[transformxy]
    \fill[white, semitransparent] (-1.5,-1.5) rectangle (1.5,1.5);
    \draw[step=1cm, grid lines] (-1.5,-1.5) grid (1.5,1.5);
  \end{scope}

  \draw[seq4] (4,4,4) -- (0,0,0);
  \draw[vector] (0,0,0) -- (1,1,1);
  \draw[densely dotted] (1,1,0) -- (1,1,1);
  \node[resetxy,seq4] at (1.5,1.5) {$\Col A$};
  \point at (0,0,0);
\end{tikzpicture}

\bigskip

\begin{minipage}{.7\linewidth}
\pause
Let's compute the null space:
\begin{webonly}
\displayskips{1pt}
\[ A\vec{x y} = \vec{x+y x+y x+y}. \]
This zero if and only if $x=-y$.  So
\[ \Nul A = \left\{ \vec{x y}\text{ in }\R^2\mid y=-x \right\}. \]
This defines a line in $\R^2$:
\end{webonly}
\end{minipage}
\hfill\pause
\begin{tikzpicture}[scale=.5, baseline]
  \useasboundingbox (-3,-3) -- (3,1);
  \draw[grid lines] (-3,-3) grid (3,3);
  \draw[seq4] (3,-3) -- (-3,3);
  \point at (0,0);
  \node[seq4,fill=white, inner sep=1pt] at (-.7,2) {$\Nul A$};
\end{tikzpicture}

\end{frame}


%%%%%%%%%%%%%%%%%%%%%%%%%%%%%%%%%%%%%%%%%%%%%%%%%%%%%%%%%%%%%%%%%%%

\begin{frame}
\frametitle{The Null Space is a Span}

The column space of a matrix $A$ is defined to be a span (of the columns).

\pause\medskip
The null space is defined to be the solution set to $Ax=0$.
\pause
It is a subspace, so it is a span.

\pause
\begin{ques}
  How to find vectors which span the null space?
\end{ques}

\pause
\alert{Answer:} Parametric vector form!
\pause
We know that the solution set to $Ax=0$ has a parametric form that looks like
\[ x_3\vec{1 2 1 0} + x_4\vec{-2 3 0 1}\quad
\parbox{2.5cm}{\centering if, say, $x_3$ and $x_4$ are the free variables. \pause So}
\quad
\Nul A = \Span\left\{ \vec{1 2 1 0},\, \vec{-2 3 0 1} \right\}.
\]

\pause\medskip
Refer back to the slides for \S1.5 (Solution Sets).

\pause\medskip
\alert{Note:} It is much easier to define the null space first as a
subspace, then find spanning vectors \emph{later}, if we need them.  This is one
reason subspaces are so useful.

\end{frame}


%%%%%%%%%%%%%%%%%%%%%%%%%%%%%%%%%%%%%%%%%%%%%%%%%%%%%%%%%%%%%%%%%%%

\begin{frame}
\frametitle{The Null Space is a Span}
\framesubtitle{Example, revisited}

Find vector(s) that span the null space of $A = \mat{1 1; 1 1; 1 1}$.

\begin{webonly}
\medskip
The reduced row echelon form is
$\mat{1 1; 0 0; 0 0}$.

\medskip
This gives the equation $x + y = 0$, or
\[ \syseq{x = -y; y = y}
\quad\longsquiggly[parametric vector form]\quad
\vec{x y} = y\vec{-1 1}. \]
\end{webonly}

\begin{minipage}[t]{.6\linewidth}
  \webonlycmd{
  The null space is
  \[ \Nul A = \Span\left\{ \vec{-1 1} \right\}. \]}
\end{minipage}\pause
\begin{tikzpicture}[scale=.5, baseline=1cm]
  \draw[grid lines] (-3,-3) grid (3,3);
  \draw[seq4] (3,-3) -- (-3,3);
  \draw[vector] (0,0) -- (-1,1);
  \point at (0,0);
  \node[seq4,fill=white, inner sep=1pt] at (-.7,2) {$\Nul A$};
  \end{tikzpicture}

\end{frame}


%%%%%%%%%%%%%%%%%%%%%%%%%%%%%%%%%%%%%%%%%%%%%%%%%%%%%%%%%%%%%%%%%%%

\begin{frame}
\frametitle{Subspaces}
\framesubtitle{Summary}

\vskip 5mm

\begin{bluebox}[How do you check if a subset is a subspace?]{.9\linewidth}
  \smallskip
  \begin{itemize}
  \item<+-> Is it a span?  Can it be written as a span?
  \item<+-> Can it be written as the column space of a matrix?
  \item<+-> Can it be written as the null space of a matrix?
  \item<+-> Is it all of $\R^n$ or the zero subspace $\{0\}$?
  \item<+-> Can it be written as a type of subspace that we'll learn about later
    (eigenspaces, \ldots)?
  \end{itemize}

  \smallskip\uncover<+->{%
  If so, then it's automatically a subspace.}

  \smallskip\uncover<+->{%
  If all else fails:}
  \begin{itemize}
  \item<+-> Can you verify directly that it satisfies the three defining
    properties?
  \end{itemize}
\end{bluebox}

\end{frame}


%%%%%%%%%%%%%%%%%%%%%%%%%%%%%%%%%%%%%%%%%%%%%%%%%%%%%%%%%%%%%%%%%%%

\begin{frame}
\frametitle{Basis of a Subspace}

What is the \emph{smallest number} of vectors that are needed to span a subspace?

\pause\bigskip
\namedbox{basis}{%
\begin{minipage}{\textwidth}\vskip -.14cm
\begin{defn}
  Let $V$ be a subspace of $\R^n$.  A \textbf{basis} of $V$ is a set of vectors
  $\{v_1,v_2,\ldots,v_m\}$ in $V$ such that:
  \begin{enumerate}
    \pause
  \item $V = \Span\{v_1,v_2,\ldots,v_m\}$, and
    \pause
  \item $\{v_1,v_2,\ldots,v_m\}$ is linearly independent.
  \end{enumerate}
  \pause
  The number of vectors in a basis is the \textbf{dimension} of $V$, and is
  written $\dim V$.
\end{defn}
\end{minipage}
}
\begin{tikzpicture}[remember picture, overlay]
  \node<6->[redbox, fit=(basis)] (basisbox) {};
  \node<6->[red, rotate=-90, anchor=south, font=\small, text width=2cm, align=center]
    at (basisbox.east) {Note the big red border here};
\end{tikzpicture}

\begin{uncoverenv}<7->
\pause[7]\bigskip
\alert{Why} is a basis the smallest number of vectors needed to span?

\pause\medskip
Recall: \emph{linearly independent} means that every time you add another
vector, the span gets bigger.

\pause\medskip
Hence, if we remove any vector, the span gets \emph{smaller\/}: so any smaller set
can't span $V$.

\pause
\begin{bluebox}[Important]{.8\textwidth}
  A subspace has \emph{many different} bases, but they all have the same
  number of vectors (see the exercises in \S2.9).
\end{bluebox}

\end{uncoverenv}

\end{frame}


%%%%%%%%%%%%%%%%%%%%%%%%%%%%%%%%%%%%%%%%%%%%%%%%%%%%%%%%%%%%%%%%%%%

\begin{frame}
\frametitle{Bases of $\R^2$}

\vskip-.2cm
\begin{ques}
  What is a basis for $\R^2$?
\end{ques}

\begin{minipage}[t]{.6\linewidth}
  \uncover<2->{We need two vectors that \emph{span $\R^2$} and are
  \emph{linearly independent}.}
  \uncover<3->{$\{e_1,e_2\}$ is one basis.}
  \begin{enumerate}
  \item<4-> They span: ${a\choose b} =\uncover<5->{ae_1 + be_2.}$
  \item<6-> They are linearly independent because they are not collinear.
  \end{enumerate}
\end{minipage}
\hfill
\begin{uncoverenv}<3->
\begin{tikzpicture}[picture align top, scale=.75]
  \useasboundingbox (-2,-2) rectangle (2,1);
  \draw[grid lines] (-2,-2) grid (2,2);
  \draw[thick vector] (0,0) -- (1,0) node[anchor=west,whitebg] {$e_1$};
  \draw[thick vector] (0,0) -- (0,1) node[anchor=south,whitebg] {$e_2$};
  \point at (0,0);
\end{tikzpicture}
\end{uncoverenv}

\pause[7]\smallskip
\begin{ques}
  What is another basis for $\R^2$?
\end{ques}

\begin{minipage}[t]{.6\linewidth}
  \uncover<8->{Any two nonzero vectors that are not collinear.}
  \uncover<9->{$\bigl\{{1\choose 0},{1\choose 1}\bigr\}$ is also a basis.}
  \begin{enumerate}
  \item<10-> They span: $\begin{psmm}1&1\\0&1\end{psmm}$ has a pivot in every row.
  \item<11-> They are linearly independent: $\begin{psmm}1&1\\0&1\end{psmm}$ has
    a pivot in every column.
  \end{enumerate}
\end{minipage}
\hfill
\begin{uncoverenv}<9->
\begin{tikzpicture}[picture align top, scale=.75]
  \useasboundingbox (-2,-2) rectangle (2,1);
  \draw[grid lines] (-2,-2) grid (2,2);
  \draw[thick vector] (0,0) -- (1,0) node[anchor=west,whitebg] {$1\choose 0$};
  \draw[thick vector] (0,0) -- (1,1) node[anchor=south,whitebg] {$1\choose 1$};
  \point at (0,0);
\end{tikzpicture}
\end{uncoverenv}

\end{frame}


%%%%%%%%%%%%%%%%%%%%%%%%%%%%%%%%%%%%%%%%%%%%%%%%%%%%%%%%%%%%%%%%%%%

\begin{frame}
\frametitle{Bases of $\R^n$}

The unit coordinate vectors
\[ e_1 = \vec{1 0 \vdots, 0 0},\quad e_2 = \vec{0 1 \vdots, 0 0},\quad\ldots,\quad
   e_{n-1} = \vec{0 0 \vdots, 1 0},\quad e_n = \vec{0 0 \vdots, 0 1} \]
are a basis for $\R^n$.
\pause
\begin{enumerate}
\item They span: \namedbox{identity}{$I_n$} has a pivot in every row.
\pause
\begin{tikzpicture}[blue!50, remember picture, overlay]
  \path (identity.north) ++(1cm,.25cm)
    node[font=\small, anchor=west] (expl)
      {The identity matrix has columns $e_1,e_2,\ldots,e_n$.};
  \draw[->, shorten >=1pt] (expl.west) to[out=180,in=90] (identity.north);
\end{tikzpicture}
\pause
\item They are linearly independent: $I_n$ has a pivot in every column.
\end{enumerate}

\pause\medskip
\alert{In general:} $\{v_1,v_2,\ldots,v_n\}$ is a basis for $\R^n$ if and only
if the matrix 
\[ A = \mat{| | ,, |; v_1 v_2 \cdots, v_n; | | ,, |} \]
has a pivot in every row and every column,
i.e.\ if $A$ is \blankuntil{6}{\emph{invertible}}.

\end{frame}


%%%%%%%%%%%%%%%%%%%%%%%%%%%%%%%%%%%%%%%%%%%%%%%%%%%%%%%%%%%%%%%%%%%

\begin{frame}
\frametitle{Basis of a Subspace}
\framesubtitle{Example}

\vskip-3mm
\begin{eg}
  Let
  \[ V = \left\{ \vec{x y z} \text{ in } \R^3\bigm| x + 3y + z = 0 \right\} \qquad 
  \cB = \left\{ \vec{-3 1 0},\;\vec{0 1 -3} \right\}. \]
  Verify that $\cB$ is a basis for $V$.

  \begin{webonly}
    \displayskips{3pt}
  \begin{enumerate}\setcounter{enumi}{-1}
  \item \alert{In $V$:} both vectors are in $V$ because
    \[ -3 + 3(1) + 0 = 0 \sptxt{and} 0 + 3(1) + (-3) = 0. \]
    
  \item \alert{Span:}
    If $\vec{x y z}$ is in $V$, then $y = -\frac 13(x+z)$, so
    \vskip-3mm
    \[ \vec{x y z} = -\frac x3\vec{-3 1 0} - \frac z3\vec{0 1 -3}. \]
    
  \item \alert{Linearly independent:}
    \[ c_1\vec{-3 1 0} + c_2\vec{0 1 -3} = 0
    \implies \vec{-3c_1 c_1+c_2 -3c_2} = \vec{0 0 0}
    \implies c_1 = c_2 = 0. \]
  \end{enumerate}
  \end{webonly}
\end{eg}

\end{frame}


%%%%%%%%%%%%%%%%%%%%%%%%%%%%%%%%%%%%%%%%%%%%%%%%%%%%%%%%%%%%%%%%%%%

\begin{frame}
\frametitle{Basis for $\Nul A$}

\begin{bluebox}[Fact]{.8\linewidth}
  The vectors in the parametric vector form of the general
  solution to $Ax=0$ always form a basis for $\Nul A$.
\end{bluebox}

\pause
\begin{eg}
  \vskip-6mm
  \begin{webonly}
  \[\begin{split}
    A &= \mat[r]{1 2 0 -1; -2 -3 4 5; 2 4 0 -2}
      \quad\longsquiggly[rref]\quad
      \mat[r]{1 0 -8 -7; 0 1 4 3; 0 0 0 0} \\
    &\quad
      \longsquiggly[\parbox{\widthof{parametric}}
        {\centering parametric\\vector\\[-2pt]form}]
    \quad \namedbox{x}{x} {}= x_3\vec{8 -4 1 0} + x_4\vec{7 -3 0 1} 
      \;\longsquiggly[\parbox{\widthof{basis of}}{
        \centering basis of\\$\Nul A$\strut}]\;
      \left\{ \vec{8 -4 1 0},\,\vec{7 -3 0 1} \right\}
  \end{split}\]
  \begin{enumerate}
  \item The vectors span $\Nul A$ by construction (every solution to $Ax=0$
    has \namedbox{this}{this} form).
    \tikz[remember picture, overlay, blue!50]
      \draw[->, shorten >=1pt]
        (this.north) .. controls ++(0,1) and ++(0,-2) .. (x.south);
  \item Can you see why they are linearly independent?
    (Look at the last two rows.)
  \end{enumerate}
  \end{webonly}
\end{eg}

\end{frame}


%%%%%%%%%%%%%%%%%%%%%%%%%%%%%%%%%%%%%%%%%%%%%%%%%%%%%%%%%%%%%%%%%%%

\begin{frame}
\frametitle{Basis for $\Col A$}

\begin{bluebox}[Fact]{.8\textwidth}
  The \emph{pivot columns} of $A$ always form a basis for $\Col A$.
\end{bluebox}

\pause\medskip
\alert{Warning:} I mean the pivot columns of the \emph{original} matrix $A$, not
the row-reduced form.
\pause
(Row reduction changes the column space.)

\pause
\begin{eg}\vskip-5mm
  \begin{webonly}
  \[ A = \mat[r]{\namedbox{col1top}{1} \namedbox{col2top}{2} 0 -1; -2 -3 4 5;
    \namedbox{col1bot}{2} \namedbox{col2bot}{4} 0 -2}
  \quad\longsquiggly[rref]\quad
  \mat[r]{1 0 -8 -7; 0 1 4 3; \namedbox{rcol1}{0} \namedbox{rcol2}{0} 0 0} \]
  \begin{tikzpicture}[remember picture, overlay]
    \path ($(rcol1)!.5!(rcol2)$) ++(0,-1cm)
      node[blue!50, font=\small] (rref) {pivot columns in rref};
    \draw[->,blue!50, shorten >=1pt] (rref.north) to[out=90,in=-90] (rcol1.south);
    \draw[->,blue!50, shorten >=1pt] (rref.north) to[out=90,in=-90] (rcol2.south);
  \end{tikzpicture}
  \begin{tikzpicture}[remember picture, overlay]
    \path let \p1=($(col1bot)!.5!(col2bot)$) in (\p1 |- rref.base)
      node[blue!50, font=\small,anchor=base] (orig) {pivot columns $=$ basis};
    \node[fit=(col1top) (col1bot), draw=orange, rounded corners] (col1) {};
    \node[fit=(col2top) (col2bot), draw=orange, rounded corners] (col2) {};
    \draw[->,blue!50, shorten >=1pt] (orig.north) to[out=90,in=-90] (col1.south);
    \draw[->,blue!50, shorten >=1pt] (orig.north) to[out=90,in=-90] (col2.south);
    \draw[->,
        decoration={snake, amplitude=.4mm, segment length=1mm, post length=.5mm},
        decorate]
      (rref.west) -- (orig.east);
  \end{tikzpicture}

  \vskip 1cm
  So a basis for $\Col A$ is 
  \[ \left\{ \vec[r]{1 -2 2},\,\vec[r]{2 -3 4} \right\}. \]
  \end{webonly}
\end{eg}

\pause
\alert{Why?}  End of \S2.8, or ask in office hours.

\end{frame}


%%% Local Variables:
%%% TeX-master: "../slides"
%%% End:
