
\titleframe{Section 1.4}{The Matrix Equation $Ax=b$}

\usetikzlibrary{decorations.pathreplacing}


%%%%%%%%%%%%%%%%%%%%%%%%%%%%%%%%%%%%%%%%%%%%%%%%%%%%%%%%%%%%%%%%%%%

\begin{frame}
\frametitle{Matrix $\times$ Vector}

\vskip3mm
Let $A$ be an $\namedbox{rows}{m}\times \namedbox{cols}{n}$ matrix 
\pause
\begin{tikzpicture}[remember picture, overlay]
  \path (rows.north) ++(-1mm,5mm)
    node[left, blue!50, align=center, text width=2.5cm, inner xsep=1pt] (expl1) 
      {the first number is the number of rows};
  \draw[rounded corners, ->, blue!50, shorten >=1pt]
    (expl1.east) -| (rows.north);
  \path (cols.north) ++(11mm,5mm)
    node[right, blue!50, align=center, text width=3cm, inner xsep=1pt] (expl2) 
      {the second number is the number of columns};
  \draw[rounded corners, ->, blue!50, shorten >=1pt]
    (expl2.west) -| (cols.north);
\end{tikzpicture}%
\pause
\[ A = \mat{ 
| | {} |;
v_1 v_2 \cdots, v_n;
| | {}   |}
\qquad\text{with columns } v_1,v_2,\ldots,v_n
\]

\pause
\begin{defn}
  \def\r{\color<8->{red}}
  The \textbf{product} of $A$ with a vector $x$ in $\R^{\r n}$ is the linear combination
  \[ Ax = 
  \mat{ 
  | | {} |;
  v_1 v_2 \cdots, v_{\namedbox{cols}{\scriptstyle\r n}};
  | | {}   |} 
  \vec{x_1 x_2 \vdots, x_{\namedbox{rows}{\scriptstyle\r n}}}\;
  \namedbox{defeq}{\defeq}\;
  x_1v_1 + x_2v_2 + \cdots + x_{\r n}v_{\r n}.
 \]
 \pause
 \begin{tikzpicture}[remember picture, overlay]
   \node[fit=(defeq), draw, thick, blue!50, rounded corners, inner sep=2pt] (x) {};
   \path (x.north) ++(3mm,4mm)
     node[right, blue!50, align=center, inner xsep=1pt] (expl3)
       {this means the equality\\is a \emph{definition}};
   \draw[->, blue!50, shorten >=1pt] (expl3.west) to[out=180, in=90] (x.north);
   \path<8-> (rows.south) ++(3mm,-4mm)
     node[red, anchor=west, thin border] (eq) {these must be equal};
   \draw<8->[->, shorten >=1pt, red, rounded corners] (eq.west) -| (cols.south);
   \draw<8->[->, shorten >=1pt, red, rounded corners] (eq.west) -| (rows.south);
 \end{tikzpicture}%
 \pause
 The output is a vector in $\R^{\blankuntil{7}{m}}$.
\end{defn}

\pause[8]%
Note that the number of {\color{red}columns} of $A$ has to equal the number of
{\color{red}rows} of $x$.

\pause
\begin{eg}
  \vskip -.5cm
  \[ \mat{
4 5 6;
7 8 9}
\vec{1 2 3}
= \webonlycmd{1\vec{4 7} + 2\vec{5 8} + 3\vec{6 9} = \vec{32 50}.}
 \]
\end{eg}

\end{frame}


%%%%%%%%%%%%%%%%%%%%%%%%%%%%%%%%%%%%%%%%%%%%%%%%%%%%%%%%%%%%%%%%%%%

\begin{frame}
\frametitle{Matrix Equations}
\framesubtitle{An example}

\vskip-3mm
\begin{ques}
  Let $v_1,v_2,v_3$ be vectors in $\R^3$.
  \pause
  How can you write the vector equation
  \[ 2v_1 + 3v_2 - 4v_3 = \vec{7 2 1} \]
  in terms of matrix multiplication?
\end{ques}

\begin{webonly}
  \medskip
  \alert{Answer:}
  Let $A$ be the matrix with colums $v_1,v_2,v_3$, and let $x$ be the vector
  with entries $2,3,-4$.  Then
  \[ Ax = \mat{ | | |; v_1 v_2 v_3; | | |}\vec{2 3 -4}
  = 2v_1 + 3v_2 - 4v_3, \]
  so the vector equation is equivalent to the matrix equation
  \[ Ax = \vec{7 2 1}. \]
\end{webonly}

\end{frame}


%%%%%%%%%%%%%%%%%%%%%%%%%%%%%%%%%%%%%%%%%%%%%%%%%%%%%%%%%%%%%%%%%%%

\begin{frame}
\frametitle{Matrix Equations}
\framesubtitle{In general}

Let $v_1,v_2,\ldots,v_n$, and $b$ be vectors in $\R^m$. 
\pause
Consider the vector equation
\[ x_1v_1 + x_2v_2 + \cdots + x_nv_n = b. \]
\pause
It is equivalent to the \textbf{matrix equation}
\[ Ax = b \]
where
\pause
\[ A = \mat{ | | {} |; v_1 v_2 \cdots, v_n; | | {} |} 
\qquad\text{and}\qquad
x = \vec{x_1 x_2 \vdots, x_n}. \]
\pause
Conversely, if $A$ is any $m\times n$ matrix, then
\[ Ax = b 
\qquad \parbox{.3\textwidth}{\centering is equivalent to the vector equation}
\qquad
\pause
x_1v_1 + x_2v_2 + \cdots + x_nv_n = b \]
\pause
where $v_1,\ldots,v_n$ are
\pause 
the columns of $A$, and $x_1,\ldots,x_n$ are 
\pause
the entries of $x$.
\note{In math, define notation so equations look simpler.}

\end{frame}


%%%%%%%%%%%%%%%%%%%%%%%%%%%%%%%%%%%%%%%%%%%%%%%%%%%%%%%%%%%%%%%%%%%

\begin{frame}
\frametitle{Linear Systems, Vector Equations, Matrix Equations, $\ldots$}

We now have \emph{four} equivalent ways of writing (and thinking about) linear
systems:
\pause
\begin{enumerate}
\item As a system of equations:\namedbox{top}{\strut}
\[ \syseq{2x_1 + 3x_2 = 7; x_1 - x_2 = 5} \]
\vskip-5mm\pause
\item As an augmented matrix:
  \vskip -3mm
  \[ \amat{ 2 3 7; 1 -1 5} \]
\vskip-5mm\pause
\item As a vector equation ($x_1v_1 + \cdots + x_nv_n = b$):
  \[ x_1\vec{2 1} + x_2\vec{3 -1} = \vec{7 5} \]
\vskip-7mm\pause
\item As a matrix equation ($Ax = b$):
  \[ \mat{ 2 3; 1 -1}\vec{x_1 x_2} = \namedbox{bottom}{\vec{7 5}} \]
  \note{(4) is by definition equal to (3).}%
\end{enumerate}
\pause

In particular, \emph{all four have the same solution set}.
\pause
\begin{tikzpicture}[remember picture, overlay]
  \draw[decoration={brace, amplitude=2mm}, decorate, thick, red] 
    let \p1=($(current page.north east) - (3.8cm,0)$) in
      (\p1 |- top.north) -- 
        node[right=2mm, midway, align=justify, text width=3.3cm] 
        {We will move back and forth freely between these over and over again, for the
          rest of the semester.  Get comfortable with them now!}
      (\p1 |- bottom.south);
\end{tikzpicture}

\end{frame}


%%%%%%%%%%%%%%%%%%%%%%%%%%%%%%%%%%%%%%%%%%%%%%%%%%%%%%%%%%%%%%%%%%%

\begin{frame}
\frametitle{Matrix $\times$ Vector}
\framesubtitle{Another way}

\vskip-3mm
\note{Now is as good a time as ever$\ldots$}
\begin{defn}
  A \textbf{row vector} is a matrix with one row.
  \pause
  The product of a row vector of length $n$ and a (column) vector of length $n$
  is
  \[ \mat{ a_1 \cdots, a_n} \vec{x_1 \vdots, x_n}
  \defeq a_1x_1 + \cdots + a_nx_n. \]
  This is a \blankuntil{3}{scalar}.
\end{defn}

\pause[4]\medskip
If $A$ is an $m\times n$ matrix with rows $r_1,r_2,\ldots,r_{\blankuntil{5}m}$,
\pause
and $x$ is a vector in $\R^n$, then
\[ Ax = 
\mat[c]{ \matrow{r_1};
     \matrow{r_2};
     \vdots ;
     \matrow{r_m}}
     x
= \vec{r_1x r_2x \vdots, r_mx}
 \]
This is \pause a vector in $\R^m$ (again).

\end{frame}


%%%%%%%%%%%%%%%%%%%%%%%%%%%%%%%%%%%%%%%%%%%%%%%%%%%%%%%%%%%%%%%%%%%

\begin{frame}
\frametitle{Matrix $\times$ Vector}
\framesubtitle{Both ways}

\vskip-3mm
\begin{eg}
\vskip-5mm
  \[ 
\mat{
4 5 6;
7 8 9}
\vec{1 2 3} = 
\begin{webonly}
\vec{{\scriptscriptstyle(\,4\; 5\; 6\,)
      \Bigl(\smallveciii 123 \Bigr)}
    {\scriptscriptstyle(\,7\; 8\; 9\,)
      \Bigl(\smallveciii 123 \Bigr)}} 
  = \vec{4\cdot1+5\cdot2+6\cdot3 7\cdot1+8\cdot2+9\cdot3}
  = \vec{32 50}.
\end{webonly}
 \]
\pause
 Note this is the same as before:\\[\abovedisplayskip]
\begin{webonly}
  \hbox to \linewidth{\hss
$\displaystyle\mat{
4 5 6;
7 8 9}
\vec{1 2 3}
= 1\vec{4 7} + 2\vec{5 8} + 3\vec{6 9}
= \vec{1\cdot4+2\cdot5+3\cdot6 1\cdot7+2\cdot8+3\cdot9}
= \vec{32 50}.
$\hss}
\end{webonly}
\end{eg}

\pause
\begin{bluebox}{.75\linewidth}
  Now you have \emph{two} ways of computing $Ax$.  

  \pause\medskip
  In the second, you calculate $Ax$ one entry at a time.

  \pause\medskip
  The second way is usually the most convenient, but we'll use both.
\end{bluebox}

\end{frame}


%%%%%%%%%%%%%%%%%%%%%%%%%%%%%%%%%%%%%%%%%%%%%%%%%%%%%%%%%%%%%%%%%%%

\begin{frame}
\frametitle{Spans and Solutions to Equations}

Let $A$ be a matrix with columns $v_1,v_2,\ldots,v_n$:
\[ A = \mat{ | | {} |; v_1 v_2 \cdots, v_n; | | {} |} \]

\pause\smallskip
\begin{bluebox}
  [Very Important Fact That Will Appear on Every Midterm and the Final]
  {.93\textwidth}
  \mode<handout>{\setbeamercovered{transparent}}
  \vskip-1.5mm
  \[ \begin{split}
    Ax = b &\text{ has a solution}  \\[-5mm]
    &\uncover<3-7| handout:0>{\namedbox{iff}{{}\iff{}} \text{there exist } x_1,\ldots,x_n \text{ such that }
    A\vec{x_1 x_2 \vdots, x_n} = b} \\
    &\uncover<5-7| handout:0>{\iff \text{there exist } x_1,\ldots,x_n \text{ such that }
    x_1v_1 + \cdots + x_nv_n = b} \\
    &\uncover<6-7| handout:0>{\iff b \text{ is a linear combination of } v_1,\ldots,v_n} \\
    &\uncover<7->{\namedbox{iff2}{{}\iff{}} b \text{ is in the span of the columns of } A.}
  \end{split}\]
\end{bluebox}

\uncover<9->{The last condition is geometric.}

\begin{tikzpicture}[remember picture, overlay]
  \node<4->[below=.4cm of iff.west, xshift=-8mm, blue!50] (ifflabel)
    {``if and only if''};
  \draw<4-7| handout:0>[->, shorten >=-1.5mm, blue!50]
    (ifflabel) to[in=180,out=90] (iff.west);
  \draw<5-7| handout:0>[->, shorten >=-1.5mm, blue!50]
    (ifflabel) to[in=180,out=90] (iff.west);
  \draw<8->[->, shorten >=-1.5mm, blue!50]
    (ifflabel) to[in=180,out=-90] (iff2.west);
\end{tikzpicture}
\note{Define $\iff$}

\end{frame}


%%%%%%%%%%%%%%%%%%%%%%%%%%%%%%%%%%%%%%%%%%%%%%%%%%%%%%%%%%%%%%%%%%%

\begin{frame}
\frametitle{Spans and Solutions to Equations}
\framesubtitle{Example}

\vskip-3mm
\begin{ques}
  Let $A = \mat[r]{2 1; -1 0; 1 -1}$.
  Does the equation $Ax = \vec{0 2 2}$ have a solution?
\end{ques}

\pause
\begin{center}
\begin{tikzpicture}[myxyz, scale=0.5,
    thin border nodes, baseline]
  \path[clip, resetxy] (-4,-4) rectangle (4,4);

  \def\v{(2,-1,1)}
  \def\w{(1,0,-1)}
  \def\b{(0,2,2)}

  \node[coordinate] (X) at \v {};
  \node[coordinate] (Y) at \w {};

  \begin{scope}[x=(X), y=(Y), transformxy]
    \path[clip] (-5, -5) -- (5, 5) -- (5, 10) -- (-5, 10) -- cycle;
    \fill<4->[seq4!30, nearly opaque] (-1.5,-1) rectangle (1.5,2);
    \draw<4->[step=.5cm, seq4, very thin] (-1.5,-1) grid (1.5,2);
  \end{scope}

  \begin{scope}[transformxy]
    \fill[white, nearly opaque] (-2, -2) rectangle (3, 3);
    \draw[grid lines] (-2, -2) grid (3, 3);
  \end{scope}

  \begin{scope}[x=(X), y=(Y), transformxy]
    \path[clip] (-5, -5) -- (5, 5) -- (5, -10) -- (-5, -10) -- cycle;
    \fill<4->[seq4!30, nearly opaque] (-1.5,-1) rectangle (1.5,2);
    \draw<4->[step=.5cm, seq4, very thin] (-1.5,-1) grid (1.5,2);
  \end{scope}

  \draw[vector, seq1] (0,0,0) -- 
    node [midway, above left] {$v$} \v;
  \draw[thin, densely dotted] \v -- \projxy\v;

  \draw[vector, seq2!50!white] (0,0,0) -- 
    node [midway, below left] {$w$} \w;
  \draw[thin, densely dotted, black!50!white] \w -- \projxy\w;

  \draw<3->[vector, seq3] (0,0,0) -- 
    node [midway, above right] {$b$} \b;
  \draw<3->[thin, densely dotted] \b -- \projxy\b;

  \node<4->[seq4] at (-1.5cm, 2cm) {$\Span\{v,w\}$};

  \point at (0,0,0);

\end{tikzpicture}
\qquad
\begin{minipage}[c]{0.5\textwidth}
  Columns of $A$:
  \[ \color{seq1} v = \vec{2 -1 1} \qquad 
  \color{seq2} w = \vec{1 0 -1} \]
  \begin{uncoverenv}<3->
    Output vector:
    \[ \color{seq3} b = \vec{0 2 2} \]
  \end{uncoverenv}
\end{minipage}

\end{center}

\pause[5]\smallskip%
Is $b$ contained in the span of the columns of $A$?
\pause
It sure doesn't look like it.

\pause\medskip
\alert{Conclusion:} $Ax=b$ is \emph{inconsistent}.

\end{frame}


%%%%%%%%%%%%%%%%%%%%%%%%%%%%%%%%%%%%%%%%%%%%%%%%%%%%%%%%%%%%%%%%%%%

\begin{frame}
\frametitle{Spans and Solutions to Equations}
\framesubtitle{Example, continued}

\vskip-3mm
\begin{ques}
  Let $A = \mat[r]{2 1; -1 0; 1 -1}$.
  Does the equation $Ax = \vec{0 2 2}$ have a solution?
\end{ques}

\medskip
\alert{Answer:}
Let's check by solving the matrix equation using row reduction.  

\begin{webonly}
\medskip
The first step is to put the system into an augmented matrix.
\[  \amat{2 1 0; -1 0 2; 1 -1 2}
\;\longsquiggly[row reduce]\;
  \amat{1 0 0; 0 1 0; 0 0 1} \]
The last equation is $0=1$, so the system is \emph{inconsistent}.
\end{webonly}

\pause\medskip
In other words, the matrix equation 
\[ \mat[r]{2 1; -1 0; 1 -1}x = \vec{0 2 2} \]
has no solution, as the picture shows.

\end{frame}


%%%%%%%%%%%%%%%%%%%%%%%%%%%%%%%%%%%%%%%%%%%%%%%%%%%%%%%%%%%%%%%%%%%

\begin{frame}
\frametitle{Spans and Solutions to Equations}
\framesubtitle{Example}

\vskip-3mm
\begin{ques}
  Let $A = \mat[r]{2 1; -1 0; 1 -1}$.
  Does the equation $Ax = \vec{1 -1 2}$ have a solution?
\end{ques}

\pause
\begin{center}
\begin{tikzpicture}[myxyz, scale=0.5,
    thin border nodes, baseline]
  \path[clip, resetxy] (-4,-4) rectangle (4,4);

  \def\v{(2,-1,1)}
  \def\w{(1,0,-1)}
  \def\b{(1,-1,2)}

  \node[coordinate] (X) at \v {};
  \node[coordinate] (Y) at \w {};

  \begin{scope}[x=(X), y=(Y), transformxy]
    \path[clip] (-5, -5) -- (5, 5) -- (5, 10) -- (-5, 10) -- cycle;
    \fill<4->[seq4!30, nearly opaque] (-1.5,-1.5) rectangle (1.5,2);
    \draw<4->[step=.5cm, seq4, very thin] (-1.5,-1.5) grid (1.5,2);
  \end{scope}

  \begin{scope}[transformxy]
    \fill[white, nearly opaque] (-2, -2) rectangle (3, 3);
    \draw[help lines] (-2, -2) grid (3, 3);
  \end{scope}

  \begin{scope}[x=(X), y=(Y), transformxy]
    \path[clip] (-5, -5) -- (5, 5) -- (5, -10) -- (-5, -10) -- cycle;
    \fill<4->[seq4!30, nearly opaque] (-1.5,-1.5) rectangle (1.5,2);
    \draw<4->[step=.5cm, seq4, very thin] (-1.5,-1.5) grid (1.5,2);
  \end{scope}

  \draw[vector, seq1] (0,0,0) -- 
    node [midway, below right] {$v$} \v;
  \draw[thin, densely dotted] \v -- \projxy\v;

  \draw[vector, seq2!50!white] (0,0,0) -- 
    node [midway, below left] {$w$} \w;
  \draw[thin, densely dotted, black!50!white] \w -- \projxy\w;

  \draw<3->[vector, seq3] (0,0,0) -- 
    node [midway, above left] {$b$} \b;
  \draw<3->[thin, densely dotted] \b -- \projxy\b;

  \node<4->[seq4] at (-1.75cm, 2.45cm) {$\Span\{v,w\}$};

  \point at (0,0,0);

\end{tikzpicture}
\qquad
\begin{minipage}[c]{0.5\textwidth}
  Columns of $A$:
  \[ \color{seq1} v = \vec{2 -1 1} \qquad 
  \color{seq2} w = \vec{1 0 -1} \]
  \begin{uncoverenv}<3->
    Solution vector:
    \[ \color{seq3} b = \vec{1 -1 2} \]
  \end{uncoverenv}
\end{minipage}
\end{center}

\pause[5]
Is $b$ contained in the span of the columns of $A$?
\pause
It looks like it: in fact,
\[ b = \blankuntil{7}{1} v + \blankuntil{7}{(-1)}w
\implies
x = \vec{\uncover<8->1 \uncover<8->{{-1}}}.
 \]

\end{frame}


%%%%%%%%%%%%%%%%%%%%%%%%%%%%%%%%%%%%%%%%%%%%%%%%%%%%%%%%%%%%%%%%%%%

\begin{frame}
\frametitle{Spans and Solutions to Equations}
\framesubtitle{Example, continued}

\vskip-3mm
\begin{ques}
  Let $A = \mat[r]{2 1; -1 0; 1 -1}$.
  Does the equation $Ax = \vec{1 -1 2}$ have a solution?
\end{ques}

\medskip
\alert{Answer:}
Let's do this systematically using row reduction.
\begin{webonly}
  \[ \amat{2 1 1; -1 0 -1; 1 -1 2}
  \;\longsquiggly[row reduce]\;
  \amat{1 0 1; 0 1 -1; 0 0 0} \]
This gives us
\[ x = 1\qquad y = -1. \]
\end{webonly}

\pause
This is consistent with the picture on the previous slide:
\[ 1\vec{2 -1 1} -1\vec{1 0 -1} = \vec{1 -1 2}
\qquad\text{or}\qquad\pause
A\vec{1 -1} = \vec{1 -1 2}. \]

\end{frame}


%%%%%%%%%%%%%%%%%%%%%%%%%%%%%%%%%%%%%%%%%%%%%%%%%%%%%%%%%%%%%%%%%%%

\begin{pollframe}

\smallskip
\begin{bluebox}[Poll]{.8\textwidth}
  Which of the following true statements can be checked by eyeballing them,
  \emph{without} row reduction?
  \medskip
  \begin{eAlpherate}
  \item $\vec{0 1 2}$ is in the span of
    $\;\vec{3 3 4},\; \vec{0 10 20},\; \vec{0 -1 -2}$.
  \item $\vec{0 1 2}$ is in the span of
    $\;\vec{3 3 4},\; \vec{0 5 7},\; \vec{0 6 8}$.
  \item $\vec{0 1 2}$ is in the span of
    $\;\vec{3 3 4},\; \vec{0 1 0},\; \vec{0 0 \sqrt2}$.
  \item $\vec{0 1 2}$ is in the span of
    $\;\vec{5 7 0},\; \vec{6 8 0},\; \vec{3 3 4}$.
  \end{eAlpherate}
\end{bluebox}

\note[item]{A: use negative the third}
\note[item]{C: use the second and third}
\note[item]{For the other two, draw the span (especially B)}

\end{pollframe}


%%%%%%%%%%%%%%%%%%%%%%%%%%%%%%%%%%%%%%%%%%%%%%%%%%%%%%%%%%%%%%%%%%%

\begin{frame}
\frametitle{When Solutions Always Exist}

Here are criteria for a linear system to always have a solution.

\begin{thm}
  Let $A$ be an $m\times n$ (non-augmented) matrix.  The following are
  \namedbox{equivalent}{equivalent}
  \pause
  \begin{tikzpicture}[remember picture, overlay]
    \path (equivalent.south) ++(-1.5mm,-5mm)
      node[blue!50, align=center, anchor=north] (expl)
        {recall that this means\\
          that for given $A$, either they're\\
          all true, or they're all false};
    \draw[->, blue!50] (expl.north) to[out=90,in=-90] (equivalent.south);
  \end{tikzpicture}
  \pause
  \begin{enumerate}
  \item $Ax = b$ has a solution for all $b$ in $\R^{\blankuntil{4}m}$.
    \pause[5]
  \item The span of the columns of $A$ is all of $\R^m$.
    \pause
  \item A has a pivot in each row.
  \end{enumerate}
\end{thm}

\note{Show previous slide for definition of $Ax$ to see why it's $m$ entries.}
\pause
Why is (1) the same as (2)?
\pause
This was the 
\tikz[baseline] \node[bluebox, inner sep=2mm, anchor=base] 
  {Very Important};
box from before.

\pause\medskip
Why is (1) the same as (3)?
\pause
If $A$ has a pivot in each row then its reduced row echelon form looks like this:
\[ \mat{
1   0   \star,   0   \star ;
0   1   \star , 0   \star ;
0   0   0   1   \star 
}
\pause\quad\parbox{.2\textwidth}
{\centering and $(\,A\mid b\,)$ reduces to this:}\quad
\amat[c]{
1   0   \star,   0   \star, \star ;
0   1   \star , 0   \star, \star ;
0   0   0   1   \star, \star
}.\]
\pause
There's no $b$ that makes it inconsistent, so there's always a solution.
\pause
If $A$ doesn't have a pivot in each row, then its reduced form looks like this:
\pause
\[ \mat{
1   0   \star,   0   \star ;
0   1   \star , 0   \star ;
0   0   0   0  0 
}
\pause\quad\parbox{.2\textwidth}
{\centering and this can be made inconsistent:}\quad
\amat{
1   0   \star,   0   \star, 0 ;
0   1   \star , 0   \star, 0 ;
0   0   0   0   0 16
}.
\]

\end{frame}


%%%%%%%%%%%%%%%%%%%%%%%%%%%%%%%%%%%%%%%%%%%%%%%%%%%%%%%%%%%%%%%%%%%

\begin{frame}
\frametitle{Properties of the Matrix--Vector Product}

\begin{bluebox}{.65\textwidth}
  Let $c$ be a scalar, $u,v$ be vectors, and $A$ a matrix.

  \pause\smallskip
  \begin{itemize}
  \item $A(u+v) = Au + Av$
    \pause
  \item $A(cv) = cAv$
  \end{itemize}

  \pause\smallskip
  {\small\color{black!60}See Lay, \S1.4, Theorem~5.}
\end{bluebox}

\webonlycmd{For instance, $A(3u-7v) = 3Au - 7Av$.}

\bigskip
\begin{uncoverenv}<5->
  \alert{Consequence:}
  If $u$ and $v$ are solutions to $Ax = 0$, then so is every vector in 
  $\Span\{u,v\}$.
  \pause[6]
  Why?
  \begin{webonly}
  \[ \left\{\syseq*{Au = 0; Av = 0}\right.
  \quad\implies\quad
  A(xu+yv) = xAu + yAv = x0 + y0 = 0. \]
  (Here $0$ means the zero vector.)
  \end{webonly}

  \pause

  \begin{bluebox}[Important]{.65\textwidth}
    The set of solutions to $Ax = \color{red}0$ is a span.
  \end{bluebox}

\end{uncoverenv}

\end{frame}


%%% Local Variables:
%%% TeX-master: "../slides"
%%% End:
