
% JDR: The first 1/3 of this lecture can be safely skipped if you want to get to
%   1.2 more quickly.

\titleframe{Chapter 1}{Linear Equations}
\titleframe{Section 1.1}{Systems of Linear Equations}

\def\rowop#1#2{%
  \hfill%
  \hbox to 0.2\linewidth{\hss\longsquiggly[#1]}%
  \hbox to 0.4\linewidth{\hss#2}%
  }

%%%%%%%%%%%%%%%%%%%%%%%%%%%%%%%%%%%%%%%%%%%%%%%%%%%%%%%%%%%%%%%%%%%

% Disable a bunch of these slides to get to 1.2 quicker
%\iffalse
\begin{frame}
\frametitle{One Linear Equation}

What does the solution set of a linear equation look like?

\pause\medskip
\begin{columns}[onlytextwidth]
  \column[t]{.6\linewidth}
  \vskip-\baselineskip
  \begin{itemize}
  \item $x+y=1$ 
    \pause\\[1mm]
    $\longsquiggly$ a line in the plane: \color{seq1}$y = 1-x$
  \end{itemize}
  \column[t]{.4\linewidth}\centering
  \begin{tikzpicture}[scale=.5, picture align top]
    \draw[grid lines] (-2,-2) grid (2,2);
    \draw[->] (-2,0) -- (2,0);
    \draw[->] (0,-2) -- (0,2);
    \draw[seq1, thick] (-1,2) -- (2,-1);
  \end{tikzpicture}
\end{columns}
\pause
\begin{columns}[onlytextwidth]
  \column[t]{.6\linewidth}
  \vskip-\baselineskip
  \begin{itemize}
  \item $x + y + z = 1$
    \pause\\[1mm]
    $\longsquiggly$ a plane in space: \color{seq1}$z = 1-x-y$
  \end{itemize}
  \column[t]{.4\linewidth}\centering
  \begin{tikzpicture}[scale=.25, picture align top, myxyz]
    \draw[densely dotted] (0,0,0) -- (1,0,0);
    \draw[densely dotted] (0,0,0) -- (0,1,0);
    \draw[densely dotted] (0,0,0) -- (0,0,1);
    \begin{scope}[x={(-1,0,1)}, y={(0,-1,1)}, transformxy=1]
      \fill[help lines, seq1!50, fill opacity=.7] (-2, -2) rectangle (2, 2);
      \draw[help lines, seq1!50] (-2, -2) grid (2, 2);
    \end{scope}
    \point[scale=.5] at (1,0,0);
    \point[scale=.5] at (0,1,0);
    \point[scale=.5] at (0,0,1);
    \draw[->] (1,0,0) -- (2,0,0) node[right] {$x$};
    \draw[->] (0,1,0) -- (0,2,0) node[below left] {$y$};
    \draw[->] (0,0,1) -- (0,0,2) node[above] {$z$};
  \end{tikzpicture}
\end{columns}
\pause
\begin{columns}[onlytextwidth]
  \column[c]{.6\linewidth}
  \begin{itemize}
  \item $x + y + z + w = 1$
    \pause\\[1mm]
    $\longsquiggly$ a ``$3$-plane'' in ``$4$-space''\ldots
  \end{itemize}
  \column[c]{.4\linewidth}\centering\color{black!50}
  [not pictured here]
\end{columns}
\end{frame}


%%%%%%%%%%%%%%%%%%%%%%%%%%%%%%%%%%%%%%%%%%%%%%%%%%%%%%%%%%%%%%%%%%%

\begin{frame}
\frametitle{Systems of Linear Equations}

What does the solution set of a \emph{system} of more than one linear equation
look like?

\pause\vfill
\begin{columns}[onlytextwidth]
  \column[t]{.4\linewidth}
  \[\begin{split}
    \color{seq1}x - 3y &\color{seq1}= \color{seq1}-3 \\
    \color{seq2}2x + y &\color{seq2}= \color{seq2}8
  \end{split}\]
  \webonlycmd{\noindent
  \ldots is the \emph{intersection} of two lines,
  which is a \emph{point} in this case.}
  \note{Find the intersection point}
  \pause
  \column[t]{.6\linewidth}\centering
  \begin{tikzpicture}[picture align top, scale=.095]
    \draw[grid lines] (-20,-20) rectangle (20,20);
    \clip (-20,-20) rectangle (20,20);
    \draw[->] (-20,0) -- (20,0);
    \draw[->] (0,-20) -- (0,20);
    \draw[seq1, thick] (-20,-20/3+1) -- (20,20/3+1);
    \draw[seq2, thick] (-20,40+8) -- (20,-40+8);
    \point at (3,2);
  \end{tikzpicture}
\end{columns}

\pause
\bigskip
In general it's an intersection of lines, planes, etc.

\vfill

\end{frame}


%%%%%%%%%%%%%%%%%%%%%%%%%%%%%%%%%%%%%%%%%%%%%%%%%%%%%%%%%%%%%%%%%%%

\begin{frame}
\frametitle{Kinds of Solution Sets}
  
In what other ways can two lines intersect?

\pause\vfill
\begin{columns}[onlytextwidth]
  \column[t]{.4\linewidth}
    \[\begin{split}
      \color{seq1}x - 3y &\color{seq1}= \color{seq1}-3 \\
      \color{seq2}x - 3y &\color{seq2}= \color{seq2}3
    \end{split}\]
    \webonlycmd{\noindent
    has no solution: the lines are \emph{parallel}.}
  \note[item]{Find the lack of intersection point}
  \pause
  \column[t]{.6\linewidth}\centering
  \begin{tikzpicture}[picture align top, scale=.095]
    \draw[grid lines] (-20,-20) rectangle (20,20);
    \clip (-20,-20) rectangle (20,20);
    \draw[->] (-20,0) -- (20,0);
    \draw[->] (0,-20) -- (0,20);
    \draw[seq1, thick] (-20,-20/3+1) -- (20,20/3+1);
    \draw[seq2, thick] (-20,-20/3-1) -- (20,20/3-1);
  \end{tikzpicture}
\end{columns}

\pause
\bigskip
A system of equations with no solutions is called \textbf{inconsistent}.
\note[item]{Obviously inconsistent in this case}

\vfill

\end{frame}


%%%%%%%%%%%%%%%%%%%%%%%%%%%%%%%%%%%%%%%%%%%%%%%%%%%%%%%%%%%%%%%%%%%

\begin{frame}
\frametitle{Kinds of Solution Sets}
  
In what other ways can two lines intersect?

\pause\vfill

\begin{columns}[onlytextwidth]
  \column[t]{.4\linewidth}
  \[\begin{split}
    \color{seq1}x - 3y &\color{seq1}= \color{seq1}-3 \\
    \color{seq2}2x - 6y &\color{seq2}= \color{seq2}-6
  \end{split}\]
  \webonlycmd{\noindent
  has infinitely many solutions: they are the \emph{same line}.}
  \pause
  \column[t]{.6\linewidth}\centering
  \begin{tikzpicture}[picture align top, scale=.095]
    \draw[grid lines] (-20,-20) rectangle (20,20);
    \clip (-20,-20) rectangle (20,20);
    \draw[->] (-20,0) -- (20,0);
    \draw[->] (0,-20) -- (0,20);
    \draw[seq2, double=seq1!50, double distance=1pt, thick] (-20,-20/3+1) -- (20,20/3+1);
  \end{tikzpicture}
\end{columns}

\pause
\bigskip
Note that multiplying an equation by a nonzero number gives the
\emph{same solution set}.
\pause In other words, they are \emph{equivalent} (systems
of) equations.

\vfill

\end{frame}


%%%%%%%%%%%%%%%%%%%%%%%%%%%%%%%%%%%%%%%%%%%%%%%%%%%%%%%%%%%%%%%%%%%

\begin{pollframe}

What about in three variables?
\pause
\bigskip

\begin{poll}
\begin{bluebox}[Poll]{.65\textwidth}
  In how many different ways can three planes intersect in space?

  \smallskip
  \begin{eAlpherate}
  \item One
  \item Two
  \item Three
  \item Four
  \item Five
  \item Six
  \item Seven
  \item Eight
  \end{eAlpherate}
\end{bluebox}
\end{poll}

\note{Think: how many ``ways'' can three lines intersect in $\R^2$?}

\end{pollframe}


%%%%%%%%%%%%%%%%%%%%%%%%%%%%%%%%%%%%%%%%%%%%%%%%%%%%%%%%%%%%%%%%%%%

\iffalse % This is a poll question in 1.2
\begin{frame}
\frametitle{Number of Solutions}

\begin{bluebox}{.65\textwidth}
  In general, a system of linear equations can have:

  \smallskip
  \begin{itemize}
    \pause
  \item Zero solutions (\emph{inconsistent});
    \pause
  \item One solution;
    \pause
  \item Infinitely many solutions.
  \end{itemize}
\end{bluebox}

\onslide<5->{We will verify this later on.}

\end{frame}
\fi
%\fi


%%%%%%%%%%%%%%%%%%%%%%%%%%%%%%%%%%%%%%%%%%%%%%%%%%%%%%%%%%%%%%%%%%%
% JDR: can start lecture here

\begin{frame}
\frametitle{Solving Systems of Equations}

\vskip-3mm
\begin{eg}
  Solve the system of equations
  \[\syseq{x + 2y + 3z = 6;
    2x - 3y + 2z = 14;
    3x + y - z = -2
  }\]
\end{eg}

This is the kind of problem we'll talk about for the first half of the course.

\pause

\begin{bluebox}{.6\textwidth}
  \begin{itemize}
  \item A \textbf{solution} is a list of numbers $x,y,z,\ldots$ that make
    \emph{all} of the equations true.
    \pause
  \item The \textbf{solution set} is the collection of all solutions.
    \pause
  \item \textbf{Solving} the system means finding the solution set.
  \end{itemize}
\end{bluebox}

\uncover<5->{What is a \emph{systematic} way to solve a system of equations?}

\end{frame}


%%%%%%%%%%%%%%%%%%%%%%%%%%%%%%%%%%%%%%%%%%%%%%%%%%%%%%%%%%%%%%%%%%%

\begin{frame}
\frametitle{Solving Systems of Equations}

\vskip-3mm
\begin{eg}
  Solve the system of equations
  \[\syseq{x + 2y + 3z = 6;
    2x - 3y + 2z = 14;
    3x + y - z = -2
  }\]
\end{eg}

\pause
What strategies do you know?

\begin{webonly}
  \begin{itemize}
  \item Substitution
  \item Elimination
  \end{itemize}

  Both are perfectly valid, but only elimination scales well to large numbers of
  equations.
\end{webonly}

\end{frame}


%%%%%%%%%%%%%%%%%%%%%%%%%%%%%%%%%%%%%%%%%%%%%%%%%%%%%%%%%%%%%%%%%%%

\begin{frame}
\frametitle{Solving Systems of Equations}

\vskip-3mm
\begin{eg}
  Solve the system of equations
  \[\syseq{x + 2y + 3z = 6;
    2x - 3y + 2z = 14;
    3x + y - z = -2
  }\]
\end{eg}

\alert{Elimination method:} in what ways can you manipulate the equations?

\vfill

\begin{webonly}
  \begin{itemize}
  \item Multiply an equation by a nonzero number.
      \hfill{\bf\color{seq-blue}(scale)}
  \item Add a multiple of one equation to another.
      \hfill{\bf\color{seq-blue}(replacement)} 
  \item Swap two equations.
      \hfill{\bf\color{seq-blue}(swap)}
  \end{itemize}
\end{webonly}
\vfill

\end{frame}


%%%%%%%%%%%%%%%%%%%%%%%%%%%%%%%%%%%%%%%%%%%%%%%%%%%%%%%%%%%%%%%%%%%

\begin{frame}
\frametitle{Solving Systems of Equations}

\vskip-3mm
\begin{eg}
  Solve the system of equations
  \[\syseq{x + 2y + 3z = 6;
    2x - 3y + 2z = 14;
    3x + y - z = -2
  }\]
\end{eg}\pause

\begin{webonly}
\leavevmode
\hbox to .4\textwidth{\hss\longsquiggly[\color{red}Multiply first by $-3$]}
\hskip 5mm
  $\syseq{-3x - 6y - 9z = -18;
    2x - 3y + 2z = 14;
    3x + y - z = -2
  }$\\[4mm]
\leavevmode
\hbox to .4\textwidth{\hss\longsquiggly[\color{red}Add first to third]}
\hskip 5mm
  $\syseq{-3x - 6y - 9z = -18;
    2x - 3y + 2z = 14;
    \. \+ -5y - 10z = -20
  }$
\end{webonly}

\bigskip

Now I've eliminated $x$ from the last equation!

\pause\medskip
~\ldots but there's a long way to go still.  Can we make our lives easier?

\end{frame}


%%%%%%%%%%%%%%%%%%%%%%%%%%%%%%%%%%%%%%%%%%%%%%%%%%%%%%%%%%%%%%%%%%%

\begin{frame}
\frametitle{Solving Systems of Equations}
\framesubtitle{Better notation}

It sure is a pain to have to write $x, y, z,$ and $=$ over and over again.
\pause

\bigskip
\alert{Matrix notation:} write just the numbers, in a box, instead!

\begin{center}
\bigskip
  $\syseq{x + 2y + 3z = 6;
    2x - 3y + 2z = 14;
    3x + y - z = -2
  }$
~\longsquiggly[becomes]~
$\amat{
1  2  3  6 ;
2  -3  2  14; 
3  1  -1  -2
}$
\end{center}

\pause\medskip
This is called an \textbf{(augmented) matrix}.  Our equation manipulations
become \textbf{elementary row operations}:

\note[item]{Augmented means ``remember there was an $=$ there.''}

\pause\medskip
\begin{itemize}
  \item Multiply all entries in a row by a nonzero number.
    {\hfill\bf\color{seq-blue}(scale)}
    \pause
  \item Add a multiple of each entry of one row to the
    corresponding entry in another.
      \hfill{\bf\color{seq-blue}(row replacement)}
    \pause
  \item Swap two rows.
      {\hfill\bf\color{seq-blue}(swap)}
\end{itemize}

\end{frame}


%%%%%%%%%%%%%%%%%%%%%%%%%%%%%%%%%%%%%%%%%%%%%%%%%%%%%%%%%%%%%%%%%%%

\begin{frame}
\frametitle{Row Operations}

\vskip-3mm
\begin{eg}
  Solve the system of equations
  \[\syseq{x + 2y + 3z = 6;
    2x - 3y + 2z = 14;
    3x + y - z = -2
  }\]
\end{eg}

\pause

Start:
\[\amat{
1   2   3   6 ;
2   -3   2   14 ;
3   1   -1   -2
}\]

\pause
\alert{Goal:} we want our elimination method to eventually produce a system of
equations like\\
\begin{center}
$\syseq{x \+ \. \+ \. = A; \. \+ y \+ \. = B; \. \+ \. \+ z = C}$
\pause
\hskip5mm or in matrix form,\hskip5mm
\webonlycmd{$\amat{
1   0   0   A ;
0   1   0   B ;
0   0   1   C
}$}
\end{center}
\pause
So we need to do row operations that make the start matrix look like the end
one.

\pause\smallskip
\alert{Strategy:} fiddle with it so we only have ones and zeros.
\end{frame}

%%%%%%%%%%%%%%%%%%%%%%%%%%%%%%%%%%%%%%%%%%%%%%%%%%%%%%%%%%%%%%%%%%%

\begin{frame}
\frametitle{Row Operations}
\framesubtitle{Continued}

\def\r{\color{red}}
$\amat{
\color<3->{red}1   2   3   6 ;
\namedbox{a21}{2}   -3   2   14 ;
\namedbox{a31}{3}   1   -1   -2
}$%
\begin{tikzpicture}[remember picture, overlay]
  \node<2-> [draw,thick,rounded corners,blue!50,fit=(a21) (a31)] (x) {};
  \node<2-> [blue!50, below=5mm of x, xshift=1.5cm, align=center] (expl) 
    {We want these to be zero.\\
     {\uncover<3->{So we subract multiples of the first row.}}};
  \draw<2->[->, blue!50] (expl.north) to[out=90,in=-90] (x.south);
\end{tikzpicture}%
\begin{webonly}%
\rowop{$R_2 = R_2-2R_1$}{$\amat{
      1   2   3   6 ;
      \r0   -7   -4   2 ;
      3   1   -1   -2
    }$}\\
\rowop{$R_3 = R_3-3R_1$}{$\amat{
      1   2   3   6 ;
      0   -7   -4   2 ;
      \r0   -5   -10   -20
    }$}
\end{webonly}%

\pause[4]\bigskip
$\amat{
  1\phantom-   \namedbox{b12}{\phantom-2}   3   6 ;
  0\phantom-   \namedbox{b22}{\color<6->{seq-green}-7}   -4   2 ;
  0\phantom-   \namedbox{b32}{-5}   -10   -20
}$%
\begin{tikzpicture}[remember picture, overlay]
  \node<5-> [draw, thick, rounded corners, blue!50, fit=(b12)] (x) {};
  \node<5-> [draw, thick, rounded corners, blue!50, fit=(b32)] (y) {};
  \node<5-> [blue!50, below=5mm of y] (expl) {We want these to be zero.};
  \draw<5-> [->, blue!50] 
    let \p1=($(x.west) - (3mm,0)$) in
      (expl.north -| \p1) -- (x.south west -| \p1) to[out=90,in=180] (x.west);
  \draw<5-> [->, blue!50] 
    let \p1=($(y.west) - (3mm,0)$) in
      (expl.north -| \p1) -- (y.south west -| \p1) to[out=90,in=180] (y.west);
  \node<6-> [seq-green, below=1mm of expl, xshift=5mm, align=center] (expl2) 
    {It would be nice if this were a $1$.\\
     {\uncover<7->{We could divide by $-7$, but that}}\\
     {\uncover<7->{would produce ugly fractions.}}\\[3mm]
     {\uncover<8->{\color{black}Let's swap the last two rows first.}}};
  \draw<6-> [->, seq-green] 
    let \p1=($(b22.east) + (3mm,0)$) in
      ($(expl2.north -| expl.east) + (1mm,0)$) 
        -- ($(expl.east) + (1mm,0)$)
        to[out=90,in=0] ($(expl.north east) + (-2mm,1mm)$)
        -- ($(expl.north -| \p1) + (2mm,1mm)$)
        to[out=180,in=-90] ($(expl.north -| \p1) + (0,3mm)$)
        -- ($(b22.south east -| \p1)-(0,1mm)$) 
        to[out=90,in=0] (b22.east);
\end{tikzpicture}%
\begin{webonly}%
\rowop{$R_2\ToT R_3$}{$\amat{
    1   2   3   6 ;
    0   -5   -10   -20 ;
    0   -7   -4   2
  }$}\\
\rowop{$R_2 = R_2\divsymb-5$}{$\amat{
      1   2   3   6 ;
      0   \r1   2   4 ;
      0   -7   -4   2 
    }$}\\
\rowop{$R_1 = R_1-2R_2$}{$\amat{
      1   \r0   -1   -2 ;
      0   1   2   4 ;
      0   -7   -4   2 
    }$}\\
\rowop{$R_3 = R_3+7R_2$}{$\amat{
      1   0   -1   -2 ;
      0   1   2   4 ;
      0   \r0   10   30
    }$}
\end{webonly}%

\end{frame}


%%%%%%%%%%%%%%%%%%%%%%%%%%%%%%%%%%%%%%%%%%%%%%%%%%%%%%%%%%%%%%%%%%%

\begin{frame}
\frametitle{Row Operations}
\framesubtitle{Continued}
\def\r{\color{red}}

$\amat{
    1   0   \namedbox{c13}{-1}   -2 ;
    0   1   \namedbox{c23}{2}   4 ;
    0   0   \namedbox{c33}{\color<3->{seq-green}10}   30
  }$
\begin{tikzpicture}[remember picture, overlay]
  \node<2-> [draw, thick, rounded corners, blue!50, fit=(c13) (c23)] (x) {};
  \node<2-> [blue!50, below=8mm of x, xshift=-5mm] (expl) {We want these to be zero.};
  \draw<2-> [->, blue!50] 
    let \p1=($(x.west) - (2mm,0)$) in
      (expl.north -| \p1) to[out=90,in=-135] ($(x.south west) + .5*(1mm,1mm)$);
  \node<3-> [seq-green, below=1mm of expl, xshift=8mm] (expl2) 
    {Let's make this a $1$ first.};
  \draw<3-> [->, shorten >=.5mm, seq-green] 
      ($(expl2.north -| expl.east) + (1mm,0)$) 
        -- ($(expl.east) + (1mm,0)$)
        to[out=90,in=0] ($(expl.north east) + (-2mm,1mm)$)
        -- ($(expl.north -| c33.south) + (2mm,1mm)$)
        to[out=180,in=-90] ($(expl.north -| c33.south) + (0,3mm)$)
        -- (c33.south);
\end{tikzpicture}%
\begin{webonly}%
\rowop{$R_3 = R_3\divsymb 10$}{$\amat{
        1   0   -1   -2 ;
        0   1   2   4 ;
        0   0   \r1   3
      }$}\\
\rowop{$R_1 = R_1+R_3$}{$\amat{
        1   0   \r0   1 ;
        0   1   2   4 ;
        0   0   1   3
      }$}\\
\rowop{$R_2 = R_2 - 2R_3$}{$\amat{
        1   0   0   1 ;
        0   1   \r0   -2 ;
        0   0   1   3
      }$}\\[3mm]
\rowop{translates into}{$\syseq{
    x \+ \. \+ \. = 1;
    \. \+ y \+ \. = -2;
    \. \+ \. \+ z = 3}$}
\end{webonly}%

\pause[4]%
Success!

\pause\medskip
\alert{Check:}

\bigskip\centering
$\syseq{x + 2y + 3z = 6;
  2x - 3y + 2z = 14;
  3x + y - z = -2
}$
~\longsquiggly[substitute solution]~
\begin{webonly}
  $\syseq{
    1 + 2\cdot(-2) + 3\cdot 3 = 6;
    2\cdot 1 - 3\cdot(-2) + 2\cdot 3 = 14;
    3\cdot 1 + (-2) - 3 = -2
  }$\bigcheck 
\end{webonly}

\end{frame}


%%%%%%%%%%%%%%%%%%%%%%%%%%%%%%%%%%%%%%%%%%%%%%%%%%%%%%%%%%%%%%%%%%%

\begin{frame}
\frametitle{Row Equivalence}

\begin{bluebox}[Important]{.7\textwidth}
  The process of doing row operations to a matrix does not change the solution
  set of the corresponding linear equations!
\end{bluebox}

\pause

\vfill

\begin{defn}
  Two matrices are called \textbf{row equivalent} if one can be obtained from
  the other by doing some number of elementary row operations.
\end{defn}

\pause\bigskip
So the linear equations of row-equivalent matrices have the
\emph{same solution set}.

\vfill

\end{frame}


%%%%%%%%%%%%%%%%%%%%%%%%%%%%%%%%%%%%%%%%%%%%%%%%%%%%%%%%%%%%%%%%%%%

\begin{frame}
\frametitle{A Bad Example}

\vskip-3mm
\begin{eg}
  Solve the system of equations
  \[\syseq{x + y = 2;
    3x + 4y = 5;
    4x + 5y = 9
  }\]
\end{eg}

\pause
Let's try doing row operations:

\bigskip

\def\r{\color{red}}
\begin{center}\hskip 2cm\begin{minipage}{.75\linewidth}
\pause\hfill
$\amat{
\color<4->{red}\phantom-1 1 2;
\namedbox{a21}{3} 4 5;
\namedbox{a31}{4} 5 9}$%
\begin{tikzpicture}[remember picture, overlay]
  \node<4-> [draw, thick, rounded corners, blue!50, fit=(a21) (a31)] (x) {};
  \node<4-> [blue!50, left=2mm of x, xshift=-5mm, align=center] (expl) 
    {First clear these by\\
     subtracting multiples\\
     of the first row.};
  \draw<4-> [->, blue!50] (expl.east) -- (x.west);
\end{tikzpicture}%
\begin{webonly}%
\rowop{$R_2 = R_2 - 3R_1$}{$\amat{ 
    1 1 2; {\r0} 1 {-1};
  4 5 9}$\hskip 8mm}\par
\rowop{$R_3 = R_3 - 4R_1$}{$\amat{
  1 1 2;
  0 1 {-1};
  {\r0} 1 1}$\hskip 8mm}\par\vskip3mm
\end{webonly}%
\pause[5]\hfill$\amat{
  1 1 2;
  0 \color<6->{red}1 {-1};
  0 \namedbox{b32}{1} 1}$%
\begin{tikzpicture}[remember picture, overlay]
  \node<6-> [draw, thick, rounded corners, blue!50, fit=(b32)] (y) {};
  \node<6-> [blue!50, align=center] at (y.west -| expl.south) (expl2) 
    {Now clear this by\\
     subtracting\\
     the second row.};
  \draw<6-> [->, blue!50] (expl2.east) -- (y.west);
\end{tikzpicture}%
\begin{webonly}%
\rowop{$R_3 = R_3 - R_2$}{$\amat{
  1 1 2;
  0 1 {-1};
  0 {\r0} 2}$\hskip 8mm}
\end{webonly}%
\end{minipage}\end{center}%

\end{frame}


%%%%%%%%%%%%%%%%%%%%%%%%%%%%%%%%%%%%%%%%%%%%%%%%%%%%%%%%%%%%%%%%%%%

\begin{frame}
\frametitle{A Bad Example}
\framesubtitle{Continued}

\begin{center}
  $\amat{
  1 1 2;
  0 1 -1;
  0 0 2}$
  \;\longsquiggly[translates into]\;
  \begin{webonly}\def\r{\color{red}}
    $\syseq{x + y = 2; \. \+ y = -1; \. \+ \r0 \r= \r2}$
  \end{webonly}
\end{center}

\pause\medskip
In other words, the original equations
\begin{center}
  $\syseq{x + y = 2;
    3x + 4y = 5;
    4x + 5y = 9
  }$
  \qquad have the same solutions as \qquad
  $\syseq{x + y = 2; \. \+ y = -1; \. \+ 0 = 2}$
\end{center}

\pause\medskip
But the latter system obviously has no solutions (there is no way to make them
all true), so our original system has no solutions either.

\pause
\bigskip

\begin{defn}
  A system of equations is called \textbf{inconsistent} if it has no solution.
  It is \textbf{consistent} otherwise.
\end{defn}

\end{frame}


%%% Local Variables:
%%% TeX-master: "../slides"
%%% End:
