
\titleframe{Chapter 3}{Determinants}
\titleframe{Section 3.1}{Introduction to Determinants}


%%%%%%%%%%%%%%%%%%%%%%%%%%%%%%%%%%%%%%%%%%%%%%%%%%%%%%%%%%%%%%%%%%%

\begin{frame}
\frametitle{Orientation}

\alert{Recall:} This course is about learning to:

\begin{itemize}
\item \textcolor<2->{gray}{Solve the matrix equation $Ax = b$}\\
  \pause
  We've said most of what we'll say about this topic now.
  \pause
\item Solve the matrix equation $Ax = \lambda x$ (eigenvalue problem)\\
  \pause
  We are now aiming at this.
  \pause
\item \textcolor{gray}{Almost solve the equation $Ax = b$}\\
  \pause
  This will happen later.
\end{itemize}

\pause\bigskip
The next topic is \emph{determinants}.  

\pause\bigskip
Dan Margalit has written some notes
which, in my opinion, explain the topic in a much better way than Lay does.
\pause
(Both cover the same material.)

\pause\medskip
Prof.\ Margalit's notes are the primary reference for Chapter~3.

\end{frame}


%%%%%%%%%%%%%%%%%%%%%%%%%%%%%%%%%%%%%%%%%%%%%%%%%%%%%%%%%%%%%%%%%%%

\begin{frame}
\frametitle{The Idea of Determinants}

Let $A$ be an $n\times n$ matrix.
\pause
\textcolor{red}{Determinants are only for square matrices.}

\pause\medskip
The columns $v_1,v_2,\ldots,v_n$ give you $n$ vectors in $\R^n$.
\pause
These determine a \textbf{parallelepiped} $P$.

\begin{center}
\begin{tikzpicture}[thin border nodes]
  \filldraw[fill=seq-orange, fill opacity=.2, very thin]
    (0,0) -- (1,2) -- (3,3) -- (2,1) -- cycle;
  \draw[vector] (0,0) -- node[auto]{$v_1$} (1,2);
  \draw[vector] (0,0) -- node[auto,swap]{$v_2$} (2,1);
  \node at (1.5,1.5) {$P$};

  \begin{scope}[xshift=5cm, yshift=.2cm, scale=1.8, myxyz]
    \filldraw[fill=seq-orange, opacity=.2, very thin]
      (0,0,0) -- (0,-1,0) -- (.3,-1,1) -- (.3,0,1) -- cycle;
    \filldraw[fill=seq-orange, opacity=.2, very thin]
      (0,0,0) -- (0,-1,0) -- (1,-.8,0) -- (1,.2,0) -- cycle;
    \filldraw[fill=seq-orange, opacity=.2, shift={(0,-1,0)}, very thin]
      (0,0,0) -- (1,.2,0) -- (1.3,.2,1) -- (.3,0,1) -- cycle;
    \filldraw[fill=seq-orange, fill opacity=.2, very thin]
      (0,0,0) -- (1,.2,0) -- (1.3,.2,1) -- (.3,0,1) -- cycle;
    \filldraw[fill=seq-orange, fill opacity=.2, shift={(1,.2,0)}, very thin]
      (0,0,0) -- (0,-1,0) -- (.3,-1,1) -- (.3,0,1) -- cycle;
    \filldraw[fill=seq-orange, fill opacity=.2, shift={(.3,0,1)}, very thin]
      (0,0,0) -- (0,-1,0) -- (1,-.8,0) -- (1,.2,0) -- cycle;
    \draw[vector] (0,0,0) -- node[auto,swap] {$v_1$} (1,.2,0);
    \draw[vector, opacity=.4] (0,0,0) -- node[auto,swap] {$v_2$} (0,-1,0);
    \draw[vector] (0,0,0) -- node[auto] {$v_3$} (.3,0,1);
    \node at (1.2,0,1.2) {$P$};
  \end{scope}

\end{tikzpicture}
\end{center}

\pause\medskip
\alert{Observation:} the volume of $P$ is zero $\iff$ 
\pause
the columns are \emph{linearly dependent} ($P$ is ``flat'') $\iff$
\pause
the matrix $A$ is not invertible.

\pause\bigskip
The \textbf{determinant} of $A$ will be a number $\det(A)$ whose absolute value
is the volume of $P$.
\pause
In particular, $\det(A)\neq 0\iff$ $A$ is invertible.

\end{frame}


%%%%%%%%%%%%%%%%%%%%%%%%%%%%%%%%%%%%%%%%%%%%%%%%%%%%%%%%%%%%%%%%%%%

\begin{frame}
\frametitle{Determinants of $2\times 2$ Matrices}
\framesubtitle{Revisited}

We already have a formula in the $2\times 2$ case:
\[ \def\r{\textcolor{seq-red}} \def\g{\textcolor{seq-green}}
\det\mat{\r a \g b; \g c \r d} = \r{a}\r{d} - \g{b}\g{c}. \]
\pause
What does this have to do with volumes?
\pause

\vskip-.5cm
\begin{webonly}
\begin{center}
\begin{tikzpicture}[scale=.5, picture align top, thin border nodes]
  \draw[help lines, black!25] (-1,-1) grid (4,4);
  \filldraw[fill=seq-orange, fill opacity=.2, very thin]
    (0,0) -- (2,0) -- (3,3) -- (1,3) -- cycle;
  \draw[vector] (0,0) -- node[auto,swap] {$v_1$} (2,0);
  \draw[vector] (0,0) -- node[auto] {$v_2$} (1,3);
  \draw[|<->|, very thin] (1,3.5) -- node[auto, scale=.7] {base} (3,3.5);
  \draw[|<->|, very thin] (3.5,0) --
    node[scale=.7, rotate=-90, above] {height} (3.5,3);
  \point at (0,0);
\end{tikzpicture}\quad
\begin{minipage}[t]{0.6\linewidth}
  \[ v_1 = \vec{2 0} \qquad v_2 = \vec{1 3} \]
  The area of the parallelogram is
  \[ \text{base}\times\text{height} = 2\cdot 3
  = \left\vert\det\mat{2 1; 0 3}\right\vert. \]
\end{minipage}
\end{center}
\end{webonly}

The area of the parallelogram is always $|ad-bc|$.
\pause
If $v_1$ is not on the $x$-axis: it's a fun geometry problem!

\pause\medskip
\alert{Note:} this shows $\det(A)\neq 0\iff$ $A$ is invertible in this case.
\pause
(The volume is zero if and only if the columns are collinear.)

\pause\medskip
\alert{Question:} What does the sign of the determinant mean?
\note{If you swap the vectors, you multiply by $-1$.}

\end{frame}


%%%%%%%%%%%%%%%%%%%%%%%%%%%%%%%%%%%%%%%%%%%%%%%%%%%%%%%%%%%%%%%%%%%

\begin{frame}
\frametitle{Determinants of $3\times 3$ Matrices}
Here's the formula:
\[ \det\mat{a_{11} a_{12} a_{13}; a_{21} a_{22} a_{23}; a_{31} a_{32} a_{33}}
= \begin{aligned}
&\textcolor<5->{seq-green}{
  a_{11}a_{22}a_{33} + a_{12}a_{23}a_{31} + a_{13}a_{21}a_{32}} \\
&\quad
\textcolor<5->{seq-blue}{
  -a_{13}a_{22}a_{31} - a_{11}a_{23}a_{32} - a_{12}a_{21}a_{33}}
\end{aligned} \]

\pause
How on earth do you remember this?
\pause
Draw a bigger matrix, repeating the first two columns to the right:
\pause
\[\spaligndelims\vert\vert
+ \loopmat35{
  \pgfmathtruncatemacro{\jj}{mod(\j-1, 3) + 1}
  \appendnoexp\namedbox\append{{a-\i-\j}{a_{\i\jj}}}
}
  - \loopmat35{
  \pgfmathtruncatemacro{\jj}{mod(\j-1, 3) + 1}
  \appendnoexp\namedbox\append{{b-\i-\j}{a_{\i\jj}}}
} \]
\pause
Then add the products of the downward diagonals, and subtract the product of the
upward diagonals.
\begin{tikzpicture}[remember picture, overlay,
    very thick, opacity=.75]
  \draw[seq-green] (a-1-1.north west) -- (a-3-3.south east);
  \draw[seq-green] (a-1-2.north west) -- (a-3-4.south east);
  \draw[seq-green] (a-1-3.north west) -- (a-3-5.south east);
  \draw[seq-blue]  (b-1-3.north east) -- (b-3-1.south west);
  \draw[seq-blue]  (b-1-4.north east) -- (b-3-2.south west);
  \draw[seq-blue]  (b-1-5.north east) -- (b-3-3.south west);
\end{tikzpicture}
\pause
For example,
\[\hss \det\mat[r]{5 1 0; -1 3 2; 4 0 -1} =
\begin{webonly}
\spaligndelims\vert\vert
\def\pm{\phantom-}
\mat[r]{
  \namedbox{c-1-1}{\pm5} \namedbox{c-1-2}{\pm1} \namedbox{c-1-3}{\pm0}
  \namedbox{c-1-4}{\pm5} \namedbox{c-1-5}{\pm1}; -1 \pm3 \pm2 -1 \pm3;
  \namedbox{c-3-1}{\pm4} \namedbox{c-3-2}{\pm0} \namedbox{c-3-3}{-1}
  \namedbox{c-3-4}{\pm4} \namedbox{c-3-5}{\pm0}}
= \textcolor{seq-green}{-15+8+0} \textcolor{seq-blue}{{}-0-0-1}
= -8
\end{webonly}
 \hss\]
\begin{webonly}
\begin{tikzpicture}[remember picture, overlay,
    very thick, opacity=.75]
  \draw[seq-green] (c-1-1.north west) -- (c-3-3.south east);
  \draw[seq-green] (c-1-2.north west) -- (c-3-4.south east);
  \draw[seq-green] (c-1-3.north west) -- (c-3-5.south east);
  \draw[seq-blue]  (c-1-3.north east) -- (c-3-1.south west);
  \draw[seq-blue]  (c-1-4.north east) -- (c-3-2.south west);
  \draw[seq-blue]  (c-1-5.north east) -- (c-3-3.south west);
\end{tikzpicture}
\end{webonly}

\pause\medskip
What does this have to do with volumes?
\pause
Next time.

\end{frame}


%%%%%%%%%%%%%%%%%%%%%%%%%%%%%%%%%%%%%%%%%%%%%%%%%%%%%%%%%%%%%%%%%%%

\begin{frame}
\frametitle{A Formula for the Determinant}

\vskip-1mm
When $n\geq 4$, the determinant isn't just a sum of products of diagonals.
\pause
The formula is \emph{recursive}: you compute a larger determinant in terms of
smaller ones.

\pause\medskip
First some notation.  Let $A$ be an $n\times n$ matrix.\\
\pause\medskip
$\begin{aligned} A_{ij} &= \text{$ij$th \textbf{minor} of $A$}  \\
  &= (n-1)\times(n-1)\text{ matrix you get by deleting the $i$th row and $j$th column}
\end{aligned}$\\
\pause\medskip
$\begin{aligned}
  C_{ij} &= (-1)^{i+j} \det A_{ij} \\
  &= \text{$ij$th \textbf{cofactor} of $A$}
\end{aligned}$\\
\pause\medskip
The signs of the cofactors follow a checkerboard pattern:
\[ \def\a{\textcolor{seq-green}{\pmb{+}}} \def\b{\textcolor{seq-blue}{\pmb{-}}}
\mat{\a,\b,\a,\b ; \b,\a,\b,\a ; \a,\b,\a,\b; \b,\a,\b,\a}
\qquad\text{$\color{seq-violet}\pm$ in the $ij$ entry is the sign of $C_{ij}$} \]

\pause\vskip-5mm\null
\begin{defn}
  The \textbf{determinant} of an $n\times n$ matrix $A$ is 
  \abovedisplayskip=1pt\belowdisplayskip=1pt
  \[ \det(A) = \sum_{j=1}^n a_{1j} C_{1j}
  = a_{11}C_{11} + a_{12}C_{12} + \cdots + a_{1n}C_{1n}. \]
  \pause
  This formula is called \textbf{cofactor expansion along the first row}.
\end{defn}

\end{frame}


%%%%%%%%%%%%%%%%%%%%%%%%%%%%%%%%%%%%%%%%%%%%%%%%%%%%%%%%%%%%%%%%%%%

\begin{frame}
\frametitle{A Formula for the Determinant}
\framesubtitle{$1\times 1$ Matrices}

This is the beginning of the recursion.
\pause
\[ \det(\,a_{11}\,) = a_{11}. \]

\end{frame}


%%%%%%%%%%%%%%%%%%%%%%%%%%%%%%%%%%%%%%%%%%%%%%%%%%%%%%%%%%%%%%%%%%%

\def\aijmat#1#2#3#4{\loopmat#1#2{
  \appendnoexp\namedbox\append{{#3-\i-\j}{#4_{\i\j}}}}
}

\begin{frame}
\frametitle{A Formula for the Determinant}
\framesubtitle{$2\times 2$ Matrices}
\[ A = \mat{a_{11} a_{12} ; a_{21} a_{22}} \]
\pause
The minors are:
\begin{align*}
  A_{11} &= \webonlycmd{\aijmat22aa = (\,a_{22}\,)} &
  A_{12} &= \webonlycmd{\aijmat22ba = (\,a_{21}\,)} \\
  A_{21} &= \webonlycmd{\aijmat22ca = (\,a_{12}\,)} &
  A_{22} &= \webonlycmd{\aijmat22da = (\,a_{11}\,)}
\end{align*}
\begin{webonly}
\begin{tikzpicture}[remember picture, overlay,
    decoration={zigzag,segment length=1.5mm},
    seq-red, thick, opacity=.7, line join=round,
    every node/.style={seq-green, draw, thick, circle, inner sep=.5pt}]
  \node[fit=(a-1-1)] {};
  \draw[decorate] (a-1-1.west) -- (a-1-2.east);
  \draw[decorate] (a-1-1.north) -- (a-2-1.south);

  \node[fit=(b-1-2)] {};
  \draw[decorate] (b-1-1.west) -- (b-1-2.east);
  \draw[decorate] (b-1-2.north) -- (b-2-2.south);

  \node[fit=(c-2-1)] {};
  \draw[decorate] (c-2-1.west) -- (c-2-2.east);
  \draw[decorate] (c-2-1.south) -- (c-1-1.north);

  \node[fit=(d-2-2)] {};
  \draw[decorate] (d-2-1.west) -- (d-2-2.east);
  \draw[decorate] (d-1-2.north) -- (d-2-2.south);
 
\end{tikzpicture}
\end{webonly}

\pause
The cofactors are
\begin{align*}
  C_{11} &= \webonlycmd{+\det A_{11} = a_{22}} &
  C_{12} &= \webonlycmd{-\det A_{12} = -a_{21}} \\
  C_{21} &= \webonlycmd{-\det A_{21} = -a_{12}} &
  C_{22} &= \webonlycmd{+\det A_{22} = a_{11}}
\end{align*}

\pause
The determinant is
\[ \det A = a_{11}C_{11} + a_{12}C_{12} 
= a_{11}a_{22} - a_{12}a_{21}. \]

\end{frame}

%%%%%%%%%%%%%%%%%%%%%%%%%%%%%%%%%%%%%%%%%%%%%%%%%%%%%%%%%%%%%%%%%%%

\begin{frame}
\frametitle{A Formula for the Determinant}
\framesubtitle{$3\times 3$ Matrices}

\vskip-.5cm
\[ A = \loopmat33{\append{a_{\i\j}}} \]

\pause
The top row minors and cofactors are:
\begin{align*}
  A_{11} &= \webonlycmd{\aijmat33aa = \mat{a_{22} a_{23}; a_{32} a_{33}}} &
  C_{11} &= \webonlycmd{+\det\mat{a_{22} a_{23}; a_{32} a_{33}}} \\
  A_{12} &= \webonlycmd{\aijmat33ba = \mat{a_{21} a_{23}; a_{31} a_{33}}} &
  C_{12} &= \webonlycmd{-\det\mat{a_{21} a_{23}; a_{31} a_{33}}} \\
  A_{13} &= \webonlycmd{\aijmat33ca = \mat{a_{21} a_{22}; a_{31} a_{32}}} &
  C_{13} &= \webonlycmd{+\det\mat{a_{21} a_{22}; a_{31} a_{32}}}
\end{align*}
\begin{webonly}
\begin{tikzpicture}[remember picture, overlay,
    decoration={zigzag,segment length=1.5mm},
    seq-red, thick, opacity=.7, line join=round,
    every node/.style={seq-green, draw, thick, circle, inner sep=.5pt}]
  \node[fit=(a-1-1)] {};
  \draw[decorate] (a-1-1.north) -- (a-3-1.south);
  \draw[decorate] (a-1-1.west) -- (a-1-3.east);

  \node[fit=(b-1-2)] {};
  \draw[decorate] (b-1-2.north) -- (b-3-2.south);
  \draw[decorate] (b-1-1.west) -- (b-1-3.east);

  \node[fit=(c-1-3)] {};
  \draw[decorate] (c-1-3.north) -- (c-3-3.south);
  \draw[decorate] (c-1-1.west) -- (c-1-3.east);
\end{tikzpicture}
\end{webonly}

\pause
The determinant is the same formula as before (as it turns out):
\[\begin{split}
  \det A &= a_{11}C_{11} + a_{12}C_{12} + a_{13}C_{13} \\
  &= a_{11}\det\mat{a_{22} a_{23}; a_{32} a_{33}} 
   - a_{12}\det\mat{a_{21} a_{23}; a_{31} a_{33}}
   + a_{13}\det\mat{a_{21} a_{22}; a_{31} a_{32}}
\end{split} \]
\note{Always good to have more formulas for the same thing!}

\end{frame}


%%%%%%%%%%%%%%%%%%%%%%%%%%%%%%%%%%%%%%%%%%%%%%%%%%%%%%%%%%%%%%%%%%%

\begin{frame}
\frametitle{A Formula for the Determinant}
\framesubtitle{Example}

\[\hss\begin{aligned}
  \det\mat[r]{5 1 0; -1 3 2; 4 0 -1} &=
\webonlycmd{
  5\cdot\det\mat[r]{
    \namedbox{a-1-1}{5} \namedbox{a-1-2}{1} \namedbox{a-1-3}{0};
    -1 -3 2;
    \namedbox{a-3-1}{4} \namedbox{a-3-2}{0} \namedbox{a-3-3}{-1}}
  -1\cdot\det\mat[r]{
    \namedbox{b-1-1}{5} \namedbox{b-1-2}{1} \namedbox{b-1-3}{0};
    -1 \phantom-3 2;
    \namedbox{b-3-1}{4} \namedbox{b-3-2}{0} \namedbox{b-3-3}{-1}}
  }\\
  &\webonlycmd{
    \qquad\qquad+0\cdot\det\mat[r]{
    \namedbox{c-1-1}{5} \namedbox{c-1-2}{1} \namedbox{c-1-3}{0};
    -1 \phantom-3 2;
    \namedbox{c-3-1}{4} \namedbox{c-3-2}{0} \namedbox{c-3-3}{-1}}
  }\\
  &\webonlycmd{= 5\cdot\det\mat{3 2 ; 0 -1} - 1\cdot\det\mat{-1 2; 4 -1}
  + 0\cdot\det\mat{-1 3 ; 4 0}} \\
  &\webonlycmd{= 5\cdot(-3-0) - 1\cdot(1-8)} \\
  &\webonlycmd{= -15 + 7 = -8}
\end{aligned}\hss\]
\begin{webonly}
\begin{tikzpicture}[remember picture, overlay,
    decoration={zigzag,segment length=1.5mm},
    seq-red, thick, opacity=.7, line join=round,
    every node/.style={seq-green, draw, thick, circle, inner sep=.5pt}]
  \node[fit=(a-1-1)] {};
  \draw[decorate] (a-1-1.north) -- (a-3-1.south);
  \draw[decorate] (a-1-1.west) -- (a-1-3.east);

  \node[fit=(b-1-2)] {};
  \draw[decorate] (b-1-2.north) -- (b-3-2.south);
  \draw[decorate] (b-1-1.west) -- (b-1-3.east);

  \node[fit=(c-1-3)] {};
  \draw[decorate] (c-1-3.north) -- (c-1-3.north |- c-3-3.south);
  \draw[decorate] (c-1-1.west) -- (c-1-3.east);
\end{tikzpicture}
\end{webonly}
  
\end{frame}


%%%%%%%%%%%%%%%%%%%%%%%%%%%%%%%%%%%%%%%%%%%%%%%%%%%%%%%%%%%%%%%%%%%

\begin{frame}
\frametitle{$2n-1$ More Formulas for the Determinant}

\alert{Recall:} the formula
\[ \det(A) = \sum_{j=1}^n a_{1j} C_{1j}
= a_{11}C_{11} + a_{12}C_{12} + \cdots + a_{1n}C_{1n}. \]
is called \textbf{cofactor expansion along the first row.}
\pause
Actually, you can expand cofactors along any row or column you like!
\[\begin{split}
  \det A &= \sum_{j=1}^n a_{ij} C_{ij} \quad\text{for any fixed } i \\
  \det A &= \sum_{i=1}^n a_{ij} C_{ij} \quad\text{for any fixed } j
\end{split}\]
\pause
Try this with a row or a column with a lot of zeros.

\end{frame}


%%%%%%%%%%%%%%%%%%%%%%%%%%%%%%%%%%%%%%%%%%%%%%%%%%%%%%%%%%%%%%%%%%%

\begin{frame}
\frametitle{Cofactor Expansion}
\framesubtitle{Example}

\[A = \mat{2 1 \namedbox{a-1-3}{\color<2->{seq-red}0};
  1 1 \color<2->{seq-red}0;
  5 9 \namedbox{a-3-3}1} \]
It looks easiest to expand along the
\pause
third column:
\begin{tikzpicture}[remember picture, overlay]
  \node[fit=(a-1-3) (a-3-3), draw, thick, seq-green,
    rounded corners, inner sep=2pt] {};
\end{tikzpicture}
\[\begin{aligned} \det A &=
  \webonlycmd{
    0\cdot\det\bigg( \parbox{\widthof{don't}}{\centering don't\\care} \bigg) - 
    0\cdot\det\bigg( \parbox{\widthof{don't}}{\centering don't\\care} \bigg)
  + 1\cdot\det\mat{2 1 \namedbox{b-1-3}0;
    1 1 0;
    \namedbox{b-3-1}5 9 \namedbox{b-3-3}1}} \\
  &\webonlycmd{= \det\mat{2 1; 1 1} = 2-1 = 1 }
\end{aligned}\]
\begin{webonly}
\begin{tikzpicture}[remember picture, overlay,
    decoration={zigzag,segment length=1.5mm},
    seq-red, thick, opacity=.7, line join=round,
    every node/.style={seq-green, draw, thick, circle, inner sep=.5pt}]
  \node[fit=(b-3-3)] {};
  \draw[decorate] (b-1-3.north) -- (b-3-3.south);
  \draw[decorate] (b-3-1.west) -- (b-3-3.east);
\end{tikzpicture}
\end{webonly}

\end{frame}


%%%%%%%%%%%%%%%%%%%%%%%%%%%%%%%%%%%%%%%%%%%%%%%%%%%%%%%%%%%%%%%%%%%

\begin{pollframe}

\begin{poll}
\vskip-7mm\null
\begin{bluebox}[Poll]{.8\linewidth}
  \[ \det\mat[r]{
    1  7 -5 14  3 22;
    0 -2 -3 13 11  1;
    0  0 -1 -9  7 18;
    0  0  0  3  6 -8;
    0  0  0  0  1 -11;
    0  0  0  0  0 -1
  } ={} ? \]
  \centering
  A. $-6$ \quad B. $-3$ \quad C. $-2$ \quad D. $-1$ \quad E. $1$
  \quad F. $2$ \quad G. $3$ \quad H. $6$
\end{bluebox}

\pause\small
If you expand repeatedly along the first column, you get
\[\hss\begin{aligned}
  &\uncover<2->
  {1\cdot \det\mat[r]{-2 -3 13 11 1; 0 -1 -9 7 18; 0 0 3 6 -8; 0 0 0 1 -11; 0 0 0 0 -1}}  
  \uncover<3->
  {= 1\cdot(-2)\cdot\det\mat[r]{-1 -9 7 -18; 0 3 6 -8; 0 0 1 -11; 0 0 0 1}} \\
  &\uncover<4->{\qquad= 1\cdot(-2)\cdot(-1)\cdot\det\mat[r]{3 6 -8; 0 1 -11; 0 0 -1}}
  \uncover<5->{= 1\cdot(-2)\cdot(-1)\cdot 3\cdot\det\mat{1 -11; 0 -1}} \\
  &\uncover<6->{\qquad= 1\cdot(-2)\cdot(-1)\cdot 3\cdot 1\cdot(-1) = -6}
\end{aligned}\hss\]
\end{poll}

\end{pollframe}


%%%%%%%%%%%%%%%%%%%%%%%%%%%%%%%%%%%%%%%%%%%%%%%%%%%%%%%%%%%%%%%%%%%

\begin{frame}
\frametitle{The Determinant of an Upper-Triangular Matrix}

The computation in the poll works for any matrix that is
\emph{upper-triangular} (all entries below the main diagonal are zero).

\pause\vskip .75cm
\begin{thm}
  The determinant of an upper-triangular matrix is  the product of the diagonal entries:
  \[ \det\mat{
    \namedbox{a1}{a_{11}} a_{12} a_{13} \cdots, a_{1n};
    0      \namedbox{a2}{a_{22}} a_{23} \cdots, a_{2n};
    0      0      \namedbox{a3}{a_{33}} \cdots, a_{3n};
    \vdots,\vdots,\vdots,\ddots,\vdots;
    0      0      0      \cdots, \namedbox{an}{a_{nn}}}
  = a_{11}a_{22}a_{33}\cdots a_{nn}. \]
  \begin{tikzpicture}[remember picture, overlay]
    \node foreach \i in {1,2,3,n}
       [draw, circle, seq-green, thick, inner sep=.1pt, fit=(a\i)] {};
  \end{tikzpicture}
\end{thm}

\pause
The same is true for lower-triangular matrices.  (Repeatedly expand along the
first row.) 

\end{frame}


%%%%%%%%%%%%%%%%%%%%%%%%%%%%%%%%%%%%%%%%%%%%%%%%%%%%%%%%%%%%%%%%%%%

\begin{frame}
\frametitle{A Formula for the Inverse}
\framesubtitle{For fun---from \S3.3}

For $2\times 2$ matrices we had a nice formula for the inverse:
\[ A = \mat{a b; c d} \implies
A\inv = \frac 1{ad-bc}\mat{d -b; -c a}
\pause
= \frac 1{\det A}\mat{C_{11} C_{21}; C_{12} C_{22}}. \]
\pause

\begin{thm}
This last formula works for any $n\times n$ invertible matrix $A$:
\[ A\inv = \frac 1{\det A}
\mat{
  C_{11} C_{21} C_{31} \cdots, C_{n1} ;
  C_{12} C_{22} C_{32} \cdots, C_{n2} ;
  \namedbox{x}{C_{13}} C_{23} C_{33} \cdots, C_{n3} ;
  \vdots,\vdots,\vdots,\ddots, \vdots ;
  C_{1n} C_{2n} C_{3n} \cdots, C_{nn}
}
\uncover<4->{{}=\frac 1{\det A}\bigl( C_{ij} \bigr)^T}
 \]
\end{thm}

\pause
Note that the cofactors are ``transposed'': the $(i,j)$ entry of the matrix is
$C_{ji}$.
\begin{tikzpicture}[remember picture, overlay]
  \node[draw, circle, seq-green, thick, inner sep=.1pt, fit=(x)] (y) {};
  \path (y.north west) ++(-1.5,.5) node[blue!50] (expl) {$(3,1)$ entry};
  \draw[->, blue!50, shorten >=1pt] (expl.east) to[out=0,in=145] (y.145);
\end{tikzpicture}

\pause\medskip
The proof uses Cramer's rule.  See Dan Margalit's notes on the website for a
nice explanation.

\end{frame}


%%%%%%%%%%%%%%%%%%%%%%%%%%%%%%%%%%%%%%%%%%%%%%%%%%%%%%%%%%%%%%%%%%%

\begin{frame}
\frametitle{A Formula for the Inverse}
\framesubtitle{Example}

Compute $A\inv$, where
$A = \mat{1 0 1; 0 1 1; 1 1 0}$.

\pause
The minors are:
\begin{webonly}
\begin{align*}
  A_{11} &= \mat{1 1; 1 0} &
  A_{12} &= \mat{0 1; 1 0} &
  A_{13} &= \mat{0 1; 1 1} \\
  A_{21} &= \mat{0 1; 1 0} &
  A_{22} &= \mat{1 1; 1 0} &
  A_{23} &= \mat{1 0; 1 1} \\
  A_{31} &= \mat{0 1; 1 1} &
  A_{32} &= \mat{1 1; 0 1} &
  A_{33} &= \mat{1 0; 0 1}
\end{align*}
\end{webonly}
\pause
The cofactors are (don't forget to multiply by $(-1)^{i+j}$):
\begin{webonly}
\begin{align*}
  C_{11} &= -1  &  C_{12} &=  \phantom-1  &  C_{13} &= -1 \\
  C_{21} &= \phantom-1  &  C_{22} &= -1  &  C_{23} &= -1 \\
  C_{31} &= -1  &  C_{32} &= -1  &  C_{33} &=  \phantom-1
\end{align*}
\end{webonly}
\pause
The determinant is (expanding along the first row):
\[ \det A = \webonlycmd{1\cdot C_{11} + 0\cdot C_{12} + 1\cdot C_{13} = -2} \]

\end{frame}


%%%%%%%%%%%%%%%%%%%%%%%%%%%%%%%%%%%%%%%%%%%%%%%%%%%%%%%%%%%%%%%%%%%

\begin{frame}
\frametitle{A Formula for the Inverse}
\framesubtitle{Example, continued}

Compute $A\inv$, where
$A = \mat{1 0 1; 0 1 1; 1 1 0}$.

\pause
The inverse is
\[ A\inv = \webonlycmd{\frac 1{\det A}\loopmat33{\append{C_{\j\i}}}
= -\frac12\mat[r]{-1 1 -1; 1 -1 -1; -1 -1 1}.} \]

\pause
Check:
\begin{webonly}
\[ \mat{1 0 1; 0 1 1; 1 1 0}\cdot -\frac12\mat[r]{-1 1 -1; 1 -1 -1; -1 -1 1}
= \mat{1 0 0; 0 1 0; 0 0 1}. \bigcheck[\quad] \]
\end{webonly}

\end{frame}


%%% Local Variables:
%%% TeX-master: "../slides"
%%% End:
