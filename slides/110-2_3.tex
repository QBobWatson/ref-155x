
% JDR: This should be about 1/3 of one lecture.  It's meant to be done in the
%   same class period as section 2.2.

\titleframe{Section 2.3}{Characterization of Invertible Matrices}

\usetikzlibrary{decorations.pathreplacing}


%%%%%%%%%%%%%%%%%%%%%%%%%%%%%%%%%%%%%%%%%%%%%%%%%%%%%%%%%%%%%%%%%%%

\begin{frame}
\frametitle{Invertible Transformations}

\vskip-3mm
\begin{defn}
  A transformation $T\colon\R^n\to\R^n$ is \textbf{invertible} if there exists
  another transformation $U\colon\R^n\to\R^n$ such that 
  \[ T\circ U(x) = x \sptxt{and} U\circ T(x) = x \]
  for all $x$ in $\R^n$.
  \pause
  In this case we say $U$ is the \textbf{inverse} of $T$, and we write 
  $U = T\inv$.
\end{defn}

\pause\smallskip
In other words, $T(U(x)) = x$, so $T$ ``undoes'' $U$, and likewise $U$
``undoes'' $T$.

\pause\smallskip
\begin{bluebox}[Fact]{.5\linewidth}
  A transformation $T$ is invertible if and only if it is both one-to-one and
  onto.
\end{bluebox}

\begin{webonly}
If $T$ is one-to-one and onto, this means
for every $y$ in $\R^n$, there is a unique $x$ in $\R^n$ such that $T(x) = y$.
Then $T\inv(y) = x$.
\end{webonly}

\end{frame}


%%%%%%%%%%%%%%%%%%%%%%%%%%%%%%%%%%%%%%%%%%%%%%%%%%%%%%%%%%%%%%%%%%%

\begin{frame}
\frametitle{Invertible Transformations}
\framesubtitle{Examples}

\vskip-2mm
  Let $T = $ counterclockwise rotation in the plane by $45^\circ$.  What is $T\inv$?
  
  \def\theo{\includegraphics[width=2cm]{figures/theo2.jpg}}%
\hfill\begin{tikzpicture}
  \begin{scope}
    \clip (0,0) circle[radius=1cm];
    \node (theo1) at (0,0) {\theo};
  \end{scope}
  \draw[->,opacity=.3] (-1.2,0) -- (1.2,0);
  \draw[->,opacity=.3] (0,-1.2) -- (0,1.2);

  \begin{scope}[xshift=4cm]
  \begin{scope}
    \clip (0,0) circle[radius=1cm];
    \node[rotate=45] (theo2) at (0,0) {\theo};
  \end{scope}
  \draw[->,opacity=.3] (-1.2,0) -- (1.2,0);
  \draw[->,opacity=.3] (0,-1.2) -- (0,1.2);
  \end{scope}

  \draw[->] (theo1.20) to["$T$", out=20, in=160] (theo2.115);

  \begin{scope}[xshift=8cm]
  \begin{scope}
    \clip (0,0) circle[radius=1cm];
    \node<2-> (theo3) at (0,0) {\theo};
  \end{scope}
  \draw<2->[->,opacity=.3] (-1.2,0) -- (1.2,0);
  \draw<2->[->,opacity=.3] (0,-1.2) -- (0,1.2);
  \end{scope}

  \draw<2->[->] (theo2.-25) to["$T\inv$", out=20, in=160] (theo3.160);
  \useasboundingbox (9.2,0);
\end{tikzpicture}\hfill\null\\[1mm]
\pause
$T\inv$ is \emph{clockwise} rotation by $45^\circ$.

\pause\medskip
  Let $T = $ shrinking by a factor of $2/3$ in the plane.  What is $T\inv$?

  \def\theo{\includegraphics[width=2cm]{figures/theo6.jpg}}%
\hfill\begin{tikzpicture}
  \begin{scope}
    \node (theo1) at (0,0) {\theo};
  \end{scope}
  \draw[->,opacity=.3] (-1.2,0) -- (1.2,0);
  \draw[->,opacity=.3] (0,-1.2) -- (0,1.2);

  \begin{scope}[xshift=4cm]
  \begin{scope}
    \node[scale=2/3] at (0,0) {\theo};
    \node[minimum size={2cm+.3333em}] (theo2) at (0,0) {};
  \end{scope}
  \draw[->,opacity=.3] (-1.2,0) -- (1.2,0);
  \draw[->,opacity=.3] (0,-1.2) -- (0,1.2);
  \end{scope}

  \draw[->] (theo1.20) to["$T$", out=20, in=160] (theo2.160);

  \begin{scope}[xshift=8cm]
  \begin{scope}
    \node<4-> (theo3) at (0,0) {\theo};
  \end{scope}
  \draw<4->[->,opacity=.3] (-1.2,0) -- (1.2,0);
  \draw<4->[->,opacity=.3] (0,-1.2) -- (0,1.2);
  \end{scope}

  \draw<4->[->] (theo2.20) to["$T\inv$", out=20, in=160] (theo3.160);
  \useasboundingbox (9.2,0);
\end{tikzpicture}\hfill\null\\[1mm]
\pause
$T\inv$ is \emph{stretching} by $3/2$.

\pause\medskip
Let $T = $ projection onto the $x$-axis.  What is $T\inv$?
\pause
It is not invertible: you can't undo it.

\end{frame}


%%%%%%%%%%%%%%%%%%%%%%%%%%%%%%%%%%%%%%%%%%%%%%%%%%%%%%%%%%%%%%%%%%%

\begin{frame}
\frametitle{Invertible Linear Transformations}

If $T\colon\R^n\to\R^n$ is an invertible \emph{linear} transformation with
matrix $A$, then what is the matrix for $T\inv$?

\medskip
\begin{webonly}
Let $B$ be the matrix for $T\inv$.
We know $T\circ T\inv$ has matrix $AB$, so for all $x$,
\[ ABx = T\circ T\inv(x) = x. \]
Hence $AB = I_n$, so $B = A\inv$.
\end{webonly}

\pause\vskip 1cm
\begin{bluebox}[Fact]{.85\linewidth}
  If $T$ is an invertible linear transformation with matrix $A$, then\\[1mm]
  $T\inv$ is an invertible linear transformation with matrix $A\inv$.
\end{bluebox}

\end{frame}


%%%%%%%%%%%%%%%%%%%%%%%%%%%%%%%%%%%%%%%%%%%%%%%%%%%%%%%%%%%%%%%%%%%

\begin{frame}
\frametitle{Invertible Linear Transformations}
\framesubtitle{Examples}

\displayskips{4pt}
Let $T = $ counterclockwise rotation in the plane by $45^\circ$.  Its matrix is
\webonlycmd{
\[ A = \mat{\cos(45^\circ) -\sin(45^\circ) ; \sin(45^\circ) \cos(45^\circ)}
= \frac 1{\sqrt 2}\mat{1 -1; 1 1}. \]}%
Then $T\inv = $ counterclockwise rotation by $-45^\circ$.  Its matrix is
\webonlycmd{
\[ B = \mat{\cos(-45^\circ) -\sin(-45^\circ) ; \sin(-45^\circ) \cos(-45^\circ)}
= \frac 1{\sqrt 2}\mat{1 1; -1 1}. \]}%
\rlap{\alert{Check:}}\hfill
\webonlycmd{
$\displaystyle AB = \frac 12\mat{1 -1; 1 1}\mat{1 1; -1 1} = 
\mat{1 0 ; 0 1} \bigcheck[\quad]$}
\hfill\null

\pause\medskip
Let $T = $ shrinking by a factor of $2/3$ in the plane.  Its matrix is
\webonlycmd{
\[ A = \mat{2/3 0; 0 2/3} \]}%
Then $T\inv = $ stretching by $3/2$.  Its matrix is
\webonlycmd{
\[ B = \mat{3/2 0; 0 3/2} \]}%
\rlap{\alert{Check:}}\hfill
\webonlycmd{
$\displaystyle AB = \mat{2/3 0; 0 2/3}\mat{3/2 0; 0 3/2} = \mat{1 0; 0 1}
 \bigcheck[\quad]$}
\hfill\null

\end{frame}


%%%%%%%%%%%%%%%%%%%%%%%%%%%%%%%%%%%%%%%%%%%%%%%%%%%%%%%%%%%%%%%%%%%

\begin{frame}
\frametitle{The Invertible Matrix Theorem}
\framesubtitle{A.K.A.\ The Really Big Theorem of Math~1553}

\vskip-3mm
\begin{oneoffthm}{The Invertible Matrix Theorem}
  Let $A$ be an $n\times n$ matrix, and let $T\colon\R^n\to\R^n$ be the
  linear transformation $T(x) = Ax$.  The following statements are equivalent.
  \begin{enumerate}
  \item \namedbox{first}{$A$} is invertible.
    \pause
  \item $T$ is invertible.
    \pause
  \item $A$ is row equivalent to $I_n$.
    \pause
  \item $A$ has $n$ pivots.
    \pause
  \item $Ax=0$ has only the trivial solution.
    \pause
  \item The columns of $A$ are linearly independent.
    \pause
  \item $T$ is one-to-one.
    \pause
  \item $Ax = b$ is consistent for all $b$ in $\R^n$.
    \pause
  \item The columns of $A$ span $\R^n$.
    \pause
  \item $T$ is onto.
    \pause
  \item $A$ has a left inverse (there exists $B$ such that $BA = I_n$).
    \pause
  \item $A$ has a right inverse (there exists $B$ such that $AB = I_n$).
    \pause
  \item \namedbox{last}{$A^T$} is invertible.
  \end{enumerate}
  \pause
  \begin{tikzpicture}[remember picture, overlay]
    \draw[decorate,decoration={brace, amplitude=3mm}, thick, red]
      let \p1=($(current page.east) + (-2cm,0)$) in
        (first.north -| \p1) -- (last.south -| \p1)
        node[rotate=-90, midway, anchor=south, yshift=.2cm]
          {you really have to know these};
  \end{tikzpicture}
\end{oneoffthm}
\end{frame}


%%%%%%%%%%%%%%%%%%%%%%%%%%%%%%%%%%%%%%%%%%%%%%%%%%%%%%%%%%%%%%%%%%%

\begin{frame}
\frametitle{The Invertible Matrix Theorem}
\framesubtitle{Summary}

There are two kinds of \textcolor{red}{\emph{square}} matrices:
\pause
\begin{enumerate}
\item invertible (non-singular), and
\pause
\item non-invertible (singular).
\end{enumerate}

\pause\bigskip
For invertible matrices, all statements of the Invertible Matrix Theorem are true.

\pause\bigskip
For non-invertible matrices, all statements of the Invertible Matrix Theorem are
false.

\pause\bigskip
\alert{Strong recommendation:} 
If you want to understand invertible matrices, go through all of the conditions
of the IMT and try to figure out on your own (or at least with help from the
book) why they're all equivalent.

\pause\bigskip
You know enough at this point to be able to reduce all of the statements to
assertions about the pivots of a square matrix.

\end{frame}


%%% Local Variables:
%%% TeX-master: "../slides"
%%% End:
