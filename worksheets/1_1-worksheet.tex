
\documentclass[12pt]{amsart}
\usepackage{graphicx, url, verbatim}

\def\styledir{../style/}
\usepackage{\styledir jdr-style, \styledir jdr-linalg}
\usepackage[homework]{\styledir jdr-hwexam}

\title{Math 1553 Worksheet \S1.2}
\duedate{DATE HERE}
\name{Solutions}

\answerstrue

\begin{document}

\hwtitle

\begin{problems}
\problem
  \begin{subproblems}
  \subproblem Which of the following matrices are in row echelon form?  Which
    are in reduced row echelon form?

  \subproblem Which entries are the pivots?  Which are the pivot columns?
  \end{subproblems}
\[
\ifanswers\def\r{\color{seq-red}}\else\def\r{}\fi
\ifanswers\def\b{\color{seq-blue}}\else\def\b{}\fi
\mat{
  \r1 \b0 \b0 0;
  \b0 \r1 \b0 0;
  \b0 \b0 \r1 1}
\qquad
\mat{
  \r1 \b0 1 \b0;
  \b0 \r1 1 \b0;
  \b0 \b0 0 \r1
}
\qquad
\mat{
  \r1 \b1 0 \b1 \b1;
  \b0 \r2 0 \b2 \b2;
  \b0 \b0 0 \r3 \b3;
  \b0 \b0 0 \b0 \r4
 }
\qquad
\mat{
  \r1 1 \b0 1;
  \b0 0 \r1 1;
  \b0 0 \b0 0
 }
\]

\begin{solution}
  The first, second, and fourth matrices are in reduced row echelon form; the
  last is in row echelon form.
  The pivots are in red; the other entries in the pivot columns are in blue.
\end{solution}

\ifanswers\else\vfill\fi

\problem
  \begin{subproblems}
  \subproblem Row reduce the following matrices to reduced row echelon form.
  \subproblem If these are augmented matrices for a linear system (with the last
    column being after the $=$ sign), then which are inconsistent?  Which have a
    \emph{unique} solution?
  \end{subproblems}
\ifanswers\else
\[
\mat{
  1 2 3 4;
  4 5 6 7;
  6 7 8 9
 }
\qquad
\mat{
  1 3 5 7;
  3 5 7 9;
  5 7 9 1
 }
\qquad
\mat{
3 -4 2 0;
-8 12 -4 0;
-6 8 -1 0
 }
\]

\vfill
\fi

\begin{solution}
  \def\rowop#1#2{%
    \null\hfill%
    \hbox to 0.2\linewidth{\hss\longsquiggly[#1]}%
    \hbox to 0.4\linewidth{\hskip 1em #2\hss}%
  }
  \def\r{\color{seq-red}}

  \hfill$\mat{
    1 2 3 4;
    4 5 6 7;
    6 7 8 9
  }$\rowop{$R_2 = R_2-4R_1$}{$\mat{
      1 2 3 4;
      \r0 -3 -6 -9;
      6 7 8 9
    }$}\\[1mm]%
  \rowop{$R_3 = R_3-6R_1$}{$\mat{
      1 2 3 4;
      0 -3 -6 -9;
      \r0 -5 -10 -15
    }$}\\[1mm]
  \rowop{$R_2 = R_2 \divsymb -3$}{$\mat{
      1 2 3 4;
      0 \r1 2 3;
      0 -5 -10 -15
    }$}\\[1mm]
  \rowop{$R_3 = R_3 + 5R_2$}{$\mat{
      1 2 3 4;
      0 1 2 3;
      0 \r0 0 0
    }$}\\[1mm]
  \rowop{$R_1 = R_1 - 2R_2$}{$\mat{
      1 \r0 -1 -2;
      0 1 2 3;
      0 0 0 0
    }$}\\[1mm]
  This is the reduced row echelon form.  Interpreted as an augmented matrix, it
  corresponds to the system of linear equations
  \[ \syseq{x \+ \. - z = -2; \. \+ y + 2z = 3; \. \+ \. \+ 0 = 0\rlap.} \]
  This system is consistent, but since $z$ is a free variable, it does not have
  a \emph{unique} solution.

  \bigskip
  \hfill$\mat{
    1 3 5 7;
    3 5 7 9;
    5 7 9 1
  }$\rowop{$R_2 = R_2-3R_1$}{$\mat{
      1 3 5 7;
      \r0 -4 -8 -12;
      5 7 9 1
    }$}\\[1mm]
  \rowop{$R_3 = R_3-5R_1$}{$\mat{
      1 3 5 7;
      0 -4 -8 -12;
      \r0 -8 -16 -34
    }$}\\[1mm]
  \rowop{$R_2 = R_2 \divsymb -4$}{$\mat{
      1 3 5 7;
      0 \r1 2 3;
      0 -8 -16 -34
    }$}\\[1mm]
  \rowop{$R_3 = R_3 + 8R_2$}{$\mat{
      1 3 5 7;
      0 1 2 3;
      0 \r0 0 -10
    }$}\\[1mm]
  \rowop{$R_3 = R_3\divsymb -10$}{$\mat{
      1 3 5 7;
      0 1 2 3;
      0 0 0 \r1
    }$}\\[1mm]
  \rowop{$R_1 = R_1-7R_3$}{$\mat{
      1 3 5 \r0;
      0 1 2 3;
      0 0 0 1
    }$}\\[1mm]
  \rowop{$R_2 = R_2-3R_3$}{$\mat{
      1 3 5 0;
      0 1 2 \r0;
      0 0 0 1
    }$}\\[1mm]
  \rowop{$R_1 = R_1-3R_2$}{$\mat{
      1 \r0 -1 0;
      0 1 2 0;
      0 0 0 1
    }$}\\[1mm]
  This is the reduced row echelon form.  Interpreted as an augmented matrix, it
  corresponds to the system of linear equations
  \[ \syseq{
    x \+ \. - z = 0;
    \. \+ y + 2z = 0;
    \. \+ \. \+ 0 = 1\rlap{,}
  } \]
  which is inconsistent.

  \bigskip
  \hfill$\mat{
    3 -4 2 0;
    -8 12 -4 0;
    -6 8 -1 0
  }$\rowop{$R_2 = R_2 + 3R_1$}{$\mat{
      3 -4 2 0;
      \r1 0 2 0;
      -6 8 -1 0
    }$}\\[1mm]
  \rowop{$R_1 \ToT R_2$}{$\mat{
      \r1 0 2 0;
      3 -4 2 0;
      -6 8 -1 0
    }$}\\[1mm]
  \rowop{$R_2 = R_2-3R_1$}{$\mat{
      1 0 2 0;
      \r0 -4 -4 0;
      -6 8 -1 0
    }$}\\[1mm]
  \rowop{$R_3 = R_3+6R_1$}{$\mat{
      1 0 2 0;
      0 -4 -4 0;
      \r0 8 11 0
    }$}\\[1mm]
  \rowop{$R_2 = R_2 \divsymb -4$}{$\mat{
      1 0 2 0;
      0 \r1 1 0;
      0 8 11 0
    }$}\\[1mm]
  \rowop{$R_3 = R_3-8R_2$}{$\mat{
      1 0 2 0;
      0 1 1 0;
      0 \r0 3 0
    }$}\\[1mm]
  \rowop{$R_3 = R_3 \divsymb 3$}{$\mat{
      1 0 2 0;
      0 1 1 0;
      0 0 \r1 0
    }$}\\[1mm]
  \rowop{$R_1 = R_1-2R_3$}{$\mat{
      1 0 \r0 0;
      0 1 1 0;
      0 0 1 0
    }$}\\[1mm]
  \rowop{$R_2 = R_2 - R_3$}{$\mat{
      1 0 0 0;
      0 1 \r0 0;
      0 0 1 0
    }$}\\[1mm]
  This is the reduced row echelon form.  Interpreted as an augmented matrix, it
  corresponds to the system of linear equations
  \[ \syseq{x = 0; y = 0; z = 0\rlap{,}}  \]
  which has a unique solution.

\end{solution}

\end{problems}


\end{document}
